\documentclass[11pt]{amsart}

 \usepackage{hyperref}
\usepackage{tikz}
\usetikzlibrary{arrows, decorations.markings, decorations.pathmorphing, decorations.pathreplacing, shapes.arrows, patterns, calc}
\usepackage{url}

\tikzset{% arrow close to the source: the 0.2 determines where the arrow is drawn
  ->-/.style={decoration={markings, mark=at position 0.5 with {\arrow{to}}},
              postaction={decorate}},
}

\tikzset{% arrow close to the source: the 0.2 determines where the arrow is drawn
  -<-/.style={decoration={markings, mark=at position 0.5 with {\arrow{to reversed}}},
              postaction={decorate}},
}

\tikzset{% arrow close to the source: the 0.2 determines where the arrow is drawn
  dbl->-/.style={
double, 
double equal sign distance,
shorten >= 1pt,
shorten <= 1pt,
 decoration={markings, mark=at position 0.5 with {\arrow{implies}}},
              postaction={decorate}},
}


\tikzset{% arrow close to the source: the 0.2 determines where the arrow is drawn
  dbl-<-/.style={
double, 
double equal sign distance,
shorten >= 1pt,
shorten <= 1pt,
 decoration={markings, mark=at position 0.5 with {\arrowreversed{implies}}},
              postaction={decorate}},
}






\usepackage{amsmath,amsthm}
\textwidth=14.5cm \oddsidemargin=1cm  \evensidemargin=1cm\setlength{\parskip}{10pt} \setlength{\headsep}{20pt}

\pdfmapfile{+mathpple.map}
\usepackage{mathrsfs}
\usepackage{amscd,amssymb, amsfonts, verbatim,subfigure, enumerate}
\usepackage[mathcal]{eucal}
\usepackage[super]{nth}

\usepackage{mathpazo}



\linespread{1.2}  
\usepackage{color,slashed}



\setcounter{tocdepth}{1}

\renewcommand{\c}{\mf{c}}
\newcommand{\Lap}{\tr}
\newcommand{\dbar}{\br{\partial}}
\newcommand{\Dirac}{\slashed{\partial}}
\newcommand{\RHom}{\mbb{R}\op{Hom}}
\newcommand{\CP}{\mathbb{CP}}
\newcommand{\wbar}{\br{w}} 
\newcommand{\del}{\partial}
\renewcommand{\sl}{\mathfrak{sl}}
\renewcommand{\L}{\mathscr{L}}
\newcommand{\red}[1]{{\color{red}{#1}}} 
\newcommand{\zbar}{\br{z}}
\newcommand{\so}{\mathfrak{so}}
\newcommand{\Spin}{\op{Spin}}
\newcommand{\PV}{\op{PV}}
\newcommand{\GL}{\op{GL}}
\newcommand{\h}{\mathfrak{h}}
\newcommand{\Per}{\mscr{P}}
\newcommand{\dpa}[1]{\frac{\partial}{\partial #1}}
\newcommand{\dpas}[1]{\tfrac{\partial}{\partial #1}}
\newcommand{\Res}{\op{Res}}
\newcommand{\Sup}{\op{Sup}}
\newcommand{\Sing}{\op{Sing}}
\newcommand{\Diag}{\triangle}





\newcommand{\Obs}{\op{Obs}}
\newcommand{\su}{\mathfrak{su}}
\newcommand{\Der}{\op{Der}}

\newcommand{\eps}{\epsilon}
\newcommand{\g}{\mathfrak{g}}

\newcommand{\Hol}{\op{Hol}}

\renewcommand{\Re}{\op{Re}}


\newcommand{\xto}{\xrightarrow}




\newcommand{\what}{\widehat}
\newcommand{\tr}{\triangle}



\newcommand{\til}{\widetilde}
\newcommand{\mscr}{\mathscr}
\renewcommand{\det}{\operatorname{det}}





\newcommand{\br}{\overline}

\newcommand{\iso}{\cong}
\newcommand{\C}{\mathbb C}
\newcommand{\CC}{\mathbb C}
\newcommand{\N}{\mathbb N}
\newcommand{\Q}{\mbb Q}
\newcommand{\rarr}{\rightarrow}
\newcommand{\larr}{\leftarrow}
\newcommand{\norm}[1]{\left\| #1 \right\|}
\newcommand{\Oo}{\mscr O}
\newcommand{\Z}{\mathbb Z}
\newcommand{\defeq}{\overset{\text{def}}{=}}
\newcommand{\into}{\hookrightarrow}


\newcommand{\mbf}{\mathbf}
\newcommand{\mbb}{\mathbb}
\newcommand{\mf}{\mathfrak}
\newcommand{\mc}{\mathcal}
\newcommand{\from}{\leftarrow}

\newcommand{\R}{\mbb R}
\renewcommand{\d}{\mathrm{d}}
\newcommand{\liminv}{ \varprojlim }
\newcommand{\limdir}{\varinjlim}
\newcommand{\dirlim}{\varinjlim}

\def\define{\overset{\rm def}{=}}
\def\ep{\varepsilon}
\def\zbar{{\overline{z}}}
\def\Weyl{{\rm Weyl}}
\def\Cl{{\rm C}\ell}
\def\U{{\rm U}}
\def\PV{{\rm PV}}
\def\thetabar{\Bar{\theta}}
\def\hotimes{\; \Hat{\otimes} \;} 
\def\C{{\rm C}}
\newcommand{\op}{\operatorname}
\def\ibar{{\Bar{i}}}
\def\jbar{{\Bar{j}}}
\def\CP{{\mathbb{CP}}}
\def\eps{{\epsilon}}
\def\what{\Hat}
\newcommand{\abs}[1]{\left| #1 \right|}
\newcommand{\ip}[1]{\left\langle #1 \right\rangle}
\def\oO{\cO}
\def\CP{{\mathbb{CP}}}
\def\bu{{\bullet}}
\def\DD{\mathbb{D}}
\def\cO{\mc O}
\def\ZZ{\mathbb{Z}}
\def\lie#1{\ensuremath{\mathfrak{#1}}}


\newcommand{\bfP}{\mathbf{P}}


\DeclareMathOperator*{\colim}{colim}
\DeclareMathOperator{\Aut}{Aut} \DeclareMathOperator{\End}{End}

\DeclareMathOperator{\Sym}{Sym} \DeclareMathOperator{\Hom}{Hom}
\DeclareMathOperator{\Spec}{Spec}
\DeclareMathOperator{\Diff}{Diff} 
\DeclareMathOperator{\Met}{Met} \DeclareMathOperator{\Vol}{Vol}
\DeclareMathOperator{\Tr}{Tr}

\def\brian#1{{\textcolor{blue!65!red}{BRW: {#1}}}}
\def\natalie#1{{\textcolor{green!65!black}{BRW: {#1}}}}


\def\beqn{\begin{equation}}
\def\eeqn{\end{equation}}










\newtheoremstyle{thm}% name
  {7pt}%      Space above
  {7pt}%      Space below
  {\itshape}%         Body font
  {}%         Indent amount (empty = no indent, \parindent = para indent)
  {\bf}% Thm head font
  {.}%        Punctuation after thm head
  {5pt}%     Space after thm head: " " = normal interword space;
         %       \newline = line-break
  {\thmnumber{#2 }\thmname{#1}\thmnote{ (#3)}}%         Thm head spec (can be left empty, meaning `normal')





\newtheoremstyle{def}% name
  {7pt}%      Space above
  {10pt}%      Space below
  {\itshape}%         Body font
  {}%         Indent amount (empty = no indent, \parindent = para indent)
  {\bf}% Thm head font
  {.}%        Punctuation after thm head
  {5pt}%     Space after thm head: " " = normal interword space;
         %       \newline = line-break
  {\thmnumber{#2} \thmname{#1}\thmnote{ (#3)}}%         Thm head spec (can be left empty, meaning `normal')





\newtheoremstyle{rem}% name
  {4pt}%      Space above
  {10pt}%      Space below
  {}%         Body font
  {}%         Indent amount (empty = no indent, \parindent = para indent)
  {\itshape}% Thm head font
  {:}%        Punctuation after thm head
  {3pt}%     Space after thm head: " " = normal interword space;
        %       \newline = line-break
  {}%         Thm head spec (can be left empty, meaning `normal')

\newtheoremstyle{texttheorem}% name
  {8pt}%      Space above
  {8pt}%      Space below
  {\itshape}%         Body font
  {}%         Indent amount (empty = no indent, \parindent = para indent)
  {\bf}% Thm head font
  {. \hspace{5pt}}%        Punctuation after thm head
  {3pt}%     Space after thm head: " " = normal interword space;
        %       \newline = line-break
  {}%         Thm head spec (can be left empty, meaning `normal')




\theoremstyle{thm}



\newtheorem*{claim}{Claim}
\newtheorem*{theorem*}{Theorem}
\newtheorem*{lemma*}{Lemma}
\newtheorem*{corollary*}{Corollary}
\newtheorem*{proposition*}{Proposition}
\newtheorem*{definition*}{Definition}
\newtheorem{ntheorem}{Theorem}
\newtheorem*{thmA}{Theorem A}
\newtheorem*{thmB}{Theorem B}
\newtheorem*{thmC}{Theorem C}
\newtheorem*{conjecture}{Conjecture}




\newtheorem{theorem}{Theorem}[subsection]
\newtheorem{thm-def}{Theorem/Definition}[theorem]
\newtheorem{proposition}[theorem]{Proposition}
\newtheorem{question}{Question}
\newtheorem*{question*}{Question}
\newtheorem{lemma}[theorem]{Lemma}
\newtheorem{sublemma}[theorem]{Sub-lemma}
\newtheorem{notation}[theorem]{Notation}
\newtheorem{corollary}[theorem]{Corollary}
\newtheorem{deflem}[theorem]{Definition-Lemma}
\newtheorem*{hope}{Hope}
\numberwithin{equation}{subsection}


\theoremstyle{def}
%\theoremstyle{definition}
\newtheorem{definition}[theorem]{Definition}
\newtheorem*{udefinition}{Definition}




\theoremstyle{rem}
%\declaretheorem[style=rem, numbered=no,qed=$\diamondsuit$]{remark}
%\declaretheorem[style=rem, numbered=no,qed=$\diamondsuit$]{remarks}
%\declaretheorem[style=rem, numbered=no,qed=$\diamondsuit$]{example}

\newtheorem*{remark}{Remark}
\newtheorem*{remarks}{Remarks}
\newtheorem*{example}{Example}

%\theoremstyle{texttheorem}
%\newtheorem{textlemma}[theorem]{}

%%This document

\newcommand{\cinfty}{C^{\infty}}



%% BRIAN'S STUFF
\usepackage{cmupint}
\newcommand{\sfc}{\mathsf{c}}
\newcommand{\sfb}{\mathsf{b}}
\renewcommand\div{{\partial_\Omega}}





\usepackage{stmaryrd}
\parskip=10pt
\date{}
\newcommand{\gl}{\mf{gl}}






%
\title{K3 draft}
%\author{Kevin Costello}
%\author{Natalie M. Paquette}
%\thanks{CALT-TH 2019-050}
%
%\address{Perimeter Institue for Theoretical Physics}
%\address{Walter Burke Institute for Theoretical Physics, California Institute of Technology}
%\email{kcostello@perimeterinstitute.ca}
%\email{nataliep@caltech.edu}


\begin{document}

\maketitle

\section{Twisted supergravity on ${\rm AdS}_3 \times S^3 \times K3$}

\subsection{Compactification of Kodaira--Spencer theory}

Recall that $\PV (Y) = \PV^{\bu,\bu}(Y)$ denotes the graded space given by the Dolbeault resolution of holomorphic polyvector fields on a complex manifold $Y$.

Let $Y$ be a complex surface with a fixed holomorphic symplectic structure.
The fields of Kodaira--Spencer theory on $\CC^3 \times Y$ are polyvector fields which are in the kernel of the divergence operator.
A polyvector field on $\CC^3 \times Y$ can be written as a tensor product of one on $\CC^3$ with one on $Y$.  
Polyvector fields on $Y$ are the same as differential forms, because the holomorphic symplectic form on $Y$ identifies the tangent and cotangent bundles. 
In particular, the harmonic polyvector fields are given simply by the de Rham cohomology of $Y$.  
Furthermore, polyvector fields on $Y$ which are harmonic are automatically in the kernel of the divergence operator $\div$, by standard Hodge theory arguments.   
Summarizing there is an equivalence of graded algebras
\[
\PV (\CC^3) \otimes \bigg(\ker \div |_{\PV(Y)} \bigg) \simeq \PV(\CC^3) \otimes H^\bu(X) .
\]
We will use this equivalence to describe the fields of the theory on $\CC^3$ upon compactification along $Y$. 

Let $A = H^\bu(Y)$ denote the cohomology ring of $Y$.
We are mostly interested in the case that $Y$ is a K3 surface in which case this algebra is generated by even elements $\eta, \br \eta, \eta_a$ for $a=1,\ldots 20$ subject to the relations
\beqn
\label{eqn:K3rel}
\begin{split}
\eta^2 & = \Bar{\eta}^2 = 0 \\
\eta_a \eta_b & = h_{ab} \eta \Bar{\eta} 
\end{split}
\eeqn
where $h_{ab}$ is a non-degenerate symmetric pairing on $\CC^{20}$. 
Let $I$ denote the ideal generated by these equations so that $A = \CC[\eta,\Bar{\eta}, \eta_a] / I$. 

As usual, we write the polyvector fields on $\CC^3$ in terms of a superspace by introducing odd variables $\theta^i$, $\br{\theta}_{\br{j}}$.  
Here, $\theta^i$ represents the coordinate vector field $\partial_{z_i}$ and $\br \theta_i$ represents the coordinate Dolbeault form $\d \zbar_i$. 
Then we can write the field content as a collection of superfields
\begin{equation} 
		\mu(z,\zbar,\theta^i, \br{\theta}_{\br{i}},\eta) \in \oplus_{i,j}  \PV^{i,j}(\CC^3) \otimes A .
\end{equation}
Here, we are using the shorthand $\eta$ to inform that there is a dependence on $\eta, \br{\eta}$, and $\eta_a$, $a=1,\ldots, 20$. 
As such, such a superfield decomposes in its dependencies on the generators of the cohomology of $X$ as
\begin{multline}
\mu(z,\zbar,\theta^i, \br{\theta}_{\br{i}}) \\
+ \mu_\eta (z,\zbar,\theta^i, \br{\theta}_{\br{i}}) \eta + \mu_{\br{\eta}} (z,\zbar,\theta^i, \br{\theta}_{\br{i}}) \br{\eta} + \mu^a (z,\zbar,\theta^i, \br{\theta}_{\br{i}}) \eta_a \\
+ \mu_{\eta \Bar{\eta}} (z,\zbar,\theta^i, \br{\theta}_{\br{i}}) \eta \Bar{\eta} .
\end{multline}
We emphasize that the $\eta$-variables represent harmonic polyvector fields on $X$ and so are not acted on by any differential operators along $\CC^3$. 

%The fields on $\C^3$ must satisfy the constraint 
%\begin{equation} 
%	\partial_{z_{i_1}} \mu_{i_1 \dots i_l; a}^{\br{j}_1 \dots \br{j}_k} = 0. 
%\end{equation}
The superfield satisfies the equation $\partial \mu = 0$
where, in the superspace formulation,
\begin{align} 
	\dbar & \, = \, \br{\theta}_{\br{j}} \partial_{\zbar_{\br{j}}} \\
	\partial &\, = \, \partial_{\theta^i} \partial_{z_i}.  
\end{align}
	The Lagrangian is
	\begin{equation} 
		\tfrac{1}{2} \int_{\CC^{3 \mid 6}}  \mu \dbar \partial^{-1} \mu \d^3 z \d^3 \zbar \d^3 \theta \d^3 \br{\theta} |_{\eta \br{\eta}} + \frac{1}{6} \int_{\CC^{3 \mid 6}} \mu^3 \d^3 z \d^3 \zbar \d^3 \theta \d^3 \br{\theta} |_{\eta \br{\eta}}  
	\end{equation}
where the $(-)|_{\eta \br{\eta}}$ means we pick up only the $\eta \br{\eta}$ component.

Before we turn to the computation of the backreaction, we will simplify the field content somewhat, following \cite{CostelloGaiotto}.  
We note that the coefficient of $\theta^1 \theta^2 \theta^3$ does not appear in the kinetic term in the action.  
This field does not propagate, so we can (and will) impose the additional constraint
\begin{equation}\label{eq:nonprop} 
	\partial_{\theta^1} \partial_{\theta^2} \partial_{\theta^3} \mu (z,\zbar,\theta,\br{\theta},\eta) = 0. 
\end{equation}

Next, let us expand the superfield $\mu$ only in the $\theta^i$ variables:
\begin{equation} 
	\mu = \mu(z,\zbar,\br{\theta},\eta) + \mu_{i}(z,\zbar,\br{\theta},\eta) \theta^i + \dots 
\end{equation}
We note that the constraint $\partial \mu_{ij} = 0$ implies that there is some super-field
\begin{equation} 
	\what{\mu}_{ijk}(z,\zbar,\br{\theta},\eta) = 	\alpha(z,\zbar,\br{\theta},\eta) \eps_{ijk}   
\end{equation}
so that $\partial_{z_i} \what{\mu}_{ijk} = \mu_{jk}$.

It is convenient to rephrase the theory in terms of the field $\alpha(z,\zbar,\br{\theta},\eta)$, which has no holomorphic index. 
We will also change notation and let $\gamma(z,\zbar,\br{\theta},\eta)$ be the term with no $\theta^i$ dependence in the superfield $\mu(z,\zbar,\theta,\br{\theta},\eta)$.  

In summary, we have the following fundamental superfields in the compactified theory on $\CC^3$:
\begin{itemize}
\item $\mu_i (z,\zbar,\br{\theta}, \eta) \theta^i$ which we identify with an element in the graded space
\beqn
\mu \in \PV^{1,\bu}(\CC^3) \otimes A .
\eeqn
\item $\alpha (z,\zbar,\br \theta, \eta)$ which we identify with an element of the graded space
\beqn
\alpha \in \Omega^{0,\bu}(\CC^3) \otimes A .
\eeqn
\item $\gamma(z,\zbar,\br \theta, \eta)$ which we also identify with an element of the graded space
\beqn
\gamma \in \Omega^{0,\bu}(\CC^3) \otimes A .
\eeqn
\end{itemize}
In terms of these fields, the Lagrangian is
\begin{multline} 
	\tfrac{1}{2}\int_{\CC^{3|3}}   \eps^{ijk} \dbar \mu_{i} (\partial^{-1}  \mu)_{jk} \, \d^3 z \d^3 \zbar \d^3 \br{\theta} |_{\eta \br \eta}   + \int  \alpha \dbar \gamma \d^3 z \d^3 \zbar \d^3 \br{\theta} |_{\eta \br{\eta}} 
	\\
	+ \tfrac{1}{6} \int_{\CC^{3|3}}  \eps_{ijk} \mu_{i}\mu_{i} \mu_{c,i} \, \d^3 z \d^3 \br{z} \d^3 \br{\theta} |_{\eta \br{\eta}} + \int  \alpha \mu_i \partial_{z_i}  \gamma \, \d^3 z \d^3 \zbar \d^3 \br{\theta} |_{\eta \br{\eta}} .
\end{multline} 
In this expression we project onto the component $\eta \br \eta$ and perform the Berezin integral along the $\br \theta$ coordinate.

Just as when we twist a field theory, when we twist a supergravity theory the ghost number of the twisted theory  is a mixture of the ghost number and a $U(1)_R$-charge of the original physical theory. To define a consistent ghost number, one can choose any $U(1)_R$ in the physical theory under which the supercharge has weight $1$.  In general, there are many ways to do this.  It is convenient for us to make the following assignments of ghost number.
\begin{enumerate} 
	\item The variables $\eta_a$ \brian{finish}
	\item The anti-commuting variables $\br{\theta}_i$ have ghost number $1$.
	\item The fields $\alpha$, $\gamma$ have ghost number $-1$, and are fermionic.
\end{enumerate}

\subsection{Backreaction as an infinitesimal deformation} 
\label{sec:conifold}

In the type IIB theory on $\CC^3 \times K3$ we consider a D1--D5 system where the D1 branes wrap 
\[
\CC \times 0 \times \{x\} \subset \CC \times \CC^2 \times K3 
\]
for some $y \in K3$ and the D5 branes wrap 
\[
\CC \times 0 \times K3 \subset \CC \times \CC^2 \times K3 .
\]
We can apply a duality to turn this into a D3 brane system which wraps 
\[
\CC \times 0 \times \Sigma \subset \CC \times \CC^2 \times K3 
\]
for some two-cycle $\Sigma \subset K3$. 
\brian{say this precisely}

In the last section, we argued that the dimensional reduction along a $K3$ surface becomes an extended version of Kodaira--Spencer theory where the extra fields are labeled by the cohomology of the surface.
Upon compactification, the D3 system becomes a system of $B$-type branes in this extended version of Kodaira--Spencer theory. 

The charge of these branes is labeled by a cohomology class 
\beqn
F \in H^2(K3) \subset A .
\eeqn
We denote 
\beqn
N \define \ip{F, F}
\eeqn
using the inner product on $H^2(Y)$. 
Explicitly, if $F = f \eta + \br f \br \eta + f^a \eta_a$ for $f, \br f, f_a$ complex numbers, then $N = f \br f + f^{a} f^{b} h_{ab}$ where $h_{ab}$ is the fixed non-degenerate symmetric pairing. 
\brian{Can we say how $N$ is related to $N_1$ and $N_5$? This would require tracing through some dualities carefully...}

Let's choose coordinates $z,w_1,w_2$ on $\CC^3$, where the branes wrap the $z$-plane along $w_1=w_2=0$. 
Including the backreaction will deform the geometry away from the locus of the brane. 
Before backreacting, we should say what geometry is actually being deformed. 
Recall that in the case of ordinary Kodaira--Spencer theory on $\CC^3$, it was shown in \cite{CGhol} that the backreaction of B-branes along $\CC \subset\CC^3$ deformed the complex structure on $\CC^3 \setminus \CC$ to the {\em deformed conifold} $SL_2(\CC)$. 

Our case is similar in that the branes are supported along the same locus as in \cite{CGhol}.
The difference is that we are working with a bigger space of fields, roughly extended by the cohomology of the $K3$ surface. 
We will now show that in the case of type IIB compactified on a K3 surface, the backreaction determines an {\em infinitesimal} deformation of the complex manifold $\CC^3 \setminus \CC$ over the fat point $\Spec A$ where $A = H^\bu(K3)$. 
 
If $A$ is any local ring, an infinitesimal deformation of a complex manifold $M_0$ over $\Spec A$ is an element 
\beqn
\mu_{def} \in \PV^{1,1}(M_0) \otimes \mathfrak{m}_A 
\eeqn
satisfying the Maurer--Cartan equation.
In our case, $M_0 = \CC^3 \setminus \CC$ and $\mu_{def}$ is a supergravity field sourced by the branes. 
The Maurer--Cartan equation is the equation of motion for $\mu_{def}$. 

%Recall that the cohomology ring of a K3 surface (with complex coefficients) is generated by even elements $\eta, \br \eta, \eta_a$, where $a=1,\ldots,20$ subject to the relations \eqref{eqn:K3rel}. 
%These relations define a quadric $Z$ inside $\CC^{22}$. 
%Thus, in total, we will see that the backreaction will deform the complex structure on 
%\[
%(\CC^3 \setminus \CC) \times Z \subset \CC^3 \times \CC^{22} .
%\]
%Before deforming, there is an obvious projection map 
%\[
%(\CC^3 \setminus \CC) \times Z \to Z 
%\] 
%whose fiber is a Calabi--Yau manifold equipped with the standard holomorphic volume form $\Omega_{\CC^3} = \d z \d w_1 \d w_2$. 
Forgetting the usual $2\ZZ$-grading, the cohomology $A$ of a $K3$ surface is a local ring.
Following \cite{CGhol}, the backreaction of this system of branes introduces a twisted supergravity field
\[
\mu_{BR} \in \br\PV^{1,1}(\CC^3) \otimes A 
\]
which we can identify with an element of $\br\Omega^{2,1}(\CC^3) \otimes A$ using the Calabi--Yau form on $\CC^3$. 
This field satisfies the following equations of motion
\beqn
\label{eqn:mcbr}
	\begin{split}
		\dbar \mu_{BR}  & = F \, \delta_{\CC \subset \CC^3} \\
		\del \mu_{BR} & = 0 .
	\end{split}
\eeqn
For quantization we will also impose the gauge fixing condition that $\dbar^\ast \mu_{BR}(\eta_a) = 0$. 	
There is a unique solution to the above equations given by
\beqn
\mu_{BR} = \frac{\eps^{ij} \br w_i \d \br w_j}{|w|^4} \partial_z \otimes F .
\eeqn
Note that this field is of the form $\mu_{BR,0} \otimes F$ where $\mu_{BR,0}$ is the Beltrami differential which gives rise to the deformed conifold \cite{CGhol}---all of the dependence on the parameters $\eta, \br \eta, \eta_a$ is in the cohomology class~$F$.

Equations \eqref{eqn:mcbr} imply that $\mu_{BR}$ determines an infinitesimal deformation of~$\CC^3 \setminus \CC$ over $\Spec A$. 
The Kodaira--Spencer map associated to this infinitesimal deformation is of the form
\[
KS \colon T_{\Spec A} \to H^1(\CC^3 \setminus \CC, T) 
\]
and simply maps a derivation $\delta$ of $A$ to the class 
\[
\delta(F) \left[\frac{\eps^{ij} \br w_i \d \br w_j}{|w|^4} \del_z \right] \in H^1(\CC^3 \setminus \CC, T) .
\] 
\brian{what more to say?}
%This Beltrami differential defines a new complex structure on the complex manifold $(\CC^3 \setminus \CC) \times Z$ where $Z$ is the quadric inside $\CC^{22}$ defined by \eqref{eqn:K3rel}. 
%Note that \brian{how to say things remain algebraic along $Z$?}
%A function $\Phi(z,\br z, w_i, \br w_i , \eta, \br \eta, \eta_a)$ is holomorphic if the following equations hold
%\beqn
%	\begin{split}
%		\del_{\br z} \Phi & = 0 \\
%		\del_{\br w_1} \Phi - F \frac{\br w_2}{\norm{w}^2} \del_z \Phi & = 0 \\
%		\del_{\br w_2} \Phi + F \frac{\br w_1}{\norm{w}^2} \del_z \Phi & = 0 .
%	\end{split}
%\eeqn
%It is easy to check that the following functions are holomorphic for the deformed complex structure 
%\beqn
%	\begin{split}
%		u_1 & = w_1 z - F \frac{\br w_2}{\norm{w}^2} \\
%		u_2 & = w_2 z + F \frac{\br w_1}{\norm{w}^2} .
%	\end{split}
%\eeqn
%
%In total, we are describing the affine variety $X_F$ inside of $\CC^4 \times \CC^{22}$ defined by the equations 
%\beqn
%\label{eqn:conifold}
%	\begin{split}
%		\eta^2 = \br \eta^2 & = 0 \\
%		\eta_a \eta_b & = h_{ab} \eta \br \eta \\
%		w_1 u_2 - w_2 u_1 & = F . 
%	\end{split}
%\eeqn
%Here, $(u_1,u_2,w_1,w_2)$ are the holomorphic coordinates on $\CC^4$ and $\eta,\br \eta, \eta_a$ are the holomorphic coordinates on $\CC^{22}$. 
%Denote by $X_F^0 \subset X_F$ the open subset where $u_i,w_j$ are all not zero. 
%
%We observe that upon deforming the complex structure, there is still a fibration
%\[
%X_{F}^0 \to Z 
%\]
%whose fibers are Calabi--Yau three-folds. 
%The volume form in the coordinates $z, w_1,w_2$ is unchanged when we deform the complex structure.
%This is because the Beltrami differential $\mu_{BR}$ is divergence-free. 
%In the coordinates $u_i, w_i$ the holomorphic volume on the fibers of this projection reads
%\beqn
%\Omega|_{w_1 \ne 0} = \frac{\d u_1 \d w_1 \d w_2}{w_1}, \qquad \Omega|_{w_2 \ne 0} = \frac{\d u_2 \d w_1 \d w_2}{w_2} . 
%\eeqn
%
%In \cite{CP1} the first two authors considered Kodaira--Spencer theory on the complex surface $T^4$ and found that the resulting backreaction only gave finite-order corrections to operator products on flat space. 
%We are in a very similar situation here in the case of Kodaira--Spencer theory compactified on a K3 surface. 
%The flat space background corresponds to $F = 0$. 
%Now, since $F \in H^2(Y)$, we see that $F^3 = 0$ in the ring $A = H^\bu (Y)$.
%Thus, only the only powers of $F$ that appear in corrections to the flat space OPE are $F$ and $F^2$. 

%\subsection{Partial compactification of the extended conifold}
%\label{sec:compact} 
%
%The ordinary deformed conifold is biholomorphic to $SL_2(\CC)$. 
%One can view this as a subvariety of $\CC^4$ with coordinates $u_i,w_j$ satisfying $\eps^{ij} u_i w_j = N$. 
%To compactify this subvariety the idea is to introduce an additional homogeneous coordinate $\lambda$ and consider the projective variety in $\CP^4$ defined by the equation $\eps^{ij} u_i w_j = N \lambda^2$ where the $u_i,w_j,\lambda$ are now projective coordinates. 
%The boundary of this projective variety is described by the equation $\eps^{ij} u_i w_j = 0$ and is equivalent to $\CP^1 \times \CP^1$. 
%
%We use the new coordinate $z$ for the first $\CP^1$ which is the boundary of the Euclidean ${\rm AdS}_3$ space
%and the new coordinate $w$ for the second $\CP^1$.  
%Let $n$ be the coordinate normal to the boundary $\CP^1 \times \CP^1$ of $SL_2(\CC)$. 
%This coordinate has a first order pole at $z=\infty$ and $w=\infty$. 
%Let $\DD_z \times \CP^1_w \subset \CP^1_z \times \CP^1_w$ be a neighborhood of the algebraic curve $0 \times \CP^1_w$ in the boundary divisor. 
%A neighborhood of this region in the whole manifold $\br{X_{F}^0}$ is a deformation of the total space of $\cO(1) \to \DD \times \CP^1$ by the Beltrami differential 
%\[
%n^2 \d ...
%\]
%
%
%The situation here is very similar. 
%The extended conifold $X_F^0$ has a completion $\br{X_F^0}$ which is a subvariety of $\CP^4 \times Z$. 
%It is defined by modifying the last equation in \eqref{eqn:conifold} to
%\beqn
%\eps^{ij} u_i w_j = F \lambda^2 
%\eeqn
%where $\lambda$ is an additional homogenous coordinate.
%The boundary of $\br{X_F^0}$ is the subvariety of $\CP^3 \times Z$ defined by $\eps^{ij} u_i w_j = 0$ which is $\CP^1 \times \CP^1 \times Z$. 

\section{The chiral algebra on $D$-branes} 

\subsection{$D$-branes in type IIB}

We consider the situation of a $D1/D5$ brane system in the twist of type IIB on a Calabi--Yau five-fold $X$. 
For simplicity, we assume that we have a collection of $N_1 = N$ $D1$ branes supported along a complex submanifold
\[
\Sigma \subset X 
\]
together with a single $D5$ brane which is parallel to the $D1$ branes. 

We consider $D1$ branes which are a sum simple branes labeled by $\cO_\Sigma$.
The Dolbeault model for the open string fields which stretch between two such $D1$ branes is given by 
\beqn\label{eqn:open1}
\Omega^{0,\bu}(\Sigma, \underline{\rm Ext}_{\cO_X}(\cO_\Sigma)) \cong \Omega^{0,\bu}(\Sigma, \wedge^\bu N_\Sigma) 
\eeqn
where $N_\Sigma$ is the normal bundle to $\Sigma$ in $X$. 
If we take $X$ to the be the total space of the bundle $N_\Sigma$ then the Calabi--Yau condition requires $\wedge^4 N_\Sigma = K_\Sigma$. 
In the case $\Sigma = \CC$ and $X = \CC^5$ we can twist by a homomorphism $SO(2) \to SO(4)$ to 
write the open string fields \eqref{eqn:open1} as 
\beqn\label{eqn:open1a}
\Omega^{0,\bu}(\CC) \otimes \lie{gl}(N) [\ep_1,\ldots,\ep_4] [1] .
\eeqn
Here the $\ep_i$ are odd variables. 

Next, we consider $D1-D5$ strings. 
The open string fields are given by 
\beqn\label{eqn:open15}
\Omega^{0,\bu}(\CC, K^{\frac12}_\CC)  [\ep_3,\ep_4] \otimes {\rm Hom}(\CC, \CC^N) = \Omega^{0,\bu}(\CC, K^{\frac12}_\CC)  [\ep_3,\ep_4] \otimes \CC^N .
\eeqn
Together with the $D5-D1$ strings we get 
\beqn\label{eqn:open15a}
\Omega^{0,\bu}(\CC, K^{\frac12}_\CC)  [\ep_3,\ep_4] \otimes T^*\CC^N .
\eeqn

In total, we see that the open-strings of the $D1/D5$ system along $\Sigma = \CC$ are given by the Dolbeault complex valued in the following holomorphic vector bundle
\beqn
\bigg(\lie{gl}(N)[\ep_1,\ep_2][1] \oplus K^{\frac12}_\CC \otimes T^*\CC^N \bigg) \otimes \CC[\ep_3,\ep_4] .
\eeqn
The bundle in parentheses can be written as 
\brian{Now here I really want $\deg{\ep_1}=\deg{\ep_2} = +1$}
\beqn
\lie{gl}(N)[1] \oplus T^* \left(\lie{gl}(N) \oplus K^{\frac12}_\Sigma \CC^N\right) \oplus \lie{gl}(N) [-1] .
\eeqn
Up to the factor of $K^{\frac12}_\Sigma$ this is evidently the underlying vector space of the graded Lie algebra controlling Hamiltonian reduction of 
\beqn
T^*(\lie{gl}(N) \oplus \CC^N) \sslash \lie{gl}(N) .
\eeqn

\section{Supergravity states}

\subsection{Boundary conditions}

To describe the boundary conditions we will use the partial compactification of the extended deformed conifold as described in \S \ref{sec:compact}. 

Recall that there are three fundamental fields for Kodaira--Spencer theory compactified on a $K3$ surface.
Two fundamental fields $\alpha, \gamma$ are Dolbeault forms of type $(0,\bu)$ valued in the algebra $A$. 
The last fundamental field $\mu$ is a $(0,\bu)$ form valued in $T_{\CC^3} \otimes A$.
We can use the Calabi--Yau form to view $\mu$ as a Dolbeault form of type $(2,\bu)$ valued in the algebra $A$. 

\begin{itemize}
\item The vacuum boundary condition for the fields $\alpha, \gamma$ is that each are divisible by the coordinate $n$. 
\item The vacuum boundary condition for the field $\mu$ is that when viewing it as a Dolbeault form of type $(2,\bu)$ it can be expressed as a sum of terms which are each wedge products of $\d \log n, \d w, \d z , \d \br n, \d \br w , \d \br z$ with coefficients that are regular at $n = 0$. 
\end{itemize} 


%The vacuum boundary condition for the fields $\alpha$ requires that it is divisible by the coordinate $n$. 
%We can modify this at the point $z=0$ at the boundary by taking 
%\[
%\alpha = n^{-k} w^l \del_z^{r} \delta_{z=0} \otimes a .
%\]
%In this expression, $k \geq 0$ and $l \leq k$ to ensure that there are no poles along $w = \infty$. 
%Also, $a \in H^\bu(Y)$ denotes an arbitrary cohomology class for the $K3$ surface.
%The ansatz for $\gamma$ is identical.
%
%For the field $\mu$

\subsection{Enumerating states}

The supergravity states were enumerated in \cite{CP}. 
We briefly recall the results here. 

The twisted supergravity states organize into a representation for the super Lie algebra $\lie{psu}(2|2)$.
The bosonic factor of this super Lie algebra is $\lie{su}(2)_L \oplus \lie{su}(2)_R$. 
The first copy is the global conformal transformations in the $z$-plane and the second copy is the $R$-symmetry algebra which rotates the $w$-coordinate.
We take the Cartan of this Lie algebra to be spanned by $(L_0, J_0^3)$. 
  
Denote by $(\frac{\bf m}{\bf 2})_S$ the short representation of $\lie{psu}(2|2)$ whose highest weight vector has $(L_0, J_0^3)$ eigenvalue $(m/2,m/2)$ \cite{dB1}. 
As an example, the short representation $({\bf 1})_S$ consists of a boson with weight $(L_0 = 1, J_0^3 = 1)$, which in our notation corresponds to 
\beqn
\mu \sim n^{-2} \d \log n \d z \delta_{z=0}  .
\eeqn 
There are also two fermions in $({\bf 1})_S$ with weights $(3/2,1/2)$ corresponding to the states
\beqn
\alpha \sim n^{-1} \delta_{z=0} + \cdots , \quad \gamma \sim n^{-1}\delta_{z=0} + \cdots
\eeqn
and another boson of weight $(2,0)$ corresponding to 
\beqn
\mu \sim n^{-2} \d \log n \d w \delta_{z=0} + \cdots .
\eeqn 

In \cite{CP} it was shown that the space of single particle states of twisted type IIB supergravity on a $K3$ surface is given by the $\lie{psu}(2|2)$ representation 
\beqn\label{eqn:IIBstates}
\bigoplus_{m \geq 1} (\frac{\bf m}{\bf 2})_S \otimes H^\bu(K3) = \bigoplus_{m \geq 1} \bigoplus_{i,j} (\frac{\bf m}{\bf 2})_S \otimes H^{i,j} (K3)  . 
\eeqn 
This should be compared to \cite{dB1}, where the supergravity states upon localizing are found to be
\beqn\label{eqn:db1}
\bigoplus_{m \geq 0} \bigoplus_{i,j} (\frac{\bf m+i}{\bf 2})_S \otimes H^{i,j} (K3) .
\eeqn
The answers clearly agree in the range where the highest weight of the short representation is at least two. 
The low weight discrepancies break up into two types:
\begin{itemize}
\item In \cite{dB1} there is an extra factor of $({\bf 0})_S \otimes H^{0,i}(K3)$. 
So, there are two extra bosonic operators in the analysis of \cite{dB1}. 
In \cite{CP} is was pointed out that these are topological operators, annihilated by $L_{-1}$, and have nonsingular OPE with all remaining operators. 
\item 
In our analysis there is an extra factor of $(\frac{\bf 1}{\bf 2})_S \otimes H^{2,j}(K3)$. 
We can remove these two bosonic states while maintaining an $SO(21)$ symmetry. 
\end{itemize}

\subsection{The index} 

The contributions from the four towers of states in Figure \ref{fig:states} comprise short representations for $\lie{su}(2|2)$. 
Let $\chi_j (q,y)$ be the single particle index of the combination of these short representations which have spin $j$ (that is, $L_0 = j$). 

The generating function for our single-particle states is given in terms of the characters $\chi_j(q,y)$ as 
\beqn
\sum_{n,m,l} c_{sugra}(m,n,l) p^m q^n y^l = \sum_{m \geq 0} \sum_{i,j} h^{i,j} \chi_{m+i} (q,y) p^{m+1} .
\eeqn
The $c_{sugra}(m,n,l)$ appear as the exponents of the index of the full multiparticle Fock space via \cite{dB2, dmvv}:
\beqn
\prod_{m> 0, n,l} \frac{1}{(1-p^mq^ny^l)^{c_{sugra}(m,n,l)}} .
\eeqn 

\subsection{Global symmetry algebra}
For $F \in H^2(Y_{K3})$, let $X^0_F$ be the extended deformed conifold as defined in \S \ref{sec:conifold}.
Recall that $Z = {\rm Spec}(A)$ is the affine variety described by the commutative algebra $A = H^\bu(Y_{K3})$. 
Previously, we saw that the space $X_F^0$ admits a fibration $X^0_F \to Z$ fibered in Calabi--Yau three-folds. 
Let $\cO(X^0_F)$ be the algebra of functions on $X^0_F$.
By Hartog's theorem this is the algebra generated by the bosonic linear functions $u_i, w_j, \eta, \br \eta, \eta_a$ where $i,j=1,2$, $a=1,\ldots, 20$ subject to the relations
\[
\eta^2 = \br \eta^2 = \eta_a \eta_b - h_{ab} \eta \br \eta = 0, \qquad \eps^{ij} u_i w_j = F . 
\]

Let 

\section{Orbifold CFT}

\subsection{The half-twisted $\mc N=4$ CFT}

Assuming that the target manifold $M$ is K\"ahler, we can decompse the complexified bosons into $\phi^I, \phi^{\br I}$ where $I, \br I = 1,\ldots, \dim_\CC (M)$ index a local holomorphic (and anti-holomorphic) coordinate on the target.
The fermionic fields of the physical supersymmetric $\sigma$-model are sections of the following bundles
\beqn
\label{eqn:fermionsuntwist}
\begin{split}
\psi^I & \in \Gamma\left(\Sigma \, , \, K^{\frac12}_\Sigma \otimes \phi^* T^{1,0}_M\right) \\ 
\psi^{\br I} & \in \Gamma\left(\Sigma \, , \, K^{\frac12}_\Sigma \otimes \phi^* T^{0,1}_M\right) \\
\br\psi^I & \in \Gamma\left(\Sigma \, , \, \br K^{\frac12}_\Sigma \otimes \phi^* T^{1,0}_M\right) \\
\br\psi^{\br I} & \in \Gamma\left(\Sigma \, , \, \br K^{\frac12}_\Sigma \otimes \phi^* T^{0,1}_M\right). 
\end{split}
\eeqn

To pass to the half-twisted\footnote{We are referring to the twist which renders the theory holomorphic on the Riemann surface.}, we restrict to the cohomology of the supercharge $\br Q_+$ and twist the fermionic fields using a combination of R-symmetry curents. 
The more common $A$-type twist, studied in \cite{KapustinHalf}, is with respect to the current $\frac12 (J_L - J_R)$. 
Instead, we will consider the twist just by the $R$-symmetry current $-J_R$. 
This has the effect of modifying the right-moving fermions but not the left-moving fermions:
\beqn
\label{eqn:fermionstwist}
\begin{split}
\psi^I & \in \Gamma\left(\Sigma \, , \, K^{\frac12}_\Sigma \otimes \phi^* T^{1,0}_M\right) \\ 
\psi^{\br I} & \in \Gamma\left(\Sigma \, , \, K^{\frac12}_\Sigma \otimes \phi^* T^{0,1}_M\right) \\
\br\psi^I & \in \Gamma\left(\Sigma \, , \, \br K_\Sigma \otimes \phi^* T^{1,0}_M\right) \\
\br\psi^{\br I} & \in \Gamma\left(\Sigma \, , \, \phi^* T^{0,1}_M\right). 
\end{split}
\eeqn

In $\br Q_+$ cohomology the right-moving fermions are rendered trivial. 
We identify the fields which remain in $\br Q_+$ cohomology in terms of a chiral $bc \beta\gamma$ system in the following way
\beqn
\begin{split}
\beta_I & = g_{I \br J} \del_z X^{\br J} , \quad \gamma^I = X^I \\
b_I & = g_{I \br J} \psi^{\br J} , \quad c^I = \psi^I .
\end{split}
\eeqn
The free action is simply $\int_\Sigma \beta \dbar \gamma + \int_\Sigma b \dbar c$. 
Notice that in twist, the $bc$ fields still have half-integral spin as opposed to the conventional $A$ or $B$-twists. 

In our context, the target space $M$ is $\oplus_{N=1}^\infty \Sym^N (T^4)$ or $\oplus_{N=1}^\infty \Sym^N (K3)$. 
These spaces are hyperk\"ahler, so the chiral de Rham complex will acquire a symmetry by a twisted version of the $\mc N = 4$ superconformal algebra. 
Actually, since our twist only modifies the right-moving fermions, the left-moving part of the $\mc N = 4$ superconformal algebra itself remains the same as in the untwisted theory. 

\subsection{Chiral de Rham complex on hyperk\"ahler manifolds}

We describe the chiral de Rham complex on a general hyperk\"ahler manifold. 
We will specialize to the symmetric orbifold in the next section. 

We adjust indexing notation to one that is more natural from the point of view of hyperk\"ahler geometry.
Locally, the holomorphic tangent bundle of a hyperk\"ahler manifold $M$ of complex dimension $2n$ decomposes as $T^{1,0}_M = V \otimes \CC^2$ where $V$ is a holomorphic vector bundle of rank $n$. 
We let $i = 1,\ldots, n$ index a frame for the bundle $V$ and $\alpha = 1,2$ index be the $SU(2)$ doublet index. 
The fields will be written as $\gamma^{i\alpha}, \beta_i^\beta$, etc..

The nonvanishing $bc \beta\gamma$ OPEs are
\beqn
\begin{split}
\beta_i^\alpha (0) \gamma^{j\beta} (z) & \simeq \eps^{\alpha \beta} \delta_i^j \frac1z \\
b^\alpha (0) c^{j\beta} (z) & \simeq \eps^{\alpha \beta} \delta_i^j  \frac1z .
\end{split}
\eeqn

The $\mc N=4$ superconformal algebra is generated by the following fields:
\beqn
\begin{split}
T(z) & = \eps_{\alpha\beta} \beta_i^\alpha \del \gamma^{i \beta} + \frac12 \eps_{\alpha \beta} b^\alpha_i \del c^{i \beta}  \\
G^{\alpha \beta} (z) & = \beta_i^\alpha c^{i \beta} + b_i^\alpha \del \gamma^{i \beta} \\
J^a (z) & = \eps_{\alpha \beta} b_i^\alpha \sigma^{a \beta}_\gamma c^{i \gamma} .
\end{split}
\eeqn
\brian{normalize}

\subsection{The chiral de Rham complex of the symmetric orbifold}

Our eventual goal is to describe the chiral de Rham complex on the symmetric orbifold on the hyperk\"ahler manifolds $T^4$ or a $K3$ surface. 
We start by describing the chiral de Rham complex of an affine version of the symmetric orbifold, namely $\Sym^N(\CC^2)$. 

Consider the complex vector space $B_N = \lie{gl}(N) \times \CC^N$ and its cotangent space $T^* B_N$. 
We consider $T^* B_N$ as a holomorphic symplectic manifold. 
There is a Hamiltonian action of $GL(N)$ on $T^* B_N$ with moment map 
\beqn
\mu_N (v,A ; w , B) = vw + [A,B] . 
\eeqn 
The symmetric orbifold $\Sym^N (\CC^2)$ can be written as the Hamiltonian reduction of $T^*B_N$ with respect to this Hamiltonian action 
\beqn
T^* B_N \sslash_\mu GL(N) = \mu^{-1} (0) \slash GL(N) .
\eeqn 


\begin{itemize}
\item 
A map $\Sigma \to T^* B_N$ which we will decompose as a map
\[
\gamma^{\alpha}_{(r)} \colon \Sigma \to T^* \CC^N
\]
where $\alpha=1,2$ and $r = 1,\ldots, N$ and a map
\[
\Gamma^{\alpha r}_{s} \colon \Sigma \to T^* \lie{gl}(N) ,
\]
where $\alpha=1,2$ and $r,s=1,\ldots,N$. 
Similarly, there are the bosonic $(1,0)$ forms $\beta^\alpha_{(r)}$, $\mc B^{\alpha r}_s$. 
\item 
There are the fermions
\[
c^\alpha_{(r)} \in \Gamma(\Sigma, K^{\frac12}_\Sigma) \otimes T^* \CC^N ,
\]
where $\alpha=1,2$ and $r=1,\ldots N$ and 
\[
C^{\alpha r}_s \in \Gamma(\Sigma, K^{\frac12}_\Sigma) \otimes T^* \lie{gl}(N) .
\]
Similarly, there are the fermions $b^\alpha_{(r)}, B^{\alpha r}_s$.
\item
There are the $\lie{gl}(N)$ valued ghosts 
\[
\sfc^{r}_s \in \Gamma(\Sigma, \cO) \otimes \lie{gl}(N), 
\]
and their anti-ghosts $\sfb^r_s$. 
\item
Finally, there are the $\lie{gl}(N)$ valued ghosts 
\[
\til\sfc^{r}_s \in \Gamma(\Sigma , K_\Sigma) \otimes \lie{gl}(N),
\]
(note that these are of spin $+1$) and their anti-ghosts $\til \sfb_s^r$. 
\end{itemize}

In Appendix \ref{appx:cdr} we give a construction of the chiral de Rham complex on the Hamiltonian reduction of a holomorphic symplectic manifold. 
In the notation of that appendix we are using $W = T^* B_N$ and $G=GL(N)$. 
We spell out the BRST operator in this notation. 
It is given by $Q = \oint J_{BRST}(z) \, \d z$ where the BRST current is
\beqn
\label{eqn:BRSTcdr}
\begin{split}
J_{BRST} (z) & = \frac12 \Tr{\sfb \sfc^2} \\ & + \Tr{\sfc (\beta \gamma)} + \Tr{\sfc \mc B \Gamma} + \frac12 \Tr{\til \sfc (\gamma^2)} + \frac12 \Tr{\til \sfc \Gamma^2} \\ & + \Tr{\sfc (b c)} + \Tr{\sfc B C} + \frac12 \Tr{\til \sfc (c^2)} + \frac12 \Tr{\til \sfc C^2} .
\end{split}
\eeqn
In the expression above we have made use of the following notation. 
If two fields $x,y$ transform in the vector representation of $GL(N)$, we denote by $(xy) = x \otimes y \in \lie{gl}(N)$ the induced adjoint valued element (this is used in terms 2,4,6, and 8). 
\appendix

\section{Chiral de Rham complexes and BRST reduction} 
\label{appx:cdr}

Suppose that $W$ is a symplectic vector space with a Hamiltonian action by a Lie group $G$ and moment map $\mu \colon W \to \lie{g}^*$. 
We will construct a model for the chiral de Rham complex on the Hamiltonian reduction of $W$ by $G$. 

The first step is to introduce a derived model for the symplectic reduction $W \sslash_\mu G = \mu^{-1} (0) / \lie{g}$.
Introduce the following graded vector space 
\beqn
W \sslash \lie{g} \define \lie{g}[1] \oplus W \oplus \lie{g}^*[-1] 
\eeqn
whose elements we will write as triples $(\sfc, \gamma, \til \sfb)$. 
There is an $L_\infty$ structure on $W\sslash \lie{g} [-1]$ determined by the brackets 
\beqn
\begin{split}
[c_1,c_2]_2 & = [c_1,c_2]_{\lie{g}} \in \lie{g} \\
[c, \gamma_1,\ldots,\gamma_k]_{k+1} & = \ip{c, \mu_{k+1}(\gamma_1,\ldots,\gamma_k)} \in W \\
[\gamma_1,\ldots,\gamma_l]_{l} & = \mu_l(\gamma_1,\ldots,\gamma_l) \in \lie{g}^* .
\end{split}
\eeqn

Choose a symplectic basis $\{w_\alpha\}$ for $W$ where the symplectic form is $\omega_{\alpha \beta}$. 
Additionally, let $\{e_a\}$ be a basis for the Lie algebra $\lie{g}$. 
We will assume for simplicity that the moment map is quadratic of the form $\mu^{a}$ 
In particular, the $L_\infty$ structure on $W\sslash \lie{g}[-1]$ is a strict graded Lie algebra. 
The vertex algebra is generated by the following fields
\begin{itemize}
\item $\gamma^\alpha(z)$ is a field of spin zero and ghost number zero. 
$\beta^\alpha (z)$ is a field of spin $+1$ and ghost number zero. 
\item $c^\alpha(z)$ is a field of spin $+1/2$ and ghost number $+1$. 
$b^\alpha (z)$ is a field of spin $+1/2$ and ghost number $-1$. 
\item $\sfc^a(z)$ is a field of spin zero and ghost number $+1$. 
$\sfb_a(z)$ is a field of spin $+1$ and ghost number $-1$. 
\item $\til \sfc^a(z)$ is a field of spin $+1$ and ghost number $+1$. 
$\til \sfb_a(z)$ is a field of spin zero and ghost number $-1$. 
\end{itemize}

The BRST operator is given in terms of the current
\beqn
J_{BRST}(z)  = f^a_{bc} \sfb_a \sfc^b \sfc^c + \mu_{a\alpha\beta} \sfc^a \beta^\alpha \gamma^\beta + \mu_{a \alpha \beta} \til \sfc^a \gamma^\alpha  \gamma^\beta + \mu_{a\alpha\beta} \sfc^a b^\alpha c^\beta + \mu_{a \alpha \beta} \til \sfc^a c^\alpha  c^\beta 
\eeqn
as $Q = \oint J_{BRST} (z) \, \d z$. 


\end{document}
