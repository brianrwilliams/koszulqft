\documentclass[11pt]{amsart}
\usepackage{cmupint}


\usepackage{stmaryrd}
\parskip=10pt



 \usepackage{hyperref}
\usepackage{tikz}
\usetikzlibrary{arrows, decorations.markings, decorations.pathmorphing, decorations.pathreplacing, shapes.arrows, patterns, calc}
\usepackage{url}

\tikzset{% arrow close to the source: the 0.2 determines where the arrow is drawn
  ->-/.style={decoration={markings, mark=at position 0.5 with {\arrow{to}}},
              postaction={decorate}},
}

\tikzset{% arrow close to the source: the 0.2 determines where the arrow is drawn
  -<-/.style={decoration={markings, mark=at position 0.5 with {\arrow{to reversed}}},
              postaction={decorate}},
}

\tikzset{% arrow close to the source: the 0.2 determines where the arrow is drawn
  dbl->-/.style={
double, 
double equal sign distance,
shorten >= 1pt,
shorten <= 1pt,
 decoration={markings, mark=at position 0.5 with {\arrow{implies}}},
              postaction={decorate}},
}


\tikzset{% arrow close to the source: the 0.2 determines where the arrow is drawn
  dbl-<-/.style={
double, 
double equal sign distance,
shorten >= 1pt,
shorten <= 1pt,
 decoration={markings, mark=at position 0.5 with {\arrowreversed{implies}}},
              postaction={decorate}},
}






\usepackage{amsmath,amsthm}
\textwidth=14.5cm \oddsidemargin=1cm  \evensidemargin=1cm\setlength{\parskip}{10pt} \setlength{\headsep}{20pt}

\pdfmapfile{+mathpple.map}
\usepackage{mathrsfs}
\usepackage{amscd,amssymb, amsfonts, verbatim,subfigure, enumerate}
\usepackage[mathcal]{eucal}
\usepackage[super]{nth}

\usepackage{mathpazo}



\linespread{1.2}  
\usepackage{color,slashed}



\setcounter{tocdepth}{1}

\renewcommand{\c}{\mf{c}}
\newcommand{\Lap}{\tr}
\newcommand{\dbar}{\br{\partial}}
\newcommand{\Dirac}{\slashed{\partial}}
\newcommand{\RHom}{\mbb{R}\op{Hom}}
\newcommand{\CP}{\mathbb{CP}}
\newcommand{\wbar}{\br{w}} 
\newcommand{\del}{\partial}
\renewcommand{\sl}{\mathfrak{sl}}
\renewcommand{\L}{\mathscr{L}}
\newcommand{\red}[1]{{\color{red}{#1}}} 
\newcommand{\zbar}{\br{z}}
\newcommand{\so}{\mathfrak{so}}
\newcommand{\Spin}{\op{Spin}}
\newcommand{\PV}{\op{PV}}
\newcommand{\GL}{\op{GL}}
\newcommand{\h}{\mathfrak{h}}
\newcommand{\Per}{\mscr{P}}
\newcommand{\dpa}[1]{\frac{\partial}{\partial #1}}
\newcommand{\dpas}[1]{\tfrac{\partial}{\partial #1}}
\newcommand{\Res}{\op{Res}}
\newcommand{\Sup}{\op{Sup}}
\newcommand{\Sing}{\op{Sing}}
\newcommand{\Diag}{\triangle}
\newcommand{\Obs}{\op{Obs}}
\newcommand{\su}{\mathfrak{su}}
\newcommand{\Der}{\op{Der}}
\newcommand{\eps}{\epsilon}
\newcommand{\g}{\mathfrak{g}}
\newcommand{\Hol}{\op{Hol}}
\renewcommand{\Re}{\op{Re}}
\newcommand{\xto}{\xrightarrow}
\newcommand{\what}{\widehat}
\newcommand{\tr}{\triangle}
\newcommand{\til}{\widetilde}
\newcommand{\mscr}{\mathscr}
\renewcommand{\det}{\operatorname{det}}
\newcommand{\br}{\overline}
\newcommand{\iso}{\cong}
\newcommand{\C}{\mathbb C}
\newcommand{\CC}{\mathbb C}
\newcommand{\N}{\mathbb N}
\newcommand{\Q}{\mbb Q}
\newcommand{\rarr}{\rightarrow}
\newcommand{\larr}{\leftarrow}
\newcommand{\norm}[1]{\left\| #1 \right\|}
\newcommand{\Oo}{\mscr O}
\newcommand{\Z}{\mathbb Z}
\newcommand{\defeq}{\overset{\text{def}}{=}}
\newcommand{\into}{\hookrightarrow}
\newcommand{\mbf}{\mathbf}
\newcommand{\mbb}{\mathbb}
\newcommand{\mf}{\mathfrak}
\newcommand{\mc}{\mathcal}
\newcommand{\from}{\leftarrow}
\newcommand{\R}{\mbb R}
\renewcommand{\d}{\mathrm{d}}
\newcommand{\liminv}{ \varprojlim }
\newcommand{\limdir}{\varinjlim}
\newcommand{\dirlim}{\varinjlim}
\newcommand{\bfP}{\mathbf{P}}
\newcommand{\gl}{\mf{gl}}
\newcommand{\cinfty}{C^{\infty}}
\newcommand{\cN}{\mathcal{N}}

\def\define{\overset{\rm def}{=}}
\def\ep{\varepsilon}
\def\zbar{{\overline{z}}}
\def\Weyl{{\rm Weyl}}
\def\Cl{{\rm C}\ell}
\def\U{{\rm U}}
\def\PV{{\rm PV}}
\def\thetabar{\Bar{\theta}}
\def\hotimes{\; \Hat{\otimes} \;} 
\def\C{{\rm C}}
\newcommand{\op}{\operatorname}
\def\ibar{{\Bar{i}}}
\def\jbar{{\Bar{j}}}
\def\CP{{\mathbb{CP}}}
\def\eps{{\epsilon}}
\def\what{\Hat}
\newcommand{\abs}[1]{\left| #1 \right|}
\newcommand{\ip}[1]{\left\langle #1 \right\rangle}
\def\oO{\cO}
\def\CP{{\mathbb{CP}}}
\def\bu{{\bullet}}
\def\DD{\mathbb{D}}
\def\cO{\mc O}
\def\ZZ{\mathbb{Z}}
\def\lie#1{\ensuremath{\mathfrak{#1}}}
\newcommand{\sfc}{\mathsf{c}}
\newcommand{\sfb}{\mathsf{b}}
\renewcommand\div{{\partial_\Omega}}


%math operators
\DeclareMathOperator*{\colim}{colim}
\DeclareMathOperator{\Aut}{Aut} \DeclareMathOperator{\End}{End}
\DeclareMathOperator{\Sym}{Sym} \DeclareMathOperator{\Hom}{Hom}
\DeclareMathOperator{\Spec}{Spec}
\DeclareMathOperator{\Diff}{Diff} 
\DeclareMathOperator{\Met}{Met} \DeclareMathOperator{\Vol}{Vol}
\DeclareMathOperator{\Tr}{Tr}

%comments
\def\brian#1{{\textcolor{blue!65!red}{BRW: {#1}}}}
\def\natalie#1{{\textcolor{green!65!black}{BRW: {#1}}}}


%eqn macros
\newcommand\beqn{\begin{equation}}
\newcommand\eeqn{\end{equation}}


%thm styles
\newtheoremstyle{thm}% name
  {7pt}%      Space above
  {7pt}%      Space below
  {\itshape}%         Body font
  {}%         Indent amount (empty = no indent, \parindent = para indent)
  {\bf}% Thm head font
  {.}%        Punctuation after thm head
  {5pt}%     Space after thm head: " " = normal interword space;
         %       \newline = line-break
  {\thmnumber{#2 }\thmname{#1}\thmnote{ (#3)}}%         Thm head spec (can be left empty, meaning `normal')



\newtheoremstyle{def}% name
  {7pt}%      Space above
  {10pt}%      Space below
  {\itshape}%         Body font
  {}%         Indent amount (empty = no indent, \parindent = para indent)
  {\bf}% Thm head font
  {.}%        Punctuation after thm head
  {5pt}%     Space after thm head: " " = normal interword space;
         %       \newline = line-break
  {\thmnumber{#2} \thmname{#1}\thmnote{ (#3)}}%         Thm head spec (can be left empty, meaning `normal')



\newtheoremstyle{rem}% name
  {4pt}%      Space above
  {10pt}%      Space below
  {}%         Body font
  {}%         Indent amount (empty = no indent, \parindent = para indent)
  {\itshape}% Thm head font
  {:}%        Punctuation after thm head
  {3pt}%     Space after thm head: " " = normal interword space;
        %       \newline = line-break
  {}%         Thm head spec (can be left empty, meaning `normal')

\newtheoremstyle{texttheorem}% name
  {8pt}%      Space above
  {8pt}%      Space below
  {\itshape}%         Body font
  {}%         Indent amount (empty = no indent, \parindent = para indent)
  {\bf}% Thm head font
  {. \hspace{5pt}}%        Punctuation after thm head
  {3pt}%     Space after thm head: " " = normal interword space;
        %       \newline = line-break
  {}%         Thm head spec (can be left empty, meaning `normal')




\theoremstyle{thm}

\newtheorem*{claim}{Claim}
\newtheorem*{theorem*}{Theorem}
\newtheorem*{lemma*}{Lemma}
\newtheorem*{corollary*}{Corollary}
\newtheorem*{proposition*}{Proposition}
\newtheorem*{definition*}{Definition}
\newtheorem{ntheorem}{Theorem}
\newtheorem*{thmA}{Theorem A}
\newtheorem*{thmB}{Theorem B}
\newtheorem*{thmC}{Theorem C}
\newtheorem*{conjecture}{Conjecture}

\newtheorem{theorem}{Theorem}[subsection]
\newtheorem{thm-def}{Theorem/Definition}[theorem]
\newtheorem{prop}[theorem]{Proposition}
\newtheorem{question}{Question}
\newtheorem*{question*}{Question}
\newtheorem{lemma}[theorem]{Lemma}
\newtheorem{sublemma}[theorem]{Sub-lemma}
\newtheorem{notation}[theorem]{Notation}
\newtheorem{corollary}[theorem]{Corollary}
\newtheorem{deflem}[theorem]{Definition-Lemma}
\newtheorem*{hope}{Hope}
\numberwithin{equation}{subsection}


\theoremstyle{def}
%\theoremstyle{definition}
\newtheorem{definition}[theorem]{Definition}
\newtheorem*{udefinition}{Definition}


\theoremstyle{rem}
%\declaretheorem[style=rem, numbered=no,qed=$\diamondsuit$]{remark}
%\declaretheorem[style=rem, numbered=no,qed=$\diamondsuit$]{remarks}
%\declaretheorem[style=rem, numbered=no,qed=$\diamondsuit$]{example}

\newtheorem*{remark}{Remark}
\newtheorem*{remarks}{Remarks}
\newtheorem*{example}{Example}

%\theoremstyle{texttheorem}
%\newtheorem{textlemma}[theorem]{}

%%This document


\title{K3 draft}
\date{\today}

%\author{Kevin Costello}
%\author{Natalie M. Paquette}
%\thanks{CALT-TH 2019-050}
%
%\address{Perimeter Institue for Theoretical Physics}
%\address{Walter Burke Institute for Theoretical Physics, California Institute of Technology}
%\email{kcostello@perimeterinstitute.ca}
%\email{nataliep@caltech.edu}


\begin{document}

\maketitle

\tableofcontents

\section{Twisted supergravity on a $K3$ surface}

\subsection{Kodaira--Spencer theory and the closed string $B$-model}

\subsection{Compactification of Kodaira--Spencer theory}

Recall that $\PV (Y) = \PV^{\bu,\bu}(Y)$ denotes the graded space given by the Dolbeault resolution of holomorphic polyvector fields on a complex manifold $Y$.

Let $Y$ be a complex surface with a fixed holomorphic symplectic structure.
The fields of Kodaira--Spencer theory on $\CC^3 \times Y$ are polyvector fields which are in the kernel of the divergence operator.
A polyvector field on $\CC^3 \times Y$ can be written as a tensor product of one on $\CC^3$ with one on $Y$.  
Polyvector fields on $Y$ are the same as differential forms, because the holomorphic symplectic form on $Y$ identifies the tangent and cotangent bundles. 
In particular, the harmonic polyvector fields are given simply by the de Rham cohomology of $Y$.  
Furthermore, polyvector fields on $Y$ which are harmonic are automatically in the kernel of the divergence operator $\div$, by standard Hodge theory arguments.   
Summarizing there is an equivalence of graded algebras
\[
\PV (\CC^3) \otimes \bigg(\ker \div |_{\PV(Y)} \bigg) \simeq \PV(\CC^3) \otimes H^\bu(X) .
\]
We will use this equivalence to describe the fields of the theory on $\CC^3$ upon compactification along $Y$. 

Let $A = H^\bu(Y)$ denote the cohomology ring of $Y$.
We are mostly interested in the case that $Y$ is a K3 surface in which case this algebra is generated by even elements $\eta, \br \eta, \eta_a$ for $a=1,\ldots 20$ subject to the relations
\beqn
\label{eqn:K3rel}
\begin{split}
\eta^2 & = \Bar{\eta}^2 = 0 \\
\eta_a \eta_b & = h_{ab} \eta \Bar{\eta} 
\end{split}
\eeqn
where $h_{ab}$ is a non-degenerate symmetric pairing on $\CC^{20}$. 
Let $I$ denote the ideal generated by these equations so that $A = \CC[\eta,\Bar{\eta}, \eta_a] / I$. 

As usual, we write the polyvector fields on $\CC^3$ in terms of a superspace by introducing odd variables $\theta^i$, $\br{\theta}_{\br{j}}$.  
Here, $\theta^i$ represents the coordinate vector field $\partial_{z_i}$ and $\br \theta_i$ represents the coordinate Dolbeault form $\d \zbar_i$. 
Then we can write the field content as a collection of superfields
\begin{equation} 
		\mu(z,\zbar,\theta^i, \br{\theta}_{\br{i}},\eta) \in \oplus_{i,j}  \PV^{i,j}(\CC^3) \otimes A .
\end{equation}
Here, we are using the shorthand $\eta$ to inform that there is a dependence on $\eta, \br{\eta}$, and $\eta_a$, $a=1,\ldots, 20$. 
As such, such a superfield decomposes in its dependencies on the generators of the cohomology of $X$ as
\begin{multline}
\mu(z,\zbar,\theta^i, \br{\theta}_{\br{i}}) \\
+ \mu_\eta (z,\zbar,\theta^i, \br{\theta}_{\br{i}}) \eta + \mu_{\br{\eta}} (z,\zbar,\theta^i, \br{\theta}_{\br{i}}) \br{\eta} + \mu^a (z,\zbar,\theta^i, \br{\theta}_{\br{i}}) \eta_a \\
+ \mu_{\eta \Bar{\eta}} (z,\zbar,\theta^i, \br{\theta}_{\br{i}}) \eta \Bar{\eta} .
\end{multline}
We emphasize that the $\eta$-variables represent harmonic polyvector fields on $X$ and so are not acted on by any differential operators along $\CC^3$. 

%The fields on $\C^3$ must satisfy the constraint 
%\begin{equation} 
%	\partial_{z_{i_1}} \mu_{i_1 \dots i_l; a}^{\br{j}_1 \dots \br{j}_k} = 0. 
%\end{equation}
The superfield satisfies the equation $\partial \mu = 0$
where, in the superspace formulation,
\begin{align} 
	\dbar & \, = \, \br{\theta}_{\br{j}} \partial_{\zbar_{\br{j}}} \\
	\partial &\, = \, \partial_{\theta^i} \partial_{z_i}.  
\end{align}
	The Lagrangian is
	\begin{equation} 
		\tfrac{1}{2} \int_{\CC^{3 \mid 6}}  \mu \dbar \partial^{-1} \mu \d^3 z \d^3 \zbar \d^3 \theta \d^3 \br{\theta} |_{\eta \br{\eta}} + \frac{1}{6} \int_{\CC^{3 \mid 6}} \mu^3 \d^3 z \d^3 \zbar \d^3 \theta \d^3 \br{\theta} |_{\eta \br{\eta}}  
	\end{equation}
where the $(-)|_{\eta \br{\eta}}$ means we pick up only the $\eta \br{\eta}$ component.

Before we turn to the computation of the backreaction, we will simplify the field content somewhat, following \cite{CostelloGaiotto}.  
We note that the coefficient of $\theta^1 \theta^2 \theta^3$ does not appear in the kinetic term in the action.  
This field does not propagate, so we can (and will) impose the additional constraint
\begin{equation}\label{eq:nonprop} 
	\partial_{\theta^1} \partial_{\theta^2} \partial_{\theta^3} \mu (z,\zbar,\theta,\br{\theta},\eta) = 0. 
\end{equation}

Next, let us expand the superfield $\mu$ only in the $\theta^i$ variables:
\begin{equation} 
	\mu = \mu(z,\zbar,\br{\theta},\eta) + \mu_{i}(z,\zbar,\br{\theta},\eta) \theta^i + \dots 
\end{equation}
We note that the constraint $\partial \mu_{ij} = 0$ implies that there is some super-field
\begin{equation} 
	\what{\mu}_{ijk}(z,\zbar,\br{\theta},\eta) = 	\alpha(z,\zbar,\br{\theta},\eta) \eps_{ijk}   
\end{equation}
so that $\partial_{z_i} \what{\mu}_{ijk} = \mu_{jk}$.

It is convenient to rephrase the theory in terms of the field $\alpha(z,\zbar,\br{\theta},\eta)$, which has no holomorphic index. 
We will also change notation and let $\gamma(z,\zbar,\br{\theta},\eta)$ be the term with no $\theta^i$ dependence in the superfield $\mu(z,\zbar,\theta,\br{\theta},\eta)$.  

In summary, we have the following fundamental superfields in the compactified theory on $\CC^3$:
\begin{itemize}
\item $\mu_i (z,\zbar,\br{\theta}, \eta) \theta^i$ which we identify with an element in the graded space
\beqn
\mu \in \PV^{1,\bu}(\CC^3) \otimes A .
\eeqn
\item $\alpha (z,\zbar,\br \theta, \eta)$ which we identify with an element of the graded space
\beqn
\alpha \in \Omega^{0,\bu}(\CC^3) \otimes A .
\eeqn
\item $\gamma(z,\zbar,\br \theta, \eta)$ which we also identify with an element of the graded space
\beqn
\gamma \in \Omega^{0,\bu}(\CC^3) \otimes A .
\eeqn
\end{itemize}
In terms of these fields, the Lagrangian is
\begin{multline} 
	\tfrac{1}{2}\int_{\CC^{3|3}}   \eps^{ijk} \dbar \mu_{i} (\partial^{-1}  \mu)_{jk} \, \d^3 z \d^3 \zbar \d^3 \br{\theta} |_{\eta \br \eta}   + \int  \alpha \dbar \gamma \d^3 z \d^3 \zbar \d^3 \br{\theta} |_{\eta \br{\eta}} 
	\\
	+ \tfrac{1}{6} \int_{\CC^{3|3}}  \eps_{ijk} \mu_{i}\mu_{i} \mu_{c,i} \, \d^3 z \d^3 \br{z} \d^3 \br{\theta} |_{\eta \br{\eta}} + \int  \alpha \mu_i \partial_{z_i}  \gamma \, \d^3 z \d^3 \zbar \d^3 \br{\theta} |_{\eta \br{\eta}} .
\end{multline} 
In this expression we project onto the component $\eta \br \eta$ and perform the Berezin integral along the $\br \theta$ coordinate.

Just as when we twist a field theory, when we twist a supergravity theory the ghost number of the twisted theory  is a mixture of the ghost number and a $U(1)_R$-charge of the original physical theory. To define a consistent ghost number, one can choose any $U(1)_R$ in the physical theory under which the supercharge has weight $1$.  In general, there are many ways to do this.  It is convenient for us to make the following assignments of ghost number.
\begin{enumerate} 
	\item The variables $\eta_a$ \brian{finish}
	\item The anti-commuting variables $\br{\theta}_i$ have ghost number $1$.
	\item The fields $\alpha$, $\gamma$ have ghost number $-1$, and are fermionic.
\end{enumerate}

\subsection{Backreaction as an infinitesimal deformation} 
\label{sec:conifold}

In the type IIB theory on $\CC^3 \times K3$ we consider a D1--D5 system where the D1 branes wrap 
\[
\CC \times 0 \times \{x\} \subset \CC \times \CC^2 \times K3 
\]
for some $y \in K3$ and the D5 branes wrap 
\[
\CC \times 0 \times K3 \subset \CC \times \CC^2 \times K3 .
\]
We can apply a duality to turn this into a D3 brane system which wraps 
\[
\CC \times 0 \times \Sigma \subset \CC \times \CC^2 \times K3 
\]
for some two-cycle $\Sigma \subset K3$. 
\brian{say this precisely}

In the last section, we argued that the dimensional reduction along a $K3$ surface becomes an extended version of Kodaira--Spencer theory where the extra fields are labeled by the cohomology of the surface.
Upon compactification, the D3 system becomes a system of $B$-type branes in this extended version of Kodaira--Spencer theory. 

The charge of these branes is labeled by a cohomology class 
\beqn
F \in H^2(K3) \subset A .
\eeqn
We denote 
\beqn
N \define \ip{F, F}
\eeqn
using the inner product on $H^2(Y)$. 
Explicitly, if $F = f \eta + \br f \br \eta + f^a \eta_a$ for $f, \br f, f_a$ complex numbers, then $N = f \br f + f^{a} f^{b} h_{ab}$ where $h_{ab}$ is the fixed non-degenerate symmetric pairing. 
\brian{Can we say how $N$ is related to $N_1$ and $N_5$? This would require tracing through some dualities carefully...}

Let's choose coordinates $z,w_1,w_2$ on $\CC^3$, where the branes wrap the $z$-plane along $w_1=w_2=0$. 
Including the backreaction will deform the geometry away from the locus of the brane. 
Before backreacting, we should say what geometry is actually being deformed. 
Recall that in the case of ordinary Kodaira--Spencer theory on $\CC^3$, it was shown in \cite{CGhol} that the backreaction of B-branes along $\CC \subset\CC^3$ deformed the complex structure on $\CC^3 \setminus \CC$ to the {\em deformed conifold} $SL_2(\CC)$. 

Our case is similar in that the branes are supported along the same locus as in \cite{CGhol}.
The difference is that we are working with a bigger space of fields, roughly extended by the cohomology of the $K3$ surface. 
We will now show that in the case of type IIB compactified on a K3 surface, the backreaction determines an {\em infinitesimal} deformation of the complex manifold $\CC^3 \setminus \CC$ over the fat point $\Spec A$ where $A = H^\bu(K3)$. 
 
If $A$ is any local ring, an infinitesimal deformation of a complex manifold $M_0$ over $\Spec A$ is an element 
\beqn
\mu_{def} \in \PV^{1,1}(M_0) \otimes \mathfrak{m}_A 
\eeqn
satisfying the Maurer--Cartan equation.
In our case, $M_0 = \CC^3 \setminus \CC$ and $\mu_{def}$ is a supergravity field sourced by the branes. 
The Maurer--Cartan equation is the equation of motion for $\mu_{def}$. 
%Recall that the cohomology ring of a K3 surface (with complex coefficients) is generated by even elements $\eta, \br \eta, \eta_a$, where $a=1,\ldots,20$ subject to the relations \eqref{eqn:K3rel}. 
%These relations define a quadric $Z$ inside $\CC^{22}$. 
%Thus, in total, we will see that the backreaction will deform the complex structure on 
%\[
%(\CC^3 \setminus \CC) \times Z \subset \CC^3 \times \CC^{22} .
%\]
%Before deforming, there is an obvious projection map 
%\[
%(\CC^3 \setminus \CC) \times Z \to Z 
%\] 
%whose fiber is a Calabi--Yau manifold equipped with the standard holomorphic volume form $\Omega_{\CC^3} = \d z \d w_1 \d w_2$. 
The cohomology ring $A$ of a $K3$ surface is a local ring.
Following \cite{CGhol}, the backreaction of this system of branes introduces a twisted supergravity field
\[
\mu_{BR} \in \br\PV^{1,1}(\CC^3) \otimes A 
\]
which we can identify with an element of $\br\Omega^{2,1}(\CC^3) \otimes A$ using the Calabi--Yau form on $\CC^3$. 
This field satisfies the following equations of motion
\beqn
\label{eqn:mcbr}
	\begin{split}
		\dbar \mu_{BR}  & = F \, \delta_{\CC \subset \CC^3} \\
		\del \mu_{BR} & = 0 .
	\end{split}
\eeqn
For quantization we will also impose the gauge fixing condition that $\dbar^\ast \mu_{BR}(\eta_a) = 0$. 	
There is a unique solution to the above equations given by
\beqn
\mu_{BR} = \frac{\eps^{ij} \br w_i \d \br w_j}{|w|^4} \partial_z \otimes F .
\eeqn
Note that this field is of the form $\mu_{BR,0} \otimes F$ where $\mu_{BR,0}$ is the Beltrami differential which gives rise to the deformed conifold \cite{CGhol}---all of the dependence on the parameters $\eta, \br \eta, \eta_a$ is in the cohomology class~$F$.

Equations \eqref{eqn:mcbr} imply that $\mu_{BR}$ determines an infinitesimal deformation of~$\CC^3 \setminus \CC$ over $\Spec A$. 
The Kodaira--Spencer map associated to this infinitesimal deformation is of the form
\[
KS \colon T_{\Spec A} \to H^1(\CC^3 \setminus \CC, T) 
\]
and simply maps a derivation $\delta$ of $A$ to the class 
\[
\delta(F) \left[\frac{\eps^{ij} \br w_i \d \br w_j}{|w|^4} \del_z \right] \in H^1(\CC^3 \setminus \CC, T) .
\] 
\brian{what more to say?}
%This Beltrami differential defines a new complex structure on the complex manifold $(\CC^3 \setminus \CC) \times Z$ where $Z$ is the quadric inside $\CC^{22}$ defined by \eqref{eqn:K3rel}. 
%Note that \brian{how to say things remain algebraic along $Z$?}
%A function $\Phi(z,\br z, w_i, \br w_i , \eta, \br \eta, \eta_a)$ is holomorphic if the following equations hold
%\beqn
%	\begin{split}
%		\del_{\br z} \Phi & = 0 \\
%		\del_{\br w_1} \Phi - F \frac{\br w_2}{\norm{w}^2} \del_z \Phi & = 0 \\
%		\del_{\br w_2} \Phi + F \frac{\br w_1}{\norm{w}^2} \del_z \Phi & = 0 .
%	\end{split}
%\eeqn
%It is easy to check that the following functions are holomorphic for the deformed complex structure 
%\beqn
%	\begin{split}
%		u_1 & = w_1 z - F \frac{\br w_2}{\norm{w}^2} \\
%		u_2 & = w_2 z + F \frac{\br w_1}{\norm{w}^2} .
%	\end{split}
%\eeqn
%
%In total, we are describing the affine variety $X_F$ inside of $\CC^4 \times \CC^{22}$ defined by the equations 
%\beqn
%\label{eqn:conifold}
%	\begin{split}
%		\eta^2 = \br \eta^2 & = 0 \\
%		\eta_a \eta_b & = h_{ab} \eta \br \eta \\
%		w_1 u_2 - w_2 u_1 & = F . 
%	\end{split}
%\eeqn
%Here, $(u_1,u_2,w_1,w_2)$ are the holomorphic coordinates on $\CC^4$ and $\eta,\br \eta, \eta_a$ are the holomorphic coordinates on $\CC^{22}$. 
%Denote by $X_F^0 \subset X_F$ the open subset where $u_i,w_j$ are all not zero. 
%
%We observe that upon deforming the complex structure, there is still a fibration
%\[
%X_{F}^0 \to Z 
%\]
%whose fibers are Calabi--Yau three-folds. 
%The volume form in the coordinates $z, w_1,w_2$ is unchanged when we deform the complex structure.
%This is because the Beltrami differential $\mu_{BR}$ is divergence-free. 
%In the coordinates $u_i, w_i$ the holomorphic volume on the fibers of this projection reads
%\beqn
%\Omega|_{w_1 \ne 0} = \frac{\d u_1 \d w_1 \d w_2}{w_1}, \qquad \Omega|_{w_2 \ne 0} = \frac{\d u_2 \d w_1 \d w_2}{w_2} . 
%\eeqn
%
%In \cite{CP1} the first two authors considered Kodaira--Spencer theory on the complex surface $T^4$ and found that the resulting backreaction only gave finite-order corrections to operator products on flat space. 
%We are in a very similar situation here in the case of Kodaira--Spencer theory compactified on a K3 surface. 
%The flat space background corresponds to $F = 0$. 
%Now, since $F \in H^2(Y)$, we see that $F^3 = 0$ in the ring $A = H^\bu (Y)$.
%Thus, only the only powers of $F$ that appear in corrections to the flat space OPE are $F$ and $F^2$. 

%\subsection{Partial compactification of the extended conifold}
%\label{sec:compact} 
%
%The ordinary deformed conifold is biholomorphic to $SL_2(\CC)$. 
%One can view this as a subvariety of $\CC^4$ with coordinates $u_i,w_j$ satisfying $\eps^{ij} u_i w_j = N$. 
%To compactify this subvariety the idea is to introduce an additional homogeneous coordinate $\lambda$ and consider the projective variety in $\CP^4$ defined by the equation $\eps^{ij} u_i w_j = N \lambda^2$ where the $u_i,w_j,\lambda$ are now projective coordinates. 
%The boundary of this projective variety is described by the equation $\eps^{ij} u_i w_j = 0$ and is equivalent to $\CP^1 \times \CP^1$. 
%
%We use the new coordinate $z$ for the first $\CP^1$ which is the boundary of the Euclidean ${\rm AdS}_3$ space
%and the new coordinate $w$ for the second $\CP^1$.  
%Let $n$ be the coordinate normal to the boundary $\CP^1 \times \CP^1$ of $SL_2(\CC)$. 
%This coordinate has a first order pole at $z=\infty$ and $w=\infty$. 
%Let $\DD_z \times \CP^1_w \subset \CP^1_z \times \CP^1_w$ be a neighborhood of the algebraic curve $0 \times \CP^1_w$ in the boundary divisor. 
%A neighborhood of this region in the whole manifold $\br{X_{F}^0}$ is a deformation of the total space of $\cO(1) \to \DD \times \CP^1$ by the Beltrami differential 
%\[
%n^2 \d ...
%\]
%
%
%The situation here is very similar. 
%The extended conifold $X_F^0$ has a completion $\br{X_F^0}$ which is a subvariety of $\CP^4 \times Z$. 
%It is defined by modifying the last equation in \eqref{eqn:conifold} to
%\beqn
%\eps^{ij} u_i w_j = F \lambda^2 
%\eeqn
%where $\lambda$ is an additional homogenous coordinate.
%The boundary of $\br{X_F^0}$ is the subvariety of $\CP^3 \times Z$ defined by $\eps^{ij} u_i w_j = 0$ which is $\CP^1 \times \CP^1 \times Z$. 

\section{The twisted symmetric orbifold CFT}

Supergravity on $AdS_3 \times S^3 \times X$, where $X$ is either $T^4$ or a $K3$ surface, is expected to be holographically dual to a particular two-dimensional superconformal field theory (SCFT). 
Here we review this system of interest, following \cite{Davidetal} and references therein.
We also review the notion of the chiral de Rham complex of a manifold, and its generalization to orbifolds, in order to discuss the chiral algebra associated with the half-twist of the $Sym^N(T^4)$ SCFT, putatively dual to Kodaira-Spencer theory on the superconifold. Of course, this SCFT is the IR limit of the field theory that arises from the zero modes of the open strings on the D1-D5 branes. 
The lowest-lying modes of open strings, which provide an effective field theory description of the D1 and D5-branes, naturally furnish a gauge theory whose IR limit we are primarily interested in.  
The D5-D5 strings give rise to a six-dimensional supersymmetric $U(N_5)$ gauge theory preserving 16 supercharges. 
When all the D-branes are coincident the gauge theory is in the Higgs phase and when some of the adjoint scalars in the field theory acquire a vev, corresponding to transverse separation of the branes, the theory is in the Coulomb phase. 
We will focus on the Higgs branch of the gauge theory throughout, which involves turning on a nonvanishing Fayet-Iliopoulos parameter (dually, NS B-field). 
We reduce four directions of the gauge theory on 
\[
X = T^4 \quad \text{or} \quad K3
\]
which results in an effective two-dimensional $U(N_5)$ gauge theory which preserves 16 supercharges.
The D1-D1 strings similarly produce a $U(N_1)$ gauge theory preserving 16 supercharges. 
More interesting are the D1-D5 and D5-D1 strings, which break the total supersymmetry down to 8 supercharges (though more supersymmetries will be obtained in the near-horizon/low energy limits, so that the dual pair of theories has 16 supersymmetries overall). 
These strings produce matter multiplets transforming in the bifundamental representations of the gauge groups. 

On the Higgs branch, one must solve the vanishing of the bosonic potential (i.e. D-flatness equations) modulo the gauge symmetries $U(N_1)\times U(N_5)$ to obtain the moduli space. 
If one imagined that both sets of D-branes were supported on a noncompact six-dimensional space, these D-flatness equations can be rewritten to reproduce the ADHM equations for $N_1$ instantons of a six-dimensional $U(N_5)$ gauge theory a la \cite{WittenADHM}. 
In fact, it has been argued that the instanton moduli space is the more accurate description of the dual field theory, so that one should study the moduli space of $N_1$ instantons of a $U(N_5)$ gauge theory on $T^4$, i.e. the Hilbert scheme of $N_1 N_5$ points on $T^4$ \footnote{Throughout this note we ignore the center of mass factor of the moduli space that produces a $\tilde{T}^4$ factor, for some $\tilde{T}^4$ not necessarily the same as the compactification $T^4$. The relationship between the two tori is clarified in \cite{GiveonKutasovSeiberg}.}. The (conformally invariant limit of the) gauge theory description is expected to only capture the regime of vanishing size instantons (i.e. when the hypermultiplets have small vevs). One can understand that the gauge theory description is approximate by noticing that the Yang-Mills couplings are given in terms of the $T^4$ volume $V$ and string coupling as $g_1^2 = g_s (2 \pi \alpha'), g_5^2 = g_s V/(\alpha' (2\pi)^3)$ so for energies much smaller than the inverse string length the gauge theories are strongly coupled \cite{Davidetal}. 


To get the SCFT we take an IR limit, which would be dual to a near-horizon limit from the closed string point of view. In this limit, the gauge theory moduli space becomes the target space of the low-energy sigma-model. It has been argued that the correct instanton moduli space is a smooth deformation of the symmetric product theory $Sym^{N_1 N_5}(\tilde{T}^4)/S_{N_1 N_5}$. Indeed, there is a point in the SCFT moduli space (far from the supergravity point itself) where the theory takes precisely the symmetric orbifold form. The orbifold point is the analogue of free Yang-Mills theory in the perhaps more-familiar $AdS_5\times S^5$/ 4d $\mc N =4$ SYM duality, and is dual to a stringy point in moduli space which has been explored extensively in recent years \cite{Eberhardtetal}.
 

\subsection{The symmetric orbifold SCFT}

With this background in mind, we go directly to the SCFT description of the orbifold point in moduli space. We will be particularly interested in protected, moduli-independent quantities that can still be compared to the supergravity point in moduli space. We will take the branes to be supported on $\R \times S^1$ after $T^4$ compactification, so that the CFT is defined on the cylinder. On the cylinder, the NS sector corresponds to anti-periodic boundary conditions on the fermions. The sigma model is then the $\mc N = (4,4)$ theory whose bosonic fields are valued in maps from $S^1 \rightarrow Sym^N(T^4)$.  

The physical SCFT has R-symmetries $SO(4) \simeq SU(2)_L \times SU(2)_R$ dual to rotations of the $S^3$ and symmetries under a global $SO(4)_I \simeq SU(2)_a \times SU(2)_b$ of transverse rotations; this latter symmetry is broken by compactification on $T^4$. The latter $SO(4)_I$, although broken by the background, is still often used to organize the field content of the theory, and acts as an outer automorphism on the $\mc N=(4, 4)$ superconformal algebra. As is well known, the isometries of $AdS_3 \times S^3$ are $SL(2, \R) \times SL(2, \R) \times SO(4)$ form the bosonic part of the supergroup $SU(1,1|2) \times SU(1,1|2)$ which preserve the supergravity vacuum and form the anomaly-free global subalgebra of the $\mc N= (4,4)$ superconformal algebra.

The orbifold theory can be described in terms of free fields on $N:= N_1N_5$ copies of the $T^4$ theory. We write $SU(2)_a \times SU(2)_b$ doublet indices as $A, \dot{B}$, $SU(2)_L\times SU(2)_R$ doublet indices as $\alpha, \dot{\beta}$. $SO(4)_I$ vector indices will be denoted by $i,j$, etc and subscripts $(r), r= 1,\ldots, N$ label the orbifold copy number.  

Each $T^4$ theory has four free bosons $X^i_{(r)}$ and eight free fermions, the left-movers $\psi^{\alpha \dot{A}}_{(r)}(z)$ and right-movers $\bar{\psi}^{\dot{\alpha} \dot{A}}_{(r)}(\bar{z})$, for fixed copy $(r)$ that satisfy the reality conditions
\begin{align*}
\psi^{\dagger}_{\alpha \dot{A}} &= -\epsilon_{\alpha \beta}\epsilon_{\dot{A} \dot{B}}\psi^{\beta \dot{B}} \\ 
\bar{\psi}^{\dagger}_{\dot{\alpha} \dot{A}} &= -\epsilon_{\dot{\alpha} \dot{\beta}}\epsilon_{\dot{A} \dot{B}}\bar{\psi}^{\dot{\beta} \dot{B}}.
\end{align*}

In terms of the free fields, we construct the holomorphic $\mc N=4$ superconformal algebra generators (similar expressions hold for the right-movers). In what follows, we have implicitly performed the diagonal sum over the copy index of all fields to obtain $S_{N_1 N_5}$-invariant expressions:
\begin{align*}
J^{a}(z) &= {1 \over 4} \epsilon_{\dot{A} \dot{B}} \psi^{\alpha \dot{A}}\epsilon_{\alpha\beta}(\sigma^{* a})^{\beta}_{\gamma} \psi^{\gamma \dot{B}}\\
G^{\alpha A}(z) &= \psi^{\alpha \dot{A}}\left[\partial X \right]^{\dot{B}A}\epsilon_{\dot{A} \dot{B}}\\
T(z) &= {1 \over 2} \epsilon_{\dot{A}\dot{B}}\epsilon_{AB}\left[\partial X\right]^{\dot{A}A}\left[\partial X \right]^{\dot{B}B} + {1\over 2}\epsilon_{\alpha \beta}\epsilon_{\dot{A}\dot{B}} \psi^{\alpha \dot{A}}\partial \psi^{\beta \dot{B}}
\end{align*} with $a$ an $SU(2)_L$ triplet index and using the notation $\left[ X \right]^{A \dot{A}} = {1 \over \sqrt{2}}X^i (\sigma^i)^{\dot{A} A}$ using the usual Pauli matrices plus $\sigma^4 = i \mbb 1_2$.

The free fields are normalized in the usual way,
\begin{align*}
\langle X^i(z) X^j(w) \rangle &= -2 \delta^{ij}\log|z-w| \\
\langle \psi^{\alpha \dot{A}} \psi^{\beta \dot{B}} \rangle &= - {\epsilon^{\alpha \beta}\epsilon^{\dot{A}\dot{B}} \over z-w}
\end{align*}
using which one can verify that the generators for each copy indeed satisfy the OPEs for the $\mc N=4$ superconformal algebra with $c=6$ and the diagonal sum for $c=6N$ \textcolor{red}{Kodaira-Spencer theory seems to miss the $\partial J$ terms in the TJ, GG OPEs below. Do they come from the diagrams including a single line sourcing the Beltrami differential?}:
\begin{align*}
J^a(z)J^b(w) &\sim  {c \over 12}{\delta^{ab} \over (z-w)^2} + i \epsilon^{ab}_c {J^c(w) \over z-w}\\
J^a(z)G^{\alpha A}(w) &\sim {1 \over 2} (\sigma^{*a})^{\alpha}_{\beta} {G^{\beta A}(w) \over z-w}\\
G^{\alpha A}(z)G^{\beta B}(w) &\sim  - \epsilon^{AB}\epsilon^{\alpha \beta}{T(w) \over z-w} - {c \over 3}{\epsilon^{AB} \epsilon^{\alpha \beta} \over (z-w)^3} + \epsilon^{AB}\epsilon^{\beta\gamma}(\sigma^{*a})^{\alpha}_{\gamma}\left({2 J^a(w) \over (z-w)^2} + {\partial J^a(w) \over z-w} \right)\\
T(z)J^a(w) &\sim {J^a(w) \over (z-w)^2} + {\partial J^a(w) \over z-w}\\
T(z)G^{\alpha A}(w)&\sim { {3 \over 2} G^{\alpha A}(w) \over (z-w)^2} + {\partial G^{\alpha A}(w) \over z-w}\\
T(z)T(w) &\sim {c \over 2}{1 \over (z-w)^4} + 2 {T(w) \over (z-w)^2} + {\partial T(w) \over z-w}.
\end{align*}
It is also easy to derive the OPEs of these generators with the basic free primaries:
\begin{align*}
J^a(z)\psi^{\alpha \dot{A}}(w) &\sim {1 \over 2}(\sigma^{*a})^{\alpha}_{\beta}{\psi^{\beta \dot{A}}(w) \over z-w}\\
G^{\alpha A}(z)\left[\partial X(w)\right]^{\dot{B}B} &\sim \epsilon^{AB}\left({\psi^{\alpha \dot{B}}(w) \over (z-w)^2} + {\partial \psi^{\alpha \dot{B}}(w) \over z-w} \right)\\
G^{\alpha A}(z)\psi^{\beta \dot{A}}(w) &\sim \epsilon^{\alpha \beta}{\left[\partial X(w) \right]^{\dot{A}A} \over z-w}\\
T(z) \left[\partial X(w) \right]^{\dot{A}A} &\sim { \left[\partial X(w) \right]^{\dot{A}A} \over (z-w)^2} + {\left[\partial^2 X(w) \right]^{\dot{A}A} \over z-w}\\
T(z) \psi^{\alpha \dot{A}} &\sim {{1\over2} \psi^{\alpha \dot{A}}(w) \over (z-w)^2} + {\partial \psi^{\alpha \dot{A}}(w) \over z-w}.
\end{align*}
 As always, we define the modes $\mc O_m$ of a field $\mc O(z)$ in terms of its weight $\Delta$:
 \begin{equation}
 \mc O_m = \oint {dz \over 2\pi i} \mc O(z) z^{\Delta + m-1}.
 \end{equation}
 It is easy from the above OPEs to get the mode algebra of the $\mc N=4$ superconformal algebra. For simplicity, we will just record the mode algebra of the global subalgebra generated by $\left\lbrace J_0^a, G^{\alpha A}_{\pm 1/2}, L_0, L_{\pm 1} \right\rbrace$, which has its Cartan subalgebra generated by $J_0^3, L_0$: 
 \begin{align}
 \left[L_0, L_{\pm 1} \right]&= \mp L_{\pm}\\
 \left[L_1, L_{-1}\right] &= 2 L_0 \\
 \left[J^a_0, J^b_0 \right]&= i \epsilon^{a b}_c J^c_0 \\
 \left\lbrace  G^{\alpha A}_{1/2}, G^{\beta B}_{-1/2}\right\rbrace &= \epsilon^{AB}\epsilon^{\beta\gamma}(\sigma^{*a})^{\alpha}_{\gamma}J^a_0 -\epsilon^{AB}\epsilon^{\alpha \beta}L_0 \\
 \left\lbrace  G^{\alpha A}_{-1/2}, G^{\beta B}_{1/2}\right\rbrace &= -\epsilon^{AB}\epsilon^{\beta\gamma}(\sigma^{*a})^{\alpha}_{\gamma}J^a_0 -\epsilon^{AB}\epsilon^{\alpha \beta}L_0 \\
 \left[L_0, G^{\alpha A}_{\pm 1/2} \right]&= \mp G^{\alpha A}_{\pm 1/2}\\
 \left[L_1, G^{\alpha A}_{1/2}\right]&= \left[L_{-1}, G^{\alpha A}_{-1/2} \right] = 0 \\
 \left[L_{\pm1}, G^{\alpha A}_{\mp 1/2} \right]&= \pm G^{\alpha A}_{\pm 1/2} \\
 \left[J^a_{0}, G^{\alpha A}_{\pm n} \right]&= {1 \over 2}(\sigma^{*a})^{\alpha}_{\beta}G^{\beta A}_{\pm n} 
 \end{align} These commutators generate $\mf psu(1,1|2)$. Notice that there is no anomaly $c = 6 N$ in the global subalgebra. 
 
 
\subsection{Chiral primaries \& short multiplets in the symmetric orbifold}
\textcolor{red}{Chiral primaries themselves have an OPE $J[m, 0]J[n, 0] \sim regular$, but still have interesting three point functions from the regular terms $O^i O^j \sim C^{ij}_k O^k$... can we recover the chiral ring coefficients from KS somehow?}
From studying the algebra above it is easy to derive that a primary $\phi$ that also satisfies the condition $G^{+ A}_{-1/2}|\phi \rangle = 0, A=1,2$ satisfies $h=j$ (for $L_0$ eigenvalue $h$ and $J_0^3$ eigenvalue $j$) and is called a chiral primary. The quantum numbers and two and three-point functions among chiral primaries are protected quantities as one moves in moduli space and can be matched to the corresponding quantities at the supergravity point. Anti-chiral primaries are defined similarly and satisfy $h = -j$. In the full physical theory, one combines left and right-moving (anti)chiral primaries: $(c, c), (a,a), (a,c), (c, a)$. We will first focus on chiral primaries in the holomorphic half of the SCFT. 

Chiral primaries can arise in the twisted and untwisted sectors of the orbifold. In an $n$-twisted sector (cyclically permuting $n$ copies of the $T^4$ SCFT), the weights of the chiral primaries are bounded: ${n-1 \over 2} \leq h \leq {n+1 \over 2}$. The chiral primaries are explicitly constructed as follows, starting with a twist field. Consider the twist field $\sigma_{l+1}(z)$ which cyclically permutes $l+1$ copies of the holomorphic SCFT as one moves around the point $z$ in the base space; it creates the ground state of the twisted sector by acting on the original NS vacuum. It has weight $h = {6 \over 24}( (l+ 1) - {1 \over l+1})$, but no charge. To make a chiral primary from this state, we must dress it with modes of $J^+ \sim \psi^{+ \dot{1}} \psi^{+ \dot{2}}$, which carry $SU(2)_L$ charge. In particular, using the fact that operators in the twisted sector are fractionally moded we can build the chiral primaries \cite{LuninMathur}:
\begin{align*}
\sigma^0_{l+1}&:=  J^+_{-{l-1 \over l+1}}J^{+}_{-{l-3 \over l+1}} \ldots J^+_{-{1 \over l+1}}\sigma_{l+1}, \ \ \ l+1 \text { odd} \\
\sigma^0_{l+1}&:= J^+_{-{l-1 \over l+1}}J^{+}_{-{l-3 \over l+1}} \ldots J^+_{-{2 \over l+1}}S^+_{l+1}\sigma_{l+1}, \ \ \ l+1 \text { even}
\end{align*} which have $h=j= l/2$. The spin fields $S^+_{l+1}$ map the NS sector vacuum to the R sector, in order to restore the overall periodicity of the fermions as it traverses the length of the long closed string. These single long string states map to single particle states in the supergravity theory. 


From this basic chiral primary, we can create three additional chiral primaries by acting with the fermions, which also map to single particle supergravity states:
\begin{align*}
&\sigma^0_{l+1},  &\qquad h=j= l/2 \\
&\psi^{+ \dot{1}}\sigma^0_{l+1}, &\qquad h=j= (l+1)/2\\
&\psi^{+ \dot{2}}\sigma^0_{l+1} , &\qquad h=j= (l+1)/2\\
&\psi^{+ \dot{1}}\psi^{+ \dot{2}}\sigma^0_{l+1},   &\qquad h=j= (l+2)/2.
\end{align*}
Combining this construction on the left and right enables one to construct 16 $(c,c)$ primaries, which can be mapped to cohomology classes of the target space when viewing the fermions as differential forms. (Again, remember that we are implicitly summing over copy indices so that in the $n$th twisted sector we have e.g. $\psi^{+ \dot{1}} = \sum_{r=1}^n \psi^{+ \dot{1}}_{(r)}$). Of course, when $l=0$, the basic chiral primary is just the NS sector vacuum with $h=j=0$. 

The chiral primaries are part of supermultiplets. These $SU(1,1|2)$ multiplets arise from acting on the chiral primaries with modes of the global subalgebra: the chiral primaries are precisely the highest weight states of short $SU(1,1|2)$ representations \footnote{Long $SU(1,1|2)$ representations can be obtained by acting with the global modes on global primary fields, i.e. fields annihilated by $L_1, G^{\alpha A}_{+1/2}$; there are 16 states per long multiplet.}. Schematically, one can view a short multiplet as associating to each chiral primary $c$ 4 $\mathfrak{sl}(2)$ primary fields that are also $SU(2)_L$ highest weight states: $|c \rangle, G^{- 1}_{-1/2}|c \rangle, G^{-2}_{-1/2}|c \rangle,  \\
G^{-1}_{-1/2}G^{-2}_{-1/2}|c \rangle + {1 \over 2h}J^-_0 L_{-1}|c\rangle$ . To fill out the rest of the short multiplet, one acts on each of these four states with an arbitrary number of $L_{-1}$ generators, as well as with repeated applications of $J_0^-$ to fill out each $SU(2)_L$ multiplet. 

When the chiral primary has weight $h \leq 1/2$, the representation is further truncated and does not contain the $G^{-1}_{-1/2}G^{-2}_{-1/2}|c \rangle + {1 \over 2h}J^-_0 L_{-1}|c\rangle$ state, so is sometimes called an ultra-short representation. 

One can also construct anti-chiral primaries with $h=-j$ in the holomorphic (or anti-holomorphic with $\bar{h}=-\bar{j}$) sector. The construction is almost identical to the chiral primary case (again, see \cite{LuninMathur} for details):
\begin{align*}
\tilde{\sigma}^0_{l+1}&:=  J^-_{-{l-1 \over l+1}}J^{-}_{-{l-3 \over l+1}} \ldots J^-_{-{1 \over l+1}}\sigma_{l+1}, \ \ \ l+1 \text { odd} \\
\tilde{\sigma}^0_{l+1}&:= J^-_{-{l-1 \over l+1}}J^{-}_{-{l-3 \over l+1}} \ldots J^-_{-{2 \over l+1}}S^-_{l+1}\sigma_{l+1}, \ \ \ l+1 \text { even}.
\end{align*} One can again act on these basic primaries with the fermions. Two point functions of chiral and anti-chiral primaries are non-vanishing (in contrast to c-c and a-a two-point functions) and can always be normalized to unity when the operators are unit-separated.

We also note briefly that the exactly marginal operators, which form a basis for the tangent space of the moduli space, can be found in such multiplets. In particular, marginal operators that preserve $\mc{N}=(4,4)$ supersymmetry must be $SU(2)_L$ singlets with $h=\bar{h}=1$. Therefore, they must be in the multiplets with highest weight states (combining now holomorphic and anti-holomorphic sectors) $\sigma^{+ \dot{+} }_{2}, \psi^{+\dot{A}}\bar{\psi}^{\dot{+}\dot{B}}$. Each of these five operators gives four such states, corresponding to 20 marginal operators.

\subsection{The chiral de Rham complex} 

One can start with the single-particle states furnished by chiral primaries and their $SU(1,1,|2)$ descendents and construct a Fock space. 
These are dual to $1/4$-BPS states in the full physical SCFT, and their graded dimension gives rise to the elliptic genus of the model.
The $1/4$-BPS states are captured by the so-called half-twist of the supersymmetric model which we now recall.

The two-dimensional $\cN=(4,4)$ $\sigma$-model admits a half-twist along the lines of \cite{Kapustin, Witten} which results in a purely chiral theory.
For this purpose, it can be convenient to recombine the fermions into vectors, and complexify the bosons $X$ so that we chose local holomorphic and anti-holomorphic coordinates on the target space: $\phi^i, \phi^{\bar{i}}$. 
Then, explicitly, the fermionic fields are sections of the following bundles: 
\begin{align*}
\Psi^i &\in \Gamma(K^{1/2}\otimes \phi^*(T^{(1,0)}M))\\
\Psi^{\bar{i}}&\in \Gamma(K^{1/2}\otimes \phi^*(T^{(0,1)}M))\\
\bar{\Psi}^{i} &\in \Gamma(\bar{K}^{1/2}\otimes \phi^*(T^{(1,0)}M))\\
\bar{\Psi}^{\bar{i}} &\in \Gamma(\bar{K}^{1/2}\otimes \phi^*(T^{(0,1)}M))
\end{align*}  
where as before the left-movers are given by $\Psi$ and the right-movers by $\bar{\Psi}$.

To pass to the half-twisted model, we will restrict to the cohomology of the supercharge $\bar{Q}_+$, after we twist with a certain combination of R-symmetry currents. It is common, as in  \cite{Kapustin}, to perform the A-type twist by the current ${1 \over 2}(J_L - J_R)$ before passing to cohomology. We will instead simply consider a twist by $-J_R$, on the right-movers only, so that the twisted fields live in the following spaces of sections:

\begin{align*}
\Psi^i &\in \Gamma(K^{1/2}\otimes \phi^*(T^{(1,0)}M))\\
\Psi^{\bar{i}} &\in \Gamma(K^{1/2}\otimes \phi^*(T^{(0,1)}M))\\
\bar{\Psi}^i &\in \Gamma(\bar{K} \otimes\phi^*(T^{(1,0)}M))\\
\bar{\Psi}^{\bar{i}} &\in  \Gamma(\phi^*(T^{(0,1)}M))
\end{align*} 

We then make the standard local identifications of fields in the twisted theory with those of ${\rm dim}_{\C}M$ copies of a free $bc\beta\gamma$ system (though again, we stress, the $bc$ fields are just ordinary fermions):
\begin{align*}
\beta_i &\equiv g_{i \bar{j}}\partial_z X^{\bar{j}} \\
\gamma^i &\equiv \phi^i \\
b_i &\equiv g_{i \bar{j}} \Psi^{\bar{j}} \\
c^i &\equiv \Psi^i.
\end{align*}
On a local patch $U \subset M$ we can also take $g_{i\bar{j}} = \delta_{i \bar{j}}$.
Notice that in the standard treatment of the half-twisted model, where the twist is performed using the A-model current, the left-moving fermions transform instead as $\Psi^i \in \Gamma(\phi^* T^{(1, 0)}M), g_{i \bar{j}}\Psi^{\bar{j}} \in \Gamma(K \otimes \phi^* T^{(0, 1)}M)$, rendering the $bc$ fields of spin $(1, 0)$, respectively. In our case, the spins remain half-integral.

The nontrivial operators in the $\bar{Q}_+ = g_{i \bar{j}}\tilde{\Psi}^{\bar{j}} \partial_{\bar{z}}X^i$ cohomology are those of left and right-moving conformal dimensions $(n, 0), n \geq 0$ \cite{Tan,CostelloHol,ESW}. 
The operators of dimension $(0, 0)$ (i.e. the operators forming the ground ring), in particular, have an interpretation as $(0,k)$-forms on the target space. 
The operators of dimension $(n > 0, 0)$ are given by $(0, k)$-forms valued in various tensor product bundles arising from insertions of $\partial_z X^i, g_{i \bar{j}}\partial_z X^{\bar{j}}, g_{i \bar{j}}\partial_z \Psi^{\bar{j}}$. In terms of the physical operators (post-twist), the operators in $\bar{Q}_+$-cohomology will composites of 1.) polynomials in the left-moving fermions and in arbitrary numbers of their holomorphic derivatives 2.) some function of the scalar fields and arbitrary numbers of their holomorphic derivatives 3.) The field $\Psi^{\bar{i}}$, though none of its derivatives (since, by its equation of motion, the holomorphic derivatives of $\Psi^{\bar{i}}$ may be expressed in terms of the aforementioned fields and their derivatives only). Call such an operator $\mathcal{F}$. One can further study which such operators can be constructed globally. Using standard techniques from cohomology reveals that that the Dolbeault cohomology describing the local operators $H_{\bar{\partial}}^{(0,k)}(M, \mathcal{F})=0$ for $k>0$, so that we disallow operators $\mathcal{F}$ that contain $\Psi^{\bar{i}}$. Translating this over to the $bc\beta\gamma$ language, we have that the relevant operators are nothing but functions of $b, c, \beta, \gamma$ and their holomorphic derivatives.

In the present context, our target space $M$ is given by $\oplus_{a=1}^{\infty} Sym^a (T^4)$ or $\oplus_{a=1}^{\infty} Sym^a(K3)$. These spaces are hyperkahler, so the chiral de Rham complex has  (in general, a twisted version of) $\mathcal{N}=4$ supersymmetry \cite{Heluanietal}.

The basic, non vanishing OPEs for $bc\beta\gamma$ systems are
\begin{align*}
b_i(z)c^j(w) = c^j(z)b_i(w) &\sim {\delta^{j}_i \over z - w} \\
\beta_i(z)\gamma^j(w) = -\gamma^j(z)\beta_i(w) &\sim {-\delta^{j}_i \over z - w}
\end{align*}

As explained in previous sections, odd spin operators in the symmetric orbifold theories will be built up from the Ramond sector vacuum, so we will also need the OPEs between the ghosts and the operator $\Sigma(z)$ (often called the spin field) that maps $|0\rangle_{NS} \rightarrow |0\rangle_{R}$:
\begin{align*}
\beta(z)\Sigma(w) &\sim {\tilde{\Sigma}(w) \over (z - w)^{1/2}}\\
\gamma(z)\Sigma(w) &\sim 0.
\end{align*}


\subsection{Elliptic genera for K3 surfaces}

Consider the chiral half of the $\mc N= (4,4)$ $\sigma$-model on the symmetric orbifold  $\Sym^N X$ where $X$ is $T^4$ or a $K3$ surface. 
After performing the half-twist, this is all that remains of the supersymmetric $\sigma$-model.
 According to \cite{DMVV} we can regard the direct sum of the vacuum modules of the chiral algebras of $\Sym^N X$ as being itself a Fock space. The generators of this Fock space are given by the single string states. These single string states are the analog of single trace operators in a gauge theory, and will ultimately be matched with single-particle states in the holographic dual.

Let $c(n,m)$ be the super dimension of the space of operators in supersymmetric $\sigma$-model into $X$, which are of weight $n$ under $L_0$ and of weight $m$ under the action of the Cartan of $SU(2)_R$.  
Let $q,y$ be fugacities for $L_0$ and the Cartan of $SU(2)_R$, respectively---the elliptic genus $\chi(X;q,y)$ is a series in these variables.  
Of course, for $X = T^4$ the elliptic genus vanishes, so it is safe to assume from hereon that $X$ is a K3 surface.

Introducing another parameter $p$, which keeps track of the symmetric power, we can consider the generating series
\begin{equation} 
	\sum_{n \geq 0} p^n \chi(\Sym^n X; q,y) 
\end{equation}
%where $\chi$ indicates the character of the vacuum module of the $bc\beta\gamma$\footnote{We will call the system a $bc\beta\gamma$ system by a slight abuse of terminology: we will not employ the A-type twist on the left-movers when restricting to $\bar{Q}_+$ cohomology, so that the would-be $bc$ fields have fermionic statistics and spin.} on $\Sym^n X$.  
The main result of \cite{deBoer,Verlinde?} is an expression for this generating series
\begin{equation} 
	\sum_{n} p^n \chi(\Sym^n X; q,y) = \prod_{l,m \geq 0,n >0 } \frac{1}{(1 - p^n q^m y^l)^{c(nm,l)}}
\end{equation}
where $c(m,l)$ is a function of the quantity $4m-l^2$.
In other words, we can interpret the direct sum of the vacuum modules of the $\Sym^n X$ $\sigma$-model as being the Fock space generated by a trigraded super vector space 
\begin{equation} 
	V =\oplus_{n \ge 0,m,l} V_{n,m,l} 
\end{equation}
where the super dimension of $V_{n,m,l}$ is $c(nm,l)$.

Setting $V_n = \oplus_{m,l} V_{n,m,l}$, we see that $V_n$ is isomorphic to the vacuum module of the $bc\beta\gamma$ system on the original surface $X$, except with a different conformal structure.  
A state of the $\sigma$-model into $X$ of spin $k$ is of spin $k/n$ in $V_n$.  

The states in $V_N$ will play the role of the single-trace operators in the large $N$ limit of the $\Sym^N X$ $\sigma$-model.   
These states can be understood geometrically as follows---let us focus just on the $S^1$-modes of this $\sigma$-model.
A map $S^1 \to \Sym^N X$ is the same as an $N$-fold cover $M \to S^1$ together with a map $M \to X$.  
Therefore, the Hilbert space of the $\sigma$-model on $\Sym^N X$ decomposes over sectors corresponding to the topological type of this $N$-fold cover, which are labelled by partitions of $N$. 
The single string sector is the sector corresponds to $M$ being connected. 
This means that the monodromy of the cover $M \to  S^1$ is conjugate to the length $N$ cycle of type $(1 \dots N)$ in the symmetric group $S_N$.  

Since the $N$-fold cover of $S^1$ corresponding to the single trace sector is connected, the Hilbert space of the single-trace sector is isomorphic to that of the original $\sigma$-model into $X$.  
However, the conformal structure is different---a rotation along $S^1$ in this $\sigma$-model rotates the total space $1/N$ times.
This tells us that an operator in the single-trace sector carries spin $1/N$ times that of the corresponding state of the original $\sigma$-model. 
The projection onto $\Z_N$-invariant states ultimately restores integrality of the spin. 
In particular, the generating function of elliptic genera of $\Sym^N X$ decomposes as
\begin{equation}
\sum_{N\geq 0}p^N \chi(\Sym^N X; q, y) = \prod_{n>0}\sum_{N \geq 0}p^{n N}\chi(\Sym^{N}\mathcal{H}^{\Z_n}_{(n)}; q, y)
\end{equation}
with $\sum_{N \geq 0}p^{n N}\chi(\Sym^{N}\mathcal{H}^{\Z_n}_{(n)}; q, y) = \prod_{l, m\geq0}{1 \over (1 - p q^m y^l)^{c(mn, l)}}$. Here, $\mathcal{H}_{(n)}$ is the Hilbert space of a single long string on $X$ of length $n$ with winding number $1/n$. 

We can extract the $N \rightarrow \infty$ limit of this expression, following the logic employed in \cite{de Boer, MAGOO, BKKP}. First, in preparation for comparison to supergravity, we perform spectral flow\footnote{We shift the overall power of $q$ by $q^{c/24}$ so that the vacuum occurs at $q^0$.} to the NS sector:
\begin{align*}
\sum_{N \geq 0}p^N \chi_{NS}(\Sym^N X; q, y) & = \sum_{N\geq 0}p^N \chi(\Sym^N X; q, y \sqrt{q}) y^N q^{N/2} \\
&= \prod_{\substack{n \geq 0 \\ m \geq 0, m \in \Z \\ l \in \Z}} \frac{1}{(1 - p^n q^{m + l/2 + n/2} y^{l + n})^{c(nm,l)}} \\
&= \prod_{\substack{n \geq0 \\ m' \geq |l'|/2, \ 2 m' \in \Z_{\geq 0} \\ l' \in \Z, \ m' - l'/2 \in \Z_{\geq 0}}} \frac{1}{(1 - p^n q^{m'} y^{l'})^{c(nm' - nl'/2,n-l')}}.
\end{align*}

At any power of $q$, there will be contributions from terms of the form ${1 \over (1 - p y^{l'})^{c(-l'/2, l'-1)}}$. The only nonvanishing such term in our case when $m'=0$ is ${1 \over (1 - p)^2}$. We wish to isolate the coefficients of all terms of the form $q^a y^b p^N$ for $a \ll N$. Taylor expanding ${1 \over (1-p)^2}$ and extracting the desired coefficient gives $N h(a, b) + \mathcal{O}(N^0)$ where $h(a, b)$ is the coefficient of $q^a y^b$ in
\begin{equation}\nonumber
\prod_{\substack{m' \geq |l'|/2, \ 2 m' \in \Z_{\geq 0} \\ l' \in \Z, \  m' - l'/2 \in \Z_{\geq 0}}}{1 \over (1 - q^{m'} y^{l'})^{f(m', l')}}
\end{equation}with $f(m', l'):= \sum_{n >0}c(n(m' -  l'/2), l' - n)$.  The coefficients $c(M, L)$ vanish for $4M-L^2 < -1$ so for $m' \geq 1$ the sum truncates to $f(m', l') = \sum_{n=1}^{4m'}c(n(m' -  l'/2), l' - n)$.

Hence, we can get a finite contribution upon dividing by $N$. 

We can also write out the non-vanishing $f(m', l')$ more explicitly, recalling that the coefficients are constrained to lie in the following range of the Jacobi variable: $-2m' \leq l' \leq 2m', l' \equiv 2 m' \mod 2$. Reproducing the elementary manipulations in Appendix A of \cite{BKKP} (in particular, using the fact that $c(N, L)$ depends only on $4N-L^2$ and $L \ {\rm mod} \ 2$) allows us to rewrite the sum as
\begin{equation}\label{eq:fml2}
f(m', l') = \left( \sum_{\tilde{n} \in \Z}c(m'^2 - l'^2/4, \tilde{n}) \right) - c(0, l'),
\end{equation} where $n':= n - 2m$ in the first term. 
The first term is non-vanishing only when $l' = \pm 2 m'$ and then it reduces to the Witten index of K3, i.e. $f(m', \pm 2m') = 24$ for general $m'$. Otherwise, we have $f(m', l') = -c(0, l')$. When $m' \in \mathbb{Z}$ the nonvanishing such term is $-c(0, 0) = -20$, and when $m' \in \Z + 1'/2$ we have $-c(0, 1) = -2$ and $-c(0, -1) = -2$. 

In sum, we obtain
\begin{align*}
{\rm lim}_{N \rightarrow \infty}{\chi_{NS}(\Sym^{N} X; q, y) \over N} &= \prod_{k \geq 1}{(1 - q^k)^{20}(1 - q^{k-1/2}y^{-1})^2(1 - q^{k-1/2}y)^2 \over (1 - q^{k/2}y^k)^{24}(1 - q^{k/2}y^{-k})^{24}} \\
&= 1 + \left({22 \over y} + 22 y \right)q^{1/2} + \left({277 \over y^2} + 464 + 277 y^2 \right)q + \text{O}(q^{3/2}).
\end{align*} 
We will denote this large $N$ limit by $\chi_{NS}(\Sym^\infty X ; q,y)$. 
In particular, for there are two bosonic towers corresponding to (anti)chiral primary states and three fermionic towers corresponding to (derivatives of) the states capturing the cohomology of a single copy of K3. At $k=1$, there is a cancellation to ${(1 - q)^{20} \over (1 - q^{1/2}y)^{22}(1 - q^{1/2}y^{-1})^{22}}$.

Throughout this derivation, we have used the coefficients $c(m, l)$ that appear in the expansion of the K3 elliptic genus in the Ramond sector: $\sum_{m \geq 0, l \in \mathbb{Z}} c(m, l) q^m y^l$. We can also rewrite things slightly in terms of the coefficients of the NS sector elliptic genus expansion: $\sum_{2 m' \in \mathbb{Z}_{\geq 0}, l' \in \mathbb{Z}} \mathcal{C}(m', l') q^{m'} y^{l'} = 2 + (20/y + 20 y)q^{1/2} + (2/y^2 - 128 + 2 y^2)q + \ldots$ using the half-integral spectral flow relation on the coefficients of the elliptic genera: $\mathcal{C}(m', l') = c(m'-l'/2, l'-1)$. 

%Applying spectral flow to the second quantized elliptic genus and reindexing $n,m,l$ gives:
%\begin{equation}
%\prod_{\substack{n \geq0 \\ m \in \Z_{\geq 0} \\ l \in \Z}} \frac{1}{(1 - p^n q^{m} y^{l})^{c(nm - nl/2,n-l)}} = \prod_{\substack{n \geq0 \\ m \geq |l|/2, \ 2 m \in \Z_{\geq 0} \\ l \in \Z, \ m - l/2 \in \Z_{\geq 0}}} \frac{1}{(1 - p^n q^{m} y^{l})^{\mathcal{C}(n m - n l/2 - n/2 + l/2 + 1/2, l - n + 1)}}.
%\end{equation}
%The same set of manipulations as before give (again with $m \in \mathbb{Z}_{\geq 0}/2, l \in \mathbb{Z}, -2m \leq l \leq 2m, l \equiv 2 m (\text{ mod } 2)$)
%\begin{equation}
%f(m, l) = \left(\sum_{n'= - \infty}^{\infty}\mathcal{C}(m^2 - l^2/4 + (n' + 1)/2, n' + 1) \right) - \mathcal{C}((l + 1)/2, l + 1)
%\end{equation} to the same conclusion. As before, $n':= n - 2m$. In particular, when $l = \pm 2 m$, the sum only gets nonvanishing contributions from $n' = -1, 0, 1$, which contribute $2, 20, 2$, respectively. 

Let us unpack those contributions a bit more, starting from the expression 
\begin{align*}\label{eq:fml1}
f(m', l') &= \sum_{n=1}^{4m'}c(n(m'-l'/2), l'-n) \\
&= \sum_{n=1}^{4m'}\mathcal{C}(n(m'-l'/2 - 1/2) + l'/2 + 1/2, l' - n +1).
\end{align*} In the second line, we have used spectral flow to rewrite the sum in terms of the NS sector elliptic genus.


Let us henceforth drop the primes on our variables, for ease of notation, and then take for example the states $l = 2 m$. The summand is nonvanishing only for $\mathcal{C}(1/2, 1)= 20, \mathcal{C}(0, 0)=2, \mathcal{C}(1, 2)= 2$ and, of course, have contributions which sum to 24. The solutions to these conditions occur at $n=2m, n = 1 + 2m, n=2m -1$ $ \ (\forall m >0, 2m \in \Z_{\geq 0})$ and these values of $n$ do appear in the sum. The corresponding states therefore have quantum numbers $(n, m, l) = 20\times(j, j/2, j), 2\times(j + 1, j/2, j), 2\times(j-1, j/2, j), \ j \in \Z_{>0}$ and come from chiral primary states in the physical theory. In the next section, we will recall the corresponding states in the physical theory \cite{luninmathur, others...}. Similarly, for $l = -2m$ we will obtain contributions from anti-chiral primary states \textcolor{blue}{finish}. 

We can also discuss the origin of the terms of negative multiplicity, which contribute to the numerator of the index. Taking $f(m,l)$ as it is written in \ref{eq:fml1} and studying $l=0, k \in \mathbb{Z}_{\geq 1}$, there is a cancellation between the $n=4k$ term, $\mathcal{C}(4k(k-1/2) + 1/2,-4k+1)=20$, and the rest of the terms, which sum to $-40$. Similar cancellations apply to $l=\pm 1, m = k-1/2, k \in \mathbb{Z}_{\geq 1}$, wherein the last terms of the sum are always $\mathcal{C}(4(k-1)^2, 4-4k) = 2, \mathcal{C}((1-2k)^2, 2-4k) = 2$, respectively, and the remainder of the terms produce coefficients summing to -4. Alternatively, one can use manipulations on Jacobi form coefficients to rewrite $f(m, l)$ as in \ref{eq:fml2}:
\begin{equation}
f(m', l') =\left(\sum_{\tilde{n}= - \infty}^{\infty}\mathcal{C}(m'^2 - l'^2/4 + (\tilde{n} + 1)/2, \tilde{n} + 1) \right) - \mathcal{C}((l' + 1)/2, l' + 1)
\end{equation} with $\tilde{n} = n-2m$ as before. The first term sums to zero when $l=0, m= k$ and for $l=\pm 1, m = k-1/2$, and the second gives the desired negative coefficients.


\textcolor{blue}{finish discussion of quantum numbers of states contributing to the infinite-N index}

\subsection{Branes in twisted supergravity}
\brian{Here I wanted to derive from first-principles the dual chiral CFT using the topological string package.}

We consider the situation of a $D1/D5$ brane system in the twist of type IIB on a Calabi--Yau five-fold $X$. 
For simplicity, we assume that we have a collection of $N_1 = N$ $D1$ branes supported along a complex submanifold
\[
\Sigma \subset X 
\]
together with a single $D5$ brane which is parallel to the $D1$ branes. 

We consider $D1$ branes which are a sum simple branes labeled by $\cO_\Sigma$.
The Dolbeault model for the open string fields which stretch between two such $D1$ branes is given by 
\beqn\label{eqn:open1}
\Omega^{0,\bu}(\Sigma, \underline{\rm Ext}_{\cO_X}(\cO_\Sigma)) \cong \Omega^{0,\bu}(\Sigma, \wedge^\bu N_\Sigma) 
\eeqn
where $N_\Sigma$ is the normal bundle to $\Sigma$ in $X$. 
If we take $X$ to the be the total space of the bundle $N_\Sigma$ then the Calabi--Yau condition requires $\wedge^4 N_\Sigma = K_\Sigma$. 
In the case $\Sigma = \CC$ and $X = \CC^5$ we can twist by a homomorphism $SO(2) \to SO(4)$ to 
write the open string fields \eqref{eqn:open1} as 
\beqn\label{eqn:open1a}
\Omega^{0,\bu}(\CC) \otimes \lie{gl}(N) [\ep_1,\ldots,\ep_4] [1] .
\eeqn
Here the $\ep_i$ are odd variables. 

Next, we consider $D1-D5$ strings. 
The open string fields are given by 
\beqn\label{eqn:open15}
\Omega^{0,\bu}(\CC, K^{\frac12}_\CC)  [\ep_3,\ep_4] \otimes {\rm Hom}(\CC, \CC^N) = \Omega^{0,\bu}(\CC, K^{\frac12}_\CC)  [\ep_3,\ep_4] \otimes \CC^N .
\eeqn
Together with the $D5-D1$ strings we get 
\beqn\label{eqn:open15a}
\Omega^{0,\bu}(\CC, K^{\frac12}_\CC)  [\ep_3,\ep_4] \otimes T^*\CC^N .
\eeqn

In total, we see that the open-strings of the $D1/D5$ system along $\Sigma = \CC$ are given by the Dolbeault complex valued in the following holomorphic vector bundle
\beqn
\bigg(\lie{gl}(N)[\ep_1,\ep_2][1] \oplus K^{\frac12}_\CC \otimes T^*\CC^N \bigg) \otimes \CC[\ep_3,\ep_4] .
\eeqn
The bundle in parentheses can be written as 
\brian{Now here I really want $\deg{\ep_1}=\deg{\ep_2} = +1$}
\beqn
\lie{gl}(N)[1] \oplus T^* \left(\lie{gl}(N) \oplus K^{\frac12}_\Sigma \CC^N\right) \oplus \lie{gl}(N) [-1] .
\eeqn
Up to the factor of $K^{\frac12}_\Sigma$ this is evidently the underlying vector space of the graded Lie algebra controlling Hamiltonian reduction of 
\beqn
T^*(\lie{gl}(N) \oplus \CC^N) \sslash \lie{gl}(N) .
\eeqn

\subsection{The large $N$ limit}

\brian{Use LQT to give a first-principles description of the large N CFT. Discuss relationship to elliptic genus computed above.}

\section{Enumerating twisted supergravity states}

\subsection{Enumerating states in Kodaira--Spencer theory}

\brian{just review work of Costello and Gaiotto}

To describe the boundary conditions we will use the partial compactification of the extended deformed conifold as described in \S \ref{sec:compact}. 

Recall that there are three fundamental fields for Kodaira--Spencer theory.
Two fundamental fields $\alpha, \gamma$ are Dolbeault forms of type $(0,\bu)$.
The last fundamental field $\mu$ is a $(0,\bu)$ form valued in in the holomorphic tangent bundle.
We can use the Calabi--Yau form to view $\mu$ as a Dolbeault form of type $(2,\bu)$.

\begin{itemize}
\item The vacuum boundary condition for the fields $\alpha, \gamma$ is that each are divisible by the coordinate $n$. 
\item The vacuum boundary condition for the field $\mu$ is that when viewing it as a Dolbeault form of type $(2,\bu)$ it can be expressed as a sum of terms which are each wedge products of $\d \log n, \d w, \d z , \d \br n, \d \br w , \d \br z$ with coefficients that are regular at $n = 0$. 
\end{itemize} 

Denote by $\left(\mathbf{\frac{m}{2}}\right)_S$ the short representation of $\lie{psu}(1,1|2)$ whose highest weight vector has $(J_0^3,L_0)$ eigenvalues $(\frac{m}{2}, \frac{m}{2})$. 
Denote by $y$ the fugacity for the $U(1)$ symmetry $2J_0^3$ and $q$ the fugacity for the $U(1)$ symmetry $L_0$.
Let 
\beqn
D = (1-q)(1-q^{1/2} y)(1-q^{-1/2}y^{-1}) .
\eeqn

Let's first consider the case where the internal manifold is just a point.
This is just the topological string with $B$-branes wrapping 
\beqn
\C \subset \C^3 .
\eeqn 

\begin{prop}
The single particle states for Kodaira--Spencer theory on $\C^3$ with branes wrapping $\C \subset \C^3$ decompose as
\beqn
\oplus_{m \geq 1} \left(\mathbf{\frac{m}{2}}\right)_S 
\eeqn
From this, we deduce that the single particle index for Kodaira--Spencer theory on $\C^3$ is
\beqn
\frac{q^2 - 3 q + q^{1/2}(y+y^{-1})}{D} .
\eeqn
\end{prop}

\begin{itemize} 
\item State $\mu \sim n^{-k} \d \log n \d w_1 \delta_{z=0}$.
For $k \geq 1$ these even states and their descendants contribute
\beqn
\frac{y q^{1/2}}{D} 
\eeqn
to the single particle index. 
\item Lowest lying state $\mu \sim n^{-k} \d \log n \d w_2 \delta_{z=0}$.
For $k \geq 1$ these even states and their descendants contribute
\beqn
\frac{y^{-1} q^{1/2}}{D} 
\eeqn
to the single particle index. 
\item Lowest lying state $\mu \sim n^{-k} \d \log n \d z \delta_{z=0}$.
For $k \geq 2$ these even states and their descendants contribute 
\beqn
\frac{q^2}{D} 
\eeqn
to the single particle index. 
\item State $\alpha \sim n^{1-k}\delta_{z=0}$. 
For $k \geq 1$ these odd states and their descendants contribute 
\beqn
- \frac{q}{D} .
\eeqn
to the single particle index.
\item State $\gamma \sim n^{1-k}\delta_{z=0}$. 
For $k \geq 1$ these odd states and their descendants contribute 
\beqn
- \frac{q}{D} .
\eeqn
to the single particle index.
\item \item State $\nu \sim n^{1-k}\delta_{z=0}$. 
For $k \geq 1$ these odd states contribute 
\beqn
- \frac{q}{D} .
\eeqn
to the single particle index.
\end{itemize}

In total we find that the multi particle gravitational index is 
\beqn
\frac{q^2 - 3q + q^{1/2} (y+y^{-1})}{(1-q)(1-q^{1/2} y)(1-q^{-1/2}y^{-1})} = \frac{y q^{1/2}}{1-y q^{1/2}} + \frac{y^{-1} q^{1/2}}{1-y^{-1}  q^{1/2}} - \frac{q}{1-q} .
\eeqn



%The vacuum boundary condition for the fields $\alpha$ requires that it is divisible by the coordinate $n$. 
%We can modify this at the point $z=0$ at the boundary by taking 
%\[
%\alpha = n^{-k} w^l \del_z^{r} \delta_{z=0} \otimes a .
%\]
%In this expression, $k \geq 0$ and $l \leq k$ to ensure that there are no poles along $w = \infty$. 
%Also, $a \in H^\bu(Y)$ denotes an arbitrary cohomology class for the $K3$ surface.
%The ansatz for $\gamma$ is identical.
%
%For the field $\mu$

\subsection{The twisted supergravity elliptic genus}

The supergravity states were enumerated in \cite{CPkoszul}. 
We briefly recall the results here. 

The twisted supergravity states organize into a representation for the super Lie algebra $\lie{psu}(1,1|2)$.
The bosonic factor of this super Lie algebra is $\lie{su}(2)_L \times \lie{su}(2)_R$. 
The first copy is the global conformal transformations in the $z$-plane and the second copy is the $R$-symmetry algebra which rotates the $w$-coordinate.
We take the Cartan of this Lie algebra to be generated by $(L_0, J_0^3)$. 
  
Denote by $(\frac{\bf m}{\bf 2})_S$ the short representation of $\lie{psu}(1,1|2)$ whose highest weight vector has $(L_0, J_0^3)$ eigenvalue $(m/2,m/2)$ \cite{dB1}. 
As an example, the short representation $({\bf 1})_S$ consists of a boson with weight $(L_0 = 1, J_0^3 = 1)$, which in our notation corresponds to 
\beqn
\mu \sim n^{-2} \d \log n \d z \delta_{z=0}  .
\eeqn 
There are also two fermions in $({\bf 1})_S$ with weights $(3/2,1/2)$ corresponding to the states
\beqn
\alpha \sim n^{-1} \delta_{z=0} + \cdots , \quad \gamma \sim n^{-1}\delta_{z=0} + \cdots
\eeqn
and another boson of weight $(2,0)$ corresponding to 
\beqn
\mu \sim n^{-2} \d \log n \d w \delta_{z=0} + \cdots .
\eeqn 

We consider twisted type IIB supergravity on a Calabi--Yau surface $X$, where $X$ could be $T^4$ or a $K3$ surface. 

\begin{prop}[\cite{CPkoszul}]
The supergravity states for the $D1-D5$ brane system in twisted type IIB supergravity on a compact Calabi--Yau surface $X$ decompose as
\beqn\label{eqn:IIBstates}
\bigoplus_{m \geq 1} (\frac{\bf m}{\bf 2})_S \otimes H^\bu(X) = \bigoplus_{m \geq 1} \bigoplus_{i,j} (\frac{\bf m}{\bf 2})_S \otimes H^{i,j} (X)  . 
\eeqn 
In particular, when $X$ is a $K3$ surface the single particle twisted supergravity index is 
\beqn\label{eqn:sugra_index}
f_{KS}(q,y) = 24 \frac{q^2 - 3 q + q^{1/2}(y+y^{-1})}{D} .
\eeqn
\end{prop} 

This result should be compared to \cite{dB1}, where the space of supergravity states upon supersymmetric localization (that is, the chiral half of the supergravity states) is found to be
\beqn\label{eqn:db1}
\bigoplus_{m \geq 0} \bigoplus_{i,j} (\frac{\bf m+i}{\bf 2})_S \otimes H^{i,j} (X) .
\eeqn
The answers agree in the range where the highest weight of the short representation is at least two. 
The low weight discrepancies break up into two types:
\begin{itemize}
\item In \cite{dB1} there is an extra factor of $({\bf 0})_S \otimes H^{0,i}(X)$. 
So, in the case that $X$ is a $K3$ surface there are two extra bosonic operators in the analysis of \cite{dB1}. 
In \cite{CP} is was pointed out that these are topological operators, annihilated by $L_{-1}$, and have nonsingular OPE with all remaining operators. 
\item 
In our analysis there is an extra factor of $(\frac{\bf 1}{\bf 2})_S \otimes H^{2,j}(X)$. 
In the case that $X$ is a $K3$ surface one can remove these two bosonic states while maintaining an $SO(21)$ symmetry. 
\end{itemize}

Denote the single particle index of the supergravity states, described in equation \eqref{eqn:db1}, by $f_{sugra}(q,y)$. 
One of the main results of \cite{dB1} is that the corresponding multiparticle index agrees with the large $N$ elliptic genus of the orbifold CFT of a $K3$ surface
\beqn
\chi_{NS}(\Sym^\infty X ; q,y) = {\rm PExp} \left[f_{sugra}(q,y) \right]
\eeqn
where ${\rm PExp}$ is the plethystic exponential.
For $X$ a $K3$ surface, the states $(\mathbf{\frac12})_S \otimes H^{2,\bu}(X)$ contribute the single particle index
\beqn
2 f_1 (q,y) = \frac{2}{1-q}\left(-2 q + q^{1/2}(y+y^{-1})\right) .
\eeqn
If we subtract this from the supergravity index we find an exact match with the supergravity index computed by \cite{dB1}
\beqn
f_{sugra}(q,y) = f_{KS}(q,y) - 2 f_1(q,y) .
\eeqn

\subsection{Global symmetry algebra}
For $F \in H^2(Y_{K3})$, let $X^0_F$ be the extended deformed conifold as defined in \S \ref{sec:conifold}.
Recall that $Z = {\rm Spec}(A)$ is the affine variety described by the commutative algebra $A = H^\bu(Y_{K3})$. 
Previously, we saw that the space $X_F^0$ admits a fibration $X^0_F \to Z$ fibered in Calabi--Yau three-folds. 
Let $\cO(X^0_F)$ be the algebra of functions on $X^0_F$.
By Hartog's theorem this is the algebra generated by the bosonic linear functions $u_i, w_j, \eta, \br \eta, \eta_a$ where $i,j=1,2$, $a=1,\ldots, 20$ subject to the relations
\[
\eta^2 = \br \eta^2 = \eta_a \eta_b - h_{ab} \eta \br \eta = 0, \qquad \eps^{ij} u_i w_j = F . 
\]

Let 


\end{document}
