\documentclass[11pt]{amsart}

\usepackage{setspace}
\usepackage{hyperref}
\usepackage{amsmath,amsthm}

\date{\today}
\title{cdr}

\def\define{\overset{\rm def}{=}}
\def\ep{\varepsilon}
\def\zbar{{\overline{z}}}
\def\Weyl{{\rm Weyl}}
\def\Cl{{\rm C}\ell}
\def\U{{\rm U}}
\def\fgl{{\mathfrak{gl}}}
\def\GL{{\rm GL}}
\def\sfc{{\mathsf{c}}}
\def\sfb{{\mathsf{b}}}
\def\sfB{{\mathsf{B}}}
\def\sfC{{\mathsf{C}}}
\renewcommand{\op}{\operatorname}

\def\brian#1{{\textcolor{blue!65!red}{BRW: {#1}}}}
\def\natalie#1{{\textcolor{green!65!black}{BRW: {#1}}}}

%\usepackage[upint]{stix}
\usepackage{cmupint}

\begin{document}
\maketitle

\spacing{1.25}

Our goal is to describe the global sections of the chiral de Rham complex on the symmetric orbifold $\Sym^N (\CC^2)$. 

First, we realize the symmetric orbifold as a Hamiltonian reduction. 
Consider the affine space
\[
\cB \define \CC^2 \times \fgl(N) 
\]
and the total space of the cotangent bundle $\T^* \cB$.

There is a diagonal action of the group $\GL(N)$ on $\T^*\cB$ defined by the formula
\[
g \cdot (v, A ; w, B) = (g \cdot v, g A g^{-1} ; g^{-1} w, g^{-1} B g) .
\]
The map $\mu \colon \T^* \cB \to \fgl(N) \simeq \fgl(N)^*$ defined by
\begin{equation}\label{eqn:moment}
\mu (v, A ; w, B) = v \otimes w + [A,B] 
\end{equation}
is a moment map for this group action. 
The space $\Sym^N (\CC^2)$ is the Hamiltonian reduction with respect to this moment map. 

\subsection{BRST operator}
We consider the chiral de Rham complex on the space $\T^*\cB$ which has an induced Hamiltonian action by the $\gl(N)$ affine Kac--Moody algebra. 
We are interested in the resulting BRST reduction. 

Let us set up some notation. 
The graded vertex algebra underlying this BRST reduction is generated by operators in the table below. 
In the table $i,j = 1,\ldots, N$. 

\begin{table}
\begin{tabular}{c|c|c|cccccc}
& {\rm ghost} & {\rm spin} \\
\hline
$\gamma^i$ & $0$ &  $0$ \\
$ \beta_j$ & $0$ &  $1$ \\
\hline
$\psi^i$ & $+1$ &  $0$ \\
$\chi_j $ & $-1$ &  $1$ \\
\hline
$\Gamma^i_j$ & $0$ &  $0$ \\
$ B^i_j $ & $0$ &  $1$ \\
\hline
$\Psi^i_j$ & $+1$ &  $0$ \\
$ {\rm X}^i_j $ & $-1$ &  $1$ \\
\hline
$\sfc^i_j$ & $+1$ &  $0$ \\
$ \sfb^i_j $ & $-1$ &  $1$ \\
\hline
$\sfC^i_j$ & $0$ &  $0$ \\
$ \sfB^i_j $ & $0$ &  $1$ \\
%$\partial^\ell \chi_b$ & $0$ & odd & $\ell+\frac12$
\end{tabular}
\end{table}


These operators satisfy the following OPE
\[
\beta_i (0) \gamma^j (z) \simeq \delta_i^j \frac1z ; \quad \chi_i (0) \psi^j (z) \simeq \delta^j_i \frac1z  .
\]
\[
B_i^j (0) \Gamma^k_\ell (z) \simeq \delta_i^k \delta_\ell^j \frac1z ; \quad {\rm X}_i^j (0) \Psi^k_\ell (z) \simeq \delta_i^k \delta_\ell^j \frac1z  .
\]
\[
\sfb_i^j (0) \sfc^k_\ell (z) \simeq \delta_i^k \delta_\ell^j \frac1z ; \quad \sfB_i^j (0) \sfC^k_\ell (z) \simeq \delta_i^k \delta_\ell^j \frac1z  .
\]

Using the moment map $\mu$ as in \eqref{eqn:moment}, we can read off the BRST current is
\begin{align*}
J_{\rm BRST} & = \beta_j  \gamma^j \sfc_i^j + \chi_j  \psi^i \sfc_i^j + \op{Tr} B \Gamma \sfc + \op{Tr}  {\rm X}\Psi \sfc \\
& + \beta_j  \psi^i \sfC_i^j + \chi_j \gamma^i \sfC_i^j  + \op{Tr} B \Psi \sfC + \op{Tr} {\rm X} \Gamma \sfC  \\
& + \frac12 \op{Tr} \sfb \sfc^2 + \frac12 \op{Tr} \sfB  \sfC \sfc
\end{align*}
The BRST operator is $Q = \oint J_{\rm BRST} (z)\, \d z$.

For $\Phi$ any operator we set $\Phi_n = \oint z^{-n-1} \Phi(z) \d z$, this is the $n$th mode of $\Phi$.
The vacuum module underlying the vertex algebra described above is generated by modes $\beta_n, \gamma_n, \ldots, \sfB_n, \sfC_n, \sfb_n, \sfc_n$ for $n \geq 0$. 
\brian{I think we want to take $\GL(N)$-invariants and only $\sfc_n$ for $n \geq 1$.}



\subsection{Classical limit}




\end{document}

\end{document}