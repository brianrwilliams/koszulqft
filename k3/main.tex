\documentclass[11pt]{amsart}


\usepackage{macros_k3}
%\usepackage{hyperref}


%\setcounter{secnumdepth}{4}

\usepackage{subfiles}

\def\brian#1{{\textcolor{blue!65!red}{BW: {#1}}}}
\def\natalie#1{{\textcolor{green!65!black}{NP: {#1}}}}


\title{Twisted holography on AdS$_3 \times S^3 \times K3$ \& the planar chiral algebra}
\author{Kevin Costello}
\author{Víctor E. Fernández}
\author{Natalie M. Paquette}
\author{Brian R. Williams}

%
%\address{Perimeter Institue for Theoretical Physics}
%\address{Walter Burke Institute for Theoretical Physics, California Institute of Technology}
%\email{kcostello@perimeterinstitute.ca}
%\email{nataliep@caltech.edu}

\begin{document}

\maketitle

\begin{abstract} %required
In this work, we revisit and elaborate on twisted holography for AdS$_3 \times S^3 \times X$ with $X= T^4$, K3, with a particular focus on K3. We describe the twist of supergravity, identify the corresponding (generalization of) BCOV theory, and enumerate twisted supergravity states. We use this knowledge, and the technique of Koszul duality, to obtain the $N \rightarrow \infty$, or planar, limit of the chiral algebra of the dual CFT. The resulting symmetries are strong enough to fix planar 2 and 3-point functions in the twisted theory or, equivalently, in a 1/4-BPS subsector of the original duality. This technique can in principle be used to compute corrections to the chiral algebra perturbatively in $1/N$.
\end{abstract}

\tableofcontents

\section{Introduction \& Summary}

Twisted holography \cite{CLsugra, CostelloGaiotto, CP} is a proposal to access protected quantities on both sides of a holographic duality. While twists of field theory have been studied for a long time, and correspond to restricting to the cohomology of a chosen supercharge, twisting a supergravity or (spacetime) string theory involves turning on a background vev for the bosonic ghost associated to the corresponding supertranslation \cite{CLsugra}. Many choices of twists are possible, corresponding to the family of nilpotent supercharges available in the supersymmetry or superconformal algebra. One interesting, and relatively accessible, class of twists are those which endow the surviving local operators with the structure of a chiral algebra. In four real dimensions, such a twist has been an area of active recent inquiry \cite{Beem:2013sza} and was applied to the twisted holography of 4d $\mathcal{N}=4$ super Yang-Mills in \cite{CostelloGaiotto}. In two real dimensions, such a twist is simply the half-twist \cite{Witten, Kapustin}, and does not change the effective dimensionality of the twisted field theory or its bulk dual. We will explore this relatively simple twist in the context of (top-down models of) AdS$_3$/CFT$_2$, in particular AdS$_3 \times S^3 \times$ K3. Many similar results for the case when K3 is replaced by $T^4$ have already appeared in the companion paper \cite{CP}. 
It is important to note, however, that the half-twisted theory (equivalently, the minimal, holomorphic twist in two dimensions) is sensitive to nonperturbative corrections, such as worldsheet instantons, which makes studying a global description of the twist of the SCFT on K3 from first principles somewhat challenging. The mathematical version of this statement is that the chiral de Rham complex of a nontrivial compact manifold is given locally as a sheaf of free vertex algebras, but gluing these local descriptions together is not easy. Although we will derive such a local description of the twist on the field theory side, we emphasize a way to circumvent the global challenge: given the description of a twisted supergravity theory, one may apply the operation of Koszul duality to obtain the chiral algebra of the dual field theory. In particular, the global subalgebra of the chiral algebra, which can also be deduced by considering vacuum-preserving diffeomorphisms of the bulk geometry, appears in this construction. That the mathematical operation of Koszul duality may govern part of the holographic dictionary in twisted holography was first suggested in \cite{CostelloM2} and successfully applied to AdS/CFT in \cite{CP}. For a review of Koszul duality and further citations, see \cite{PW}.

The plan of this paper is as follows. In \S \ref{sec:sugra} we will give our description of the holomorphic twist of IIB supergravity in six dimensions (upon compactification on K3). We describe how our twisted action can be obtained by integrating out the vev of a bosonic superghost. We then derive the backreacted geometry in the presence of the twisted D1-D5 system. In \S \ref{sec:enumerate}, we enumerate the states in twisted supergravity and reproduce the elliptic genus computation of \cite{deBoerSUGRA, deBoerEG} in this language. In \S \ref{sec:CFT}, we review the computation of the $N \rightarrow \infty$ elliptic genus from the orbifold SCFT Sym$^N$(K3) and its matching with the supergravity computation, and twist a local model of the B-brane D1-D5 brane system. This twist recovers the expected description of the chiral de Rham complex of Sym$^N(\mathbb{C}^2)$ (i.e. the half-twist of the symmetric orbifold SCFT) in the infrared. The Loday-Quillen-Tsygan theorem, which is a natural tool in the large-$N$ limit of twisted holography (e.g. \cite{Zeng:2023lox}, \cite{ginot2022large}), gives equivalent results in the $N \rightarrow \infty$ limit to this local model of the twist, but has not yet been developed mathematically for the global K3 geometry. Consequently, in \S \ref{sec:tree} and \S \ref{sec:br} we turn our attention to the determination of the planar chiral algebra of the dual field theory from Koszul duality, first studying the chiral algebra Koszul dual of twisted IIB supergravity on $\C^2 \times$ K3 and then incorporating the effects of the D-brane backreaction using a perturbative Feynman diagrammatic approach; while incorporating the effects of backreaction perturbatively from flat space would normally involve the summation of an infinite number of diagrams, the problem simplifies dramatically in twisted holography. There are a finite number of nonzero diagrams at each order in $N$ \cite{CP}, and only 3 in the planar limit. We also comment on the global subalgebras of the chiral algebras dual to the symmetries of the flat space and backreacted (i.e. holographic) geometries, respectively.



\subfile{content/sugra}

\subfile{content/enumerate}

\subfile{content/cft}

\subfile{content/tree}

\subfile{content/br}

\subfile{content/brappendix}

\bibliographystyle{JHEP}
\bibliography{k3refs}


\end{document}
