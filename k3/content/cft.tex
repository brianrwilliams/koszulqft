\documentclass[../main.tex]{subfiles}

\begin{document}

\section{The twisted symmetric orbifold CFT}

Supergravity on $AdS_3 \times S^3 \times X$, where $X$ is either $T^4$ or a $K3$ surface, is expected to be holographically dual to a particular two-dimensional superconformal field theory (SCFT). Though our primary interest in this note is $K3$, with the $T^4$ case studied in \cite{CP}, we can be agnostic about $X$ for many aspects of the analysis. 

We will briefly review this system of interest, following \cite{Davidetal} and references therein, with a focus towards applying the holomorphic twist to this system and isolating the 1/4-BPS states. Of course, this SCFT is the IR limit of the field theory that arises from the zero modes of the open strings on the D1-D5 branes. 
The lowest-lying modes of open strings, which provide an effective field theory description of the D1 and D5-branes, naturally furnish a gauge theory whose IR limit we are primarily interested in. One can study the twist, which is insensitive to RG flow, of both the UV D1-D5 gauge theory or the symmetric orbifold CFT, and we will sketch salient features of both perspectives. 

We recall that the D5-D5 strings give rise to a six-dimensional supersymmetric $U(N_5)$ gauge theory preserving 16 supercharges. 
When all the D-branes are coincident the gauge theory is in the Higgs phase and when some of the adjoint scalars in the field theory acquire a vev, corresponding to transverse separation of the branes, the theory is in the Coulomb phase. 
We will focus on the Higgs phase of the gauge theory throughout \footnote{See \cite{Budzik:2022hcd} for a recent analysis of twisted holography in the Coulomb phase.}, which involves turning on a nonvanishing Fayet-Iliopoulos parameter (dually, NS B-field). 
We reduce four directions of the gauge theory on 
\[
X = T^4 \quad \text{or} \quad K3
\]
which results in an effective two-dimensional $U(N_5)$ gauge theory which preserves 16 supercharges.
The D1-D1 strings similarly produce a $U(N_1)$ gauge theory preserving 16 supercharges. 
More interesting are the D1-D5 and D5-D1 strings, which break the total supersymmetry down to 8 supercharges (though there will be a supersymmetry enhancement in the near-horizon/low energy limits, so that the holographic dual pair of theories has 16 supersymmetries overall). 
These strings produce matter multiplets transforming in the bifundamental representations of the gauge groups. 

On the Higgs branch, one must solve the vanishing of the bosonic potential (i.e. D-flatness equations) modulo the gauge symmetries $U(N_1)\times U(N_5)$ to obtain the moduli space. 
If one imagined that both sets of D-branes were supported on a noncompact six-dimensional space, these D-flatness equations can be rewritten to reproduce the ADHM equations for $N_1$ instantons of a six-dimensional $U(N_5)$ gauge theory a la \cite{WittenADHM}. So far, we have a description of the dual field theory in terms of an instanton moduli space, namely the moduli space of $N_1$ instantons of a $U(N_5)$ gauge theory on $X$, for which a useful model is the Hilbert scheme of $N_1 N_5$ points on $X$ \footnote{Throughout this note we ignore the center of mass factor of the moduli space that produces a $\tilde{X}$ factor, for some $\tilde{X}$ not necessarily the same as the compactification $X$. The relationship between the two manifolds in the $T^4$ case is clarified in \cite{GiveonKutasovSeiberg}.}. The (conformally invariant limit of the) gauge theory description is expected to only capture the regime of vanishing size instantons (i.e. when the hypermultiplets have small vevs). One can understand that the gauge theory description is approximate by noticing that the Yang-Mills couplings are given in terms of the $X$ volume $V$ and string coupling as $g_1^2 = g_s (2 \pi \alpha'), g_5^2 = g_s V/(\alpha' (2\pi)^3)$ so for energies much smaller than the inverse string length the gauge theories are strongly coupled \cite{Davidetal}. 


To get the SCFT we take an IR limit, which would be dual to a near-horizon limit from the closed string point of view. In this limit, the gauge theory moduli space becomes the target space of the low-energy sigma-model. It has been argued that the correct instanton moduli space is a smooth deformation of the symmetric product theory $Sym^{N_1 N_5}(\tilde{X})/S_{N_1 N_5}$ \footnote{Here we are taking both $N_1, N_5$ large.}. Indeed, there is a point in the SCFT moduli space (far from the supergravity point itself) where the theory takes precisely the symmetric orbifold form. The orbifold point is the analogue of free Yang-Mills theory in the perhaps more-familiar $AdS_5\times S^5$/ 4d $\mc N =4$ SYM duality, and is dual to a stringy point in moduli space which has been explored extensively in recent years (see, e.g., \cite{Eberhardt:2021vsx, Eberhardt:2019ywk, Eberhardt:2018ouy}).

As usual, one can focus on moduli-independent quantities to provide preliminary matches between the supergravity and orbifold points, such as the signed count of 1/4-BPS states at large-N, via the elliptic genus. The elliptic genus matches the corresponding count of BPS (or equivalently, twisted) supergravity states \cite{deBoerEG}, which we reproduced in the previous section. We review the infinite$-N$ elliptic genus computation and its matching to the twisted supergravity index below. This matching follows from the formal equivalence of the elliptic genus to the vacuum character of the chiral algebra in the holomorphic twist; this quantity is also sometimes referred to as the partition function of the half-twisted theory. 

It would be preferable to ``categorify'' the standard elliptic genus computation, and reproduce it directly from the twisted CFT perspective using the holomorphic twist of the symmetric orbifold CFT \footnote{Of course, whenever one wants to match more refined observables than the elliptic genus from the symmetric orbifold theory to the supergravity point (rather than the stringy dual of \cite{Eberhardt:2018ouy}), one must deal with moduli-dependence, e.g. \cite{Taylor:2007hs} .}.  As we mentioned, in two dimensions this is also known as the half-twist \cite{Witten, Kapustin}. It is well-known that the half-twist of a sigma-model can be mathematically formulated as the chiral de Rham complex \cite{Kapustin, Malikovetal, Tan}, and indeed this is precisely what our holomorphic twist captures. 

Unfortunately, obtaining a global description of the half-twist on a curved, compact manifold is a nonperturbative computation subject to worldsheet instanton corrections, and so prohibitively difficult with current technology. We will first review some aspects of the holomorphic twist from the perspective of the UV brane worldvolume gauge theory, and then discuss the connection to the half-twist/chiral de Rham complex of the symmetric orbifold SCFT, explaining their formal equivalence. When discussing the chiral de Rham complex, we must approximate K3 as $\mathbb{C}^2$, which provides a partial matching with twisted supergravity.  
 

\subsection{Branes in twisted supergravity}

We have already recollected the proposal of \cite{CLsugra} that the twist of type IIB supergravity is equivalent to the topological $B$-model on a Calabi--Yau fivefold.
At the level of branes, this proposal further asserts that $D(2k-1)$-branes in type IIB corresponds to topological $B$-branes.
We use that perspective here to deduce the worldvolume CFT of the twist of the $D1/D5$ system in type IIB supergravity.

We consider the system of $D1/D5$ branes in the twist of type IIB on a Calabi--Yau fivefold $Z$. 
For simplicity, we assume that we have a collection of $N_1 = N$ $D1$ branes supported along a closed Riemann surface
\[
\Sigma \subset Z \]
together with a single $D5$ brane which is parallel to the $D1$ branes. 

In topological string theory, one views branes as objects in some category.
Morphisms between objects represent open strings stretching between two branes.
In particular, a general feature of topological string theory is that the open string fields which start and end on the same brane can be described in terms of the algebra of derived endomorphisms of the object representing the brane.
Indeed, following \cite{WittenOpen}, one constructs a Chern--Simons theory based off of this derived algebra of endomorphisms where the gauge fields are degree one elements in the algebra of derived endomorphisms.
In the $B$-model, the category is the category of coherent sheaves on the Calabi--Yau manifold.
Fields of the corresponding open-string field theory (which start and on on the same brane) are given as holomorphic sections of the sheaf of derived endomorphisms.
Following \cite{CLsugra}, we will use a Dolbeault model which resolves a sheaf of holomorphic sections to describe the space of fields as the cohomological shift by one of the Dolbeault resolutions of derived endomorphsism.

We consider $D1$ branes that are a sum of simple branes labeled by the structure sheaf~$\cO_\Sigma$.
In particular, $N$ such $D1$ branes are represented by the object $\cO_{\Sigma}^{\oplus N}$ in the category of quasi-coherent sheaves on the Calabi--Yau fivefold $Z$.
A model for the sheaf of derived endomorphisms of $\cO_\Sigma$ is the holomorphic sections of the exterior algebra of the normal bundle $\cN_{\Sigma}$ of $\Sigma$ in $Z$.
A model for the sheaf of derived endomorphisms of a stack of $N$ such branes is therefore
\beqn
\text{Ext}_{\cO_Z}\left(\cO_\Sigma^{\oplus N} \right) \simeq \lie{gl}(N) \otimes \wedge^\bu \cN_\Sigma .
\eeqn
Thus, the Dolbeault model for the open string fields which stretch between two such $D1$ branes is given by
\beqn\label{eqn:open1}
\Omega^{0,\bu}\left(\Sigma, \lie{gl}(N) \otimes \wedge^\bu \cN_\Sigma\right) [1] .
\eeqn
If we take $X$ to the be the total space of the bundle $\cN_\Sigma$ then the Calabi--Yau condition requires $\wedge^4 N_\Sigma = K_\Sigma$. 
In the case $\Sigma = \C$ and $Z = \C^5$ 
%we can twist by a homomorphism $SO(2) \to SO(4)$ to 
we can write the open string fields \eqref{eqn:open1} as 
\beqn\label{eqn:open1a}
\Omega^{0,\bu}\left(\C , \lie{gl}(N) [\ep_1,\ldots,\ep_4] \right) [1] .
\eeqn
Here the $\ep_i$ are odd variables that carry spin $1/4$, meaning they transform as constant sections of the bundle $K_\C^{1/4}$.
This is precisely the field content of the holomorphic twist of two-dimensional $\cN=(8,8)$ pure gauge theory which is the worldvolume theory living on a stack of $D1$ branes in twisted supergravity on flat space.

Next, we consider $D1-D5$ strings. 
The open string fields are given by 
\beqn
\label{eqn:open15}
\Omega^{0,\bu}\left(\Sigma, \underline{\op{Ext}}_{\cO_X} \left(\cO_Z , \cO_\Sigma^{\oplus N}\right) \right) .
\eeqn
Again, on flat space with $\Sigma = \C$ this can be written in a more explicit way as
\beqn\label{eqn:open15}
\Omega^{0,\bu}\left(\C, K^{1/2}_\C  [\ep_3,\ep_4]\right) \otimes {\rm Hom}(\C, \C^N) = \Omega^{0,\bu}\left(\C, K^{1/2}_\C  [\ep_3,\ep_4]\right) \otimes \C^N .
\eeqn
Together with the $D5-D1$ strings we get 
\beqn\label{eqn:open15a}
\Omega^{0,\bu}\left(\C, K^{1/2}_\C [\ep_3,\ep_4]\right)   \otimes T^*\C^N .
\eeqn

In total, we see that the open-strings of the $D1/D5$ system along $\Sigma = \C$ are given by the Dolbeault complex valued in the following holomorphic vector bundle
\beqn
\bigg(\lie{gl}(N)[\ep_1,\ep_2][1] \oplus K^{1/2}_\C \otimes T^*\C^N \bigg) \otimes \C[\ep_3,\ep_4] .
\eeqn
If we choose twisting data so that the odd variables carry degree $\deg{\ep_1}=\deg{\ep_2} = +1$ then the bundle in parentheses can be written as
\beqn
\lie{gl}(N)[1] \oplus K^{1/2}_\Sigma \otimes T^* \left(\lie{gl}(N) \oplus  \C^N\right) \oplus \lie{gl}(N) [-1] .
\eeqn
The first summand represents the ghosts of the holomorphic CFT and the last summand the anti-ghosts.
The gauge symmetry in the middle term is induced from the standard 
action of $\lie{gl}(N)$ on $T^* \left(\lie{gl}(N) \oplus \C^N\right)$ by Hamiltonian vector fields (this is induced from the adjoint $+$ fundamental action on the base of the cotangent bundle). 
Thus, we see that this model describes ($K^{1/2}_\Sigma$-twisted) holomorphic maps from $\Sigma$ into the well-known GIT description of the symmetric orbifold $\Sym^{N} \C^2$.
That is, the worldvolume theory living on a stack of twisted $D1$ branes is the holomorphic $\sigma$-model of maps into the target $\Sym^N \C^2$.

This analysis happened entirely in flat space.
The $D1$ branes wrapped
\beqn
\C \times 0 \times 0 \times 0 \times 0 \subset \C^5
\eeqn
while the $D5$ brane wrapped
\beqn
\C \times \C^2 \times 0 \times 0 \subset \C^5 .
\eeqn
If we instead replace this $\C^2$ by a compact Calabi--Yau twofold $X$ then the above computation leads us to the well-established expectation that the worldvolume theory, after twisting, is a holomorphic $\sigma$-model with target $\Sym^N X$.
\brian{elaborate}


\subsection{The elliptic genus}

We now briefly recall the elliptic genus computation using the orbifold point in the string moduli space, which reproduces signed counts of 1/4-BPS states in the SCFT. This is formally equal to the partition function of the chiral de Rham complex, or holomorphically twisted theory on the same underlying space. We will take the branes to be supported on $\R \times S^1$ after compactification on $X$, so that the CFT is defined on the cylinder. On the cylinder, the NS sector corresponds to anti-periodic boundary conditions on the fermions. The sigma model is then the $\mc N = (4,4)$ theory whose bosonic fields are valued in maps from $S^1 \rightarrow Sym^N(X)$.  

The physical SCFT has R-symmetries $SO(4) \simeq SU(2)_L \times SU(2)_R$ dual to rotations of the $S^3$ and symmetries under a global $SO(4)_I \simeq SU(2)_a \times SU(2)_b$ of transverse rotations; this latter symmetry is broken by compactification on $X$. The latter $SO(4)_I$, although broken by the background, is still often used to organize the field content of the compactified, and acts as an outer automorphism on the $\mc N=(4, 4)$ superconformal algebra. As is well known, the isometries of $AdS_3 \times S^3$ are $SL(2, \R) \times SL(2, \R) \times SO(4)$ form the bosonic part of the supergroup $SU(1,1|2) \times SU(1,1|2)$ which preserve the supergravity vacuum and form the anomaly-free global subalgebra of the $\mc N= (4,4)$ superconformal algebra.

%The CFT when $X= T^4$ is particularly simple, as the orbifold theory can be described in terms of free fields on $N:= N_1N_5$ copies of the $T^4$ theory \cite{Davidetal}. We write $SU(2)_a \times SU(2)_b$ doublet indices as $A, \dot{B}$, $SU(2)_L\times SU(2)_R$ doublet indices as $\alpha, \dot{\beta}$. $SO(4)_I$ vector indices will be denoted by $i,j$, etc and subscripts $(r), r= 1,\ldots, N$ label the orbifold copy number.  

%Each $T^4$ theory has four free bosons $X^i_{(r)}$ and eight free fermions, the left-movers $\psi^{\alpha \dot{A}}_{(r)}(z)$ and right-movers $\bar{\psi}^{\dot{\alpha} \dot{A}}_{(r)}(\bar{z})$, for fixed copy $(r)$ that satisfy the reality conditions
%\begin{align*}
%\psi^{\dagger}_{\alpha \dot{A}} &= -\epsilon_{\alpha \beta}\epsilon_{\dot{A} \dot{B}}\psi^{\beta \dot{B}} \\ 
%\bar{\psi}^{\dagger}_{\dot{\alpha} \dot{A}} &= -\epsilon_{\dot{\alpha} \dot{\beta}}\epsilon_{\dot{A} \dot{B}}\bar{\psi}^{\dot{\beta} \dot{B}}.
%\end{align*}
%
%In terms of the free fields, we construct the holomorphic $\mc N=4$ superconformal algebra generators (similar expressions hold for the right-movers). In what follows, we have implicitly performed the diagonal sum over the copy index of all fields to obtain $S_{N_1 N_5}$-invariant expressions:
%\begin{align*}
%J^{a}(z) &= {1 \over 4} \epsilon_{\dot{A} \dot{B}} \psi^{\alpha \dot{A}}\epsilon_{\alpha\beta}(\sigma^{* a})^{\beta}_{\gamma} \psi^{\gamma \dot{B}}\\
%G^{\alpha A}(z) &= \psi^{\alpha \dot{A}}\left[\partial X \right]^{\dot{B}A}\epsilon_{\dot{A} \dot{B}}\\
%T(z) &= {1 \over 2} \epsilon_{\dot{A}\dot{B}}\epsilon_{AB}\left[\partial X\right]^{\dot{A}A}\left[\partial X \right]^{\dot{B}B} + {1\over 2}\epsilon_{\alpha \beta}\epsilon_{\dot{A}\dot{B}} \psi^{\alpha \dot{A}}\partial \psi^{\beta \dot{B}}
%\end{align*} with $a$ an $SU(2)_L$ triplet index and using the notation $\left[ X \right]^{A \dot{A}} = {1 \over \sqrt{2}}X^i (\sigma^i)^{\dot{A} A}$ using the usual Pauli matrices plus $\sigma^4 = i \mbb 1_2$.
%
%The free fields are normalized in the usual way,
%\begin{align*}
%\langle X^i(z) X^j(w) \rangle &= -2 \delta^{ij}\log|z-w| \\
%\langle \psi^{\alpha \dot{A}} \psi^{\beta \dot{B}} \rangle &= - {\epsilon^{\alpha \beta}\epsilon^{\dot{A}\dot{B}} \over z-w}
%\end{align*}

Part of the underlying chiral algebra of the $\mc{N}=(4, 4)$ SCFT OPEs is the usual holomorphic $\mc N=4$ superconformal algebra with $c=6N$ (which can be explicitly constructed as a diagonal sum over the $N$ copies of the seed $c=6$ sigma models):
\begin{align*}
J^a(z)J^b(w) &\sim  {c \over 12}{\delta^{ab} \over (z-w)^2} + i \epsilon^{ab}_c {J^c(w) \over z-w}\\
J^a(z)G^{\alpha A}(w) &\sim {1 \over 2} (\sigma^{*a})^{\alpha}_{\beta} {G^{\beta A}(w) \over z-w}\\
G^{\alpha A}(z)G^{\beta B}(w) &\sim  - \epsilon^{AB}\epsilon^{\alpha \beta}{T(w) \over z-w} - {c \over 3}{\epsilon^{AB} \epsilon^{\alpha \beta} \over (z-w)^3} + \epsilon^{AB}\epsilon^{\beta\gamma}(\sigma^{*a})^{\alpha}_{\gamma}\left({2 J^a(w) \over (z-w)^2} + {\partial J^a(w) \over z-w} \right)\\
T(z)J^a(w) &\sim {J^a(w) \over (z-w)^2} + {\partial J^a(w) \over z-w}\\
T(z)G^{\alpha A}(w)&\sim { {3 \over 2} G^{\alpha A}(w) \over (z-w)^2} + {\partial G^{\alpha A}(w) \over z-w}\\
T(z)T(w) &\sim {c \over 2}{1 \over (z-w)^4} + 2 {T(w) \over (z-w)^2} + {\partial T(w) \over z-w}.
\end{align*}
Above, we have written $SU(2)_a \times SU(2)_b$ doublet indices as $A, \dot{B}$ and $SU(2)_L\times SU(2)_R$ doublet indices as $\alpha, \dot{\beta}$. 

%$SO(4)_I$ vector indices will be denoted by $i,j$, etc and subscripts $(r), r= 1,\ldots, N$ label the orbifold copy number.  

There is, of course, also a right-moving copy in the full SCFT, though only the chiral half above will be accessible in the holomorphic twist. 

%It is also easy to derive the OPEs of these generators with the basic free primaries:
%\begin{align*}
%J^a(z)\psi^{\alpha \dot{A}}(w) &\sim {1 \over 2}(\sigma^{*a})^{\alpha}_{\beta}{\psi^{\beta \dot{A}}(w) \over z-w}\\
%G^{\alpha A}(z)\left[\partial X(w)\right]^{\dot{B}B} &\sim \epsilon^{AB}\left({\psi^{\alpha \dot{B}}(w) \over (z-w)^2} + {\partial \psi^{\alpha \dot{B}}(w) \over z-w} \right)\\
%G^{\alpha A}(z)\psi^{\beta \dot{A}}(w) &\sim \epsilon^{\alpha \beta}{\left[\partial X(w) \right]^{\dot{A}A} \over z-w}\\
%T(z) \left[\partial X(w) \right]^{\dot{A}A} &\sim { \left[\partial X(w) \right]^{\dot{A}A} \over (z-w)^2} + {\left[\partial^2 X(w) \right]^{\dot{A}A} \over z-w}\\
%T(z) \psi^{\alpha \dot{A}} &\sim {{1\over2} \psi^{\alpha \dot{A}}(w) \over (z-w)^2} + {\partial \psi^{\alpha \dot{A}}(w) \over z-w}.
%\end{align*}
% As always, we define the modes $\mc O_m$ of a field $\mc O(z)$ in terms of its weight $\Delta$:
% \begin{equation}
% \mc O_m = \oint {dz \over 2\pi i} \mc O(z) z^{\Delta + m-1}.
% \end{equation}

 It is easy from the above OPEs to get the mode algebra of the $\mc N=4$ superconformal algebra. For simplicity, we will just record the mode algebra of the global subalgebra generated by $\left\lbrace J_0^a, G^{\alpha A}_{\pm 1/2}, L_0, L_{\pm 1} \right\rbrace$, which has its Cartan subalgebra generated by $J_0^3, L_0$: 
 \begin{align}
 \left[L_0, L_{\pm 1} \right]&= \mp L_{\pm}\\
 \left[L_1, L_{-1}\right] &= 2 L_0 \\
 \left[J^a_0, J^b_0 \right]&= i \epsilon^{a b}_c J^c_0 \\
 \left\lbrace  G^{\alpha A}_{1/2}, G^{\beta B}_{-1/2}\right\rbrace &= \epsilon^{AB}\epsilon^{\beta\gamma}(\sigma^{*a})^{\alpha}_{\gamma}J^a_0 -\epsilon^{AB}\epsilon^{\alpha \beta}L_0 \\
 \left\lbrace  G^{\alpha A}_{-1/2}, G^{\beta B}_{1/2}\right\rbrace &= -\epsilon^{AB}\epsilon^{\beta\gamma}(\sigma^{*a})^{\alpha}_{\gamma}J^a_0 -\epsilon^{AB}\epsilon^{\alpha \beta}L_0 \\
 \left[L_0, G^{\alpha A}_{\pm 1/2} \right]&= \mp G^{\alpha A}_{\pm 1/2}\\
 \left[L_1, G^{\alpha A}_{1/2}\right]&= \left[L_{-1}, G^{\alpha A}_{-1/2} \right] = 0 \\
 \left[L_{\pm1}, G^{\alpha A}_{\mp 1/2} \right]&= \pm G^{\alpha A}_{\pm 1/2} \\
 \left[J^a_{0}, G^{\alpha A}_{\pm n} \right]&= {1 \over 2}(\sigma^{*a})^{\alpha}_{\beta}G^{\beta A}_{\pm n} 
 \end{align} These commutators generate $\mf psu(1,1|2)$. Notice that there is no anomaly $c = 6 N$ in the global subalgebra. 
 
 Later in the text, we will reproduce the chiral algebra of the symmetric orbifold SCFT using the method of Koszul duality. The global subalgebra can be readily reproduced at tree-level in the planar limit; to obtain the centrally extended chiral algebra, including the holomorphic $\mc{N}=4$ superconformal algebra, we will incorporate a certain class of planar Feynman diagrams sensitive to the backreaction. 
 
%Next, we will briefly recapitulate the SCFT states which contribute to the infinite-N limit of the elliptic genus, and their avatars in the holomorphic twist. 
% 
%\subsection{Chiral primaries \& short multiplets in the symmetric orbifold}
%\textcolor{red}{Chiral primaries themselves have an OPE $J[m, 0]J[n, 0] \sim regular$, but still have interesting three point functions from the regular terms $O^i O^j \sim C^{ij}_k O^k$... can we recover the chiral ring coefficients from KS somehow?}
%From studying the algebra above it is easy to derive that a primary $\phi$ that also satisfies the condition $G^{+ A}_{-1/2}|\phi \rangle = 0, A=1,2$ satisfies $h=j$ (for $L_0$ eigenvalue $h$ and $J_0^3$ eigenvalue $j$) and is called a chiral primary. The quantum numbers and two and three-point functions among chiral primaries have been extensively studied in the literature, and are protected quantities as one moves in moduli space and can be matched to the corresponding quantities at the supergravity point (e.g. \cite{}). Anti-chiral primaries are defined similarly and satisfy $h = -j$. In the full physical theory, one combines left and right-moving (anti)chiral primaries: $(c, c), (a,a), (a,c), (c, a)$. For the holomorphic twist, we will focus on chiral primaries in the holomorphic half of the SCFT. 
%
%Chiral primaries can arise in the twisted and untwisted sectors of the orbifold. In an $n$-twisted sector (cyclically permuting $n$ copies of the $X$ SCFT), the weights of the chiral primaries are bounded: ${n-1 \over 2} \leq h \leq {n+1 \over 2}$. The chiral primaries are explicitly constructed as follows, starting with a twist field. Consider the twist field $\sigma_{l+1}(z)$ which cyclically permutes $l+1$ copies of the holomorphic SCFT as one moves around the point $z$ in the base space; it creates the ground state of the twisted sector by acting on the original NS vacuum. It has weight $h = {6 \over 24}( (l+ 1) - {1 \over l+1})$, but no charge. To make a chiral primary from this state, we must dress it with modes of $J^+$, which carry $SU(2)_L$ charge. In particular, using the fact that operators in the twisted sector are fractionally moded we can build the chiral primaries \cite{LuninMathur, LuninMathurSusy}:
%\begin{align*}
%\sigma^0_{l+1}&:=  J^+_{-{l-1 \over l+1}}J^{+}_{-{l-3 \over l+1}} \ldots J^+_{-{1 \over l+1}}\sigma_{l+1}, \ \ \ l+1 \text { odd} \\
%\sigma^0_{l+1}&:= J^+_{-{l-1 \over l+1}}J^{+}_{-{l-3 \over l+1}} \ldots J^+_{-{2 \over l+1}}S^+_{l+1}\sigma_{l+1}, \ \ \ l+1 \text { even}
%\end{align*} which have $h=j= l/2$. The spin fields $S^+_{l+1}$ map the NS sector vacuum to the R sector, in order to restore the overall periodicity of the fermions as it traverses the length of the long closed string. These single long string states map to single particle states in the supergravity theory. 
%
%
%From this basic chiral primary, we can create three additional chiral primaries by acting with the generators of $H^{\bullet, 0}(X)$, which also map to single particle supergravity states. For example, when $X=T^4$ these are simply represented by free fermions:
%\begin{align*}
%&\sigma^0_{l+1},  &\qquad h=j= l/2 \\
%&\psi^{+ \dot{1}}\sigma^0_{l+1}, &\qquad h=j= (l+1)/2\\
%&\psi^{+ \dot{2}}\sigma^0_{l+1} , &\qquad h=j= (l+1)/2\\
%&\psi^{+ \dot{1}}\psi^{+ \dot{2}}\sigma^0_{l+1},   &\qquad h=j= (l+2)/2.
%\end{align*}
%Combining this construction on the left and right enables one to construct 16 $(c,c)$ primaries, which can be mapped to cohomology classes of the target space when viewing the fermions as differential forms. (Again, remember that we are implicitly summing over copy indices so that in the $n$th twisted sector we have e.g. $\psi^{+ \dot{1}} = \sum_{r=1}^n \psi^{+ \dot{1}}_{(r)}$). Of course, when $l=0$, the basic chiral primary is just the NS sector vacuum with $h=j=0$. 
%
%The chiral primaries are part of supermultiplets. These $SU(1,1|2)$ multiplets arise from acting on the chiral primaries with modes of the global subalgebra: the chiral primaries are precisely the highest weight states of short $SU(1,1|2)$ representations \footnote{Long $SU(1,1|2)$ representations can be obtained by acting with the global modes on global primary fields, i.e. fields annihilated by $L_1, G^{\alpha A}_{+1/2}$; there are 16 states per long multiplet.}. Schematically, one can view a short multiplet as associating to each chiral primary $c$ 4 $\mathfrak{sl}(2)$ primary fields that are also $SU(2)_L$ highest weight states: $|c \rangle, G^{- 1}_{-1/2}|c \rangle, G^{-2}_{-1/2}|c \rangle,  \\
%G^{-1}_{-1/2}G^{-2}_{-1/2}|c \rangle + {1 \over 2h}J^-_0 L_{-1}|c\rangle$ . To fill out the rest of the short multiplet, one acts on each of these four states with an arbitrary number of $L_{-1}$ generators, as well as with repeated applications of $J_0^-$ to fill out each $SU(2)_L$ multiplet. 
%
%When the chiral primary has weight $h \leq 1/2$, the representation is further truncated and does not contain the $G^{-1}_{-1/2}G^{-2}_{-1/2}|c \rangle + {1 \over 2h}J^-_0 L_{-1}|c\rangle$ state, so is sometimes called an ultra-short representation. 
%
%One can also construct anti-chiral primaries with $h=-j$ in the holomorphic (or anti-holomorphic with $\bar{h}=-\bar{j}$) sector. The construction is almost identical to the chiral primary case (again, see \cite{LuninMathur} for details):
%\begin{align*}
%\tilde{\sigma}^0_{l+1}&:=  J^-_{-{l-1 \over l+1}}J^{-}_{-{l-3 \over l+1}} \ldots J^-_{-{1 \over l+1}}\sigma_{l+1}, \ \ \ l+1 \text { odd} \\
%\tilde{\sigma}^0_{l+1}&:= J^-_{-{l-1 \over l+1}}J^{-}_{-{l-3 \over l+1}} \ldots J^-_{-{2 \over l+1}}S^-_{l+1}\sigma_{l+1}, \ \ \ l+1 \text { even}.
%\end{align*} One can again act on these basic primaries with the fermions (more generally, differential forms). Two point functions of chiral and anti-chiral primaries are non-vanishing (in contrast to c-c and a-a two-point functions) and can always be normalized to unity when the operators are unit-separated.
%
%We also note briefly that the exactly marginal operators, which form a basis for the tangent space of the moduli space, can be found in such multiplets. In particular, marginal operators that preserve $\mc{N}=(4,4)$ supersymmetry must be $SU(2)_L$ singlets with $h=\bar{h}=1$. Therefore, they must be in the multiplets with highest weight states (combining now holomorphic and anti-holomorphic sectors) $\sigma^{+ \dot{+} }_{2}, \psi^{+\dot{A}}\bar{\psi}^{\dot{+}\dot{B}}$. Each of these five operators gives four such states, corresponding to 20 marginal operators.
%
%\subsection{The chiral de Rham complex} 
%We now turn to the holomorphic twist of the symmetric orbifold SCFT, which is mathematically codified in the chiral de Rham complex. Physically, the holomorphic twist will localize to 1/4-BPS states in the SCFT though, as is well known \cite{deBoerEG}, the inifinite-N limit of the elliptic genus, which is nothing but the partition function of the holomorphically twisted theory (or equivalently the vacuum character of the resulting chiral algebra) can be understood in terms of counts of chiral primary states. In the subsequent section, we will obtain the infinite-N limit of the full chiral algebra itself, not just its vacuum character, from Koszul duality techniques. 
%
%The chiral de Rham complex is a sheaf of super vertex algebras defined on any manifold \cite{Malikovetal}.
%To an open coordinate neighborhood in the target, the vertex algebra is a $bc\beta\gamma$ system.
%Its global sections over the target is a super vertex algebra which is notoriously difficult to describe explicitly \cite{Moonshine stuff?}.
%On the other hand, properties of this vertex algebra are known.
%For complex manifolds this vertex algebra is known to have an action by the $\cN=2$ superconformal algebra.
%The graded character of this super vertex algebra returns the elliptic genus of the complex manifold; in this sense the chiral de Rham complex is a vertex algebra refinement of the elliptic genus.
%If the complex manifold is Calabi--Yau then this symmetry is further enhanced to the $\cN=4$ superconformal algebra and hence makes manifest the decomposition of the elliptic genus for a Calabi--Yau manifold into characters of irreducible representations of this superconformal algebra.
%
%
%We review how the chiral de Rham complex arises as the vertex algebra associated to the holomorphic twist, or half twist, which captures capturing $1/4$-BPS states of the supersymmetric $\sigma$-model on a complex manifold.
%
%The chiral de Rham complex is a sheaf of super vertex algebras defined on any manifold \cite{MSV,??}.
%To an open coordinate neighborhood in the target, the vertex algebra is a $bc\beta\gamma$ system.
%Its global sections over the target is a super vertex algebra which is notoriously difficult to describe explicitly \cite{Moonshine stuff?}.
%On the other hand, properties of this vertex algebra are known.
%For complex manifolds this vertex algebra is known to have an action by the $\cN=2$ superconformal algebra.
%The graded character of this super vertex algebra returns the elliptic genus of the complex manifold; in this sense the chiral de Rham complex is a vertex algebra refinement of the elliptic genus.
%If the complex manifold is Calabi--Yau then this symmetry is further enhanced to the $\cN=4$ superconformal algebra and hence makes manifest the decomposition of the elliptic genus for a Calabi--Yau manifold into characters of irreducible representations of this superconformal algebra.
%
%
%We review how the chiral de Rham complex arises as the vertex algebra associated to the holomorphic twist, or half twist, which captures capturing $1/4$-BPS states of the supersymmetric $\sigma$-model on a complex manifold.
%
%One can start with the single-particle states furnished by chiral primaries and their $SU(1,1,|2)$ descendents and construct a Fock space. 
%These are dual to $1/4$-BPS states in the full physical SCFT, and their graded dimension gives rise to the elliptic genus of the model.
%The $1/4$-BPS states are captured by the so-called half-twist of the supersymmetric model which we now recall.
%
%The two-dimensional $\cN=(2,2)$ $\sigma$-model admits a half-twist along the lines of \cite{Kapustin, Witten} which results in a purely holomorphic theory.
%For this purpose, it can be convenient to recombine the fermions into vectors, and complexify the bosons $X$ so that we chose local holomorphic and anti-holomorphic coordinates on the target space: $\phi^i, \phi^{\bar{i}}$. 
%Then, explicitly, the fermionic fields are sections of the following bundles: 
%\begin{align*}
%\Psi^i &\in \Gamma(K^{1/2}\otimes \phi^*(T^{(1,0)}M))\\
%\Psi^{\bar{i}}&\in \Gamma(K^{1/2}\otimes \phi^*(T^{(0,1)}M))\\
%\bar{\Psi}^{i} &\in \Gamma(\bar{K}^{1/2}\otimes \phi^*(T^{(1,0)}M))\\
%\bar{\Psi}^{\bar{i}} &\in \Gamma(\bar{K}^{1/2}\otimes \phi^*(T^{(0,1)}M))
%\end{align*}  
%where as before the left-movers are given by $\Psi$ and the right-movers by $\bar{\Psi}$.
%
%To pass to the half-twisted model, we will restrict to the cohomology of the supercharge $\bar{Q}_+$, after we twist with a certain combination of R-symmetry currents. It is common, as in  \cite{Kapustin}, to perform the A-type twist by the current ${1 \over 2}(J_L - J_R)$ before passing to cohomology. We will instead simply consider a twist by $-J_R$, on the right-movers only, so that the twisted fields live in the following spaces of sections:
%
%\begin{align*}
%\Psi^i &\in \Gamma(K^{1/2}\otimes \phi^*(T^{(1,0)}M))\\
%\Psi^{\bar{i}} &\in \Gamma(K^{1/2}\otimes \phi^*(T^{(0,1)}M))\\
%\bar{\Psi}^i &\in \Gamma(\bar{K} \otimes\phi^*(T^{(1,0)}M))\\
%\bar{\Psi}^{\bar{i}} &\in  \Gamma(\phi^*(T^{(0,1)}M))
%\end{align*} 
%
%We then make the standard local identifications of fields in the twisted theory with those of ${\rm dim}_{\C}M$ copies of a free $bc\beta\gamma$ system (though again, we stress, the $bc$ fields are just ordinary fermions):
%\begin{align*}
%\beta_i &\equiv g_{i \bar{j}}\partial_z X^{\bar{j}} \\
%\gamma^i &\equiv \phi^i \\
%b_i &\equiv g_{i \bar{j}} \Psi^{\bar{j}} \\
%c^i &\equiv \Psi^i.
%\end{align*}
%On a local patch $U \subset M$ we can also take $g_{i\bar{j}} = \delta_{i \bar{j}}$.
%Notice that in the standard treatment of the half-twisted model, where the twist is performed using the A-model current, the left-moving fermions transform instead as $\Psi^i \in \Gamma(\phi^* T^{(1, 0)}M), g_{i \bar{j}}\Psi^{\bar{j}} \in \Gamma(K \otimes \phi^* T^{(0, 1)}M)$, rendering the $bc$ fields of spin $(1, 0)$, respectively. In our case, the spins remain half-integral.
%
%The nontrivial operators in the $\bar{Q}_+ = g_{i \bar{j}}\tilde{\Psi}^{\bar{j}} \partial_{\bar{z}}X^i$ cohomology are those of left and right-moving conformal dimensions $(n, 0), n \geq 0$ \cite{Tan,CostelloHol,ESW}. 
%The operators of dimension $(0, 0)$ (i.e. the operators forming the ground ring), in particular, have an interpretation as $(0,k)$-forms on the target space. 
%The operators of dimension $(n > 0, 0)$ are given by $(0, k)$-forms valued in various tensor product bundles arising from insertions of $\partial_z X^i, g_{i \bar{j}}\partial_z X^{\bar{j}}, g_{i \bar{j}}\partial_z \Psi^{\bar{j}}$. In terms of the physical operators (post-twist), the operators in $\bar{Q}_+$-cohomology will composites of 1.) polynomials in the left-moving fermions and in arbitrary numbers of their holomorphic derivatives 2.) some function of the scalar fields and arbitrary numbers of their holomorphic derivatives 3.) The field $\Psi^{\bar{i}}$, though none of its derivatives (since, by its equation of motion, the holomorphic derivatives of $\Psi^{\bar{i}}$ may be expressed in terms of the aforementioned fields and their derivatives only). Call such an operator $\mathcal{F}$. One can further study which such operators can be constructed globally. Using standard techniques from cohomology reveals that that the Dolbeault cohomology describing the local operators $H_{\bar{\partial}}^{(0,k)}(M, \mathcal{F})=0$ for $k>0$, so that we disallow operators $\mathcal{F}$ that contain $\Psi^{\bar{i}}$. Translating this over to the $bc\beta\gamma$ language, we have that the relevant operators are nothing but functions of $b, c, \beta, \gamma$ and their holomorphic derivatives.
%
%In the present context, our target space $M$ is given by $\oplus_{a=1}^{\infty} Sym^a (T^4)$ or $\oplus_{a=1}^{\infty} Sym^a(K3)$. These spaces are hyperkahler, so the chiral de Rham complex has  (in general, a twisted version of) $\mathcal{N}=4$ supersymmetry \cite{Heluanietal}.

It is difficult to perform explicit computations in the holomorphic twist beyond a local (flat space) model, even for a single copy of $X$. Rather than try to work with the full chiral de Rham complex directly, we will outline the matching of (counts of) states between twisted supergravity and twisted CFT (via the elliptic genus). Then we will turn to the determination of the OPEs in the holomorphically twisted theory in the $N \rightarrow \infty$ limit by applying Koszul duality to our twisted supergravity theory. 

%It is difficult, however, to move beyond a local model, even for a single copy of $X$ due to nonperturbative effects. The basic, non vanishing OPEs for $bc\beta\gamma$ systems are
%\begin{align*}
%b_i(z)c^j(w) = c^j(z)b_i(w) &\sim {\delta^{j}_i \over z - w} \\
%\beta_i(z)\gamma^j(w) = -\gamma^j(z)\beta_i(w) &\sim {-\delta^{j}_i \over z - w}
%\end{align*}
%
%As explained in previous sections, odd spin operators in the symmetric orbifold theories will be built up from the Ramond sector vacuum, so we will also need the OPEs between the ghosts and the operator $\Sigma(z)$ (often called the spin field) that maps $|0\rangle_{NS} \rightarrow |0\rangle_{R}$:
%\begin{align*}
%\beta(z)\Sigma(w) &\sim {\tilde{\Sigma}(w) \over (z - w)^{1/2}}\\
%\gamma(z)\Sigma(w) &\sim 0.
%\end{align*}


%\subsection{Elliptic genera for K3 surfaces}

Consider the chiral half of the $\mc N= (4,4)$ $\sigma$-model on the symmetric orbifold  $\Sym^N X$ where $X$ is $T^4$ or a $K3$ surface. 
After performing the half-twist, this is all that remains of the supersymmetric $\sigma$-model.
 According to \cite{DMVV} we can regard the direct sum of the vacuum modules of the chiral algebras of $\Sym^N X$, for each $N$, as being itself a Fock space. The generators of this Fock space are given by the single string states. These single string states are the analog of single trace operators in a gauge theory, and will ultimately be matched with single-particle states in the holographic dual.

Let us brefly recall the computation of the elliptic genus. Let $c(n,m)$ be the super-dimension of the space of operators in supersymmetric $\sigma$-model into $X$, which are of weight $n$ under $L_0$ and of weight $m$ under the action of the Cartan of $SU(2)_R$.  
Let $q,y$ be fugacities for $L_0$ and the Cartan of $SU(2)_R$, respectively---the elliptic genus $\chi(X;q,y)$ is a series in these variables.  
Of course, for $X = T^4$ the elliptic genus vanishes \footnote{One could instead consider the modified elliptic genus for $T^4$, which is enriched with additional insertions of the fermion number operator to absorb the fermionic zero modes.}, so we will now fix $X = K3$.

Introducing another parameter $p$, which keeps track of the symmetric power, we can consider the generating series
\begin{equation} 
	\sum_{n \geq 0} p^n \chi(\Sym^n X; q,y) 
\end{equation}
%where $\chi$ indicates the character of the vacuum module of the $bc\beta\gamma$\footnote{We will call the system a $bc\beta\gamma$ system by a slight abuse of terminology: we will not employ the A-type twist on the left-movers when restricting to $\bar{Q}_+$ cohomology, so that the would-be $bc$ fields have fermionic statistics and spin.} on $\Sym^n X$.  
The main result of \cite{deBoerEG, DMVV} is an expression for this generating series
\begin{equation} 
	\sum_{n} p^n \chi(\Sym^n X; q,y) = \prod_{l,m \geq 0,n >0 } \frac{1}{(1 - p^n q^m y^l)^{c(nm,l)}}
\end{equation}
where $c(m,l)$ is a function of the quantity $4m-l^2$.
In other words, we can interpret the direct sum of the vacuum modules of the $\Sym^n X$ $\sigma$-models as being the Fock space generated by a trigraded super-vector space 
\begin{equation} 
	V =\oplus_{n \ge 0,m,l} V_{n,m,l} 
\end{equation}
where the super-dimension of $V_{n,m,l}$ is $c(nm,l)$.

Setting $V_n = \oplus_{m,l} V_{n,m,l}$, we see that $V_n$ is isomorphic to the vacuum module of the $bc\beta\gamma$ system, or chiral de Rham complex, on the original surface $X$, except with a different conformal structure: a state of the $\sigma$-model into $X$ of spin $k$ is of spin $k/n$ in $V_n$.  

The states in $V_N$ will play the role of the single-trace operators in the large $N$ limit of the $\Sym^N X$ $\sigma$-model.   
These states can be understood geometrically as follows---let us focus just on the $S^1$-modes of this $\sigma$-model.
A map $S^1 \to \Sym^N X$ is the same as an $N$-fold cover $M \to S^1$ together with a map $M \to X$.  
Therefore, the Hilbert space of the $\sigma$-model on $\Sym^N X$ decomposes over sectors corresponding to the topological type of this $N$-fold cover, which are labelled by partitions of $N$. 
The single string sector is the sector that corresponds to $M$ being connected. 
This means that the monodromy of the cover $M \to  S^1$ is conjugate to the length $N$ cycle of type $(1 \dots N)$ in the symmetric group $S_N$.  

Since the $N$-fold cover of $S^1$ corresponding to the single trace sector is connected, the Hilbert space of the single-trace sector is isomorphic to that of the original $\sigma$-model into $X$.  
However, the conformal structure is different---a rotation along $S^1$ in this $\sigma$-model rotates the total space $1/N$ times.
This tells us that an operator in the single-trace sector carries spin $1/N$ times that of the corresponding state of the original $\sigma$-model. 
The projection onto $\Z_N$-invariant states ultimately restores integrality of the spin. 
In particular, the generating function of elliptic genera of $\Sym^N X$ decomposes as
\begin{equation}
\sum_{N\geq 0}p^N \chi(\Sym^N X; q, y) = \prod_{n>0}\sum_{N \geq 0}p^{n N}\chi(\Sym^{N}\mathcal{H}^{\Z_n}_{(n)}; q, y)
\end{equation}
with $\sum_{N \geq 0}p^{n N}\chi(\Sym^{N}\mathcal{H}^{\Z_n}_{(n)}; q, y) = \prod_{l, m\geq0}{1 \over (1 - p q^m y^l)^{c(mn, l)}}$. Here, $\mathcal{H}_{(n)}$ is the Hilbert space of a single long string on $X$ of length $n$ with winding number $1/n$. 

We can extract the $N \rightarrow \infty$ limit of this expression, following the logic employed in \cite{deBoerEG, MAGOO, BKKP}, particularly \cite{BKKP}. First, in preparation for comparison to supergravity, we perform spectral flow\footnote{We shift the overall power of $q$ by $q^{c/24}$ so that the vacuum occurs at $q^0$.} to the NS sector:
\begin{align*}
\sum_{N \geq 0}p^N \chi_{NS}(\Sym^N X; q, y) & = \sum_{N\geq 0}p^N \chi(\Sym^N X; q, y \sqrt{q}) y^N q^{N/2} \\
&= \prod_{\substack{n \geq 0 \\ m \geq 0, m \in \Z \\ l \in \Z}} \frac{1}{(1 - p^n q^{m + l/2 + n/2} y^{l + n})^{c(nm,l)}} \\
&= \prod_{\substack{n \geq0 \\ m' \geq |l'|/2, \ 2 m' \in \Z_{\geq 0} \\ l' \in \Z, \ m' - l'/2 \in \Z_{\geq 0}}} \frac{1}{(1 - p^n q^{m'} y^{l'})^{c(nm' - nl'/2,n-l')}}.
\end{align*}

At any power of $q$, there will be contributions from terms of the form ${1 \over (1 - p y^{l'})^{c(-l'/2, l'-1)}}$. The only nonvanishing such term in our case when $m'=0$ is ${1 \over (1 - p)^2}$. We wish to isolate the coefficients of all terms of the form $q^a y^b p^N$ for $a \ll N$. Taylor expanding ${1 \over (1-p)^2}$ and extracting the desired coefficient gives $N h(a, b) + \mathcal{O}(N^0)$ where $h(a, b)$ is the coefficient of $q^a y^b$ in
\begin{equation}\nonumber
\prod_{\substack{m' \geq |l'|/2, \ 2 m' \in \Z_{\geq 0} \\ l' \in \Z, \  m' - l'/2 \in \Z_{\geq 0}}}{1 \over (1 - q^{m'} y^{l'})^{f(m', l')}}
\end{equation}with $f(m', l'):= \sum_{n >0}c(n(m' -  l'/2), l' - n)$.  The coefficients $c(M, L)$ vanish for $4M-L^2 < -1$ so for $m' \geq 1$ the sum truncates to $f(m', l') = \sum_{n=1}^{4m'}c(n(m' -  l'/2), l' - n)$.

Hence, we can get a finite contribution upon dividing by $N$. 

We can also write out the non-vanishing $f(m', l')$ more explicitly, recalling that the coefficients are constrained to lie in the following range of the Jacobi variable: $-2m' \leq l' \leq 2m', l' \equiv 2 m' \mod 2$. Reproducing the elementary manipulations in Appendix A of \cite{BKKP} (in particular, using the fact that $c(N, L)$ depends only on $4N-L^2$ and $L \ {\rm mod} \ 2$) allows us to rewrite the sum as
\begin{equation}\label{eq:fml1}
f(m', l') = \left( \sum_{\tilde{n} \in \Z}c(m'^2 - l'^2/4, \tilde{n}) \right) - c(0, l'),
\end{equation} where $n':= n - 2m$ in the first term. 
The first term is non-vanishing only when $l' = \pm 2 m'$ and then it reduces to the Witten index of K3, i.e. $f(m', \pm 2m') = 24$ for general $m'$. Otherwise, we have $f(m', l') = -c(0, l')$. When $m' \in \mathbb{Z}$ the nonvanishing such term is $-c(0, 0) = -20$, and when $m' \in \Z + 1'/2$ we have $-c(0, 1) = -2$ and $-c(0, -1) = -2$. 

In sum, we obtain
\begin{align}
{\rm lim}_{N \rightarrow \infty}{\chi_{NS}(\Sym^{N} X; q, y) \over N} &= \prod_{k \geq 1}{(1 - q^k)^{20}(1 - q^{k-1/2}y^{-1})^2(1 - q^{k-1/2}y)^2 \over (1 - q^{k/2}y^k)^{24}(1 - q^{k/2}y^{-k})^{24}} \\
&= 1 + \left({22 \over y} + 22 y \right)q^{1/2} + \left({277 \over y^2} + 464 + 277 y^2 \right)q + \text{O}(q^{3/2}).
\end{align} 
We will denote this large $N$ limit by $\chi_{NS}(\Sym^\infty X ; q,y)$. 
In particular, for there are two bosonic towers corresponding to (anti)chiral primary states and three fermionic towers corresponding to (derivatives of) the states capturing the cohomology of a single copy of K3. At $k=1$, there is a cancellation to ${(1 - q)^{20} \over (1 - q^{1/2}y)^{22}(1 - q^{1/2}y^{-1})^{22}}$.

We observe that this expression for the large $N$ limit of the elliptic genus agrees exactly with the plethystic exponential of the single particle twisted supergravity index we computed in \eqref{eqn:sugraindex}. One can easily see this by using the definition of the plethystic exponential 
\begin{equation}
\textrm{PE}[f](q, y) = \textrm{exp}\left(\sum_{k=1}^{\infty}{f(q^k, y^k) \over k}\right)
\end{equation} and rewriting the infinite-N elliptic genus as $\textrm{PE}[f_{CFT}](q, y)$ in terms of the function
\begin{equation}
f_{CFT}(q, y) = \sum_{m=1}^{\infty} 24 (q^{1/2} y)^m + 24 (q^{1/2}y^{-1})^m -20 q^m - 2 q^{m-1/2} y - 2 q^{m-1/2} y^{-1} ,
\end{equation} which can be immediately matched with $\textrm{PE}[f_{sugra}](q, y)$. 
This is significant evidence that our twisted supergravity theory is dual to (a point on the moduli space of) the large $N$ limit of the symmetric orbifold CFT. 

\textcolor{red}{We can reinstate some of the commented-out text below this line if and only if it is useful when discussing the match to the LQT approach.}
%Throughout this derivation, we have used the coefficients $c(m, l)$ that appear in the expansion of the K3 elliptic genus in the Ramond sector: $\sum_{m \geq 0, l \in \mathbb{Z}} c(m, l) q^m y^l$. We can also rewrite things slightly in terms of the coefficients of the NS sector elliptic genus expansion: $\sum_{2 m' \in \mathbb{Z}_{\geq 0}, l' \in \mathbb{Z}} \mathcal{C}(m', l') q^{m'} y^{l'} = 2 + (20/y + 20 y)q^{1/2} + (2/y^2 - 128 + 2 y^2)q + \ldots$ using the half-integral spectral flow relation on the coefficients of the elliptic genera: $\mathcal{C}(m', l') = c(m'-l'/2, l'-1)$. 
%
%%Applying spectral flow to the second quantized elliptic genus and reindexing $n,m,l$ gives:
%%\begin{equation}
%%\prod_{\substack{n \geq0 \\ m \in \Z_{\geq 0} \\ l \in \Z}} \frac{1}{(1 - p^n q^{m} y^{l})^{c(nm - nl/2,n-l)}} = \prod_{\substack{n \geq0 \\ m \geq |l|/2, \ 2 m \in \Z_{\geq 0} \\ l \in \Z, \ m - l/2 \in \Z_{\geq 0}}} \frac{1}{(1 - p^n q^{m} y^{l})^{\mathcal{C}(n m - n l/2 - n/2 + l/2 + 1/2, l - n + 1)}}.
%%\end{equation}
%%The same set of manipulations as before give (again with $m \in \mathbb{Z}_{\geq 0}/2, l \in \mathbb{Z}, -2m \leq l \leq 2m, l \equiv 2 m (\text{ mod } 2)$)
%%\begin{equation}
%%f(m, l) = \left(\sum_{n'= - \infty}^{\infty}\mathcal{C}(m^2 - l^2/4 + (n' + 1)/2, n' + 1) \right) - \mathcal{C}((l + 1)/2, l + 1)
%%\end{equation} to the same conclusion. As before, $n':= n - 2m$. In particular, when $l = \pm 2 m$, the sum only gets nonvanishing contributions from $n' = -1, 0, 1$, which contribute $2, 20, 2$, respectively. 
%
%Let us unpack those contributions a bit more, starting from the expression 
%\begin{align}\label{eq:fml2}
%f(m', l') &= \sum_{n=1}^{4m'}c(n(m'-l'/2), l'-n) \\
%&= \sum_{n=1}^{4m'}\mathcal{C}(n(m'-l'/2 - 1/2) + l'/2 + 1/2, l' - n +1).
%\end{align} In the second line, we have used spectral flow to rewrite the sum in terms of the NS sector elliptic genus.
%
%
%Let us henceforth drop the primes on our variables, for ease of notation, and then take for example the states $l = 2 m$. The summand is nonvanishing only for $\mathcal{C}(1/2, 1)= 20, \mathcal{C}(0, 0)=2, \mathcal{C}(1, 2)= 2$ and, of course, have contributions which sum to 24. The solutions to these conditions occur at $n=2m, n = 1 + 2m, n=2m -1$ $ \ (\forall m >0, 2m \in \Z_{\geq 0})$ and these values of $n$ do appear in the sum. The corresponding states therefore have quantum numbers $(n, m, l) = 20\times(j, j/2, j), 2\times(j + 1, j/2, j), 2\times(j-1, j/2, j), \ j \in \Z_{>0}$ and come from chiral primary states in the physical theory. Similarly, for $l = -2m$ we will obtain contributions from anti-chiral primary states \textcolor{blue}{finish}. 
%
%We can also discuss the origin of the terms of negative multiplicity, which contribute to the numerator of the index. Taking $f(m,l)$ as it is written in \eqref{eq:fml1} and studying $l=0, k \in \mathbb{Z}_{\geq 1}$, there is a cancellation between the $n=4k$ term, $\mathcal{C}(4k(k-1/2) + 1/2,-4k+1)=20$, and the rest of the terms, which sum to $-40$. Similar cancellations apply to $l=\pm 1, m = k-1/2, k \in \mathbb{Z}_{\geq 1}$, wherein the last terms of the sum are always $\mathcal{C}(4(k-1)^2, 4-4k) = 2, \mathcal{C}((1-2k)^2, 2-4k) = 2$, respectively, and the remainder of the terms produce coefficients summing to -4. Alternatively, one can use manipulations on Jacobi form coefficients to rewrite $f(m, l)$ as in \eqref{eq:fml2}:
%\begin{equation}
%f(m', l') =\left(\sum_{\tilde{n}= - \infty}^{\infty}\mathcal{C}(m'^2 - l'^2/4 + (\tilde{n} + 1)/2, \tilde{n} + 1) \right) - \mathcal{C}((l' + 1)/2, l' + 1)
%\end{equation} with $\tilde{n} = n-2m$ as before. The first term sums to zero when $l=0, m= k$ and for $l=\pm 1, m = k-1/2$, and the second gives the desired negative coefficients.
%
%
%\textcolor{blue}{finish discussion of quantum numbers of states contributing to the infinite-N index}

\subsection{The large $N$ limit}

\brian{Use LQT to give a first-principles description of the large N CFT. Discuss relationship to elliptic genus computed above.}

\end{document}