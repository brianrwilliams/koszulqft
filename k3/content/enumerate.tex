\documentclass[../main.tex]{subfiles}

\begin{document}

\section{Enumerating twisted supergravity states}

\subsection{Enumerating states in Kodaira--Spencer theory}

\brian{just review work of Costello and Gaiotto}

To describe the boundary conditions we will use the partial compactification of the extended deformed conifold as described in \S \ref{sec:compact}. 

Recall that there are three fundamental fields for Kodaira--Spencer theory.
Two fundamental fields $\alpha, \gamma$ are Dolbeault forms of type $(0,\bu)$.
The last fundamental field $\mu$ is a $(0,\bu)$ form valued in in the holomorphic tangent bundle.
We can use the Calabi--Yau form to view $\mu$ as a Dolbeault form of type $(2,\bu)$.

\begin{itemize}
\item The vacuum boundary condition for the fields $\alpha, \gamma$ is that each are divisible by the coordinate $n$. 
\item The vacuum boundary condition for the field $\mu$ is that when viewing it as a Dolbeault form of type $(2,\bu)$ it can be expressed as a sum of terms which are each wedge products of $\d \log n, \d w, \d z , \d \br n, \d \br w , \d \br z$ with coefficients that are regular at $n = 0$. 
\end{itemize} 

Denote by $\left(\mathbf{\frac{m}{2}}\right)_S$ the short representation of $\lie{psu}(1,1|2)$ whose highest weight vector has $(J_0^3,L_0)$ eigenvalues $(\frac{m}{2}, \frac{m}{2})$. 
Denote by $y$ the fugacity for the $U(1)$ symmetry $2J_0^3$ and $q$ the fugacity for the $U(1)$ symmetry $L_0$.
Let 
\beqn
D = (1-q)(1-q^{1/2} y)(1-q^{1/2}y^{-1}) .
\eeqn

Let's first consider the case where the internal manifold is just a point.
This is just the topological string with $B$-branes wrapping 
\beqn
\C \subset \C^3 .
\eeqn 

\begin{prop}
The single particle states for Kodaira--Spencer theory on $\C^3$ with branes wrapping $\C \subset \C^3$ decompose as
\beqn
\oplus_{m \geq 1} \left(\mathbf{\frac{m}{2}}\right)_S 
\eeqn
From this, we deduce that the single particle index for Kodaira--Spencer theory on $\C^3$ is
\beqn
\frac{q^2 - 3 q + q^{1/2}(y+y^{-1})}{D} .
\eeqn
\end{prop}

\begin{itemize} 
\item State $\mu \sim n^{-k} \d \log n \d w_1 \delta_{z=0}$.
For $k \geq 1$ these even states and their descendants contribute
\beqn
\frac{y q^{1/2}}{D} 
\eeqn
to the single particle index. 
\item Lowest lying state $\mu \sim n^{-k} \d \log n \d w_2 \delta_{z=0}$.
For $k \geq 1$ these even states and their descendants contribute
\beqn
\frac{y^{-1} q^{1/2}}{D} 
\eeqn
to the single particle index. 
\item Lowest lying state $\mu \sim n^{-k} \d \log n \d z \delta_{z=0}$.
For $k \geq 2$ these even states and their descendants contribute 
\beqn
\frac{q^2}{D} 
\eeqn
to the single particle index. 
\item State $\alpha \sim n^{1-k}\delta_{z=0}$. 
For $k \geq 1$ these odd states and their descendants contribute 
\beqn
- \frac{q}{D} .
\eeqn
to the single particle index.
\item State $\gamma \sim n^{1-k}\delta_{z=0}$. 
For $k \geq 1$ these odd states and their descendants contribute 
\beqn
- \frac{q}{D} .
\eeqn
to the single particle index.
\item \item State $\nu \sim n^{1-k}\delta_{z=0}$. 
For $k \geq 1$ these odd states contribute 
\beqn
- \frac{q}{D} .
\eeqn
to the single particle index.
\end{itemize}

In total we find that the multi particle gravitational index is 
\beqn
\frac{q^2 - 3q + q^{1/2} (y+y^{-1})}{(1-q)(1-q^{1/2} y)(1-q^{-1/2}y^{-1})} = \frac{y q^{1/2}}{1-y q^{1/2}} + \frac{y^{-1} q^{1/2}}{1-y^{-1}  q^{1/2}} - \frac{q}{1-q} .
\eeqn



%The vacuum boundary condition for the fields $\alpha$ requires that it is divisible by the coordinate $n$. 
%We can modify this at the point $z=0$ at the boundary by taking 
%\[
%\alpha = n^{-k} w^l \del_z^{r} \delta_{z=0} \otimes a .
%\]
%In this expression, $k \geq 0$ and $l \leq k$ to ensure that there are no poles along $w = \infty$. 
%Also, $a \in H^\bu(Y)$ denotes an arbitrary cohomology class for the $K3$ surface.
%The ansatz for $\gamma$ is identical.
%
%For the field $\mu$

\subsection{The twisted supergravity elliptic genus}

The supergravity states were enumerated in \cite{CPkoszul}. 
We briefly recall the results here. 

The twisted supergravity states organize into a representation for the super Lie algebra $\lie{psu}(1,1|2)$.
The bosonic factor of this super Lie algebra is $\lie{su}(2)_L \times \lie{su}(2)_R$. 
The first copy is the global conformal transformations in the $z$-plane and the second copy is the $R$-symmetry algebra which rotates the $w$-coordinate.
We take the Cartan of this Lie algebra to be generated by $(L_0, J_0^3)$. 
  
Denote by $(\frac{\bf m}{\bf 2})_S$ the short representation of $\lie{psu}(1,1|2)$ whose highest weight vector has $(L_0, J_0^3)$ eigenvalue $(m/2,m/2)$ \cite{dB1}. 
As an example, the short representation $({\bf 1})_S$ consists of a boson with weight $(L_0 = 1, J_0^3 = 1)$, which in our notation corresponds to 
\beqn
\mu \sim n^{-2} \d \log n \d z \delta_{z=0}  .
\eeqn 
There are also two fermions in $({\bf 1})_S$ with weights $(3/2,1/2)$ corresponding to the states
\beqn
\alpha \sim n^{-1} \delta_{z=0} + \cdots , \quad \gamma \sim n^{-1}\delta_{z=0} + \cdots
\eeqn
and another boson of weight $(2,0)$ corresponding to 
\beqn
\mu \sim n^{-2} \d \log n \d w \delta_{z=0} + \cdots .
\eeqn 

We consider twisted type IIB supergravity on a Calabi--Yau surface $X$, where $X$ could be $T^4$ or a $K3$ surface. 

\begin{prop}[\cite{CPkoszul}]
The supergravity states for the $D1-D5$ brane system in twisted type IIB supergravity on a compact Calabi--Yau surface $X$ decompose as
\beqn\label{eqn:IIBstates}
\bigoplus_{m \geq 1} (\frac{\bf m}{\bf 2})_S \otimes H^\bu(X) = \bigoplus_{m \geq 1} \bigoplus_{i,j} (\frac{\bf m}{\bf 2})_S \otimes H^{i,j} (X)  . 
\eeqn 
In particular, when $X$ is a $K3$ surface the single particle twisted supergravity index is 
\beqn\label{eqn:sugra_index}
f_{KS}(q,y) = 24 \frac{q^2 - 3 q + q^{1/2}(y+y^{-1})}{D} .
\eeqn
\end{prop} 

This result should be compared to \cite{dB1}, where the space of supergravity states upon supersymmetric localization (that is, the chiral half of the supergravity states) is found to be
\beqn\label{eqn:db1}
\bigoplus_{m \geq 0} \bigoplus_{i,j} (\frac{\bf m+i}{\bf 2})_S \otimes H^{i,j} (X) .
\eeqn
The answers agree in the range where the highest weight of the short representation is at least two. 
The low weight discrepancies break up into two types:
\begin{itemize}
\item In \cite{dB1} there is an extra factor of $({\bf 0})_S \otimes H^{0,i}(X)$. 
So, in the case that $X$ is a $K3$ surface there are two extra bosonic operators in the analysis of \cite{dB1}. 
In \cite{CP} is was pointed out that these are topological operators, annihilated by $L_{-1}$, and have nonsingular OPE with all remaining operators. 
\item 
In our analysis there is an extra factor of $(\frac{\bf 1}{\bf 2})_S \otimes H^{2,j}(X)$. 
In the case that $X$ is a $K3$ surface one can remove these two bosonic states while maintaining an $SO(21)$ symmetry. 
\end{itemize}

Denote the single particle index of the supergravity states, described in equation \eqref{eqn:db1}, by $f_{sugra}(q,y)$. 
One of the main results of \cite{dB1} is that the corresponding multiparticle index agrees with the large $N$ elliptic genus of the orbifold CFT of a $K3$ surface
\beqn
\chi_{NS}(\Sym^\infty X ; q,y) = {\rm PExp} \left[f_{sugra}(q,y) \right]
\eeqn
where ${\rm PExp}$ is the plethystic exponential.
For $X$ a $K3$ surface, the states $(\mathbf{\frac12})_S \otimes H^{2,\bu}(X)$ contribute the single particle index
\beqn
2 f_1 (q,y) = \frac{2}{1-q}\left(-2 q + q^{1/2}(y+y^{-1})\right) .
\eeqn
If we subtract this from the supergravity index we find an exact match with the supergravity index computed by \cite{dB1}
\beqn\label{eqn:sugraindex}
f_{sugra}(q,y) = f_{KS}(q,y) - 2 f_1(q,y) .
\eeqn

\subsection{Global symmetry algebra}

In this section we characterize the global symmetry algebra of the dual CFT from the point of view of the gravitational, or Kodaira--Spencer, theory following \cite{CP,CGhol}.
The global symmetry algebra is, by definition, a subalgebra of the modes of the CFT which preserve the vacuum at both $0$ and $\infty$.
Explicitly, if $\cO$ is an operator of spin $\Delta$, then the modes
\beqn
\oint z^m \cO(z) \d z
\eeqn
for $0 \leq m \leq 2 \Delta - 2$ close as an algebra and preserve the vacua at $0,\infty$.
Generally, the global symmetry algebra is a subalgebra of the mode algebra of the vertex algebra.
For us, it can be expressed as the universal enveloping algebra of a particular Lie superalgebra.

Following a similar argument as in \cite{CP}, one finds that the global symmetry algebra is the enveloping algebra of a Lie superalgebra of the form
\beqn
\op{Vect}_0 \left(X^0 \slash \Spec(A)\right) \oplus \cO(X^0) \otimes \Pi \C^2 ,
\eeqn
where:
\begin{itemize}
\item $X^0$ is the extended conifold defined as a family over $\Spec(A)$ where we have removed the singular locus, see section \ref{sec:conifold}. 
\item $\cO(X^0)$ denotes the algebra of holomorphic functions on $X^0$.
By Hartog's theorem this is the algebra generated by the bosonic linear functions $u_i, w_j, \eta, \br \eta, \eta_a$ where $i,j=1,2$, $a=1,\ldots, 20$ subject to the relations
\[
\eta^2 = \br \eta^2 = \eta_a \eta_b - h_{ab} \eta \br \eta = 0, \qquad \eps^{ij} u_i w_j = F . 
\]
\item $\op{Vect}_0\left(X^0 \slash \Spec(A)\right)$ is the Lie algebra of divergence-free holomorphic vector fields which point in the direction of the fibers of $X^0 \to \Spec(A)$ (those holomorphic vector fields preserving the holomorphic volume form on the fibers).
\item $\Pi(-)$ denotes parity shift, so that this is a Lie superalgebra.
\item The nontrivial Lie brackets (and anti-brackets) are:
\beqn
\begin{aligned} 
\,[V, V'] & = \text{commutator of vector fields} \\
[V,f] & = V(f) \\
[f_i, g_j] & = \eps_{ij} \Omega^{-1} \left( \del f_i \wedge \del g_j\right) 
\end{aligned}
\eeqn
where $V \in \left(X^0 \slash \Spec(A)\right)$, $f_i,g_j \in \cO(X^0) \otimes \Pi \C^2$.
\end{itemize}

This characterization of the global symmetry algebra will follow from our computation of the (tree level) OPE's of the boundary CFT.
In the remainder of this section we highlight some important properties of this global symmetry algebra.

Following the exact same argument as in section 5.1 of \cite{CP}, one observes that $\lie{psl}(2|2)$ is embedded as a finite-dimensional subalgebra of the global symmetry algebra.
The even part $\lie{sl}(2) \oplus \lie{sl}(2)$ consists of the complex $4 \times 4$ matrices which preserve the quadratic form $u_1 w_2 - u_2 w_1$; every such matrix gives rise to a vector field on $X^0$ which points along the fibers of $X^0 \to \Spec A$ and preserves the volume form.
The eight odd elements arise from the following linear holomorphic functions 
\beqn
(u_i,0), (w_j,0), (0,u_i), (0,w_j), \quad i,j=1,2
\eeqn
thought of as elements of $\cO(X^0) \otimes \Pi \C^2$.
Notice that this $\lie{psl}(2|2)$ does not depend on the cohomology of the K3 surface.

As $\lie{sl}(2) \oplus \lie{sl}(2)$ representations one has
\beqn
\op{Vect}_0 \left(X^0 \slash \Spec(A)\right) \cong \op{Vect}_0(Z) \otimes A \\
\eeqn
and
\beqn
\cO(X^0) \cong \cO(Z) \otimes A
\eeqn
where $Z$ is the ordinary (bosonic) conifold $u_1 w_2 - u_2 w_1 = 0$.


\end{document}