\documentclass[../main.tex]{subfiles}

\begin{document}

\section{Enumerating twisted supergravity states}

We have derived our twisted supergravity theory in the backreacted geometry; we will refer to the latter henceforth as the K3 conifold, adapting the terminology of \cite{CP}. Our theory conjecturally captures a protected subsector of IIB supergravity on AdS$_3 \times S^3 \times K3$ (which we will refer to as the untwisted theory), and we would like to perform some sanity checks of this conjecture. In particular, in this section we demonstrate that the partition function of twisted supergravity states reproduces the seminal count of 1/4-BPS KK modes in the untwisted theory \cite{deBoerSUGRA}. The methods in this section are a slight modification of those in \cite{CostelloGaiotto, CP}. 

\subsection{Inclusion of boundary divisors}\label{sec:compact}

In order to enumerate twisted supergravity states, we must understand the boundary divisors of the K3 conifold, which are the geometric support for the asymptotic scattering states that participate in (the holomorphic analogue of) Witten diagram computations\footnote{While we will not study bulk scattering directly in this work, it would be interesting to explore methods to make such bulk computations more efficient, perhaps by generalizing the technology of \cite{Budzik:2022mpd, Budzik:2023xbr} to curved backgrounds.}. 

The idea is to compactify the K3 conifold $X^0$ to a super-projective variety $\overline{X^0}$ inside $\mathbb{CP}^4 \times \textrm{Spec}(A).$\footnote{Note that other compactifications are possible, depending on one's application. In \cite{Burns}, the deformed conifold $SL(2, \mathbb{C})$ was not compactified to a quadric, as here, but instead was compactified inside the blow up of a flag variety. That compactification was the one compatible with the symmetries inherent from viewing the deformed conifold as the twistor space of 4d Burns space, which has isometry group $SU(2) \times U(1)$. It would be interesting to extend the analysis of \cite{Burns} to the conifolds of \cite{CP} and the present article, and view them as twistor spaces in turn.}. We give the $\mathbb{CP}^4$ homogeneous coordinates $U_i, W_i, Z$, so that we can complete the K3 conifold defined by equation \ref{eq:k3conifold} to 
\begin{equation}
\epsilon^{ij}U_i W_j = F Z^2.
\end{equation}
The boundary is then at $Z=0$, given by $\epsilon^{ij}U_i W_j = 0$, which is the variety $\mathbb{CP}^1 \times \mathbb{CP}^1 \times \textrm{Spec}(A)$ \subset \mathbb{CP}^3 \times \textrm{Spec}(A)$. As in \cite{CostelloGaiotto}, the two $\mathbb{CP}^1$'s may be understood, respectively, as the 2-sphere boundary of AdS$_3$, and the $S^2$ base of the $S^3$ factor, viewed as a Hopf fibration. Each $\mathbb{CP}^1$ is naturally acted on by a copy of SL$_2$.

To determine the complex structure in the neighborhood of the boundary, we must find coordinates which are holomorphic in the deformed geometry, as described in the previous section. To start, we can endow the two $\mathbb{CP}^1$'s with holomorphic coordinates $w, z$ and anti-holomorphic coordinates $\bar{w}, \bar{z}$ (in addition to the coordinates $\eta$ on $\textrm{Spec}(A)$), and take the $\mathbb{CP}^1$ with coordinates $z, \bar{z}$ to be the boundary of AdS$_3$ on which the dual twisted SCFT will live. In addition, we can specify a coordinate normal to the two boundary spheres by $n$, which has a simple pole at $z=\infty$ and at $w = \infty$. We need to specify the behavior of Kodaira-Spencer fields at $n=0$, where the complement of $n=0$ is the uncompactified K3 conifold. In these coordinates, the holomorphic volume form is 
\begin{equation}
\Omega = -{dn dw dz \over n^3} + {F \over n}{dn dw d\bar{w} \over (1 + |w|^2)^2}.
\end{equation} With these coordinates, one can straightforwardly define twisted supergravity states via the usual AdS/CFT extrapolate dictionary. 

However, this naive coordinate system is not holomorphic. Rather, the complex structure is deformed by the Beltrami differential
\begin{equation}
F n^2 d\bar{w}{1 \over (1 + |w|^2)^2}\partial_z
\end{equation}
Holomorphic functions in the neighborhood of the boundary are given by 
\begin{align}
w_1&:= {1 \over n} \\
w_2&:={w \over n} \\
u_1&:={z \over n} - F n {\bar{w} \over (1 + |w|^2)^2} \\
u_2&:={w z \over n} + F n {1 \over (1 + |w|^2)^2}.
\end{align}
Notice that these coordinates have poles at $n=0$ and satisfy $u_2 w_1 - u_1 w_2 = F$. Moreover, in these coordinates the holomorphic volume form again takes the canonical form
\begin{equation}
\Omega = {du_1 dw_1 dw_2 \over w_1}.
\end{equation}




\subsection{Enumerating states in Kodaira--Spencer theory}

To describe boundary conditions on the fields in our theory, we can use the partial compactification of the K3 conifold described in \S \ref{sec:compact}. All that remains is, following the usual AdS/CFT prescription, to specify vacuum boundary conditions for our Kodaira-Spencer supergravity fields. Then, our twisted supergravity states are solutions to the equation of motion that satisfy these vacuum boundary conditions except at a point on the conformal boundary of the AdS$_3$ factor, say $z_*$. In other words, twisted supergravity states are, as usual, local modifications of the boundary conditions, which are equivalent to boundary operators placed along  $\mathbb{CP}^1_w \times \left\lbrace z_* \right\rbrace$. 

Recall that there are three fundamental fields for Kodaira--Spencer theory.
Two fundamental fields $\alpha, \gamma$ are Dolbeault forms of type $(0,\bu)$.
The last fundamental field $\mu$ is a $(0,\bu)$ form valued in in the holomorphic tangent bundle.
We can use the Calabi--Yau form to view $\mu$ as a Dolbeault form of type $(2,\bu)$.

\begin{itemize}
\item The vacuum boundary condition for the fields $\alpha, \gamma$ is that each are divisible by the coordinate $n$. That is, we require these fields to vanish on the boundary divisor.
\item The vacuum boundary condition for the field $\mu$ is that, when viewing it as a Dolbeault form of type $(2,\bu)$, it can be expressed as a sum of terms which are each wedge products of $\d \log n, \d w, \d z , \d \br n, \d \br w , \d \br z$ with coefficients that are regular at $n = 0$. (Notice that we allow this field to have logarithmic poles on the boundary divisor, although one may also choose to impose the more restrictive condition that $\mu$ is a regular Dolbeault form). 
\end{itemize} 

We can now enumerate the supergravity states that satisfy these boundary conditions except for at a point-localized disturbance or source. The computation directly parallels \cite{CostelloGaiotto, CP}, to which we refer the reader for more details. 

Denote by $\left(\mathbf{\frac{m}{2}}\right)_S$ the short representation of $\lie{psu}(1,1|2)$ whose highest weight vector has $(J_0^3,L_0)$ eigenvalues $(\frac{m}{2}, \frac{m}{2})$. 
Denote by $y$ the fugacity for the $U(1)$ symmetry $2J_0^3$ and $q$ the fugacity for the $U(1)$ symmetry $L_0$.
Let \natalie{This looks a little different than the expressions Im used to (see my paper with Kevin for the characters as I know them)}
\beqn
D = (1-q)(1-q^{1/2} y)(1-q^{1/2}y^{-1}) .
\eeqn

Let's first consider the case where the internal manifold is just a point.
This is just the topological string with $B$-branes wrapping 
\beqn
\C \subset \C^3 .
\eeqn 
We can give the complex plane holomorphic coordinate $z$ and the other planes holomorphic coordinates  $w_1, w_2$. \natalie{was this the notation you intended? But then why is n needed?}
\begin{prop}
The single particle states for Kodaira--Spencer theory on $\C^3$ with branes wrapping $\C \subset \C^3$ decompose as
\beqn
\oplus_{m \geq 1} \left(\mathbf{\frac{m}{2}}\right)_S 
\eeqn
From this, we deduce that the single particle index for Kodaira--Spencer theory on $\C^3$ is
\beqn
\frac{q^2 - 3 q + q^{1/2}(y+y^{-1})}{D} .
\eeqn
\end{prop}

\begin{itemize} 
\item State $\mu \sim n^{-k} \d \log n \d w_1 \delta_{z=0}$.
For $k \geq 1$ these even states and their descendants contribute
\beqn
\frac{y q^{1/2}}{D} 
\eeqn
to the single particle index. 
\item Lowest lying state $\mu \sim n^{-k} \d \log n \d w_2 \delta_{z=0}$.
For $k \geq 1$ these even states and their descendants contribute
\beqn
\frac{y^{-1} q^{1/2}}{D} 
\eeqn
to the single particle index. 
\item Lowest lying state $\mu \sim n^{-k} \d \log n \d z \delta_{z=0}$.
For $k \geq 2$ these even states and their descendants contribute 
\beqn
\frac{q^2}{D} 
\eeqn
to the single particle index. 
\item State $\alpha \sim n^{1-k}\delta_{z=0}$. 
For $k \geq 1$ these odd states and their descendants contribute 
\beqn
- \frac{q}{D} .
\eeqn
to the single particle index.
\item State $\gamma \sim n^{1-k}\delta_{z=0}$. 
For $k \geq 1$ these odd states and their descendants contribute 
\beqn
- \frac{q}{D} .
\eeqn
to the single particle index.
\item State $\nu \sim n^{1-k}\delta_{z=0}$. 
For $k \geq 1$ these odd states contribute 
\beqn
- \frac{q}{D} .
\eeqn
to the single particle index.
\end{itemize}

In total we find that the single-particle gravitational index is 
\beqn
\frac{q^2 - 3q + q^{1/2} (y+y^{-1})}{(1-q)(1-q^{1/2} y)(1-q^{-1/2}y^{-1})} = \frac{y q^{1/2}}{1-y q^{1/2}} + \frac{y^{-1} q^{1/2}}{1-y^{-1}  q^{1/2}} - \frac{q}{1-q} .
\eeqn



%The vacuum boundary condition for the fields $\alpha$ requires that it is divisible by the coordinate $n$. 
%We can modify this at the point $z=0$ at the boundary by taking 
%\[
%\alpha = n^{-k} w^l \del_z^{r} \delta_{z=0} \otimes a .
%\]
%In this expression, $k \geq 0$ and $l \leq k$ to ensure that there are no poles along $w = \infty$. 
%Also, $a \in H^\bu(Y)$ denotes an arbitrary cohomology class for the $K3$ surface.
%The ansatz for $\gamma$ is identical.
%
%For the field $\mu$

\subsection{The twisted supergravity elliptic genus}

The supergravity states were enumerated in \cite{CP}. We briefly recall the results here. 

The twisted supergravity states organize into a representation for the super Lie algebra $\lie{psu}(1,1|2)$.
The bosonic factor of this super Lie algebra is $\lie{su}(2)_L \times \lie{su}(2)_R$. 
The first copy is the global conformal transformations in the $z$-plane and the second copy is the $R$-symmetry algebra which rotates the $w$-coordinate.
We take the Cartan of this Lie algebra to be generated by $(L_0, J_0^3)$. 
  
Denote by $(\frac{\bf m}{\bf 2})_S$ the short representation of $\lie{psu}(1,1|2)$ whose highest weight vector has $(L_0, J_0^3)$ eigenvalue $(m/2,m/2)$ \cite{deBoerEG}. 
As an example, the short representation $({\bf 1})_S$ consists of a boson with weight $(L_0 = 1, J_0^3 = 1)$, which in our notation corresponds to 
\beqn
\mu \sim n^{-2} \d \log n \d z \delta_{z=0}  .
\eeqn 
There are also two fermions in $({\bf 1})_S$ with weights $(3/2,1/2)$ corresponding to the states
\beqn
\alpha \sim n^{-1} \delta_{z=0} + \cdots , \quad \gamma \sim n^{-1}\delta_{z=0} + \cdots
\eeqn
and another boson of weight $(2,0)$ corresponding to 
\beqn
\mu \sim n^{-2} \d \log n \d w \delta_{z=0} + \cdots .
\eeqn 

Here, the ellipses denote additional terms required to express the fields in the holomorphic coordinates of the deformed geometry (see \cite{CP} for the complete expressions in the $T^4$ case). In particular, only a finite number of terms are required to correct the holomorphicity of these expressions, due to the fact that the relations imposed on the coordinates of $\textrm{Spec}(A)$ cause the expansions in the $\eta$'s to truncate. 

We consider twisted type IIB supergravity on a Calabi--Yau surface $X$, where $X$ could be $T^4$ or a $K3$ surface. 

\begin{prop}[\cite{CP}]
The supergravity states for the D1-D5 brane system in twisted type IIB supergravity on a compact Calabi--Yau surface $X$ decompose as
\beqn\label{eqn:IIBstates}
\bigoplus_{m \geq 1} (\frac{\bf m}{\bf 2})_S \otimes H^\bu(X) = \bigoplus_{m \geq 1} \bigoplus_{i,j} (\frac{\bf m}{\bf 2})_S \otimes H^{i,j} (X)  . 
\eeqn 
In particular, when $X$ is a $K3$ surface the single particle twisted supergravity index is 
\beqn\label{eqn:sugra_index}
f_{KS}(q,y) = 24 \frac{q^2 - 3 q + q^{1/2}(y+y^{-1})}{D} .
\eeqn
\end{prop} 

This result should be compared to \cite{deBoerEG}, where the space of supergravity states upon supersymmetric localization (that is, the chiral half of the supergravity states) is found to be
\beqn\label{eqn:db1}
\bigoplus_{m \geq 0} \bigoplus_{i,j} (\frac{\bf m+i}{\bf 2})_S \otimes H^{i,j} (X) .
\eeqn
The answers agree in the range where the highest weight of the short representation is at least two. 
The low weight discrepancies break up into two types:
\begin{itemize}
\item In \cite{deBoerEG} there is an extra factor of $({\bf 0})_S \otimes H^{0,i}(X)$. 
So, in the case that $X$ is a $K3$ surface there are two extra bosonic operators in the analysis of \cite{deBoerEG}. 
In \cite{CP} it was pointed out that these are topological operators, annihilated by $L_{-1}$, and have nonsingular OPE with all remaining operators. \\
Notice that these states are removed by hand from the infinite-$N$ ${\rm Sym}^N(K3)$ elliptic genus in \cite{deBoerEG} (as we will review below), because their degeneracy scales with $N$. Though they naturally appear on the SCFT side, and in particular are well-defined for any finite $N$, the Kodaira-Spencer theory does not see them. 
\item 
In our analysis there is an extra factor of $(\frac{\bf 1}{\bf 2})_S \otimes H^{2,j}(X)$. 
In the case that $X$ is a $K3$ surface one can remove these two bosonic states while maintaining an $SO(21)$ symmetry. \natalie{Can we say any more about this? Are they also topological or something? Does the BV formalism shed any light?}
\end{itemize}

Denote the single particle index of the supergravity states, described in equation \eqref{eqn:db1}, by $f_{sugra}(q,y)$. 
One of the main results of \cite{deBoerEG} is that the corresponding multiparticle index agrees with the large $N$ elliptic genus of the orbifold CFT of a $K3$ surface
\beqn
\chi_{NS}(\Sym^\infty X ; q,y) = {\rm PExp} \left[f_{sugra}(q,y) \right]
\eeqn
where ${\rm PExp}$ is the plethystic exponential defined by ${\rm PExp}\left[f(x) \right] = {\rm exp}\left(\sum_{k=1}^{\infty}\right {f(x^k) \over k})$, which effects a ``multi-particling'' operation.
For $X$ a $K3$ surface, the states $(\mathbf{\frac12})_S \otimes H^{2,\bu}(X)$ contribute the single particle index
\beqn
2 f_1 (q,y) = \frac{2}{1-q}\left(-2 q + q^{1/2}(y+y^{-1})\right) .
\eeqn
If we subtract this from the supergravity index we find an exact match with the supergravity index computed by \cite{deBoerEG}:
\beqn\label{eqn:sugraindex}
f_{sugra}(q,y) = f_{KS}(q,y) - 2 f_1(q,y) .
\eeqn

\subsection{Global symmetry algebra}

In this section we characterize the global symmetry algebra of the dual CFT at infinite $N$ from the point of view of the gravitational, or Kodaira--Spencer, theory following \cite{CP,CostelloGaiotto}. 
The global symmetry algebra is, by definition, a subalgebra of the modes of the operators\footnote{Again, we work with operators that survive in the planar limit; in the gauge theory context, these would be the single trace operators.} of the CFT which preserve the vacuum at both $0$ and $\infty$. 
Explicitly, if $\cO$ is an operator of spin $\Delta$, then the modes
\beqn
\oint z^m \cO(z) \d z
\eeqn
for $0 \leq m \leq 2 \Delta - 2$ close as an algebra and preserve the vacua at $0,\infty$.
Generally, the global symmetry algebra is a subalgebra of the mode algebra of the vertex algebra.
For us, it can be expressed as the universal enveloping algebra of a particular Lie superalgebra.

From the Kodaira-Spencer theory perspective, these are infinitesimal gauge symmetries which preserve the vacuum solutions to the equations of motion on the K3 conifold. 
Following a similar argument as in \cite{CP}, one finds that the global symmetry algebra is the enveloping algebra of a Lie superalgebra of the form
\beqn
\op{Vect}_0 \left(X^0 \slash \Spec(A)\right) \oplus \cO(X^0) \otimes \Pi \C^2 ,
\eeqn
where:
\begin{itemize}
\item $X^0$ is the extended conifold defined as a family over $\Spec(A)$ where we have removed the singular locus, (i.e. the K3 conifold); see section \ref{sec:conifold}. 
\item $\cO(X^0)$ denotes the algebra of holomorphic functions on $X^0$.
By Hartog's theorem this is the algebra generated by the bosonic linear functions $u_i, w_j, \eta, \br \eta, \eta_a$ where $i,j=1,2$, $a=1,\ldots, 20$ subject to the relations
\[
\eta^2 = \br \eta^2 = \eta_a \eta_b - h_{ab} \eta \br \eta = 0, \qquad \eps^{ij} u_i w_j = F . 
\]
\item $\op{Vect}_0\left(X^0 \slash \Spec(A)\right)$ is the Lie algebra of divergence-free holomorphic vector fields which point in the direction of the fibers of $X^0 \to \Spec(A)$ (those holomorphic vector fields preserving the holomorphic volume form on the fibers).
\item $\Pi(-)$ denotes parity shift, so that this is a Lie superalgebra.
\item The nontrivial Lie brackets (and anti-brackets) are:
\beqn
\begin{aligned} 
\,[V, V'] & = \text{commutator of vector fields} \\
[V,f] & = V(f) \\
[f_i, g_j] & = \eps_{ij} \Omega^{-1} \left( \del f_i \wedge \del g_j\right) 
\end{aligned}
\eeqn
where $V \in \left(X^0 \slash \Spec(A)\right)$, $f_i,g_j \in \cO(X^0) \otimes \Pi \C^2$.
\end{itemize}

This characterization of the global symmetry algebra will follow from our computation of the (tree level) OPE's of the boundary CFT. As in the examples of \cite{CostelloGaiotto, CP}, this global symmetry algebra is large enough to fix the \textit{planar} 2 and 3-point functions\footnote{In fact, at infinite$-N$ the 2-point functions in \cite{CP} were shown to vanish; the same argument goes through in this case.}.

In the remainder of this section we highlight some important properties of this global symmetry algebra.

Following the exact same argument as in section 5.1 of \cite{CP}, one observes that $\lie{psl}(2|2)$ is embedded as a finite-dimensional subalgebra of the global symmetry algebra.
The even part $\lie{sl}(2) \oplus \lie{sl}(2)$ consists of the complex $4 \times 4$ matrices which preserve the quadratic form $u_1 w_2 - u_2 w_1$; every such matrix gives rise to a vector field on $X^0$ which points along the fibers of $X^0 \to \Spec A$ and preserves the volume form.
The eight odd elements arise from the following linear holomorphic functions 
\beqn
(u_i,0), (w_j,0), (0,u_i), (0,w_j), \quad i,j=1,2
\eeqn
thought of as elements of $\cO(X^0) \otimes \Pi \C^2$.
Notice that this $\lie{psl}(2|2)$ does not depend on the cohomology of the K3 surface.

As $\lie{sl}(2) \oplus \lie{sl}(2)$ representations there are isomorphisms
\beqn
\op{Vect}_0 \left(X^0 \slash \Spec(A)\right) \cong \op{Vect}_0(Z) \otimes A \\
\eeqn
and
\beqn
\cO(X^0) \cong \cO(Z) \otimes A
\eeqn
where $Z$ is the ordinary (bosonic) conifold $u_1 w_2 - u_2 w_1 = 0$.


\end{document}