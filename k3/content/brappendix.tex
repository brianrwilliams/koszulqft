%\documentclass[a4paper,11pt]{article}
%\pdfoutput=1 

\documentclass[../main.tex]{subfiles}

%\usepackage{jheppub} 
%\usepackage[T1]{fontenc} 
%\usepackage{caption}
%\usepackage{subcaption}
%\usepackage{hyperref}
%\usepackage{revsymb}
%\usepackage{mwe}
%\usepackage{simpler-wick}

\newcommand{\cI}{\mathcal{I}}
\begin{document}

\appendix

\section{Loop computations involving backreaction}

\subsection{Backreaction in holomorphic Chern--Simons}
\label{appx:hcsbr}

Let $X = (z,x) = (z, x_1,x_2), Y = (w,y) = (w,y_1,y_2)$.
We compute the integral
\beqn
\int_{(X,Y) \in \C^3_1 \times \C^3_2} A_1(X) \, \omega (x) \,  \del_z \del_w P(X,Y) \, \omega(y) \, A_2(Y) ,
\eeqn
where $A_i$ are $(0,1)$-forms on $\C^3$, and $P(X,Y) = P(X-Y)$ is as in equation \eqref{eqn:propagatorCS}.
Plugging in $A = x_1 \d \br z$ and $B = y_2 \d \br w$ this integral becomes $\int_{z,w} \d z \, \d w \,  \del_z \del_w I(z,w)$ where
\beqn\label{eq:integral1}
I (z,w) \define (\br z - \br w)\int_{\C^2 \times \C^2} \d^4 x \d^4 y \frac{[\br x \br y] x_1 y_2}{\|x\|^4 (|z-w|^2 + \|x-y\|^2)^3 \|y\|^4} .
\eeqn
We compute $I(z,w)$ as a function of the difference $z-w$. Note that there is an additional factor over ${1 \over (2 \pi)^4}$ arising from the propagator and $\omega$ which we have suppressed, and will restore at the end. 

First, we perform the integration along $y \in \C^2$.
Using Feynman's trick we have
\begin{multline}
\int_{\C^2} \d^4 y \frac{[\br x \br y] y_2}{(|z-w|^2 + \|x-y\|^2)^3 \|y\|^4} \\ = \frac{4!}{2!} \int_0^1 \d t \, t^2 (1-t) \int_{\C^2} \d^4 y \frac{[\br x \br y] y_2}{\left(t |z-w|^2 + t \|x-y\|^2 + (1-t)\|y\|^2 \right)^5} .
\end{multline}
Introduce the new variable $\til y = y - t x$.
The the right hand side becomes
\beqn
12 \int_0^1 \d t \, t^2 (1-t) \int_{\C^2} \d^4 \til y \frac{[\br x (\br {\til y} + t \br x)] (y_2+t x_2)}{\left(\|\til y\|^2 + t(1-t)\|x\|^2 + t |z-w|^2 \right)^5} 
\eeqn
Changing to polar coordinates and first computing the residue we see that only terms invariant under $U(1) \times U(1)$ rotations of $\C^2$ will contribute to this integral.
The $U(1) \times U(1)$ invariant part of the numerator is $\br x_1 |\til y_2|^2$.
After computing the residue along both the $\til y_1$ and $\til y_2$ directions the integral then becomes
\beqn
12 (-2 \pi \im)^2 \br x_1 \int_{0}^1 \d t \, t^2 (1-t) \int_{(0,\infty) \times (0,\infty)} \d^2 \rho \,\frac{\rho_2}{(\rho_1 + \rho_2 + t(1-t)\|x\|^2 + t |z-w|^2)^5} .
\eeqn

Performing the integration over $(0,\infty) \times (0 , \infty)$ we obtain 
\beqn
\frac{(-2 \pi \im)^2}{2} \br x_1 \int_0^1 \frac{1-t}{(|z-w|^2 + (1-t)\|x\|^2)^2} %= \frac{(-2 \pi \im)^2}{2} \br x_1 \left(- \frac{1}{\|x\|^2 (|z-w|^2 + \|x\|^2)} + \cdots\right) .
\eeqn

Returning to the original integral we must now compute
\beqn
\int_{0}^1 \d t \, (1-t) \int_{\C^2} \d^4 x \, \frac{|x_1|^2}{\|x\|^4 (|z-w|^2 + (1-t)\|x\|^2)^2} ,
\eeqn
We compute the integral over $x$.

Using the Feynman trick again we have
\begin{multline}
\int_{\C^2} \d^4 x \, \frac{|x_1|^2}{\|x\|^4 (|z-w|^2 + (1-t)\|x\|^2)^2} \\ = \int_0^1 \d s \, s (1-s)
\int_{\C^2} \d^4 x \, \frac{|x_1|^2}{\left(s |z-w|^2 + (1-ts)\|x\|^2\right)^4} .
\end{multline}
After computing the angular integrations this becomes
\beqn
(-2 \pi \im)^2 \int_{(0,\infty) \times (0,\infty)} \d^2 \rho \frac{\rho_1}{(s|z-w|^2 + (1-ts)(\rho_1 + \rho_2))^4} \\ = \frac{1}{s(1-ts)^3 |z-w|^2}   
\eeqn

Finally, plugging back into the original expression we have
\beqn
I(z,w) = \frac{(- 2 \pi i)^2}{z-w} \int_{0}^1 \d t \int_0^1 \d s \frac{(1-t)(1-s)}{(1-ts)^3} .
\eeqn
The integral over $t,s$ gives ${1 \over 2}$. Combining all the resulting factors from the preceding computations, and reinstating the propagator normalization, we therefore have
\beqn
I(z, w) = {(- 2 \pi i)^4 \over 2}{1 \over (2 \pi)^4}{1 \over 2 (z- w)} = {1 \over 4(z-w)}.
\eeqn
%If $\til Y = (tw - tz, y - tx)$ then
%\beqn
%\|\til Y\|^2 + t (1-t) (|z-w|^2 + \|x\|^2) = t \|X - Y\|^2 + (1-t) \|y\|^2 .
%\eeqn
%In terms of the coordinate $\til y$, the invariant piece of 
%\beqn
%[\br x \br y] y_2
%\eeqn
%is $\br x_1 |\til y_2|^2$.

\subsection{The central term in Kodaira--Spencer theory}\label{appx:ksbr}

Let the notation for the coordinates $X,Y$ be as in the last section.
We will compute the integral
\beqn
\int_{X, Y} \mu_1(X) \, \mu_{BR} (x) \,  \bP (X,Y) \, \mu_{BR}(y) \, \mu_2(Y) .
\eeqn
Without loss of generality, we plug in the test functions
\beqn
\mu_1 (X) = x_1 \del_{x_1} \d \zbar, \quad \mu_2 (Y) = y_2 \del_{y_2} \d \wbar .
\eeqn
The vector field type is determined by the symmetry of the graph while the powers of the holomorphic coordinates $x,y$ which appear are determined by the scaling properties of the propagator and backreaction.

Notice that $\mu_{BR}(x)$ is proportional to the differential form $\ep_{ij} \br x_i  \d \br x_j$ and similarly for $\mu_{BR}(y)$.
Thus, for these test functions only the $\del_{x_1-y_1} \del_{x_2-y_2}$ part of the BCOV propagator $\bP(X,Y)$ will contribute to this integral.
Furthermore, the terms in the BCOV propagator proportional to $\d\zbar - \d \wbar$ will not contribute by type reasons.
Simplifying, we see that for this choice of test functions this integral becomes $\int_{z,w} \d z \, \d w \, I(z,w)$ where
\beqn\label{eq:integral2}
I (z,w) \define (\br z - \br w)^2 \int_{\C^2_x \times \C^2_y} \d^4 x \d^4 y \frac{[\br x \br y] x_1 y_2}{\|x\|^4 (|z-w|^2 + \|x-y\|^2)^4 \|y\|^4} 
\eeqn
where we have again suppressed the constant factors from the propagator and $\omega$, to be restored at the end. 
We remark that the factor $(\zbar-\wbar)^2$ comes from the BCOV propagator.
We compute $I(z,w)$ as a function of the difference $z-w$.

First, we perform the integration along $y \in \C^2$.
Using Feynman's trick we have
\begin{multline}
\int_{\C^2} \d^4 y \frac{[\br x \br y] y_2}{(|z-w|^2 + \|x-y\|^2)^4 \|y\|^4} \\ = {5! \over 3!} \int_0^1 \d t \, t^3 (1-t) \int_{\C^2} \d^4 y \frac{[\br x \br y] y_2}{\left(t |z-w|^2 + t \|x-y\|^2 + (1-t)\|y\|^2 \right)^6} .
\end{multline}
Introduce the new variable $\til y = y - t x$.
The the right hand side becomes
\beqn
20 \int_0^1 \d t \, t^3 (1-t) \int_{\C^2} \d^4 \til y \frac{[\br x (\br {\til y} + t \br x)] (y_2+t x_2)}{\left(\|\til y\|^2 + t(1-t)\|x\|^2 + t |z-w|^2 \right)^6} 
\eeqn
The $U(1) \times U(1)$ invariant part of the numerator is $\br x_1 |\til y_2|^2$.
After computing the residue along both the $\til y_1$ and $\til y_2$ directions the integral becomes
\beqn
20 (- 2 \pi i)^2\br x_1 \int_{0}^1 \d t \, t^3 (1-t) \int_{(0,\infty) \times (0,\infty)} \d^2 \rho \,\frac{\rho_2}{(\rho_1 + \rho_2 + t(1-t)\|x\|^2 + t |z-w|^2)^6} .
\eeqn
Performing the integration over $(0,\infty) \times (0 , \infty)$ we obtain 
\beqn
(-2 \pi i)^2 {5! \over 3!}{2! \over 5!} \br x_1 \int_0^1 \frac{1-t}{(|z-w|^2 + (1-t)\|x\|^2)^3} %= \frac{(-2 \pi \im)^2}{2} \br x_1 \left(- \frac{1}{\|x\|^2 (|z-w|^2 + \|x\|^2)} + \cdots\right) .
\eeqn

Returning to the original integral we must now compute (suppressing the overall constant factors for the moment)
\beqn
\int_{0}^1 \d t \, (1-t) \int_{\C^2} \d^4 x \, \frac{|x_1|^2}{\|x\|^4 (|z-w|^2 + (1-t)\|x\|^2)^3} ,
\eeqn
We compute the integral over $x$ as above to obtain
\beqn
(- 2 \pi i)^4{4 \over 4!}\frac{1}{|z-w|^4} \int_{0}^1 \d t \int_{0}^1 \d s \, \frac{(1-t)(1-s)}{(1-ts)^3}  
\eeqn
and hence, putting all the pieces together,
\beqn
I(z,w) = {3 \over (2 \pi )^4}{(- 2 \pi i)^4 \over 6}\frac{1}{2 (z-w)^2} = {1 \over 4(z- w)^2}.
\eeqn
Upon changing to the basis of on-shell generators (i.e. currents sourcing the properly constrained Kodaira-Spencer fields), we will recover precisely the canonical Kac-Moody algebra at the expected level ${ N \over 2}$.

\subsection{Evaluating more general holomorphic integrals}\label{appx:integral}

In the previous two appendices, we computed some holomorphic integrals which can deform a Koszul dual chiral algebra on a case-by-case basis. However, these integrals admit more general closed forms, and it is convenient to calculate them once and for all. 

For later convenience, in this appendix we will evaluate several holomorphic integrals, which are common to 1-loop Koszul duality computations in holomorphic theories; in the next appendix, we will specialize the general form we derive here to compute the remaining box diagrams which deforms our planar chiral algebra. 

We will display several different presentations of the required integral, which differ in whether the defect coordinate of a bulk vertex is constrained to coincide with a defect operator location (as when we have a bulk to boundary propagator), or not (as when we have backreaction legs). Throughout this appendix, we employ the same notation as in \S \ref{sec:br}.

\subsubsection{Presentation 1: Integral over $d^4x d^4 y$} 

We would like to obtain an expression of the form $\int dz dw I(z, w)$, where $I(z, w)$ is itself an integral over the four transverse directions $d^4x d^4y$. For notational expedience, let us strip off some overall factors which do not partake in the $d^4x d^4 y$ integral, in particular: any functions of $\bar{z}, \bar{w}$ which come from expanding the propagators, and any overall multiplicative constants which come from the normalizations of the propagators and the backreaction fields. We call this stripped-down integral $\mathcal{I}^1(z, w)$, and turn to its evaluation. (Of course, one must reinstate these factors at the end, and then perform the final integral over $dz dw$ to complete the determination of the OPE). 

We begin with an integral of the form:
\begin{equation}
    \mathcal{I}^1(\vec{j};\vec{k},\vec{l};\vec{m}, \vec{n}) = \underset{\mathbb{C}^2}{\int} \frac{ (x^1)^{k_1} (x^2)^{k_2}  (\overline{x}^1)^{l_1} (\overline{x}^2)^{l_2}}{(\vert \vert x \vert \vert^2)^{j_1}} \mathcal{I}_y (\vec{j};\vec{m}, \vec{n}) d^4x \label{eq:main type 1}
\end{equation}
where $\vec{k},\vec{l},\vec{m},\vec{n} \in (\mathbf{Z}_{\geq 0})^2, \vec{j} \in (\mathbf{Z}_{>0})^3, X = (z,x^{\dot{\alpha}}), Y = (w,y^{\dot{\alpha}})$ and:  
\begin{equation}
    \mathcal{I}_y (\vec{j};\vec{m}, \vec{n}) = \underset{\mathbb{C}^2}{\int} \frac{[\overline{x},\overline{y}] (y^1)^{m_1} (y^2)^{m_2} (\overline{y}^1)^{n_1} (\overline{y}^2)^{n_2}}{(\vert \vert X-Y \vert \vert^2)^{j_2} (\vert \vert y \vert \vert^2)^{j_3}} d^4y.
\end{equation}
We have also made the following definition: 
\begin{equation}
    [\overline{x},\overline{y}] = \overline{x}^1 \overline{y}^2 - \overline{x}^2 \overline{y}^1
\end{equation}
We first integrate over $d^4 y$. Using Feynman's trick,
\begingroup \allowdisplaybreaks \begin{align}
    \mathcal{I}_y (\vec{j};\vec{m}, \vec{n}) &= \bigg( \frac{\Gamma(j_2+j_3)}{\Gamma(j_2) \Gamma(j_3)} \bigg) \int_0^1 dt t^{j_2-1} (1-t)^{j_3-1} \underset{\mathbb{C}^2}{\int} \frac{[\overline{x},\overline{y}] (y^1)^{m_1} (y^2)^{m_2} (\overline{y}^1)^{n_1} (\overline{y}^2)^{n_2}}{(t\vert \vert X-Y \vert \vert^2 + (1-t) \vert \vert y \vert \vert^2)^{j_2+j_3}} d^4y \notag
\end{align} \endgroup
Next, we shift the integration variable $y$, $y \to y+tX$, and use the binomial theorem:
\begingroup \allowdisplaybreaks \begin{align}
    \mathcal{I}_y (\vec{j};\vec{m}, \vec{n}) &= \bigg( \frac{\Gamma(j_2+j_3)}{\Gamma(j_2) \Gamma(j_3)} \bigg) \int_0^1 dt t^{j_2-1} (1-t)^{j_3-1} \sum_{i=1}^{2} \sum_{a_i = 0}^{m_i} \sum_{b_i = 0}^{n_i} {m_i \choose a_i} {n_i \choose b_i} \times \label{eq. type 1 eq 2}\\
    &\times (tx^1)^{m_1-a_1}(tx^2)^{m_2-a_2} (t\overline{x}^1)^{n_1-b_1}(t\overline{x}^2)^{n_2-b_2} \underset{\mathbb{C}^3}{\int} \frac{[\overline{x},\overline{y}] (y^1)^{a_1} (y^2)^{a_2}(\overline{y}^1)^{b_1} (\overline{y}^2)^{b_2}}{(t\vert z-w \vert + \vert \vert y \vert \vert^2+t(1-t)\vert \vert x \vert \vert^2)^{j_2+j_3}} d^4y \notag
\end{align} \endgroup

The integral over y only receives contributions from those terms that are invariant under phase rotations of $y^{\dot{\alpha}}$. Let us make the following convenient definition
for the summations:
\begingroup \allowdisplaybreaks \begin{align}
    \sum_{(a_1,a_2)}^{(\vec{m},\vec{n})} \equiv &\bigg( \sum_{a_1=0}^{\text{Min}[m_1,n_1]} \sum_{a_2=1}^{\text{Min}[m_2,n_2+1]} {n_1 \choose a_1}  {n_2 \choose a_2-1}-\sum_{a_1=1}^{\text{Min}[m_1,n_1+1]} \sum_{a_2=0}^{\text{Min}[m_2,n_2]} {n_1 \choose a_1-1}  {n_2 \choose a_2}\bigg) {m_1 \choose a_1} {m_2 \choose a_2} \notag
\end{align} \endgroup
using which, eq.(\ref{eq. type 1 eq 2}) reduces to
\begingroup \allowdisplaybreaks \begin{align}
\mathcal{I}_y (\vec{j};\vec{m}, \vec{n}) &= \bigg( \frac{\Gamma(j_2+j_3)}{\Gamma(j_2) \Gamma(j_3)} \bigg) \sum_{(a_1,a_2)}^{(\vec{m},\vec{n})} \int_0^1 dt t^{j_2+m_1+m_2+n_1+n_2-2a_1-2a_2-1} (1-t)^{j_3-1}  \times \\
    &\times (x^1)^{m_1-a_1}(x^2)^{m_2-a_2} (\overline{x}^1)^{n_1+1-a_1}(\overline{x}^2)^{n_2+1-a_2} \times \notag\\
    & \times (-2 \pi i)^2 (t\vert z-w \vert^2+t(1-t) \vert \vert x \vert \vert^2)^{2+a_1+a_2-j_2-j_3} \int_0^{\infty} \frac{(r^1)^{a_1} (r^2)^{a_2} }{(r^1+r^2+1)^{j_2+j_3}} dr^1 dr^2 \notag
\end{align} \endgroup
where we introduced radial coordinates $r^i = \vert y^i \vert^2/(t\vert z-w\vert^2+t(1-t)\vert \vert X-W \vert \vert^2)$, and we integrated over $d\theta^i$. \\ \\
Integrating over $dr^i$ and grouping terms, this simplifies to:
\begingroup \allowdisplaybreaks \begin{align}
\mathcal{I}_y (\vec{j};\vec{m}, \vec{n}) &= \bigg( \frac{(-2\pi i)^2}{\Gamma(j_2) \Gamma(j_3)} \bigg) \sum_{(a_1,a_2)}^{(\vec{m},\vec{n})} \Gamma(a_1+1) \Gamma(a_2+1) \Gamma(j_2+j_3-2-a_1-a_2) \times \notag \\
\times& (x^1)^{m_1-a_1} (x^2)^{m_2-a_2} (\overline{x}^1)^{n_1+1-a_1} (\overline{x}^2)^{n_2+1-a_2} \int_0^1 dt \frac{t^{2+m_1+m_2+n_1+n_2-a_1-a_2-j_3} (1-t)^{j_3-1}}{(t\vert z-w \vert^2+t(1-t) \vert \vert x \vert \vert^2)^{j_2+j_3-2-a_1-a_2}} \notag
\end{align} \endgroup
\\ 
We now at last have the following integral, which we must integrate over $d^4x$:
\begin{equation}
    \mathcal{I}_x (\vec{j};\vec{k},\vec{l};\vec{m}, \vec{n}) = \underset{\mathbb{C}^2}{\int} \frac{ (x^1)^{k_1+m_1-a_1} (x^2)^{k_2+m_2-a_2} (\overline{x}^1)^{l_1+n_1+1-a_1} (\overline{x}^2)^{l_2+n_2+1-a_2}}{(\vert \vert x \vert \vert^2)^{j_1} (\vert z-w \vert^2+(1-t) \vert \vert x \vert \vert^2)^{j_2+j_3-2-a_1-a_2}} d^4x
\end{equation}
The steps we need to follow to perform this integral are identical to those of the $d^4y$
integral: Feynman’s trick, shifting the integration variable, and only retaining those terms which are invariant under phase rotations of $x^{\dot{\alpha}}$. We present the final result:
\begingroup \allowdisplaybreaks \begin{align}
\mathcal{I}^1(\vec{j};\vec{k},\vec{l};\vec{m}, \vec{n}) &= \bigg(\frac{(2\pi)^4}{\Gamma(j_1) \Gamma(j_2) \Gamma(j_3)}\bigg) \frac{\Gamma(j_1+j_2+j_3-4-k_1-k_2-m_1-m_2)}{(\vert z-w \vert^2)^{j_1+j_2+j_3-4-k_1-k_2-m_1-m_2}} \delta_{k_i+m_i}^{l_i+n_i+1} \times \notag \\
&\times \sum_{(a_1,a_2)}^{(\vec{m},\vec{n})} \Gamma(a_1+1) \Gamma(a_2+1) \Gamma(k_1+m_1+1-a_1) \Gamma(k_2+m_2+1-a_2) \times \notag \\
&\times \int_0^1 \int_0^1 ds dt \frac{t^{p_1} (1-t)^{j_3-1} s^{p_2} (1-s)^{j_1-1}}{(1-st)^{p_3}}
\end{align} \endgroup
where we have made the following definitions:
\begingroup \allowdisplaybreaks \begin{align}
    p_1 &= 2+m_1+m_2+n_1+n_2-a_1-a_2-j_3 \\ p_2 &= 1+k_1+k_2+m_1+m_2-a_1-a_2-j_1 \\ p_3 &= 2+k_1+k_2+m_1+m_2-a_1-a_2.
\end{align} \endgroup

\newline

To connect to what we have previously determined in Appendices \ref{appx:hcsbr}, \ref{appx:ksbr}, let us take several specializations of this general form. 

\begin{enumerate}
\item Consider $\vec{j} = (2, 3, 2), \vec{k} = (1, 0), \vec{l} = \vec{n} = 0, \vec{m}= (0, 1)$. The integral becomes
\begin{equation}
    \mathcal{I}^1(z, w) = \underset{(\mathbb{C}^2)^2}{\int} \frac{[\bar{x}, \bar{y}] x_1 y_2}{(\vert \vert x \vert \vert^2)^2 (\vert \vert X - Y \vert \vert^2)^3 (\vert \vert y \vert \vert^2)^2 } d^4x d^4 y.
\end{equation}
With these parameters, the general form of our integral becomes
\begin{equation}
\mathcal{I}^1(z, w) = {(- 2 \pi i)^4 \over 4}{1 \over |z-w|^2}.
\end{equation}
This integral is precisely that in equation \ref{eq:integral1}, except with the anti-holomorphic $(\bar{z}-\bar{w})$ factor from the holomorphic Chern-Simons propagator stripped off. We also must reinstate an overall constant ${1 \over 2 \pi^4}$ coming from the normalization of the propagator and the backreaction field. To get our final answer, we simply reinstate them to recover
\begin{equation}
\mathcal{I}(z, w) = {1 \over 4 (z- w)}.
\end{equation}

\item Next consider $\vec{j} = (2, 4, 2), \vec{k} = (1, 0), \vec{l} = \vec{n} = 0, \vec{m}= (0, 1)$: 
\begin{equation}
    \mathcal{I}(z, w) = \underset{(\mathbb{C}^2)^2}{\int} \frac{[\bar{x}, \bar{y}] x_1 y_2}{(\vert \vert x \vert \vert^2)^2 (\vert \vert X - Y \vert \vert^2)^4 (\vert \vert y \vert \vert^2)^2 } d^4x d^4 y.
\end{equation}
With these parameters, the general form of our integral becomes
\begin{equation}
\mathcal{I}^{1}(z, w) = {(- 2 \pi i)^4 \over 12}\left({1 \over |z-w|^2}\right)^2.
\end{equation}
This is (up to our stripped off factors) the integral we needed to compute the central term in our Kodaira-Spencer theory, equation \ref{eq:integral2}. We now simply reinstate the factors that depend on $\bar{z}, \bar{w}$ from the propagator, i.e. $(\bar{z} - \bar{w})^2$. To get the correct normalization for the OPE, we must also reinstate the constant factors which constitute the overall normalizations of $P, \omega$ (${3 \over 4 \pi^2}$, ${1 \over (2 \pi)^2}$, respectively), which we have so far suppressed. 

The result may now be plugged into an integral over $d z dw$, with a point-splitting regulator, to complete the determination of the central term in the OPE, as in \S \ref{sec:br}.
\end{enumerate}

\subsubsection{Presentation 2: Integral over $d^6X d^4y$}

Next, let us consider an integral of the form:
\begin{equation}
    \mathcal{I}^2(\vec{j};\vec{k},\vec{l};\vec{m}, \vec{n}) = \underset{\mathbb{C}^3}{\int} \frac{ (x^0)^{k_0} (x^1)^{k_1} (x^2)^{k_2} (\overline{x}^0)^{l_0} (\overline{x}^1)^{l_1} (\overline{x}^2)^{l_2}}{(\vert \vert Z-X \vert \vert^2)^{j_1}} \mathcal{I}_y (\vec{j};\vec{m}, \vec{n}) d^6X \label{eq:main type 2}
\end{equation}
where $\vec{k},\vec{l} \in (\mathbf{Z}_{\geq 0})^3, \vec{m},\vec{n} \in (\mathbf{Z}_{\geq 0})^2, \vec{j} \in (\mathbf{Z}_{>0})^3, Z = (z,0), X = (x^0,x^{\dot{\alpha}}), Y = (w,y^{\dot{\alpha}})$ and:  
\begin{equation}
    \mathcal{I}_y (\vec{j};\vec{m}, \vec{n}) = \underset{\mathbb{C}^2}{\int} \frac{[\overline{x},\overline{y}] (y^1)^{m_1} (y^2)^{m_2} (\overline{y}^1)^{n_1} (\overline{y}^2)^{n_2}}{(\vert \vert X-Y \vert \vert^2)^{j_2} (\vert \vert y \vert \vert^2)^{j_3}} d^4y
\end{equation}
The steps we need to follow to perform this integral are identical to those of the previous integral, the only difference being that we must also integrate over $d^2x^0$. We present the result: 
\begingroup \allowdisplaybreaks \begin{align}\label{eq:noncentralintegral}
\mathcal{I}^2(\vec{j};\vec{k},\vec{l};\vec{m}, \vec{n}) &= \bigg(\frac{(2\pi)^5}{\Gamma(j_1) \Gamma(j_2) \Gamma(j_3)}\bigg) \sum \frac{\Gamma(j_1+j_2+j_3-5-a_0-k_1-k_2-m_1-m_2)}{(\vert z-w \vert^2)^{j_1+j_2+j_3-5-a_0-k_1-k_2-m_1-m_2}} \delta_{k_i+m_i}^{l_i+n_i+1} \times \notag \\
&\times  \Gamma(a_0+1) \Gamma(a_1+1) \Gamma(a_2+1) \Gamma(k_1+m_1+1-a_1) \Gamma(k_2+m_2+1-a_2) \times \notag \\
&\times z^{b_0} \overline{z}^c w^{k_0-a_0-b_0} \overline{w}^{l_0-a_0-c} \int_0^1 \int_0^1 ds dt \frac{t^{p_1} (1-t)^{j_3-1} s^{p_2} (1-s)^{p_3}}{(1-st)^{p_4}}
\end{align} \endgroup
where we have made the following definitions:
\begingroup \allowdisplaybreaks \begin{align}
\sum & = \sum_{(a_1,a_2)}^{(\vec{m},\vec{n})} \sum_{a_0}^{\text{Min}[k_0,l_0]} \sum_{b_0=0}^{k_0-a_0} \sum_{c=0}^{l_0-a_0} {k_0 \choose a_0} {l_0 \choose a_0} {k_0-a_0 \choose b_0} {l_0-a_0 \choose c} \\
    p_1 &= 2+m_1+m_2+n_1+n_2-a_1-a_2-j_3 \\ p_2 &= 2+k_0+l_0+k_1+k_2+m_1+m_2-a_0-a_1-a_2-b_0-c-j_1 \\  p_3 &= 5+k_1+k_2+m_1+m_2-j_2-j_3+a_0+b_0+c \\ p_4 &= 2+k_1+k_2+m_1+m_2-a_1-a_2.
\end{align} \endgroup
%
We will now use the result of this integral to compute the integral associated to our final planar box diagram. 

\subsection{Non-central terms in Kodaira--Spencer theory}\label{appx:noncentral}

Let us fix our notation for the bulk coordinates $X = (x^0, x^i), Y = (w, y^i)$, similarly to Appendix \ref{appx:ksbr}. Additionally, we fix a coordinate $Z = (z, 0)$ along the brane. 

Our task is to compute the integral
\begin{equation}
    \underset{(\mathbb{C})^3 \times (\mathbb{C})^2}{\int} D_{a,b} \bold{P}(Z,X) \mu_1(X) \bold{P}(X,Y) \mu_2(Y) \mu_{BR}(y)
\end{equation} as a function of $z$.
Without loss of generality (i.e. other choices will lead to a vanishing integral), we can consider the test functions
\begin{equation}
    \mu_1(X) = f(X) \partial_{x_1} d\overline{x}^0 \quad \quad \mu_2(Y) = g(Y) \partial_{y_2} d\overline{y}^0.
\end{equation} Here, $f, g$ are holomorphic functions of their respective arguments.
We will also leave the order of the derivatives transverse to the brane, $a, b$, general.


%We will compute the integral
%\beqn
%\int_{X, Y} D_{a,b} \bP(U,X) \mu(X) \bP(X,Y) \mu(Y)  \mu_{BR}(y) 
%\eeqn
%as a function of $u$.
%
%First, we set $a=b=0$, so that the denominator in the above integral is
%\beqn
%\|U-X\|^8 \|X-Y\|^8 \|Y\|^4 .
%\eeqn
%Without loss of generality, we consider the test functions
%\beqn
%\mu_1 (X) = f(X) \del_{x_1} \d \zbar, \quad \mu_2 (Y) = g(Y) \del_{y_2} \d \wbar ,
%\eeqn
%where $f,g$ are holomorphic functions of $X,Y$ respectively.

Then, the anti-holomorphic differential form appearing in the integral above is
\begin{multline}
\left(\eps_{ijk} (\br Z_i - \br X_i) \d (\br Z_j - \br X_k) \d (\br Z_k - \br X_k)\right) \d \overline{x}^0\\ \left(\eps_{lmn} (\br X_l - \br Y_l) \d (\br X_m - \br Y_m) \d (\br X_n - \br Y_n)\right)\d \wbar \left(\eps_{pq} \br y_p \d \br y_q \right).
\end{multline}
This is a nonzero multiple of
\begin{multline}
\left(\eps_{ij} \br x_i \d \br x_j \d \br u\right) \d \overline{x}^0 \left(\eps_{mn} (\br x^0 - \br w) (\d \br x_m - \d \br y_m) (\d \br x_n - \d \br y_n) \right)
\d \br w \left(\ep_{pq} \br y_p  \d \br y_q\right) = 
\# [\br x \br y] (\br x^0 - \br w) \d \br z \d^3 X \d^3 Y .
\end{multline}

Recall that the BCOV propagator is a bivector.
With the choice of input functions, only certain $\del^2$-components of the propagator will contribute nontrivially.
For example, if we take the $\del_{z-x^0} \del_{x_i}$ component of $\bP(Z,X)$ then only the component $\del_{x_1 - y_1} \del_{x_2-y_2}$ component of $\bP(X,Y)$ may contribute nontrivially.
This combination contributes the analytic factor $\ep_{ij} \br x_j (\br x^0 - \br w)$ to the integral, since these appear with opposite signs when $i=1,2$ these contributions cancel.
If we take the $\del_{x_1} \del_{x_2}$ component of $\bP(Z,X)$ then only the $\del_{x^0-w} \del_{x_2-y_2}$ component of $\bP(X,Y)$ may contribute nontrivially.
This contributes the analytic factor $(\br z - \br x^0) (\br x_1 - \br y_1)$ to the integral.

%In total, we see that the integral simplifies to $\d \br z$ times 
%\beqn
%\int_{X,Y} \frac{(\br z - \br x^0) (\br x^0 - \br w) [\br x \br y] (\br x_1 - \br y_1) f(X) g(Y)}{\|Z-X\|^8 \|X-Y\|^8 \|Y\|^4} \d^3 X \d^3 Y
%\eeqn

In total, we find that we must compute the following integral:
\begin{equation}
    \tau = \underset{(\mathbb{C})^3 \times (\mathbb{C})^2}{\int} \frac{[\overline{x}, \overline{y}] (\overline{x}^1)^a (\overline{x}^2)^b (\overline{z}-\overline{x}^0) (\overline{x}^0 - \overline{w})(\overline{x}^1-\overline{y}^1) f(X) g(Y)}{(\vert \vert Z-
X \vert \vert^2)^{4+a+b}(\vert \vert X-Y \vert \vert^2)^4 (\vert \vert y \vert \vert^2)^2} d^6X d^4y
\end{equation}
We will now specialize to the holomorphic test functions $f(X) = (x^1)^{2+a}, \text{ } g(Y) = (w)^l (y^2)^{1+b}$, and define:
\begin{equation}
    \vec{j} = (4+a+b,4,2) \quad \quad \vec{k} = (0,(x^1)^{2+a},0) \quad \quad \vec{m} = (0, (y^2)^{1+b}) \quad \quad \vec{n} = (n_1, 0) 
\end{equation}
\begin{equation}
    \mathcal{A} = \{ (a+1,0),(a,1)\}.
\end{equation}
Then, our integral, $\tau$, is given by a linear combinations of our results in eq.(\ref{eq:noncentralintegral}):
\begin{align} 
    \tau = \sum_{(l_1,n_1) \in \mathcal{A}} (-1)^{n_1} \bigg( &- \overline{z} \overline{w} \mathcal{I}^2(\vec{j};\vec{k}; \vec{l} = (0,l_1,b);\vec{m};\vec{n}) +(\overline{z}+\overline{w}) \mathcal{I}^2(\vec{j};\vec{k}; \vec{l} = (1,l_1,b);\vec{m};\vec{n}) \notag \\  &-\mathcal{I}^2(\vec{j};\vec{k}; \vec{l} = (2,l_1,b);\vec{m};\vec{n})\bigg) 
\end{align}
Defining $\gamma(p_1,p_2,p_3,p_4,q_1)$ as:
\begin{equation}
    \gamma(p_1,p_2,p_3,p_4,q_1) = \int_0^1 \int_0^1 ds dt \frac{t^{p_1} (1-t)^{p_2} s^{p_3} (1-s)^{p_4}}{(1-st)^{q_1}}
\end{equation}
we find that our linear combination simplifies to:
\begin{align}\label{eq:noncentralanswer}
    \tau = \bigg( \frac{(2\pi)^5 \Gamma(3+a) \Gamma(2+b)}{\Gamma(4+a+b) \Gamma(4) \Gamma(2)}&\bigg) \bigg( \frac{1}{z-w} \bigg)^2 \sum_{(l_1,n_1) \in \mathcal{A}} (-1)^{n_1} \sum_{l_0=0}^2 (\delta_{l_0,0}-\delta_{l_0,2}) \bigg(\frac{1}{2}\bigg)^{l_0} \times \\
    & \times \sum_{c=0}^{l_0} (-1)^{l_0-c} {l_0 \choose c} \gamma(b+n_1,1,l_0-c,2+a+b+c,4+a+b). \notag
\end{align}

We can take this result, reinstate the overall normalization factors from the propagators and backreaction terms, and perform the remaining $\int dz dw$ integral over the brane coordinates to complete our determination of the OPE. 


\end{document}
