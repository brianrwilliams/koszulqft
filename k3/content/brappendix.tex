\documentclass[a4paper,11pt]{article}
\pdfoutput=1 

\documentclass[../main.tex]{subfiles}

%\usepackage{jheppub} 
%\usepackage[T1]{fontenc} 
%\usepackage{caption}
%\usepackage{subcaption}
%\usepackage{hyperref}
%\usepackage{revsymb}
%\usepackage{mwe}
%\usepackage{simpler-wick}

\newcommand{\cI}{\mathcal{I}}
\begin{document}

\appendix

\section{Loop computations involving backreaction}

\subsection{Backreaction in holomorphic Chern--Simons}
\label{appx:hcsbr}

Let $X = (z,x) = (z, x_1,x_2), Y = (w,y) = (w,y_1,y_2)$.
We compute the integral
\beqn
\int_{(X,Y) \in \C^3_1 \times \C^3_2} A_1(X) \, \omega (x) \,  \del_z \del_w P(X,Y) \, \omega(y) \, A_2(Y) ,
\eeqn
where $A_i$ are $(0,1)$-forms on $\C^3$, and $P(X,Y) = P(X-Y)$ is as in equation \eqref{eqn:propagatorCS}.
Plugging in $A = x_1 \d \br z$ and $B = y_2 \d \br w$ this integral becomes $\int_{z,w} \d z \, \d w \,  \del_z \del_w I(z,w)$ where
\beqn
I (z,w) \define (\br z - \br w)\int_{\C^2 \times \C^2} \d^4 x \d^4 y \frac{[\br x \br y] x_1 y_2}{\|x\|^4 (|z-w|^2 + \|x-y\|^2)^3 \|y\|^4} .
\eeqn
We compute $I(z,w)$ as a function of the difference $z-w$. Note that there is an additional factor over ${1 \over (2 \pi)^4}$ arising from the propagator and $\omega$ which we have suppressed, and will restore at the end. 

First, we perform the integration along $y \in \C^2$.
Using Feynman's trick we have
\begin{multline}
\int_{\C^2} \d^4 y \frac{[\br x \br y] y_2}{(|z-w|^2 + \|x-y\|^2)^3 \|y\|^4} \\ = \frac{4!}{2!} \int_0^1 \d t \, t^2 (1-t) \int_{\C^2} \d^4 y \frac{[\br x \br y] y_2}{\left(t |z-w|^2 + t \|x-y\|^2 + (1-t)\|y\|^2 \right)^5} .
\end{multline}
Introduce the new variable $\til y = y - t x$.
The the right hand side becomes
\beqn
12 \int_0^1 \d t \, t^2 (1-t) \int_{\C^2} \d^4 \til y \frac{[\br x (\br {\til y} + t \br x)] (y_2+t x_2)}{\left(\|\til y\|^2 + t(1-t)\|x\|^2 + t |z-w|^2 \right)^5} 
\eeqn
Changing to polar coordinates and first computing the residue we see that only terms invariant under $U(1) \times U(1)$ rotations of $\C^2$ will contribute to this integral.
The $U(1) \times U(1)$ invariant part of the numerator is $\br x_1 |\til y_2|^2$.
After computing the residue along both the $\til y_1$ and $\til y_2$ directions the integral then becomes
\beqn
12 (-2 \pi \im)^2 \br x_1 \int_{0}^1 \d t \, t^2 (1-t) \int_{(0,\infty) \times (0,\infty)} \d^2 \rho \,\frac{\rho_2}{(\rho_1 + \rho_2 + t(1-t)\|x\|^2 + t |z-w|^2)^5} .
\eeqn

Performing the integration over $(0,\infty) \times (0 , \infty)$ we obtain 
\beqn
\frac{(-2 \pi \im)^2}{2} \br x_1 \int_0^1 \frac{1-t}{(|z-w|^2 + (1-t)\|x\|^2)^2} %= \frac{(-2 \pi \im)^2}{2} \br x_1 \left(- \frac{1}{\|x\|^2 (|z-w|^2 + \|x\|^2)} + \cdots\right) .
\eeqn

Returning to the original integral we must now compute
\beqn
\int_{0}^1 \d t \, (1-t) \int_{\C^2} \d^4 x \, \frac{|x_1|^2}{\|x\|^4 (|z-w|^2 + (1-t)\|x\|^2)^2} ,
\eeqn
We compute the integral over $x$.

Using the Feynman trick again we have
\begin{multline}
\int_{\C^2} \d^4 x \, \frac{|x_1|^2}{\|x\|^4 (|z-w|^2 + (1-t)\|x\|^2)^2} \\ = \int_0^1 \d s \, s (1-s)
\int_{\C^2} \d^4 x \, \frac{|x_1|^2}{\left(s |z-w|^2 + (1-ts)\|x\|^2\right)^4} .
\end{multline}
After computing the angular integrations this becomes
\beqn
(-2 \pi \im)^2 \int_{(0,\infty) \times (0,\infty)} \d^2 \rho \frac{\rho_1}{(s|z-w|^2 + (1-ts)(\rho_1 + \rho_2))^4} \\ = \frac{1}{s(1-ts)^3 |z-w|^2}   
\eeqn

Finally, plugging back into the original expression we have
\beqn
I(z,w) = \frac{(- 2 \pi i)^2}{z-w} \int_{0}^1 \d t \int_0^1 \d s \frac{(1-t)(1-s)}{(1-ts)^3} .
\eeqn
The integral over $t,s$ gives ${1 \over 2}$. Combining all the resulting factors from the preceding computations, and reinstating the propagator normalization, we therefore have
\beqn
I(z, w) = {(- 2 \pi i)^4 \over 2}{1 \over (2 \pi)^4}{1 \over 2 (z- w)} = {1 \over 4(z-w)}.
\eeqn
%If $\til Y = (tw - tz, y - tx)$ then
%\beqn
%\|\til Y\|^2 + t (1-t) (|z-w|^2 + \|x\|^2) = t \|X - Y\|^2 + (1-t) \|y\|^2 .
%\eeqn
%In terms of the coordinate $\til y$, the invariant piece of 
%\beqn
%[\br x \br y] y_2
%\eeqn
%is $\br x_1 |\til y_2|^2$.

\subsection{The central term in Kodaira--Spencer theory}\label{appx:ksbr}

Let the notation for the coordinates $X,Y$ be as in the last section.
We will compute the integral
\beqn
\int_{X, Y} \mu_1(X) \, \mu_{BR} (x) \,  \bP (X,Y) \, \mu_{BR}(y) \, \mu_2(Y) .
\eeqn
Without loss of generality, we plug in the test functions
\beqn
\mu_1 (X) = x_1 \del_{x_1} \d \zbar, \quad \mu_2 (Y) = y_2 \del_{y_2} \d \wbar .
\eeqn
The vector field type is determined by the symmetry of the graph while the powers of the holomorphic coordinates $x,y$ which appear are determined by the scaling properties of the propagator and backreaction.

Notice that $\mu_{BR}(x)$ is proportional to the differential form $\ep_{ij} \br x_i  \d \br x_j$ and similarly for $\mu_{BR}(y)$.
Thus, for these test functions only the $\del_{x_1-y_1} \del_{x_2-y_2}$ part of the BCOV propagator $\bP(X,Y)$ will contribute to this integral.
Furthermore, the terms in the BCOV propagator proportional to $\d\zbar - \d \wbar$ will not contribute by type reasons.
Simplifying, we see that for this choice of test functions this integral becomes $\int_{z,w} \d z \, \d w \, I(z,w)$ where
\beqn
I (z,w) \define (\br z - \br w)^2 \int_{\C^2_x \times \C^2_y} \d^4 x \d^4 y \frac{[\br x \br y] x_1 y_2}{\|x\|^4 (|z-w|^2 + \|x-y\|^2)^4 \|y\|^4} 
\eeqn
where we have again suppressed the constant factors from the propagator and $\omega$, to be restored at the end. 
We remark that the factor $(\zbar-\wbar)^2$ comes from the BCOV propagator.
We compute $I(z,w)$ as a function of the difference $z-w$.

First, we perform the integration along $y \in \C^2$.
Using Feynman's trick we have
\begin{multline}
\int_{\C^2} \d^4 y \frac{[\br x \br y] y_2}{(|z-w|^2 + \|x-y\|^2)^4 \|y\|^4} \\ = {5! \over 3!} \int_0^1 \d t \, t^3 (1-t) \int_{\C^2} \d^4 y \frac{[\br x \br y] y_2}{\left(t |z-w|^2 + t \|x-y\|^2 + (1-t)\|y\|^2 \right)^6} .
\end{multline}
Introduce the new variable $\til y = y - t x$.
The the right hand side becomes
\beqn
20 \int_0^1 \d t \, t^3 (1-t) \int_{\C^2} \d^4 \til y \frac{[\br x (\br {\til y} + t \br x)] (y_2+t x_2)}{\left(\|\til y\|^2 + t(1-t)\|x\|^2 + t |z-w|^2 \right)^6} 
\eeqn
The $U(1) \times U(1)$ invariant part of the numerator is $\br x_1 |\til y_2|^2$.
After computing the residue along both the $\til y_1$ and $\til y_2$ directions the integral becomes
\beqn
20 (- 2 \pi i)^2\br x_1 \int_{0}^1 \d t \, t^3 (1-t) \int_{(0,\infty) \times (0,\infty)} \d^2 \rho \,\frac{\rho_2}{(\rho_1 + \rho_2 + t(1-t)\|x\|^2 + t |z-w|^2)^6} .
\eeqn
Performing the integration over $(0,\infty) \times (0 , \infty)$ we obtain 
\beqn
(-2 \pi i)^2 {5! \over 3!}{2! \over 5!} \br x_1 \int_0^1 \frac{1-t}{(|z-w|^2 + (1-t)\|x\|^2)^3} %= \frac{(-2 \pi \im)^2}{2} \br x_1 \left(- \frac{1}{\|x\|^2 (|z-w|^2 + \|x\|^2)} + \cdots\right) .
\eeqn

Returning to the original integral we must now compute (suppressing the overall constant factors for the moment)
\beqn
\int_{0}^1 \d t \, (1-t) \int_{\C^2} \d^4 x \, \frac{|x_1|^2}{\|x\|^4 (|z-w|^2 + (1-t)\|x\|^2)^3} ,
\eeqn
We compute the integral over $x$ as above to obtain
\beqn
(- 2 \pi i)^4{4 \over 4!}\frac{1}{|z-w|^4} \int_{0}^1 \d t \int_{0}^1 \d s \, \frac{(1-t)(1-s)}{(1-ts)^3}  
\eeqn
and hence, putting all the pieces together,
\beqn
I(z,w) = {3 \over (2 \pi )^4}{(- 2 \pi i)^4 \over 6}\frac{1}{2 (z-w)^2} = {1 \over 4(z- w)^2}.
\eeqn
Upon changing to the basis of on-shell generators (i.e. currents sourcing the properly constrained Kodaira-Spencer fields), we will recover precisely the canonical Kac-Moody algebra at the expected level ${ N \over 2}$.

\subsection{Evaluating more general holomorphic integrals}\label{appx:integral}

In the previous two appendices, we computed some holomorphic integrals which can deform a Koszul dual chiral algebra on a case-by-case basis. However, these integrals admit more general closed forms, and it is convenient to calculate them once and for all. 

For later convenience, in this appendix we will evaluate several general classes of holomorphic integrals, which are common to 1-loop Koszul duality computations in holomorphic theories; in the next appendix, we will specialize the general form we derive here to compute other box diagrams which deform our planar chiral algebra. 

We will evaluate three types of integrals, which differ in how many coordinates we integrate our bulk vertices over. As in the main text, one may then subsequently integrate over the defect vertex locations, if applicable, using a point splitting regulator to complete the computation of the OPE.  Throughout this appendix, we employ the same notation as in \S \ref{sec:br}.

\subsubsection{Type 1: Integral over $d^4x d^4 y$} 

We begin with an integral of the form:
\begin{equation}
    \mathcal{I}^1(\vec{j};\vec{k},\vec{l};\vec{m}, \vec{n}) = \underset{\mathbb{C}^2}{\int} \frac{ (x^1)^{k_1} (x^2)^{k_2}  (\overline{x}^1)^{l_1} (\overline{x}^2)^{l_2}}{(\vert \vert x \vert \vert^2)^{j_1}} \mathcal{I}_y (\vec{j};\vec{m}, \vec{n}) d^4x \label{eq:main type 1}
\end{equation}
where $\vec{k},\vec{l},\vec{m},\vec{n} \in (\mathbf{Z}_{\geq 0})^2, \vec{j} \in (\mathbf{Z}_{>0})^3, X = (z,x^{\dot{\alpha}}), Y = (w,y^{\dot{\alpha}})$ and:  
\begin{equation}
    \mathcal{I}_y (\vec{j};\vec{m}, \vec{n}) = \underset{\mathbb{C}^2}{\int} \frac{[\overline{x},\overline{y}] (y^1)^{m_1} (y^2)^{m_2} (\overline{y}^1)^{n_1} (\overline{y}^2)^{n_2}}{(\vert \vert X-Y \vert \vert^2)^{j_2} (\vert \vert y \vert \vert^2)^{j_3}} d^4y.
\end{equation}
We have also made the following definition: 
\begin{equation}
    [\overline{x},\overline{y}] = \overline{x}^1 \overline{y}^2 - \overline{x}^2 \overline{y}^1
\end{equation}
We first integrate over $d^4 y$. Using Feynman's trick,
\begingroup \allowdisplaybreaks \begin{align}
    \mathcal{I}_y (\vec{j};\vec{m}, \vec{n}) &= \bigg( \frac{\Gamma(j_2+j_3)}{\Gamma(j_2) \Gamma(j_3)} \bigg) \int_0^1 dt t^{j_2-1} (1-t)^{j_3-1} \underset{\mathbb{C}^2}{\int} \frac{[\overline{x},\overline{y}] (y^1)^{m_1} (y^2)^{m_2} (\overline{y}^1)^{n_1} (\overline{y}^2)^{n_2}}{(t\vert \vert X-Y \vert \vert^2 + (1-t) \vert \vert y \vert \vert^2)^{j_2+j_3}} d^4y \notag
\end{align} \endgroup
Next, we shift the integration variable $y$, $y \to y+tX$, and use the binomial theorem:
\begingroup \allowdisplaybreaks \begin{align}
    \mathcal{I}_y (\vec{j};\vec{m}, \vec{n}) &= \bigg( \frac{\Gamma(j_2+j_3)}{\Gamma(j_2) \Gamma(j_3)} \bigg) \int_0^1 dt t^{j_2-1} (1-t)^{j_3-1} \sum_{i=1}^{2} \sum_{a_i = 0}^{m_i} \sum_{b_i = 0}^{n_i} {m_i \choose a_i} {n_i \choose b_i} \times \label{eq. type 1 eq 2}\\
    &\times (tx^1)^{m_1-a_1}(tx^2)^{m_2-a_2} (t\overline{x}^1)^{n_1-b_1}(t\overline{x}^2)^{n_2-b_2} \underset{\mathbb{C}^3}{\int} \frac{[\overline{x},\overline{y}] (y^1)^{a_1} (y^2)^{a_2}(\overline{y}^1)^{b_1} (\overline{y}^2)^{b_2}}{(t\vert z-w \vert + \vert \vert y \vert \vert^2+t(1-t)\vert \vert x \vert \vert^2)^{j_2+j_3}} d^4y \notag
\end{align} \endgroup

The integral over y only receives contributions from those terms that are invariant under phase rotations of $y^{\dot{\alpha}}$. Let us make the following convenient definition
for the summations:
\begingroup \allowdisplaybreaks \begin{align}
    \sum_{(a_1,a_2)}^{(\vec{m},\vec{n})} \equiv &\bigg( \sum_{a_1=0}^{\text{Min}[m_1,n_1]} \sum_{a_2=1}^{\text{Min}[m_2,n_2+1]} {n_1 \choose a_1}  {n_2 \choose a_2-1}-\sum_{a_1=1}^{\text{Min}[m_1,n_1+1]} \sum_{a_2=0}^{\text{Min}[m_2,n_2]} {n_1 \choose a_1-1}  {n_2 \choose a_2}\bigg) {m_1 \choose a_1} {m_2 \choose a_2} \notag
\end{align} \endgroup
using which, eq.(\ref{eq. type 1 eq 2}) reduces to
\begingroup \allowdisplaybreaks \begin{align}
\mathcal{I}_y (\vec{j};\vec{m}, \vec{n}) &= \bigg( \frac{\Gamma(j_2+j_3)}{\Gamma(j_2) \Gamma(j_3)} \bigg) \sum_{(a_1,a_2)}^{(\vec{m},\vec{n})} \int_0^1 dt t^{j_2+m_1+m_2+n_1+n_2-2a_1-2a_2-1} (1-t)^{j_3-1}  \times \\
    &\times (x^1)^{m_1-a_1}(x^2)^{m_2-a_2} (\overline{x}^1)^{n_1+1-a_1}(\overline{x}^2)^{n_2+1-a_2} \times \notag\\
    & \times (-2 \pi i)^2 (t\vert z-w \vert^2+t(1-t) \vert \vert x \vert \vert^2)^{2+a_1+a_2-j_2-j_3} \int_0^{\infty} \frac{(r^1)^{a_1} (r^2)^{a_2} }{(r^1+r^2+1)^{j_2+j_3}} dr^1 dr^2 \notag
\end{align} \endgroup
where we introduced radial coordinates $r^i = \vert y^i \vert^2/(t\vert z-w\vert^2+t(1-t)\vert \vert X-W \vert \vert^2)$, and we integrated over $d\theta^i$. \\ \\
Integrating over $dr^i$ and grouping terms, this simplifies to:
\begingroup \allowdisplaybreaks \begin{align}
\mathcal{I}_y (\vec{j};\vec{m}, \vec{n}) &= \bigg( \frac{(-2\pi i)^2}{\Gamma(j_2) \Gamma(j_3)} \bigg) \sum_{(a_1,a_2)}^{(\vec{m},\vec{n})} \Gamma(a_1+1) \Gamma(a_2+1) \Gamma(j_2+j_3-2-a_1-a_2) \times \notag \\
\times& (x^1)^{m_1-a_1} (x^2)^{m_2-a_2} (\overline{x}^1)^{n_1+1-a_1} (\overline{x}^2)^{n_2+1-a_2} \int_0^1 dt \frac{t^{2+m_1+m_2+n_1+n_2-a_1-a_2-j_3} (1-t)^{j_3-1}}{(t\vert z-w \vert^2+t(1-t) \vert \vert x \vert \vert^2)^{j_2+j_3-2-a_1-a_2}} \notag
\end{align} \endgroup
\\ 
We now at last have the following integral, which we must integrate over $d^4x$:
\begin{equation}
    \mathcal{I}_x (\vec{j};\vec{k},\vec{l};\vec{m}, \vec{n}) = \underset{\mathbb{C}^2}{\int} \frac{ (x^1)^{k_1+m_1-a_1} (x^2)^{k_2+m_2-a_2} (\overline{x}^1)^{l_1+n_1+1-a_1} (\overline{x}^2)^{l_2+n_2+1-a_2}}{(\vert \vert x \vert \vert^2)^{j_1} (\vert z-w \vert^2+(1-t) \vert \vert x \vert \vert^2)^{j_2+j_3-2-a_1-a_2}} d^4x
\end{equation}
The steps we need to follow to perform this integral are identical to those of the $d^4y$
integral: Feynman’s trick, shifting the integration variable, and only retaining those terms which are invariant under phase rotations of $x^{\dot{\alpha}}$. We present the final result:
\begingroup \allowdisplaybreaks \begin{align}
\mathcal{I}^1(\vec{j};\vec{k},\vec{l};\vec{m}, \vec{n}) &= \bigg(\frac{(2\pi)^4}{\Gamma(j_1) \Gamma(j_2) \Gamma(j_3)}\bigg) \frac{\Gamma(j_1+j_2+j_3-4-k_1-k_2-m_1-m_2)}{(\vert z-w \vert^2)^{j_1+j_2+j_3-4-k_1-k_2-m_1-m_2}} \delta_{k_i+m_i}^{l_i+n_i+1} \times \notag \\
&\times \sum_{(a_1,a_2)}^{(\vec{m},\vec{n})} \Gamma(a_1+1) \Gamma(a_2+1) \Gamma(k_1+m_1+1-a_1) \Gamma(k_2+m_2+1-a_2) \times \notag \\
&\times \int_0^1 \int_0^1 ds dt \frac{t^{p_1} (1-t)^{j_3-1} s^{p_2} (1-s)^{j_1-1}}{(1-st)^{p_3}}
\end{align} \endgroup
where we have made the following definitions:
\begingroup \allowdisplaybreaks \begin{align}
    p_1 &= 2+m_1+m_2+n_1+n_2-a_1-a_2-j_3 \\ p_2 &= 1+k_1+k_2+m_1+m_2-a_1-a_2-j_1 \\ p_3 &= 2+k_1+k_2+m_1+m_2-a_1-a_2.
\end{align} \endgroup

\subsubsection{Type 2: Integral over $d^6X d^4y$}

Next, let us consider an integral of the form:
\begin{equation}
    \mathcal{I}^2(\vec{j};\vec{k},\vec{l};\vec{m}, \vec{n}) = \underset{\mathbb{C}^3}{\int} \frac{ (x^0)^{k_0} (x^1)^{k_1} (x^2)^{k_2} (\overline{x}^0)^{l_0} (\overline{x}^1)^{l_1} (\overline{x}^2)^{l_2}}{(\vert \vert Z-X \vert \vert^2)^{j_1}} \mathcal{I}_y (\vec{j};\vec{m}, \vec{n}) d^6X \label{eq:main type 2}
\end{equation}
where $\vec{k},\vec{l} \in (\mathbf{Z}_{\geq 0})^3, \vec{m},\vec{n} \in (\mathbf{Z}_{\geq 0})^2, \vec{j} \in (\mathbf{Z}_{>0})^3, Z = (z,0), X = (x^0,x^{\dot{\alpha}}), Y = (w,y^{\dot{\alpha}})$ and:  
\begin{equation}
    \mathcal{I}_y (\vec{j};\vec{m}, \vec{n}) = \underset{\mathbb{C}^2}{\int} \frac{[\overline{x},\overline{y}] (y^1)^{m_1} (y^2)^{m_2} (\overline{y}^1)^{n_1} (\overline{y}^2)^{n_2}}{(\vert \vert X-Y \vert \vert^2)^{j_2} (\vert \vert y \vert \vert^2)^{j_3}} d^4y
\end{equation}
The steps we need to follow to perform this integral are identical to those of the previous integral, the only difference being that we must also integrate over $d^2x^0$. We present the result: 
\begingroup \allowdisplaybreaks \begin{align}
\mathcal{I}^2(\vec{j};\vec{k},\vec{l};\vec{m}, \vec{n}) &= \bigg(\frac{(2\pi)^5}{\Gamma(j_1) \Gamma(j_2) \Gamma(j_3)}\bigg) \sum \frac{\Gamma(j_1+j_2+j_3-5-a_0-k_1-k_2-m_1-m_2)}{(\vert z-w \vert^2)^{j_1+j_2+j_3-5-a_0-k_1-k_2-m_1-m_2}} \delta_{k_i+m_i}^{l_i+n_i+1} \times \notag \\
&\times  \Gamma(a_0+1) \Gamma(a_1+1) \Gamma(a_2+1) \Gamma(k_1+m_1+1-a_1) \Gamma(k_2+m_2+1-a_2) \times \notag \\
&\times z^b \overline{z}^c w^{k_0-a_0-b} \overline{w}^{l_0-a_0-c} \int_0^1 \int_0^1 ds dt \frac{t^{p_1} (1-t)^{j_3-1} s^{p_2} (1-s)^{p_3}}{(1-st)^{p_4}}
\end{align} \endgroup
where we have made the following definitions:
\begingroup \allowdisplaybreaks \begin{align}
\sum & = \sum_{(a_1,a_2)}^{(\vec{m},\vec{n})} \sum_{a_0}^{\text{Min}[k_0,l_0]} \sum_{b=0}^{k_0-a_0} \sum_{c=0}^{l_0-a_0} {k_0 \choose a_0} {l_0 \choose a_0} {k_0-a_0 \choose b} {l_0-a_0 \choose c} \\
    p_1 &= 2+m_1+m_2+n_1+n_2-a_1-a_2-j_3 \\ p_2 &= 2+k_0+l_0+k_1+k_2+m_1+m_2-a_0-a_1-a_2-b-c-j_1 \\  p_3 &= 4+k_1+k_2+m_1+m_2-j_2-j_3-a-b-c \\ p_4 &= 2+k_1+k_2+m_1+m_2-a_1-a_2.
\end{align} \endgroup

\subsubsection{Type 3: Integral over $d^6X d^6 Y$}
Finally, we consider the integral where we integrate the bulk vertices over all holomorphic spacetime directions in one shot. For many purposes, this integral, while producing a somewhat more intricate final answer, is often the most natural. 
\begin{equation}
    \mathcal{I}(\vec{j};\vec{k},\vec{l};\vec{m}, \vec{n}) = \underset{\mathbb{C}^3}{\int} \frac{(x^0)^{k_0} (x^1)^{k_1} (x^2)^{k_2} (\overline{x}^0)^{l_0} (\overline{x}^1)^{l_1} (\overline{x}^2)^{l_2}}{(\vert \vert Z-X \vert \vert^2)^{j_1}} \mathcal{I}_Y (\vec{j};\vec{m}, \vec{n}) d^6X \label{eq:main I}
\end{equation}
where $\vec{k},\vec{l},\vec{m},\vec{n} \in (\mathbf{Z}_{\geq 0})^3, \vec{j} \in (\mathbf{Z}_{>0})^3, Z = (z,0^{\dot{\alpha}}), X = (x^0,x^{\dot{\alpha}}), Y = (y^0,y^{\dot{\alpha}})$, $W = (w,0^{\dot{\alpha}})$, and we have the following definitions:  
\begin{equation}
    \mathcal{I}_Y (\vec{j};\vec{m}, \vec{n}) = \underset{\mathbb{C}^3}{\int} \frac{[\overline{x},\overline{y}] (y^0)^{m_0} (y^1)^{m_1} (y^2)^{m_2} (\overline{y}^0)^{n_0} (\overline{y}^1)^{n_1} (\overline{y}^2)^{n_2}}{(\vert \vert X-Y \vert \vert^2)^{j_2} (\vert \vert Y-W \vert \vert^2)^{j_3}} d^6Y
\end{equation}
\begin{equation}
    [\overline{x},\overline{y}] = \overline{x}^1 \overline{y}^2 - \overline{x}^2 \overline{y}^1.
\end{equation}

We first integrate over $d^6 Y$. Using Feynman's trick,
\begingroup \allowdisplaybreaks \begin{align}
    \mathcal{I}_Y (\vec{j};\vec{m}, \vec{n}) &= \bigg( \frac{\Gamma(j_2+j_3)}{\Gamma(j_2) \Gamma(j_3)} \bigg) \int_0^1 dt t^{j_2-1} (1-t)^{j_3-1} \underset{\mathbb{C}^3}{\int} \frac{[\overline{x},\overline{y}] (y^0)^{m_0} (y^1)^{m_1} (y^2)^{m_2} (\overline{y}^0)^{n_0} (\overline{y}^1)^{n_1} (\overline{y}^2)^{n_2}}{(t\vert \vert X-Y \vert \vert^2 + (1-t) \vert \vert Y-W \vert \vert^2)^{j_2+j_3}} d^6Y \notag
\end{align} \endgroup
We shift the integration variable $Y$, $Y \to Y-tX-(1-t)W$, and use the binomial theorem:
\begingroup \allowdisplaybreaks \begin{align}
    \mathcal{I}_Y (\vec{j};\vec{m}, \vec{n}) &= \bigg( \frac{\Gamma(j_2+j_3)}{\Gamma(j_2) \Gamma(j_3)} \bigg) \int_0^1 dt t^{j_2-1} (1-t)^{j_3-1} \sum_{i=0}^{2} \sum_{a_i = 0}^{m_i} \sum_{b_i = 0}^{n_i} {m_i \choose a_i} {n_i \choose b_i} \times \label{eq. 2}\\
    &\times (tx^0+(1-t)w)^{m_0-a_0} (tx^1)^{m_1-a_1}(tx^2)^{m_2-a_2}(t\overline{x}^0+(1-t)\overline{w})^{n_0-b_0} (t\overline{x}^1)^{n_1-b_1}(t\overline{x}^2)^{n_2-b_2} \times \notag\\
    & \times \underset{\mathbb{C}^3}{\int} \frac{[\overline{x},\overline{y}] (y^0)^{a_0} (y^1)^{a_1} (y^2)^{a_2} (\overline{y}^0)^{b_0} (\overline{y}^1)^{b_1} (\overline{y}^2)^{b_2}}{(\vert \vert Y \vert \vert^2+t(1-t)\vert \vert X-W \vert \vert^2)^{j_2+j_3}} d^6Y \notag
\end{align} \endgroup

\\ 

The integral over Y only receives contributions from those terms that are invariant under phase rotations of $y^0$ and $y^{\dot{\alpha}}$. Let us make the following convenient definition for the summations
\begingroup \allowdisplaybreaks \begin{align}
    \sum_{(a_0,a_1,a_2)}^{(\vec{m},\vec{n})} \equiv \sum_{a_0=0}^{\text{Min}[m_0,n_0]} {m_0 \choose a_0}{n_0 \choose a_0} &\bigg( \sum_{a_1=0}^{\text{Min}[m_1,n_1]} \sum_{a_2=1}^{\text{Min}[m_2,n_2+1]} {n_1 \choose a_1}  {n_2 \choose a_2-1}  \\ &- \sum_{a_1=1}^{\text{Min}[m_1,n_1+1]} \sum_{a_2=0}^{\text{Min}[m_2,n_2]} {n_1 \choose a_1-1}  {n_2 \choose a_2}\bigg) {m_1 \choose a_1} {m_2 \choose a_2} \notag
\end{align} \endgroup
using which, eq.(\ref{eq. 2}) reduces to
\begingroup \allowdisplaybreaks \begin{align}
\mathcal{I}_Y (\vec{j};\vec{m}, \vec{n}) &= \bigg( \frac{\Gamma(j_2+j_3)}{\Gamma(j_2) \Gamma(j_3)} \bigg) \sum_{(a_0,a_1,a_2)}^{(\vec{m},\vec{n})} \int_0^1 dt t^{j_2-1} (1-t)^{j_3-1}  \times \\
    &\times (tx^0+(1-t)w)^{m_0-a_0} (tx^1)^{m_1-a_1}(tx^2)^{m_2-a_2}(t\overline{x}^0+(1-t)\overline{w})^{n_0-a_0} (t\overline{x}^1)^{n_1+1-a_1}(t\overline{x}^2)^{n_2+1-a_2} \times \notag\\
    & \times (-2 \pi i)^3 (t(1-t) \vert \vert X-W \vert \vert^2)^{3+a_0+a_1+a_2-j_2-j_3} \int_0^{\infty} \frac{(r^0)^{a_0} (r^1)^{a_1} (r^2)^{a_2} }{r^1+r^2+r^3+1)^{j_2+j_3}} dr^0 dr^1 dr^2 \notag
\end{align} \endgroup
where we introduced radial coordinates $r^i = \vert y^i \vert^2/(t(1-t)\vert \vert X-W \vert \vert^2)$, and we integrated over $d\theta^i$. \\ \\
Integrating over $dr^i$, grouping terms and using the binomial theorem, this simplifies to:
\begingroup \allowdisplaybreaks \begin{align}
\mathcal{I}_Y (\vec{j};\vec{m}, \vec{n}) &= \bigg( \frac{(-2\pi i)^3}{\Gamma(j_2) \Gamma(j_3)} \bigg) \sum_{(a_0,a_1,a_2)}^{(\vec{m},\vec{n})} \Gamma(a_0+1) \Gamma(a_1+1) \Gamma(a_2+1) \Gamma(j_2+j_3-3-a_0-a_1-a_2) \times \notag \\
\times& \sum_{b_0=0}^{m_0-a_0} \sum_{c_0=0}^{n_0-a_0} {m_0-a_0 \choose b_0} {n_0 -a_0 \choose c_0} \frac{(x^0)^{b_0} (x^1)^{m_1-a_1} (x^2)^{m_2-a_2} (\overline{x}^0)^{c_0} (\overline{x}^1)^{n_1+1-a_1} (\overline{x}^2)^{n_2+1-a_2}}{(\vert \vert X-W \vert \vert^2)^{j_2+j_3-3-a_0-a_1-a_2}} \times \notag \\
\times (w)&^{m_0-a_0-b_0} (\overline{w})^{n_0-a_0-c_0} \int_0^1 dt t^{4+m_1+m_2+n_1+n_2+a_0+b_0+c_0-j_3-a_1-a_2} (1-t)^{2+m_0+n_0+a_1+a_2-j_2-a_0-b_0-c_0} \notag
\end{align} \endgroup

We next perform the $dt$ integral,
\begingroup \allowdisplaybreaks \begin{align}
\mathcal{I}_Y (\vec{j};\vec{m}, \vec{n}) &= \bigg( \frac{(-2\pi i)^3}{\Gamma(j_2) \Gamma(j_3)} \bigg) \sum_{(a_0,a_1,a_2)}^{(\vec{m},\vec{n})} \Gamma(a_0+1) \Gamma(a_1+1) \Gamma(a_2+1) \Gamma(j_2+j_3-3-a_0-a_1-a_2) \times \notag \\
&\times \sum_{b_0=0}^{m_0-a_0}  \sum_{c_0=0}^{n_0-a_0} {m_0-a_0 \choose b_0} {n_0 -a_0 \choose c_0} \frac{(x^0)^{b_0} (x^1)^{m_1-a_1} (x^2)^{m_2-a_2} (\overline{x}^0)^{c_0} (\overline{x}^1)^{n_1+1-a_1} (\overline{x}^2)^{n_2+1-a_2}}{(\vert \vert X-W \vert \vert^2)^{j_2+j_3-3-a_0-a_1-a_2}} \times \notag \\
 & \times (w)^{m_0-a_0-b_0} (\overline{w})^{n_0-a_0-c_0} \Gamma(5+m_1+m_2+n_1+n_2+a_0+b_0+c_0-j_3-a_1-a_2) \times \notag \\
 & \times \frac{\Gamma(3+m_0+n_0+a_1+a_2-j_2-a_0-b_0-c_0)}{\Gamma(8+m_0+m_1+m_2+n_0+n_1+n_1+n_2-j_2-j_3)} \label{eq: I_Y}
\end{align} \endgroup

We now at last have the following integral, which we must integrate over $d^6X$:
\begin{equation*}
   \mathcal{I}_X(\vec{j};\vec{k},\vec{l};\vec{m}, \vec{n}) = \underset{\mathbb{C}^3}{\int} \frac{(x^0)^{k_0+b_0} (x^1)^{k_1+m_1-a_1} (x^2)^{k_2+m_2-a_2} (\overline{x}^0)^{l_0+c_0} (\overline{x}^1)^{l_1+n_1+1-a_1} (\overline{x}^2)^{l_2+n_2+1-a_2}}{(\vert \vert Z-X \vert \vert^2)^{j_1} (\vert \vert X-W \vert \vert^2)^{j_2+j_3-3-a_0-a_1-a_2}} d^6X
\end{equation*}
The steps we need to follow to perform this integral are identical to those of the $d^6Y$ integral: Feynman's trick, shifting the integration variable, using the binomial theorem, and only retaining those terms which are invariant under phase rotations of $x^0$ and $x^{\dot{\alpha}}$. We present the result:
\begingroup \allowdisplaybreaks \begin{align}
\mathcal{I}_X(\vec{j};\vec{k},\vec{l};\vec{m}, \vec{n}) &= (-2\pi i)^3 \delta_{k_1+m_1}^{l_1+n_1+1} \delta_{k_2+m_2}^{l_2+n_2+1} \frac{\Gamma(j_1+j_2+j_3-3-a_0-a_1-a_2)}{\Gamma(j_1) \Gamma(j_2+j_3-3-a_0-a_1-a_2)} \times \label{eq: I_X} \\
&\times \overset{\text{Min}[k_0+b_0,l_0+c_0]}{\sum_{d_0=0}} {k_0+b_0 \choose d_0} {l_0+c_0 \choose d_0} \frac{\Gamma(d_0+1) \Gamma(k_1+m_1+1-a_1) \Gamma(k_2+m_2+1-a_2)}{\Gamma(j_1+j_2+j_3-3-a_0-a_1-a_2)} \times \notag \\
&\times \frac{\Gamma(j_1+j_2+j_3-6-a_0-d_0-k_1-k_2-m_1-m_2)}{(\vert z-w \vert^2)^{j_1+j_2+j_3-6-d_0-a_0-k_1-k_2-m_1-m_2}} \sum_{r_0=0}^{k_0+b_0-d_0}  {k_0+b_0-d_0 \choose r_0} \times \notag \\
&\times \sum_{s_0=0}^{l_0+c_0-d_0} {l_0+c_0-d_0 \choose s_0} z^{r_0} \overline{z}^{s_0} w^{k_0+b_0-d_0-r_0} \overline{w}^{l_0+c_0-d_0-s_0} \times \notag \\
&\times \frac{\Gamma(6+d_0+a_0+k_1+k_2+m_1+m_2+r_0+s_0-j_2-j_3)}{\Gamma(9+a_0+l_0+c_0+b_0+k_0+2k_1+2k_2+2m_1+2m_2-a_1-a_2-j_1-j_2-j_3)} \times \notag \\
&\times \Gamma(3+b_0+l_0+c_0+k_0+k_1+k_2+m_1+m_2-d_0-r_0-s_0-a_1-a_2-j_1). \notag
\end{align} \endgroup
Combining our results eq.(\ref{eq: I_X}) and eq.(\ref{eq: I_X}), we thus find that eq.(\ref{eq:main I}) is given by the following expression:
\begingroup \allowdisplaybreaks \begin{align}
\mathcal{I}(\vec{j};\vec{k},\vec{l};\vec{m}, \vec{n}) &= \bigg(\frac{(-2 \pi i)^6}{\Gamma(j_1) \Gamma(j_2) \Gamma(j_3)} \bigg) \delta_{k_1+m_1}^{l_1+n_1+1} \delta_{k_2+m_2}^{l_2+n_2+1} \sum \frac{z^{r_0} \overline{z}^{s_0} w^{k_0+m_0-a_0-d_0-r_0} \overline{w}^{l_0+n_0-a_0-d_0-s_0}}{(\vert z-w \vert^2)^{j_1+j_2+j_3-6-d_0-a_0-k_1-k_2-m_1-m_2}} \times \notag \\
    &\times {m_0-a_0 \choose b_0} {n_0-a_0 \choose c_0} {k_0+b_0 \choose d_0} {l_0+c_0 \choose d_0} {k_0+b_0-d_0 \choose r_0} {l_0+c_0-d_0 \choose s_0}  \times \notag \\
    &\times \frac{\Gamma(a_0+1) \Gamma(a_1+1) \Gamma(a_2+1) \Gamma(d_0+1) \Gamma(k_1+m_1+1-a_1) \Gamma(k_2+m_2+1-a_2)}{\Gamma(8+m_0+m_1+m_2+n_0+n_1+n_2-j_1-j_2)} \times \notag \\
    &\times \frac{\Gamma(6+d_0+a_0+k_1+k_2+m_1+m_2+r_0+s_0-j_2-j_3)}{\Gamma(9+a_0+l_0+b_0+c_0+k_0+2k_1+2k_2+2m_1+2m_2-a_1-a_2-j_1-j_2-j_3)} \times \notag \\
    \times \Gamma(5+m_1&+m_2+n_1+n_2+a_0+b_0+c_0-a_1-a_2-j_3) \Gamma(3+m_0+n_0+a_1+a_2-a_0-b_0-c_0-j_2) \times \notag \\
    &\times \Gamma(j_1+j_2+j_3-6-a_0-d_0-k_1-k_2-m_1-m_2) \times \notag \\
    &\times \Gamma(3+b_0+c_0+l_0+k_0+k_1+k_2+m_1+m_2-d_0-r_0-s_0-a_1-a_2-j_1) \notag
\end{align} \endgroup
where the symbol $\sum$ is a shorthand for:
\begin{equation}
    \sum = \sum_{(a_0,a_1,a_2)}^{(\vec{m},\vec{n})} \sum_{b_0=0}^{m_0-a_0} \sum_{c_0=0}^{n_0-a_0} \overset{\text{Min}[k_0+b_0,l_0+c_0]}{\sum_{d_0=0}} \sum_{r_0=0}^{k_0+b_0-d_0} \sum_{s_0=0}^{l_0+c_0-d_0} 
\end{equation}
%\\ 
\\
A convenient specialization for us is to take $\vec{k} = (1,0,2),\vec{m}=(0,1,0),l_1=0,n_1=0$. Then, this formula reduces to:
\begingroup \allowdisplaybreaks \begin{align} 
\mathcal{I}(\vec{j};\vec{k},\vec{l};\vec{m}, \vec{n}) &= - \bigg(\frac{2! (-2 \pi i)^6}{\Gamma(j_1) \Gamma(j_2) \Gamma(j_3)} \bigg)\sum_{c=0}^{n_0} \overset{\text{Min}[1,l_0+c_0]}{\sum_{d_0=0}}  \sum_{r_0=0}^{1-d_0} \sum_{s_0=0}^{l_0+c_0-d_0} {n_0 \choose c_0} {l_0+c_0 \choose d_0} {l_0+c_0-d_0 \choose s_0} \times \label{eq: simplified general result} \\
&\times \frac{\Gamma(5+n_2+c_0-j_3) \Gamma(4+n_0-c_0-j_2) \Gamma(9+d_0+r_0+s_0-j_2-j_3)}{\Gamma(9+n_0+n_2-j_1-j_2) \Gamma(15+l_0+c_0-j_1-j_2-j_3)} \times \notag \\
&\times \Gamma(j_1+j_2+j_3-9-d_0) \Gamma(6+c_0+l_0-d_0-r_0-s_0-j_1) \notag \\
&\times \frac{z^{r_0} \overline{z}^{s_0} w^{1-d_0-r_0} \overline{w}^{l_0+n_0-d_0-s_0}}{(\vert z-w \vert^2)^{j_1+j_2+j_3-9-d_0}} \notag 
\end{align}\endgroup
%

\subsection{Non-central terms in Kodaira--Spencer theory}\label{appx:noncentral}

Let the notation for the bulk coordinates $X = (z,x),Y=(w,y)$ be as in Appendix \ref{appx:ksbr}.
Additionally we fix a coordinate $U=(u,0)$ along the brane.
We will compute the integral
\beqn
\int_{X, Y} D_{a,b} \bP(U,X) \mu(X) \bP(X,Y) \mu(Y)  \mu_{BR}(y) 
\eeqn
as a function of $u$.

First, we set $a=b=0$, so that the denominator in the above integral is
\beqn
\|U-X\|^8 \|X-Y\|^8 \|Y\|^4 .
\eeqn
Without loss of generality, we consider the test functions
\beqn
\mu_1 (X) = f(X) \del_{x_1} \d \zbar, \quad \mu_2 (Y) = g(Y) \del_{y_2} \d \wbar ,
\eeqn
where $f,g$ are holomorphic functions of $X,Y$ respectively.
Then, the anti-holomorphic differential form appearing in the integral above is
\begin{multline}
\left(\eps_{ijk} (\br U_i - \br X_i) \d (\br U_j - \br X_k) \d (\br U_k - \br X_k)\right) \d \zbar\\ \left(\eps_{lmn} (\br X_l - \br Y_l) \d (\br X_m - \br Y_m) \d (\br X_n - \br Y_n)\right)\d \wbar \left(\eps_{pq} \br y_p \d \br y_q \right)
\end{multline}
This is a nonzero multiple of
\begin{multline}
\left(\eps_{ij} \br x_i \d \br x_j \d \br u\right) \d \zbar \left(\eps_{mn} (\br z - \br w) (\d \br x_m - \d \br y_m) (\d \br x_n - \d \br y_n) \right)
\d \br w \left(\ep_{pq} \br y_p  \d \br y_q\right) = 
\# [\br x \br y] (\br z - \br w) \d \br u \d^3 X \d^3 Y .
\end{multline}

Recall that the BCOV propagator is a bivector.
With the choice of input functions, only certain $\del^2$-components of the propagator will contribute nontrivially.
For example, if we take the $\del_{u-z} \del_{x_i}$ component of $\bP(U,X)$ then only the component $\del_{x_1 - y_1} \del_{x_2-y_2}$ component of $\bP(X,Y)$ may contribute nontrivially.
This combination contributes the analytic factor $\ep_{ij} \br x_j (\br z - \br w)$ to the integral, since these appear with opposite signs when $i=1,2$ these contributions cancel.
If we take the $\del_{x_1} \del_{x_2}$ component of $\bP(U,X)$ then only the $\del_{z-w} \del_{x_1-y_1}$ component of $\bP(X,Y)$ may contribute nontrivially.
This contributes the analytic factor $(\br u - \br z) (\br x_2 - \br y_2)$ to the integral.

In total, we see that the integral simplifies to $\d \br u$ times 
\beqn
\int_{X,Y} \frac{(\br u - \br z) (\br z - \br w) [\br x \br y] (\br x_2 - \br y_2) f(X) g(Y)}{\|U-X\|^8 \|X-Y\|^8 \|Y\|^4} \d^3 X \d^3 Y
\eeqn
%We remark that the factor $(\zbar-\wbar)^2$ comes from the BCOV propagator.

We are then interested in the following integral, which occurs as a special case of the formula we derived in Appendix \ref{appx:integral}:
\begin{equation}
    \tau := \underset{(\mathbb{C}^3)^2}{\int} \frac{ [\overline{x},\overline{y}] (\overline{z}-\overline{x}^0) (\overline{x}^0-\overline{y}^0) (\overline{x}^2-\overline{y}^2) f(X) g(Y)}{(\vert \vert Z-X \vert \vert^2)^{4} (\vert \vert X-Y \vert \vert^2)^{4} (\vert \vert Y-W \vert \vert^2)^{2}} d^6X d^6Y
\end{equation}
where $Z = (z,0^{\dot{\alpha}}), X = (x^0,x^{\dot{\alpha}}), Y = (y^0,y^{\dot{\alpha}})$, $W = (0,0^{\dot{\alpha}})$, and $f(X),g(Y)$ are holomorphic test functions of $X,Y$ respectively. \\ \\
Fix the test functions $f(X) = x^0 (x^2)^2$ and $g(Y) = y^1$, and define $\vec{j} = (4,4,2), \vec{k} = (1,0,2), \vec{m} = (0,1,0), \vec{l} = (l_0,0,l_2), \vec{n} = (n_0,0,n_2)$. Define the following two sets, $L$ and $N$:
\begin{equation}
    L = \{ (1,0,1),(0,1,-1), (1,1,1), (2,0,-1)\} \quad \quad N = \{(1,0,1),(0,1,-1)\}
\end{equation}
Then, our integral, $\tau$, is given by a linear combinations of our results in eq.(\ref{eq: simplified general result}):
\begin{equation}
    \tau = \sum_{(l_0,n_0,q_0) \in L} \sum_{(l_2,n_2,q_2) \in N} (-1)^{q_0+q_2}\mathcal{I}(\vec{j};\vec{k},\vec{l};\vec{m}, \vec{n}).   
\end{equation}

\textcolor{red}{To-do: check about to make sure there is a cancellation with $\bar{z}- \bar{w}$ from earlier so that we have holomorphic poles; check and make sure I didn't skip steps from splicing things together too fast and that this still makes sense and is logical to the reader; go back into br section and summarize this result}


%and consider the integral over $X\in \C^3$:
%\begin{multline}
%\int_{X \in \C^3} \frac{(\br u - \br z) (\br z - \br w) [\br x \br y] (\br x_2 - \br y_2) x_2^2 z}{\|U - X\|^8 \|X-Y\|^8} \\= \int_{t=0}^1 \d t \int_{X \in \C^3} \frac{t^3 (1-t)^3}{\left(\|\til X\|^2 + t(1-t) \|U - Y\|^2\right)^8} \\ \times \left(\br {\til z} + (1-t)(\br w-\br u)\right) \left(\br {\til z} - t(\br w-\br u)\right) [(\br{\til x} + (1-t) \br y)] (\br {\til x}_2 + (1-t)y_2) ((\br {\til x}_2 + (1-t)y_2)^2 (\til z + t u + (1-t) w).
%\end{multline}
%
%\newpage
%
%\begin{multline}
%\int_{Y \in \C^3} \d^6 Y \frac{[\br x \br y] \br y_i y_j y_k}{\|X-Y\|^8 \|Y-W\|^4} \\
%= {5! \over 3!} \int_0^1 \d t \, t^3 (1-t) \int_{\C^2} \d^6 \til Y \frac{[\br x \br y] \br y_i y_j y_k}{(\|\til Y\|^2 + t(1-t)\|X-W\|^2)^6} .
%\end{multline}
%Where we have introduce the new variable $\til Y = Y - t X - (1-t)W$.
%Then, the right hand side becomes
%\beqn
%20 \int_0^1 \d t \, t^3 (1-t) \int_{\C^2} \d^4 \til y \frac{[\br x (\br {\til y} + t \br x)]  (\br {\til y}_i+t \br x_i) (\til y_j+tx_j)(\til y_k+tx_k)}{(\|\til Y\|^2 + t(1-t)\|X-W\|^2)^6} 
%\eeqn
%As above, only the $U(1)^2$-invariant part of the numerator will contribute nontrivially to this integral.
%There are two cases, $i=j$ or $i=k$.
%Suppose without loss of generality that $i=k$, then after computing the residue along both the $\til y_1$ and $\til y_2$ directions the integral becomes (up to a nonzero factor independent of $x,z,w$):
%\beqn
%\br x_i \int_{0}^1 \d t \, t^3 (1-t) \int_{(0,\infty) \times (0,\infty)} \d^2 \rho \,\frac{\rho_i \rho_j}{(\rho_1 + \rho_2 + \rho_3 + t(1-t)\|X-W\|^2)^6} .
%\eeqn
%Integrating over $(0,\infty)^3$ and then over $t \in (0,1)$, this is a nonzero multiple\footnote{When $i=j$ there is an additional factor of $1/2$, but since we are surpressing constants we will omit this special case} of
%\beqn
%\br x_i \frac{1}{\|X-W\|^2} .
%\eeqn
%
%It remains to compute the integral
%\beqn
%\int_{0}^1 \d t \, t (1-t) \int_{\C^2} \d^4 x \, \frac{\br x_i f(x)}{(|z'-z|^2 + \|x\|^2)^4 (|z-w|^2 + (1-t)\|x\|^2)^2} ,
%\eeqn
%which is nonzero only when $f$ is a scalar multiple of the linear function $x_i$.
%
%
%We can use Feynman's trick to write this integral as
%\beqn
%\eeqn
\end{document}
