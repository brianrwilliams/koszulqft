\documentclass[../main.tex]{subfiles}

\begin{document}

\section{Twisted supergravity in six dimensions}

\subsection{Generalities on twisting}

\subsection{Twisted supergravity}

\subsection{Kodaira--Spencer theory and IIB supergravity}

We begin with a description of the holomorphic twist of type IIB supergravity in ten dimensions.
The holomorphic supercharge used to minimally twist supergravity is invariant under $SU(5) \subset Spin(10)$, and so can be defined on any Calabi--Yau fivefold $X$.
In \cite{CLsugra}, it was conjectured that the holomorphic twist of IIB supergravity is equivalent to a certain truncation of the topological $B$-model on $X$.\footnote{This truncation was referred to as `minimal' Kodaira--Spencer theory in \textit{loc. cit.}. 
It effectively throws out on the non-propagating fields.}
In \cite{SWspinor} this conjecture was verified at the level of the free limit of type IIB supergravity.
We will assume this conjecture throughout the paper, and we will provide further justification in section \ref{s:components}.

The fields of Kodaira--Spencer theory on the Calabi--Yau fivefold $X$ are given in terms of the Dolbeault complex of polyvector fields on $X$; that is, sections of exterior powers of the holomorphic tangent bundle with values in $(0,\bu)$ Dolbeault forms:
\beqn
\PV^{i,j} (X) = \Omega^{0,j}(X, \wedge^i T X) .
\eeqn
In local holomorphic coordinates $z_1,\ldots,z_5$ such a polyvector field can be expressed as
\beqn
\mu = \mu_{j_1 \cdots j_5}^{\ibar_1\cdots \ibar_5} \d \zbar_{\ibar_1} \cdots \d \zbar_{\ibar_5} \del_{z_{j_1}} \cdots \del_{z_{j_5}} . \; \footnote{We will always omit the wedge product symbol $\wedge$.}
\eeqn

It is convenient to express polyvector fields in terms of a single super-field.
To do this, we rename $\d \zbar_{\ibar}$ as $\br \theta_{\ibar}$ and $\del_{z_{j}}$ as $\theta^j$. 
Bear in mind that $\theta$ transforms as a holomorphic vector while $\br \theta$ transforms as an anti-holomorphic covector.
With this notation, a general polyvector field 
\beqn
\mu \in \PV (X) = \oplus_{i,j} \PV^{i,j} (X)
\eeqn
can be thought of as a smooth function
\beqn
\mu = \mu(z_i, \zbar_{\ibar}, \theta^i, \br \theta_{\ibar})
\eeqn
on the superspace $\C^{5|5+5}$ where the odd cordinates are $\theta^i, \br \theta_{\ibar}$ for $i,\ibar=1,\ldots,5.$

The space of fields of Kodaira--Spencer theory is not all polyvector fields: rather, the fields are polyvector fields which satisfy the constraint that they are divergence-free with respect to the holomorphic volume form $\Omega$.
Geometrically, this means that $L_\mu \Omega = 0$ where $L_\mu$ is the Lie derivative \footnote{We recall that the Lie bracket on polyvector fields is the Schouten bracket, which reduces to the usual Lie bracket on ordinary vector fields.}; equivalently this is the condition $\div \mu = 0$ where $\div$ is the divergence operator.
In coordinates this reads
\beqn
\div = \sum_i \del_{\theta^i} \del_{z_i} .
\eeqn
In addition to $\div \mu = 0$ we also require that 
\beqn
\del_{\theta^1} \cdots \del_{\theta^5} \mu = 0,
\eeqn
which effectively throws away the top power of $T_X$. We will justify this additional condition shortly.

To define the action functional we utilize an integration map
\beqn\label{eqn:cyintegral}
\int^\Omega_X \colon \PV^{5,5}(X) \simeq C^\infty(X) \theta^1 \cdots \theta^5 \br \theta_1 \cdots \br \theta_5 \to \C
\eeqn
which is simply $\int ( \mu \vee \Omega) \wedge \Omega$, with $\Omega$ the Calabi--Yau form.
In terms of the superspace description this is simply the usual integration along $X$ together with the Berezinian integral along the odd directions.

Appearing in the Lagrangian is a non-local term involving a term proportional to $\div^{-1} \mu$.
While this is not globally well-defined, the condition that $\mu$ be in the kernel of $\div$ ensures that there exists locally such a polyvector field.

In summary, the fields of Kodaira--Spencer theory are
\beqn
\PV(X) \cap \ker \div .
\eeqn
The Lagrangian is
\beqn
\frac12 \int^\Omega_X \mu \dbar \del^{-1} \mu + \frac16 \int^\Omega_X \mu^3 
\eeqn

The conjecture originally put forth in \cite{CLsugra} is that this Lagrangian captures the supersymmetric sector of IIB supergravity as described above.
The super-field $\mu$ captures all the original fields, anti-fields, ghosts, etc. of type IIB supergravity after integrating out those fields which become massive in the holomorphic twist.
Since the field $\mu$ includes anti-fields and anti-ghosts, we can describe the BV anti-bracket in this notation.
The BV anti-bracket of two super-fields is
\beqn
\{\mu(z,\zbar,\theta,\br \theta), \mu(w,\wbar,\eta,\br \eta)\} = \del_{z_i} \del_{\theta^i} \delta(z-w) \delta(\zbar-\wbar) (\br \theta - \br \eta) (\theta - \eta) \text{Id} .
\eeqn
The appearance of the holomorphic derivative $\del_{z_i}$ in the expression above is one way to understand the appearance of the non-local kinetic term in the Lagrangian.

From this BV anti-bracket it is clear that the component of the super-field $\mu$ proportional to the top polyvector $\del_{\theta^1} \cdots \del_{\theta^5}$ does not propagate. 
It is therefore convenient to impose the additional constraint 
\beqn
\del_{\theta^1} \cdots \del_{\theta^5} \mu = 0 
\eeqn
on the fields of Kodaira--Spencer theory, as mentioned earlier.

We can avoid part of the non-locality appearing in the action by introducing a field $\Hat{\mu}_{i_1\cdots i_4} \in \PV^{4,\bu}$ which satisfies 
\beqn\label{eqn:potentialC5}
(\div \Hat{\mu})_{i_1 i_2 i_3}^\bu = \mu_{i_1 i_2 i_3}^\bu ,
\eeqn
where the bullet denotes arbitrary anti-holomorphic form type.
We can do this because we have the constraint $\div \mu = 0$.
Then, the kinetic term in the Lagrangian above can be written as 
\beqn
\label{eqn:kineticks}
\int \eps^{i_1 \cdots i_5} \eps_{\br j_1 \cdots \br j_5} \mu_{i_1} \dbar \Hat{\mu}_{i_2\cdots i_5} + \frac12 \int \eps^{i_1 \cdots i_5} \mu_{i_1 i_2} (\dbar \del^{-1}_\Omega \mu)_{i_3 i_4 i_5} .
\eeqn
This Lagrangian is still non-local, but the only non-locality involves the field $\PV^{2,\bu}(X)$.
We will see the significance of this field from the perspective of supergravity in the next subsection.
%\beqn
%\int \eps^{i_1 \cdots i_5} \eps_{\br j_1 \cdots \br j_5} \mu_{i_1}^{\br j_1} \del_{\zbar_{\jbar_2}} \Hat{\mu}_{i_2\cdots i_5}^{\jbar_3 \cdots \jbar_5} + \frac12 \int \eps^{i_1 \cdots i_5} \eps_{\br j_1 \cdots \br j_5} \mu_{i_1 i_2}^{\br j_1 \br j_2} (\dbar \div^{-1} \mu)_{i_3 i_4 i_5}^{\br j_3 \br j_4 \br j_5} + \cdots 
%\eeqn
%where the $\cdots$ denote permutations of the placement of anti-holomorphic indices.

\subsection{Matching supergravity with Kodaira--Spencer theory}

At the level of free fields, the match between the holomorphic twist of type IIB supergravity on $\R^{10}$ and Kodaira--Spencer theory has been done in \cite{SWspinor}.
Here, we spell out a precise relationship between the fields of Kodaira--Spencer theory and those of supergravity.
For simplicity we will work on flat space near the flat K\"ahler metric $g^{i \jbar}_0 = \delta^{i \jbar}$.

%In type IIB supergravity there is a subsector known as type I supergravity. 
%This theory is equipped with $N=(1,0)$ supersymmetry rather than $N=(2,0)$ supergravity for IIB supergravity.
%This theory is anomalous at one-loop, but we will only be concerned with the classical limit.
%In terms of Kodaira--Spencer theory, the fields of type I supergravity correspond to $\PV^{1,\bu}(\C^5)$ and $\PV^{3,\bu}(\C^5)$ whose fields we denote by $\mu_{i}^\bu$, $\mu_{ijk}^\bu$ respectively.

The most important bosonic field in supergravity is, of course, the metric tensor. 
As representations of $SU(5)$, the metric tensor breaks into three components: $g^{ij}, g^{i \jbar}, g^{\ibar \jbar}$.
To leading order, the components $g^{ij}, g^{i \jbar}$ are rendered massive in the twist and can hence be removed.
The remaining component of the metric corresponds to the field $\mu_{k}^{\jbar}$ in Kodaira--Spencer theory via the K\"ahler form
\beqn
g^{\ibar \jbar} \mapsto \delta^{k \ibar} \mu_k^{\jbar} .
\eeqn

The fermionic fields of type IIB supergravity include a gravitino.
In the untwisted theory the gravitino has a spinor index and a vector index.
As an $SU(5)$ representation, the 16-dimensional spinor representation $S_+$ of $SO(10)$ decomposes as a sum of three irreducible representations: the trivial representation, the exterior square of the anti-fundamental representation, and the fourth exterior power of the anti-fundamental representation:
\beqn
S_+ \simeq_{SU(5)} \C \oplus \wedge^2 \br \C^5 \oplus \wedge^4 \br \C^5 .
\eeqn
The component which survives the twist is the holomorphic vector valued in the exterior square in the above equation, and we denote this field by
\beqn
\lambda_{i}^{\jbar_1 \jbar_2} ,
\eeqn
which we can view as an element $\PV^{1,2} (\C^5)$.

The antifield to the component of the gravitino $\lambda_i^{\jbar_1 \jbar_2}$ is a tensor of the form $\lambda_{\br l_1 \br l_2}^{* k}$, where the $*$ just indicates that this is an anti-field in the physical theory.
Since the gravitino is an odd field, its anti-field has overall even parity.
It turns out that it is the derivative of this anti-field that corresponds to a field of Kodaira--Spencer theory
\beqn
\del_{z_{k_1}} \lambda^{* k_2}_{\br l_1 \br l_2} \mapsto \eps^{k_1 k_2 i_1 i_2 i_3} \eps_{\br l_1 \br l_2 \br j_1 \br j_2 \br j_3} \mu^{\br j_1 \br j_2 \br j_3}_{i_1 i_2 i_3} .
\eeqn
That is, we view the derivative of the anti-field as an element of $\PV^{3,3}$.
Following the discussion above, we can use the equation $\div \mu=0$ to replace the field $\mu^{\br j_1 \br j_2 \br j_3}_{i_1 i_2 i_3}$ by a field $\Hat{\mu}$ satisfying 
\beqn
\mu^{\br j_1 \br j_2 \br j_3}_{i_1 i_2 i_3} = \del_{z_j} \Hat{\mu}^{\br j_1 \br j_2 \br j_3}_{j i_1 i_2 i_3} .
\eeqn
Note that $\Hat{\mu}^{\br j_1 \br j_2 \br j_3}_{j i_1 i_2 i_3}$ is a field of type $\PV^{4,3}$.
Using this modified field in Kodaira--Spencer theory, we can more easily match with the anti-gravitino via
\beqn
\lambda^{* k}_{\br l_1 \br l_2} \mapsto \eps^{k i_1 i_2 i_3 i_4} \eps_{\br l_1 \br l_2 \br j_1 \br j_2 \br j_3} \mu^{\br j_1 \br j_2 \br j_3}_{k i_1 i_2 i_4} .
\eeqn

Next, let us explicitly match the holomorphic twist of type IIB supergravity with Kodaira--Spencer theory at the level of the kinetic term in the Lagrangian.
In \eqref{eqn:kineticks} we have expressed the kinetic term in the Kodaira--Spencer action as a sum of two terms.
We first show how there is a similar kinetic term involving the metric $g$ and the anti-field to the gravitino $\lambda^*$ when we twist type IIB supergravity.

Recall that the holomorphic twist amounts to assigning a certain component of the superghost a nontrivial VEV. As an $SU(5)$ representation the superghost $Q$ can be written as a sum of three tensors $Q^{(0)}, Q^{\br j_1 \br j_2}, Q^{\br j_1 \cdots \br j_4}$, which are the components of the even exterior powers of the anti-fundamental representation of $SU(5)$.
Here $Q^{(0)}$ denotes the $SU(5)$ invariant component of the superghost in the $\cN=(1,0)$ subalgebra; this is the component in which the holomorphic supercharge lives.
A term in the BV action involving $\lambda^*$ and $Q$ arises from a supersymmetric variation of the gravitino $\lambda$.

Reverting back to $SO(10)$ notation, where $a,b=1,\ldots,10$ are vector indices and $\alpha,\beta,\ldots=1,\ldots,32$ are spinor indices, the supersymmetric variation of the gravitino is of the form
\beqn
\delta \lambda_a^\alpha = \delta_{ab} (\del_{x_b} \eps^\alpha + A_{\beta}^{\alpha b} (g) \eps^\beta).
\eeqn
Here $A$ is the spin Levi-Civita tensor in the spin representation of $Spin(10)$.\footnote{We use $A$ instead of $\Gamma$ for the Levi-Civita connection to avoid confusion with $\Gamma$-matrices.}
Taking a perturbative expansion of the flat metric of the form $\delta^{ab} + g^{ab}$ and working to low order in $g^{ab}$, we can write the ordinary Levi-Civita connection as
\beqn
A_a^{bc} = \frac12 \delta_{ad} (\del_{x_c} g^{bd} + \del_b g^{cd} - \del_d g^{bc}) + O(g^2) .
\eeqn
In terms of this ordinary Levi-Civita connection, the spin Levi-Civita connection can be written in terms of $\Gamma$-matrices as
\beqn\label{eqn:christ}
A_{\beta}^{\alpha b} = \Gamma_c^{\alpha \gamma} \Gamma_{\beta \gamma}^a A_a^{bc} .
\eeqn
We are interested in the covariant derivative of the constant spinor $\eps^{(0)}$.

As before, a spinor decomposes, as an $SU(5)$ representation, into a sum of even exterior powers of the anti-fundamental representation.
The index $(0)$ represents the $SU(5)$ invariant part of the spinor.
A simple computation with $\Gamma$-matrices shows that the components of the spin Levi-Civita connection whose lower index is $(0)$ and upper index is $(\br i \br j)$ are
\begin{align*}
A_{(0)}^{(\br i \br j)k} = A_j^{\br i k} \delta^{j \br j} \\
A_{(0)}^{(\br i \br j) \br k} = A_j^{\br i \br k} \delta^{j \br j} 
\end{align*}
where the ordinary Christoffel symbols appear on the right hand side (with $SU(5)$ indices).

Plugging in \eqref{eqn:christ} we see that the desired variation of the gravitino is
\begin{align*}
\delta \lambda_k^{\br i \br j} & = \delta_{k \br k} A_{(0)}^{(\br i \br j) \br k} \eps^{(0)} \\
& = \delta_{k \br k} \delta^{j \br j} A_{j}^{\br i \br k} \eps^{(0)} \\
& = \frac12 \delta_{k \br k} \delta^{j \br j} \delta_{j \br l} \left(\delta_{\zbar_{\br k}} g^{\br l \br i} + \del_{\zbar_{\br i}} g^{\br l \br k} - \del_{\zbar_{\br l}} g^{\br i \br k} \right) \eps^{(0)} \\
& = \frac12 \delta_{k \br k} \left(\delta_{\zbar_{\br k}} g^{\br j \br i} + \del_{\zbar_{\br i}} g^{\br j \br k} - \del_{\zbar_{\br j}} g^{\br i \br k} \right) \eps^{(0)} \\ & = \eps^{\br i \br j} \delta_{k \br k} \del_{\zbar_{\br i}} g^{\br j \br k} \eps^{(0)} .
\end{align*}
In the last line we have used the fact that $\br i, \br j$ appear anti-symmetrically on the left hand side.
It follows that once we assign a nonzero VEV to the superghost $Q^{(0)}$ in the BV action there is a term of the form
\beqn
(\del_{\zbar_{\br k}} g^{\br i \br j} \delta_{l \br i}) \lambda^{* l}_{\br k \br j} .
\eeqn
This matches precisely with the first term in the Kodaira--Spencer kinetic action.

The final fields we describe in terms of the holomorphic twist are the Ramond--Ramond fields in supergravity.
These fields are magnetically sourced by $D(2k-1)$-branes and hence are forms of degree $8-2k$.
In the original presentation of Kodaira--Spencer theory, certain components of the field strengths of such forms are present as polyvector fields.
The field strength is a form of degree $9-2k$; in the holomorphic twist the component of this form which survives is of Hodge type $(5-k,4-k)$ and corresponds to polyvector field of type $(k,4-k)$ using the isomorphism
\beqn
\PV^{k,4-k} (\C^5) \simeq_\Omega \Omega^{5-k,4-k} (\C^5) \subset \Omega^{9-2k} (\R^{10}) \otimes \C
\eeqn
determined by the Calabi--Yau form.

A special Ramond--Ramond form is the four-form $C \in \Omega^4(\R^{10})$ sourced by a $D3$-brane.
Such a field is required to be `chiral' in the sense that its field strength $F = \d C$ is self-dual.
The component of the field strength
\beqn
F^{\ibar_1 \ibar_2 j_1 j_2 j_3} \in \Omega^{3,2}(\C^5)
%, F^{\ibar_1 j_1 j_2 j_3 j_4}, F^{j_1 j_2 j_3 j_4 j_5} \in \Omega^{3,2}(\C^5) \oplus \Omega^{4,1}(\C^5) \oplus \Omega^{5,0} (\C^5)
\eeqn
survives the holomorphic twist.
Using the holomorphic volume form, these components are identified with the fields
\beqn
F^{\ibar_1 \ibar_2 j_1 j_2 j_3} \mapsto \eps^{j_1 j_2 j_3 j_4 j_5} \mu_{j_4 j_5}^{\ibar_1 \ibar_2}
%F^{\ibar_1 j_1 j_2 j_3 j_4} & \mapsto \eps^{j_1 j_2 j_3 j_4 j_5} \mu_{j_5}^{\ibar_1 \ibar_2 \ibar_3 \ibar_4}
\eeqn
which is a polyvector field of type $(2,2)$.
Self-duality becomes the constraint $\del_{j} \mu_{j k}^{\ibar_1 \ibar_2} = 0$ that this polyvector field be divergence-free.
This constraint gives rise to the non-local kinetic term present in equation \eqref{eqn:kineticks}.
For more on the relationship between constraints and non-local kinetic terms we refer to \cite{SWconstraint}.

This concludes our general discussion of the twist of ten-dimensional type IIB supergravity in terms of Kodaira--Spencer theory.
We now turn to compactifications as understood in the twist. 

\subsection{Compactification of Kodaira--Spencer theory}

%Recall that $\PV (X) = \PV^{\bu,\bu}(X)$ denotes the graded vector space given by the Dolbeault resolution of holomorphic polyvector fields on a complex manifold~$X$.

Let $Y$ be a complex surface (which we will soon take to be compact) with a fixed holomorphic symplectic structure.
A general field of Kodaira--Spencer theory on $\C^3 \times Y$ is a Dolbeault-valued polyvector field which is annihilated by the divergence operator with respect to the holomorphic volume form.
A Dolbeault-valued polyvector field $\alpha^{k,\bu}$ on $\C^3 \times Y$ of type $(k,\bu)$ can be written as a tensor product of one on $\C^3$ with one on $Y$
\beqn
\alpha^{k,\bu} = \sum_{i+j=k} \beta^{i,\bu} \otimes \gamma^{j,\bu} 
\eeqn
where $\beta^{i,\bu},\gamma^{j,\bu}$ are polyvector fields of type $(i,\bu),(j,\bu)$ on $\C^3$, $Y$ respectively.
Polyvector fields on $Y$ are the same as differential forms, because the holomorphic symplectic form on $Y$ identifies the tangent and cotangent bundles. 
In particular, the harmonic polyvector fields are given simply by the de Rham cohomology of $Y$.  
Furthermore, polyvector fields on $Y$ which are harmonic are automatically in the kernel of the divergence operator $\div$, by standard Hodge theory arguments.   
To summarize, there is an equivalence of graded algebras
\[
\PV (\C^3) \otimes \bigg(\ker \div |_{\PV(Y)} \bigg) \simeq \PV(\C^3) \otimes H^\bu(Y) .
\]
We will use this equivalence to describe the fields of the theory on $\C^3$ upon compactification along $Y$.

Let $A = H^\bu(Y)$ denote the cohomology ring of $Y$ \footnote{This $A$ is not to be confused with our notation for the Levi-Civita connection above; we trust that the distinction is clear from context.}.
We are mostly interested in the case that $Y$ is a K3 surface, in which case this algebra is generated by even elements $\eta, \br \eta, \eta_a$ for $a=1,\ldots 20$ subject to the relations
\beqn
\label{eqn:K3rel}
\begin{split}
\eta^2 & = \Bar{\eta}^2 = 0 \\
\eta_a \eta_b & = h_{ab} \eta \Bar{\eta} 
\end{split}
\eeqn
where $h_{ab}$ is a non-degenerate symmetric pairing on $\C^{20}$. 
Let $I$ denote the ideal generated by these equations so that $A = \C[\eta,\Bar{\eta}, \eta_a] / I$. 

As before, we write the polyvector fields on $\C^3$ in terms of a superspace by introducing odd variables $\theta^i$, $\br{\theta}_{\br{j}}$.  
Here, $\theta^i$ represents the coordinate vector field $\partial_{z_i}$ and $\br \theta_{\br{i}}$ represents the coordinate Dolbeault form $\d \zbar_{\br{i}}$. 
Then we can write the field content as a collection of superfields
\begin{equation} 
		\mu(z,\zbar,\theta^i, \br{\theta}_{\br{i}},\eta) \in \oplus_{i,j}  \PV^{i,j}(\C^3) \otimes A .
\end{equation}
Here, we are using the shorthand $\eta$ to inform that there is a dependence on $\eta, \br{\eta}$, and $\eta_a$, $a=1,\ldots, 20$. 
As such, such a superfield decomposes in its dependencies on the generators of the cohomology of $Y$ as
\begin{multline}
\mu(z,\zbar,\theta^i, \br{\theta}_{\br{i}}) \\
+ \mu_\eta (z,\zbar,\theta^i, \br{\theta}_{\br{i}}) \eta + \mu_{\br{\eta}} (z,\zbar,\theta^i, \br{\theta}_{\br{i}}) \br{\eta} + \mu^a (z,\zbar,\theta^i, \br{\theta}_{\br{i}}) \eta_a \\
+ \mu_{\eta \Bar{\eta}} (z,\zbar,\theta^i, \br{\theta}_{\br{i}}) \eta \Bar{\eta} .
\end{multline}
We emphasize that the $\eta$-variables represent harmonic polyvector fields on $Y$ and so are not acted on by any differential operators along $\C^3$. 

%The fields on $\C^3$ must satisfy the constraint 
%\begin{equation} 
%	\partial_{z_{i_1}} \mu_{i_1 \dots i_l; a}^{\br{j}_1 \dots \br{j}_k} = 0. 
%\end{equation}
The superfield satisfies the equation $\partial \mu = 0$ \footnote{For notational simplicity, we will no longer make manifest the dependence of the divergence operator on $\Omega$.}
where, in the superspace formulation,
\begin{align} 
	\dbar & \, = \, \br{\theta}_{\br{j}} \partial_{\zbar_{\br{j}}} \\
	\partial &\, = \, \partial_{\theta^i} \partial_{z_i}.  
\end{align}
We denote by
\beqn
\int^\Omega_{\C^3,\eta} (-)|_{\eta \br \eta} \colon \PV^{3,3} \otimes A \to \eta \br \eta \PV^{3,3}  \to \C
\eeqn
the projection onto the summand $\C \eta \br \eta \subset A$ followed by integration as in \eqref{eqn:cyintegral}.

The Lagrangian is
	\begin{equation} 
		\tfrac{1}{2} \int_{\C^{3}}^\Omega  \mu \dbar \partial^{-1} \mu |_{\eta \br{\eta}} + \frac{1}{6} \int_{\C^{3}}^\Omega \mu^3 |_{\eta \br{\eta}}  
	\end{equation}
where the $(-)|_{\eta \br{\eta}}$ means we pick up only the $\eta \br{\eta}$ component.

%Before we turn to the computation of the backreaction, 
We can simplify the field content somewhat, following \cite{CostelloGaiotto} which the authors in \cite{CLsugra} refer to as minimal Kodaira--Spencer theory.
We note that the coefficient of $\theta^1 \theta^2 \theta^3$ does not appear in the kinetic term in the action.  
This field does not propagate, so we can (and will) impose the additional constraint
\begin{equation}\label{eq:nonprop} 
	\partial_{\theta^1} \partial_{\theta^2} \partial_{\theta^3} \mu (z,\zbar,\theta,\br{\theta},\eta) = 0. 
\end{equation}
This constraint only removes a single topological degree of freedom and hence will not significantly modify quantities like OPE's later on.

Next, let us expand the superfield $\mu$ only in the $\theta^i$ variables:
\begin{equation} 
	\mu = \mu(z,\zbar,\br{\theta},\eta) + \mu_{i}(z,\zbar,\br{\theta},\eta) \theta^i + \dots 
\end{equation}
We note that the constraint $\partial \mu_{ij} = 0$ implies that there is some super-field
\begin{equation} 
	\what{\mu}_{ijk}(z,\zbar,\br{\theta},\eta) = 	\alpha(z,\zbar,\br{\theta},\eta) \eps_{ijk}   
\end{equation}
so that $\partial_{z_i} \what{\mu}_{ijk} = \mu_{jk}$. 
This is parallel to the maneuver that we made for Kodaira--Spencer theory on $\C^5$ as in \eqref{eqn:potentialC5} above.

It is convenient to rephrase the theory in terms of the field $\alpha(z,\zbar,\br{\theta},\eta)$, which has no holomorphic index. 
We will also change notation and let $\gamma(z,\zbar,\br{\theta},\eta)$ be the term with no $\theta^i$ dependence in the superfield $\mu(z,\zbar,\theta,\br{\theta},\eta)$.  

In summary, we have the following fundamental superfields in the compactified theory on $\C^3$:
\begin{itemize}
\item $\mu_i (z,\zbar,\br{\theta}, \eta) \theta^i$ which we identify with an element in the graded space
\beqn
\mu \in \PV^{1,\bu}(\C^3) \otimes A [1] .
\eeqn
\item $\alpha (z,\zbar,\br \theta, \eta)$ which we identify with an element of the graded space
\beqn
\alpha \in \Omega^{0,\bu}(\C^3) \otimes A .
\eeqn
\item $\gamma(z,\zbar,\br \theta, \eta)$ which we also identify with an element of the graded space
\beqn
\gamma \in \Omega^{0,\bu}(\C^3) \otimes A [2] .
\eeqn
\end{itemize}
We explain the cohomological shifts in the next paragraph.
In terms of these fields, the Lagrangian is
\begin{multline}\label{eqn:K3action}
	\tfrac{1}{2}\int_{\C^{3}}   \eps^{ijk} \dbar \mu_{i} (\partial^{-1}  \mu)_{jk} \, \d^3 z |_{\eta \br \eta}   + \int_{\C^{3}}  \alpha \dbar \gamma \d^3 z  |_{\eta \br{\eta}} 
	\\
	+ \tfrac{1}{6} \int_{\C^{3}}  \eps_{ijk} \mu_{i}\mu_{j} \mu_{k} \, \d^3 z |_{\eta \br{\eta}} + \int_{\C^{3}}  \alpha \mu_i \partial_{z_i}  \gamma \, \d^3 z|_{\eta \br{\eta}} .
\end{multline} 
In this expression we project onto the component $\eta \br \eta$ as before.

Just as when we twist a field theory, when we twist a supergravity theory the ghost number of the twisted theory  is a mixture of the ghost number and a $U(1)_R$-charge of the original physical theory. To define a consistent ghost number, one can choose any $U(1)_R$ in the physical theory under which the supercharge has weight $1$.  In general, there are many ways to do this.  It is convenient for us to make the following assignments of ghost number.
\begin{enumerate} 
	\item The fermionic variables $\eta_a$ have ghost number $0$.
	\item The anti-commuting variables $\br{\theta}_i$ have ghost number $1$.
	\item The field $\alpha$ has ghost number zero.
	\item The field $\mu$ has ghost number $-1$ (so that the $\br \theta_i$ component has ghost number zero.
	\item The field $\gamma$ has ghost number $-2$ (so that the $\br \theta_i \br \theta_j$ component of $\gamma$ has ghost number zero).
	%\item The fields $\alpha$, $\gamma$ have ghost number $-1$, and are fermionic.
\end{enumerate}
With these choices one can check that the action \eqref{eqn:K3action} is ghost number zero.
Note that in the case $A = \C$ the choices of ghost numbers we take here is in agreement of the standard presentation of Kodaira--Spencer theory on $\C^3$ as in \cite{CostelloGaiotto}.

\subsection{Compactification and twisted multiplets}

We will next discuss the twists of matter multiplets in our compactified twisted supergravity theory. It will be useful to warm up with the twist of $\cN=(1, 0)$ supergravity before moving on to the $\cN=(2, 0)$ case we are most interested in.

In six-dimensional $\cN=(1,0)$ supersymmetry there are four multiplets which appear in the compactifications we will discuss: (i) the graviton multiplet, (ii) the vector multiplet, (iii) the tensor (or chiral two-form \cite{WittenM5}) multiplet, and (iv) the hypermultiplet.
Through the work in \cite{...}, the holomorphic twists of each of the theories associated to each of these multiplets has been characterized.
By virtue of their holomorphicity, each theory shares a linear gauge symmetry by the $\dbar$ operator, schematically of the form $\delta \Phi = \dbar \Phi$.
% (though the twist of the tensor multiplet has an additional gauge symmetry).

Here, we recall the field content of each of the twisted six-dimensional multiplets, whose origin we will review in more detail below.
\begin{itemize}
\item[(i)] The holomorphic twist of the the graviton multiplet has two fundamental fields
\beqn
(\mu, \rho, \til \alpha) \in \left(\PV^{1,\bu}(\C^3)[1] \cap \ker \div \right)^{\oplus 3} .
\eeqn
\item[(ii)] The holomorphic twist of the vector multiplet is three-dimensional holomorphic BF theory with two fundamental fields
\beqn
(A,B) \in \Omega^{0,\bu}(\C^3)[1]
\eeqn
\item[(iii)] The holomorphic twist of the tensor multiplet has as its underlying complex of fields
\beqn
\alpha \in \left(\Omega^{2,\bu}(\C^3)[1]\right) \cap \ker \del .
\eeqn
%The arrow indicated there is an additional gauge transformation for the $(3,\bu)$-form $\delta^{hol} v = \del \alpha$.
\item[(iv)] The holomorphic twist of the hypermultiplet is the higher dimensional $\beta-\gamma$ system with two fundamental fields
\beqn
(\gamma,\beta) \in \Omega^{0,\bu}(\C^3)^{\oplus 2} .
\eeqn
\end{itemize}

In this paper we are concerned with the compactification of the holomorphic twist of type IIB supergravity on a K3 surface.
Before turning towards IIB supergravity, we can instead look at the simpler type I supergravity.
Its compactification on a K3 surface is a model with 6d $\cN=(1,0)$ supersymmetry, and hence its twist should be expressed in terms of the multiplets above.


In \cite{CLtypeI,SWspinor} it is argued that the holomorphic twist of type I supergravity on a Calabi--Yau fivefold $X$ has two fundamental fields 
\begin{align*}
\mu_{type\,I} \in \Pi \PV^{1,\bu}(X) \cap \ker \div \\
\rho_{type\,I} \in \Pi \PV^{3,\bu}(X) \cap \ker \div ,
\end{align*} where $\Pi$ denotes a fermion parity shift. 
On a fivefold of the form $X = \C^3 \times Y$ where $Y$ is a K3 surface, the field $\mu_{type\,I}$ decomposes as
\beqn
\mu_{type\,I} = (\mu,\alpha; \gamma) \in \left(\PV^{1,\bu}(\C^3)[1] \oplus \PV^{1,\bu}(\C^3)[1] \otimes H^{0,2}(Y)  \right) \cap \ker \div \oplus \Omega^{0,\bu}(\C^3) \otimes H^{1,1}(Y) ,
\eeqn
where the divergence is with respect to the CY form on $\C^3$.
Similarly, if we neglect topological degrees of freedom, the field $\rho_{type\;I}$ decomposes as 
\beqn
\rho_{type\,I} = (\rho, \til \alpha, \beta) \in \left(\PV^{1,\bu}(\C^3)[1] \otimes H^{2,2}(Y) \oplus PV^{1,\bu}(\C^3)[1] \otimes H^{2,2}(Y) \right) \cap \ker \div \oplus \Omega^{0,\bu}(\C^3) \otimes H^{1,1}(Y) .
\eeqn

The fields $(\mu,\rho, \til \alpha)$ comprise the holomorphic twist of the six-dimensional graviton multiplet, denoted (i) above.
The field 
\beqn
\alpha \in \PV^{1,\bu}(\C^3)[1] \cap \ker \div \simeq \Omega^{2,\bu}(\C^3)[1] \cap \ker \del
\eeqn
comprises the holomorphic twist of a single tensor multiplet, denoted (iii) above.
Finally the fields $(\gamma, \beta) = (\gamma_a,\beta_a)$, $a=1,\ldots,20 = \dim H^{1,1}(Y)$ comprise the holomorphic twist of $20$ hyper multiplets, denoted (iv) above.
In particular, we see that in terms of multiplets the compactification of type I supergravity on a K3 surface decomposes as
\beqn
\text{type I supergravity} \rightsquigarrow (i) + (iii) + 20 \, (iv) ,
\eeqn
which is compatible with the description of the K3 compactification of the physical type I supergravity \cite{??}.

\textcolor{red}{We should probably say more here about the passage from type I to type IIB.}

\subsection{Backreaction as an infinitesimal deformation} 
\label{sec:conifold}

From now on we fix the holomorphic coordinates $(z,w_1,w_2)$ on $\C^3$.
We start with type IIB supergravity on $\C^3 \times K3$ and consider a system of $D1$--$D5$ branes where some number of $D1$ branes wrap the complex line $\{w_i=0\}$ in $\C^3$ and a point in $K3$:
\beqn
\{w_i = 0\} \times \{x\} \subset \C^3 \times K3 
\eeqn
and some number of $D5$ branes wrap the same complex line $\{w_i=0\}$ in $\C^3$ together with the entire $K3$ surface:
\beqn
\{w_i=0\} \times K3 \subset \C^3 \times K3 .
\eeqn
The effective open string theory associated to this system of branes will be supported on the intersection of this system which is simply the complex line $\{w_i=0\}$ in $\C^3$.

Using classic results \cite{Dijkgraaf:1998gf}, we can apply a duality to turn this into a D3 brane system which wraps 
\beqn
\C \times 0 \times \Sigma \subset \C^3 \times K3 
\eeqn
for a (special) Lagrangian two-cycle $\Sigma \subset K3$. This follows from the fact that any general D-brane (bound) state on K3 may be described by a Mukai vector $v$, which is a primitive vector such that $F \in \Gamma^{4, 20}, F^2>0$. Any two such vectors of equal length can be related to one another by T-duality transformations in $O(\Gamma^{4, 20})$. Of course, matching the moduli between the two duality frames can be an involved task. For our purposes, we will only need a few basic features in this frame. \footnote{A simple application of these ideas is the following. The positive-definite 4-plane which specifies the hyperk{\"a}hler structure on K3 can be decomposed into two orthogonal 2-planes which amounts to making a choice of complex structure and complexified (by the B-field) K{\"a}hler form. A quaternionic rotation of the 4-plane then exchanges the complex and K{\"a}hler structures, which amounts to a mirror symmetry transformation on the K3 surface. This will exchange the notion of B-branes and A-branes on K3, where B-branes wrap holomorphic curves (with respect to a chosen complex structure) and A-branes wrapping special Lagrangian 2-cycles. This point of view can also be reformulated as an application of the Strominger-Yau-Zaslow \cite{} picture of mirror symmetry as a composition of T-dualities acting on an elliptic fiber.}. As in our setup, B-branes (which, again, are BPS with respect to some chosen $\mathcal{N}=(2, 2)$ subalgebra of the $\mathcal{N}=(4, 4)$ superconformal algebra) on K3 surfaces can wrap not only 2-cycles, but also curves of dimension 0 and 4 (i.e. points or the entire K3 surface).} 

In the last section, we argued that the compactification along a $K3$ surface becomes an extended version of Kodaira--Spencer theory where the extra fields are labeled by the cohomology of the surface.
Upon compactification, the $D3$ system becomes a system of $B$-type branes in this extended version of Kodaira--Spencer theory.

The charge of these branes is labeled by a cohomology class 
\beqn
F \in H^2(K3) \subset A .
\eeqn
 In particular, we take $F$ to be a primitive Mukai vector, as above. We denote 
\beqn
N \define \ip{F, F}
\eeqn
using the inner product on $H^2(Y)$. 
Explicitly, if $F = f \eta + \br f \br \eta + f^a \eta_a$ for $f, \br f, f_a$ complex numbers, then $N = f \br f + f^{a} f^{b} h_{ab}$ where $h_{ab}$ is the fixed non-degenerate symmetric pairing. Then the D-brane charge is related to the number of D1-D5 branes in the original duality frame $N = N_1(N_5-1) \simeq_{N_1, N_5 \rightarrow \infty} N_1 N_5$. (To satisfy the primitivity condition, we assume $N_1, N_5$ are coprime.). \textcolor{red}{im pretty sure about the -1 but i want to double-check.}

Notice that the \textit{length} of the D-brane charge vector $F^2$ is of order $N$, so that anything proportional to the lattice vector itself scales like $\mathcal{O}(N^{1/2})$. We will always work in the supergravity limit in which any formal series in the inverse of these parameters is treated as an asymptotic series.

Generally, the backreaction deforms the geometry away from the locus of the brane. 
Before backreacting, we should say what geometry is actually being deformed. 
In the case of ordinary Kodaira--Spencer theory on $\C^3$, it was shown in \cite{CostelloGaiotto} that the backreaction of B-branes along $\C \subset\C^3$ deformed the complex structure on $\C^3 \setminus \C$ to the deformed conifold, isomorphic to $SL_2(\C)$. 

Our case is similar in that the branes are supported along the same locus as in \cite{CostelloGaiotto}.
The difference is that we are working with a bigger space of fields, roughly extended by the cohomology of the $K3$ surface.
Recall that $A = H^\bu(K3)$ denoted the cohomology ring of the $K3$ surface.
Notice that the ring $A$ is canonically augmented via integration $\int_{K3} \colon A \to \C$ and hence $A$ is a local ring whose augmentation ideal we denote $\lie{m}_A$.
A useful perspective on the extended version of Kodaira--Spencer theory we obtain by compactification along $K3$ is as a family of theories over the scheme $\op{Spec} A$.
This family has the property that over $\lie{m}_A$ we obtain ordinary Kodaira--Spencer theory.
We will see that in the case of type IIB compactified on a $K3$ surface that the backreaction determines an infinitesimal deformation of the complex manifold $\C^3 \setminus \C$ over $\Spec A$. 
 
If $A$ is any local ring, an infinitesimal deformation of a complex manifold $M_0$ over $\Spec A$ is an element 
\beqn
\mu_{def} \in \PV^{1,1}(M_0) \otimes \mathfrak{m}_A 
\eeqn
satisfying the Maurer--Cartan equation.
In our case, $M_0 = \C^3 \setminus \C$ and $\mu_{def}$ is a field sourced by the branes. 
The Maurer--Cartan equation is the equation of motion for $\mu_{def}$. 
%Recall that the cohomology ring of a K3 surface (with complex coefficients) is generated by even elements $\eta, \br \eta, \eta_a$, where $a=1,\ldots,20$ subject to the relations \eqref{eqn:K3rel}. 
%These relations define a quadric $Z$ inside $\C^{22}$. 
%Thus, in total, we will see that the backreaction will deform the complex structure on 
%\[
%(\C^3 \setminus \C) \times Z \subset \C^3 \times \C^{22} .
%\]
%Before deforming, there is an obvious projection map 
%\[
%(\C^3 \setminus \C) \times Z \to Z 
%\] 
%whose fiber is a Calabi--Yau manifold equipped with the standard holomorphic volume form $\Omega_{\C^3} = \d z \d w_1 \d w_2$. 
The cohomology ring $A$ of a $K3$ surface is a local ring.
Following \cite{CostelloGaiotto}, the backreaction of this system of branes introduces a twisted supergravity field
\beqn
\mu_{BR} \in \PV^{1,1}(\C^3 \setminus \C) \otimes A 
\eeqn
which we can identify with an element of $\Omega^{2,1}(\C^3 \setminus \C) \otimes A$ using the Calabi--Yau form on $\C^3$. 
This field satisfies the following defining equations
\beqn
\label{eqn:mcbr}
	\begin{split}
		\dbar \mu_{BR}  & = F \, \delta_{\C \subset \C^3} \\
		\del \mu_{BR} & = 0 .
	\end{split}
\eeqn
For quantization we will also impose the standard gauge fixing condition that $\dbar^\ast \mu_{BR} = 0$ in terms of the usual codifferential $\dbar^{\ast}$. 	
There is a unique solution to the above equations given by
\beqn
\mu_{BR} = \frac{\eps^{ij} \br w_i \d \br w_j}{4 \pi^2 |w|^4} \partial_z \otimes F .
\eeqn
Note that this field is of the form $\mu_{BR,0} \otimes F$ where $\mu_{BR,0} \in \PV^{1,1}$ is the Beltrami differential which gives rise to the deformed conifold \cite{CostelloGaiotto}---all of the dependence on the parameters $\eta, \br \eta, \eta_a$ is in the cohomology class~$F$.
Also we notice that $F \in \lie{m}_A$.

Equations \eqref{eqn:mcbr} imply that $\mu_{BR}$ determines an infinitesimal deformation of~$\C^3 \setminus \C$ over $\Spec A$. 
The Kodaira--Spencer map associated to this infinitesimal deformation is of the form
\[
KS \colon T_{\Spec A} \to H^1(\C^3 \setminus \C, T) 
\]
and simply maps a derivation $\delta$ of $A$ to the class 
\[
\delta(F) \left[\frac{\eps^{ij} \br w_i \d \br w_j}{|w|^4} \del_z \right] \in H^1(\C^3 \setminus \C, T) .
\] 
%\brian{what more to say?}
%This Beltrami differential defines a new complex structure on the complex manifold $(\C^3 \setminus \C) \times Z$ where $Z$ is the quadric inside $\C^{22}$ defined by \eqref{eqn:K3rel}. 
%Note that \brian{how to say things remain algebraic along $Z$?}
%A function $\Phi(z,\br z, w_i, \br w_i , \eta, \br \eta, \eta_a)$ is holomorphic if the following equations hold
%\beqn
%	\begin{split}
%		\del_{\br z} \Phi & = 0 \\
%		\del_{\br w_1} \Phi - F \frac{\br w_2}{\norm{w}^2} \del_z \Phi & = 0 \\
%		\del_{\br w_2} \Phi + F \frac{\br w_1}{\norm{w}^2} \del_z \Phi & = 0 .
%	\end{split}
%\eeqn
%It is easy to check that the following functions are holomorphic for the deformed complex structure 
%\beqn
%	\begin{split}
%		u_1 & = w_1 z - F \frac{\br w_2}{\norm{w}^2} \\
%		u_2 & = w_2 z + F \frac{\br w_1}{\norm{w}^2} .
%	\end{split}
%\eeqn
%
%In total, we are describing the affine variety $X_F$ inside of $\C^4 \times \C^{22}$ defined by the equations 
%\beqn
%\label{eqn:conifold}
%	\begin{split}
%		\eta^2 = \br \eta^2 & = 0 \\
%		\eta_a \eta_b & = h_{ab} \eta \br \eta \\
%		w_1 u_2 - w_2 u_1 & = F . 
%	\end{split}
%\eeqn
%Here, $(u_1,u_2,w_1,w_2)$ are the holomorphic coordinates on $\C^4$ and $\eta,\br \eta, \eta_a$ are the holomorphic coordinates on $\C^{22}$. 
%Denote by $X_F^0 \subset X_F$ the open subset where $u_i,w_j$ are all not zero. 
%
%We observe that upon deforming the complex structure, there is still a fibration
%\[
%X_{F}^0 \to Z 
%\]
%whose fibers are Calabi--Yau three-folds. 
%The volume form in the coordinates $z, w_1,w_2$ is unchanged when we deform the complex structure.
%This is because the Beltrami differential $\mu_{BR}$ is divergence-free. 
%In the coordinates $u_i, w_i$ the holomorphic volume on the fibers of this projection reads
%\beqn
%\Omega|_{w_1 \ne 0} = \frac{\d u_1 \d w_1 \d w_2}{w_1}, \qquad \Omega|_{w_2 \ne 0} = \frac{\d u_2 \d w_1 \d w_2}{w_2} . 
%\eeqn
%
%In \cite{CP1} the first two authors considered Kodaira--Spencer theory on the complex surface $T^4$ and found that the resulting backreaction only gave finite-order corrections to operator products on flat space. 
%We are in a very similar situation here in the case of Kodaira--Spencer theory compactified on a K3 surface. 
%The flat space background corresponds to $F = 0$. 
%Now, since $F \in H^2(Y)$, we see that $F^3 = 0$ in the ring $A = H^\bu (Y)$.
%Thus, only the only powers of $F$ that appear in corrections to the flat space OPE are $F$ and $F^2$. 

%\subsection{Partial compactification of the extended conifold}
%\label{sec:compact} 
%
%The ordinary deformed conifold is biholomorphic to $SL_2(\C)$. 
%One can view this as a subvariety of $\C^4$ with coordinates $u_i,w_j$ satisfying $\eps^{ij} u_i w_j = N$. 
%To compactify this subvariety the idea is to introduce an additional homogeneous coordinate $\lambda$ and consider the projective variety in $\CP^4$ defined by the equation $\eps^{ij} u_i w_j = N \lambda^2$ where the $u_i,w_j,\lambda$ are now projective coordinates. 
%The boundary of this projective variety is described by the equation $\eps^{ij} u_i w_j = 0$ and is equivalent to $\CP^1 \times \CP^1$. 
%
%We use the new coordinate $z$ for the first $\CP^1$ which is the boundary of the Euclidean ${\rm AdS}_3$ space
%and the new coordinate $w$ for the second $\CP^1$.  
%Let $n$ be the coordinate normal to the boundary $\CP^1 \times \CP^1$ of $SL_2(\C)$. 
%This coordinate has a first order pole at $z=\infty$ and $w=\infty$. 
%Let $\DD_z \times \CP^1_w \subset \CP^1_z \times \CP^1_w$ be a neighborhood of the algebraic curve $0 \times \CP^1_w$ in the boundary divisor. 
%A neighborhood of this region in the whole manifold $\br{X_{F}^0}$ is a deformation of the total space of $\cO(1) \to \DD \times \CP^1$ by the Beltrami differential 
%\[
%n^2 \d ...
%\]
%
%
%The situation here is very similar. 
%The extended conifold $X_F^0$ has a completion $\br{X_F^0}$ which is a subvariety of $\CP^4 \times Z$. 
%It is defined by modifying the last equation in \eqref{eqn:conifold} to
%\beqn
%\eps^{ij} u_i w_j = F \lambda^2 
%\eeqn
%where $\lambda$ is an additional homogenous coordinate.
%The boundary of $\br{X_F^0}$ is the subvariety of $\CP^3 \times Z$ defined by $\eps^{ij} u_i w_j = 0$ which is $\CP^1 \times \CP^1 \times Z$. 

\end{document}