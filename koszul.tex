\documentclass[11pt]{amsart}

\usepackage{macros, setspace}
\usepackage{hyperref}
\usepackage{amsmath,amsthm}

\author{Natalie M. Paquette and Brian R. Williams}
\date{\today}
\title{Koszul duality in QFT}

\def\define{\overset{\rm def}{=}}
\def\ep{\varepsilon}
\def\zbar{{\overline{z}}}
\def\Weyl{{\rm Weyl}}
\def\Cl{{\rm C}\ell}

\def\brian#1{{\textcolor{blue!65!red}{BRW: {#1}}}}
\def\natalie#1{{\textcolor{green!65!black}{BRW: {#1}}}}

\usepackage[upint]{stix}

\begin{document}
\maketitle

\spacing{1.25}

\section{Introduction}

\section{A biased review of Koszul duality} 

%sym and alt
%maybe a few words about mapping of modules btw derived categories? how does this relate to the line picture?
%CE(g) and U(g)

\subsection{Commutative-Lie duality}

%CE complex

\subsection{Associative-Associative duality} 

%hochschild

\subsection{A look at $\EE_n$ Koszul duality}

\subsection{Deformed Koszul duality}

\section{QFT rudiments}
%clarify: algebra of local ops vs. dense inclusion we work with in practice?

%Explain our conventions/some basic facts about the cohomological degree as ghost number, as well as the meaning of various degree shifts, which are unfamiliar to most physicists. 

\subsection{Local functionals and Maurer--Cartan elements} 



\subsection{Local operators}

Let $\Obs$ be the local operators of a one-dimensional translation invariant theory, and let $Q$ denote the BRST operator which acts on the theory.
The infinitesimal action by translations is through the time derivative $\frac{\partial}{\partial t}$. 

A one-dimensional theory is topological if there exists an endomorphism $\hat{Q}$ of the local operators, of cohomological degree $-1$, which satisfies
\[
\{Q, \hat{Q}\} = \frac{\partial}{\partial t} .
\]
In words, translations along the line are trivial in BRST cohomology. 
%Should we here contrast with notion of topological QFT coming from Q-exact stress tensors, which is a stronger statement?

%To be clarified: even ignoring BV/BRST extensions, we often have the freedom to enlarge the basic space of fields (dual to ops). E.g. U(1) gauge theory in the first order formalism enlarges from (gauge fields) to (gauge fields) x \Omega^{d-2}(M)... how is Koszul duality insensitive to this?

\subsection{Topological descent} 

Let $\Obs$ be the local operators of a one-dimensional topological field theory on $\RR$. 
Because the theory is translation invariant, we can consider the complex of {\em differential form} valued operators
\[
\Omega^\bu(\RR \, , \, \Obs) .
\]
This complex is equipped with the original BRST differential $Q$ acting on $\Obs$ as well as the de Rham differential $\d$. 

Given a local 0-form operator, $\cO = \cO^{(0)} \in \Obs$, we can use the endomorphism $\hat{Q}$ to define the following one-form valued operator
\[
\cO^{(1)} \define (\hat{Q} \cdot \cO) \d t \in \Omega^1(\RR \, , \, \Obs).
\]
The inhomogenous operator $\cO^{(0)} + \cO^{(1)}$ is closed in $\Omega^\bu(\RR \, , \, \Obs)$ if and only if the following descent equations are satisfied 
\begin{align*}
Q \cO^{(0)} & = 0 \\
\d \cO^{(0)} + Q \cO^{(1)} & = 0 .
\end{align*}

Topological descent equations were first explored by Witten in the Donaldson twist of 4d $\cN=2$ theory \cite{W88} as a means to obtain smeared gauge-invariant observables from local operators of higher ghost number. The solutions to the descent equations were constructed more generally in \cite{MW97}; they may also be familiar to string theorists from their application to the BRST cohomology of noncritical string theories, e.g. \cite{WZ92}. 

In particular, if $\{Q,\cO^{(0)}\} = 0$ then $\cO^{(1)}$ defines the following $Q$-closed local functional
\[
\int_{\RR} \cO^{(1)} .
\]
The superscript denotes a 1-form operator, now in cohomological degree (ghost number) 0, as appropriate for a gauge-invariant term in the Lagrangian. 


\subsection{Vacuums and augmentations}

%discrete choices of vacua vs. moduli spaces of vacua

\section{Line operators}

%universal line defects and Koszul duality
%stress relationship between three perspectives: MC(AxB), Hom(A!, B), couplings as local functionals from descent

\subsection*{Topological mechanics}

%free fermions/bosons in CS example? In 4d CS? Classical limit to connect Assoc/Assoc duality to Comm/Lie? 
%references for lines in 4d N=4, 2d YM, boson and fermion descriptions
%any example where we know, e.g., dim(MC(AxB)) and it's relatively tractable? If so, connect explicitly with Hom & couplings
%3d N=6 topological line?

Ordinary mechanics describes \brian{do it}
Topological mechanics, on the other hand, describes {\em locally constant} paths inside of phase space.

The most basic phase space we consider is the cotangent bundle of affine space $\T^* \RR^n$.
The fields of this system are given by functions $\gamma^i, \beta_j$ on $\RR$ where $i,j=1,\ldots n$. 
The free action is simply
\[
\int_\RR \beta_i \partial_t \gamma^i .
\]
The algebra of operators of the quantum mechanical system is simply the Weyl algebra $\Weyl(\RR^n)$.
This is the free algebra on $2n$ generators $p^i, q_j$ modulo the relation
\[
[p^i, q_j] = \delta^i_j .
\]

There is also a fermionic version of this topological mechanical system. 
Instead of taking the phase space to be $T^* \RR^n$, it is the cotangent bundle of the odd vector space $\Pi \RR^n$.
The fields describing the system are given by {\em odd} functions $\psi^i, \chi_j$ on $\RR$ where $i,j=1,\ldots n$. 
The free action is identical
\[
\int_\RR \chi_i \partial_t \psi^i .
\]
The algebra of operators of the quantum mechanical system is the Clifford algebra $\Cl(\RR^n)$ on $\RR^n$ equipped with its Euclidean metric. 



\subsection*{Coupling to bulk gauge theories}

\subsection{Coupling to line defects} 

Let's suppose we have a field theory on a manifold of the form $\RR_t \times M$ with the property that it is {\em topological} along the $\RR_t$-direction. 
Denote by $\Obs$ the local operators along $\RR_t$. 

Suppose that $\cA$ is an associative algebra which represents some other quantum mechanical system. 
Our goal is to couple $\cA$ to the original theory as a line operator along $\RR_t$. 

\begin{prop}
There is a one-to-one correspondence between the space of couplings of $\cA$ to $\Obs$ and the space of Maurer--Cartan elements in 
\[
\Obs \otimes \cA .
\]
\end{prop}

We provide a sketch of the proof of this result. 
Suppose that $\alpha \in $

%explanation about augmentation ideal in more general case should go here

%try to say something about the effect of the transverse directions on this story (topological, holomorphic, general fact alg)
%brief point about BRST vs. BV-BRST
%BV-BRST vs. BRST and `minimal' couplings

\section{``Critical'' Chern--Simons theory}

%Add a few sentences about how CS theory at the critical level can still give a nontrivial topological theory when written in the first order formalism... need to connect to the familiar Lagrangian before jumping in to the BF-type theory. 
Let $\fg$ be a Lie algebra. 
The theory has two sets of fields
\begin{align*}
A & = A_{\zbar} (z,t) \d \zbar + A_{t} (z, t) \d t \\
B & = B_z (z,t) \d z 
\end{align*}
with $A_{\zbar}, A_t, B_z$ all $\fg$-valued functions on $\RR^3$.
The action is of ``BF'' type:
\begin{align*}
\int B F_A & = \int B \d A + \frac12 B [A,A] \\
& = \int B_z \partial_t A_{\zbar} + B_z \partial_{\zbar} A_t + B_z [A_t, A_\zbar] .
\end{align*}
Notice that in the kinetic part of the action only $\partial_t$ and $\partial_{\zbar}$ derivatives appear since $B$ is a Dolbeault form of type $(1,0)$. 

The Lie algebra of gauge symmetries is the familiar one: $C^\infty(\RR^3) \otimes \fg$, i.e. smooth functions valued in $\fg$. 
An element $\fc \in C^\infty(\RR^3) \otimes \fg$ acts on the fields via the transformation rule
\begin{align*}
\delta A_\zbar & = \partial_{\zbar} \fc + [\fc, A_\zbar] \\
\delta A_t & = \partial_t \fc + [\fc,A_t] \\
\delta B_z & = [\fc, B_z] .
\end{align*}

We first treat the classical theory.
The classical BRST complex of local operators supported at $(z,t) = (0,0)$ is equivalent to the Chevalley--Eilenberg complex
\[
\clie^\bu \left(\fg[\ep][[z]]\right)
\] 
which computes Lie algebra cohomology of the ($\infty$-dimensional) Lie algebra $\fg[\ep][[z]]$. 

%Here I plan to add a few sentences explaining that CE cohomology is the usual BRST for physicists.

Pick a basis $\{e^a\}_{a = 1,\ldots, \dim \fg}$ for $\fg$ with dual basis $\{e_a\}$ for $\fg^*$. 
Explicitly, the Lie algebra cohomology representing local operators is computed by 
\[
\bigg(\Sym\left(\fg^*[\ep] [\partial_z] [-1] \right) \; , \; Q \bigg) = \bigg(\CC[\fc_a, \partial_z \fc_a , \partial^2_z \fc_a, \ldots, B_a , \partial_z B_a , \partial_z^2 B_a, \ldots]_{a=1, \ldots, \dim \fg} \; , \; Q \bigg)
\]
where the BRST operator encodes the Lie bracket on $\fg[\ep][[z]]$ and is given by \brian{write formula}

The linear local operators are spanned by 
\begin{align*}
\partial_z^n \fc_a \colon & \fc^b \mapsto \delta_a^b \partial^n_z \fc^b (z=t=0), \\
\partial_z^n B_a \colon & B^b \mapsto \delta_a^b \partial_z^n B^b_z (z=t=0) 
\end{align*}
for $n \in \ZZ_{\geq 0}, a = 1,\ldots, \dim \fg$. 
These operators are of cohomological degree $+1$, $0$ respectively. 

\subsection*{Line operators}

There is a one-to-one correspondence between classically coupled line defects along $z = 0$ described by an associative algebra $\cA$ and Maurer--Cartan elements in 
\begin{equation}\label{eqn:criticalobs}
\clie_\hbar^\bu(\fg[\ep][[z]]) \otimes \cA .
\end{equation}

The Maurer--Cartan equation reads
\begin{equation}\label{eqn:mccs}
Q_\hbar \cO^{(0)} + \cO^{(0)} \star_\hbar \cO^{(0)} = 0 .
\end{equation}
If we treat the bulk gauge field as classical, the quantum BRST operator $Q_\hbar$ is replaced by the classical BRST operator $Q$ which is simply the Chevalley--Eilenberg differential for the Lie algebra $\fg[\ep][[z]]$.
The product $\star_\hbar$ is replaced by the product $\star$ in the algebra $\cA$. 
Thus, the classical limit of \eqref{eqn:mccs} becomes
\begin{equation}\label{eqn:cmccs}
Q \cO^{(0)} + \cO^{(0)} \star \cO^{(0)} = 0 .
\end{equation}

We spell out the correspondence between such Maurer--Cartan elements and local couplings in the case that $\cA$ is the algebra of local operators of the free fermion topological mechanics system of rank $n$. 
Here, $\cA$ is isomorphic to the Clifford algebra $\Cl(\RR^n)$. 
Field theoretically, this system is described by the one-dimensional theory on $\RR$ whose fields are odd functions $\chi_i, \psi^j$ for $i,j=1,\ldots, n$. 
\brian{finish}

Consider, for instance, the local operator built from the bulk gauge theory and the free fermion theory along the line defect:
\[
\rho_j^{a,i} \fc_a \psi_i \chi^j 
\]
for some collection of coefficients $\{\rho_j^{a,i}\}$. 
Call this local operator $\cO^{(0)}$. 
Notice that $\cO^{(0)}$ is of cohomological degree $+1$.
The classical Maurer--Cartan equation \eqref{eqn:cmccs} reads
\[
f_a^{bc} \rho_j^{a,i} \fc_b \fc_c \psi_i \chi^j + .. = 0. 
\]
This equation simply says that the collection $\{\rho_j^{a,i}\}$ prescribes a representation of the Lie algebra $\fg$ on the vector space $\RR^n$. 

 


To construct the coupling, we solve the descent equations $\partial_t \cO^{(0)} + Q \cO^{(1)} = 0$.

First, notice that the local operator $\fc_a$ satisfies the following descent equation
\[
\partial_t \fc_a = Q A_{t, a} + f_{a}^{bc} \fc_b \fc_c .
\]


\section{Higher dimensional defects} 

%higher dimensional topological defects. PSM in 3dCS. 
%Kapustin--Salina.
%Check refs by Mnev et al on BV-BFV?

\begin{thebibliography}{KWWY14}

\bibitem[MW97]{MW97}
G.~W.~Moore and E.~Witten,
``Integration over the u plane in Donaldson theory,''
Adv. Theor. Math. Phys. \textbf{1}, 298-387 (1997)
doi:10.4310/ATMP.1997.v1.n2.a7
[arXiv:hep-th/9709193 [hep-th]].

\bibitem[W88]{W88}
E.~Witten,
``Topological Quantum Field Theory,''
Commun. Math. Phys. \textbf{117}, 353 (1988)
doi:10.1007/BF01223371


\bibitem[WZ92]{WZ92}
E.~Witten and B.~Zwiebach,
``Algebraic structures and differential geometry in 2-D string theory,''
Nucl. Phys. B \textbf{377}, 55-112 (1992)
doi:10.1016/0550-3213(92)90018-7
[arXiv:hep-th/9201056 [hep-th]].


\end{thebibliography}

\end{document}