\documentclass[11pt]{amsart}

\usepackage{macros, setspace}

\author{Natalie Paquette and Brian R. Williams}
\date{\today}
\title{Koszul duality in QFT}

\def\define{\overset{\rm def}{=}}
\begin{document}
\maketitle

\spacing{1.25}

\section{Introduction}

\section{A biased review of Koszul duality} 

%sym and alt
%maybe a few words about mapping of modules btw derived categories?
%CE(g) and U(g)

\subsection{Commutative-Lie duality}

%CE complex

\subsection{Associative-Associative duality} 

%hochschild

\subsection{A look at $\EE_n$ Koszul duality}

\subsection{Deformed Koszul duality}

\section{QFT rudiments}
%clarify: algebra of local ops vs. dense inclusion we work with in practice?

\subsection{Local functionals} 

\subsection{Local operators}

<<<<<<< HEAD
Let $\Obs$ be the local operators of a one-dimensional translation invariant theory. 
The infinitesimal action by translations is through the time derivative $\frac{\partial}{\partial t}$. 

A one-dimensional theory is topological if there exists an endomorphism $\eta$ of the local operators, of cohomological degree $-1$, which satisfies
\[
\{Q, \eta\} = \frac{\partial}{\partial t} .
\]

=======
%To be clarified: even ignoring BV/BRST extensions, we often have the freedom to enlarge the basic space of fields (dual to ops). E.g. U(1) gauge theory in the first order formalism enlarges from (gauge fields) to (gauge fields) x \Omega^{d-2}(M)... how is Koszul duality insensitive to this?
>>>>>>> eada762f051fe8c18801aad8542032ed6e2563ba

\subsection{Topological descent} 

Let $\Obs$ be the local operators of a one-dimensional topological field theory on $\RR_t$. 
Because the theory is translation invariant, we can consider the complex of {\em differential form} valued operators
\[
\Omega^\bu(\RR \, , \, \Obs) .
\]
This complex is equipped with the original BRST differential $\{Q, \cdot\}$ acting on $\Obs$ as well as the de Rham differential $\d$. 

Given $\cO = \cO^{(0)} \in \Obs$, we can use the endomorphism $\eta$ to define the following one-form valued operator
\[
\cO^{(1)} \define (\eta \cdot \cO) \d t \in \Omega^1(\RR \, , \, \Obs).
\]
The inhomogenous operator $\cO^{(0)} + \cO^{(1)}$ is closed in $\Omega^\bu(\RR \, , \, \Obs)$ if and only if the following descent equations are satisfies
\begin{align*}
\{Q, \cO^{(0)}\} & = 0 \\
\d \cO^{(0)} + \{Q, \cO^{(1)}\} & = 0 .
\end{align*}

In particular, if $\{Q,\cO^{(0)}\} = 0$ then $\cO^{(1)}$ defines the following $Q$-closed local functional
\[
\int_\RR \cO^{(1)} .
\]

\subsection{Coupling to line defects} 

Let's suppose we have a field theory on a manifold of the form $\RR_t \times M$ with the property that it is {\em topological} along the $\RR_t$-direction. 
Denote by $\Obs$ the local operators along $\RR_t$. 

Suppose that $A$ is an associative algebra which represents some other quantum mechanical system. 
Our goal is to couple $A$ to the original theory as a line operator along $\RR_t$. 

\begin{prop}
There is a one-to-one correspondence between the space of couplings of $A$ to $\Obs$ and the space of Maurer--Cartan elements in 
\[
A \otimes \Obs .
\]
\end{prop}

Suppose that $\alpha \in $

\subsection{Vacuums and augmentations}

%discrete choices of vacua vs. moduli spaces of vacua

\section{Line operators}

%universal line defects and Koszul duality
%stress relationship between three perspectives: MC(AxB), Hom(A!, B), couplings as local functionals from descent

\subsection{Topological quantum mechanics}

%free fermions/bosons in CS example? In 4d CS? Classical limit to connect Assoc/Assoc duality to Comm/Lie? 
%references for lines in 4d N=4, 2d YM, boson and fermion descriptions
%any example where we know, e.g., dim(MC(AxB)) and it's relatively tractable? If so, connect explicitly with Hom & couplings
%3d N=6 topological line?

\subsection{Coupling to bulk gauge theories}

%try to say something about the effect of the transverse directions on this story (topological, holomorphic, general fact alg)
%brief point about BRST vs. BV-BRST
%BV-BRST vs. BRST and `minimal' couplings

\section{Higher dimensional defects} 

%higher dimensional topological defects. PSM in 3dCS. 
%Kapustin--Salina.
%Check refs by Mnev et al on BV-BFV?

\end{document}