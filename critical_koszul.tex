\documentclass[11pt]{amsart}

\usepackage{macros}

\linespread{1.25}

%\usepackage[final]{pdfpages}

\setcounter{tocdepth}{2}
\numberwithin{equation}{section}

\def\brian{\textcolor{blue}{BW: }\textcolor{blue}}
\def\owen{\textcolor{magenta}{OG: }\textcolor{magenta}}

\def\d{{\rm d}}
\def\zbar{{\overline{z}}}
\def\dz{\d z}
\def\dzbar{{\d\zbar}}
\def\dt{\d t}
\def\del{\partial}
\def\delbar{{\overline{\partial}}}
\def\loc{{\rm loc}}
\def\GF{{\rm GF}}
\def\Vpunc{\mathring{V}}

%%Feynman diagrams
\usetikzlibrary{decorations.pathmorphing}
\usetikzlibrary{decorations.markings}
\tikzset{
	% >=stealth', %%  Uncomment for more conventional arrows
    vector/.style={decorate, decoration={snake}, draw},
	provector/.style={decorate, decoration={snake,amplitude=2.5pt}, draw},
	antivector/.style={decorate, decoration={snake,amplitude=-2.5pt}, draw},
    fermion/.style={draw=black, postaction={decorate},
        decoration={markings,mark=at position .55 with {\arrow[draw=black]{>}}}},
    fermionbar/.style={draw=black, postaction={decorate},
        decoration={markings,mark=at position .55 with {\arrow[draw=black]{<}}}},
    fermionnoarrow/.style={draw=black},
    gluon/.style={decorate, draw=black,
        decoration={coil,amplitude=4pt, segment length=5pt}},
    scalar/.style={dashed,draw=black, postaction={decorate},
        decoration={markings,mark=at position .55 with {\arrow[draw=black]{>}}}},
    scalarbar/.style={dashed,draw=black, postaction={decorate},
        dwecoration={markings,mark=at position .55 with {\arrow[draw=black]{<}}}},
    scalarnoarrow/.style={dashed,draw=black},
    electron/.style={draw=black, postaction={decorate},
        decoration={markings,mark=at position .55 with {\arrow[draw=black]{>}}}},
	bigvector/.style={decorate, decoration={snake,amplitude=4pt}, draw},
}

\begin{document}
\title{The factorization algebra of mixed BF theory}

\maketitle

\begin{prop} 
The classical and quantum factorization algebra of critical Chern--Simons theory on $\RR \times \CC$ satisfy:
\begin{enumerate}
\item[(1)]
at the classical level $\Obs^{cl}_{0}$ is a dg commutative algebra isomorphic to ${\rm C}_{\GF}^* \left(\fg[\epsilon][[z]]\right)$. Its Koszul dual is isomorphic to the enveloping algebra $U(\fg[\epsilon][[z]])$;
\item[(2)] at the quantum level $\Obs^{q}_{0}$ is a dg algebra isomorphic to an $\hbar$-deformation of ${\rm C}_{\GF}^* \left(\fg[\epsilon][[z]]\right)$.
Its Koszul dual to a graded algebra $\hbar$-deformation of $U(\fg[\epsilon][[z]])$. 
\brian{Possible higher order corrections will be more involved, and maybe not something we should get into.}
\end{enumerate}
\end{prop}

\begin{rmk} 
Upon turning on the Chern-Simons deformation, the $\hbar$-deformation of $U(\fg[\epsilon][[z]])$, as stated in the proposition, further deforms to the ordinary universal enveloping algebra $U(\fg)[[\hbar]]$ with the obvious associative product. 
We provide more details below. 
\end{rmk}

\subsection{}

In this section we begin by recounting some basic properties of the factorization algebra of mixed BF theory on the manifold $\CC \times \RR$. 
Recall, the fields of the theory are 
\begin{align*}
\alpha & \in \Omega^{0,*}(\CC) \widehat{\otimes}_\pi \Omega^*(\RR)[1] \otimes \fg \\
\beta & \in \Omega^{1,*}(\CC) \widehat{\otimes}_\pi \Omega^*(\RR) \otimes \fg
\end{align*}

Suppose $D(z_0) \subset \CC$ is a disk centered at $z_0$ in $\CC$. 
Then, there are quasi-isomorphisms 
\begin{align*}
\cO^{hol}(D) [1] \otimes \fg & \xto{\simeq} \Omega^{0,*}(\CC) \widehat{\otimes}_\pi \Omega^*(\RR)[1] \otimes \fg \\
\Omega^{1,hol}(D) \otimes \fg & \xto{\simeq} \Omega^{1,*}(\CC) \widehat{\otimes}_\pi \Omega^*(\RR) \otimes \fg
\end{align*}
In the first line, the map sends a holomorphic function $f : D \to \CC$ and Lie algebra element $X \in \fg$ to the element form $f \otimes 1 \otimes X$.
The second quasi-isomorphism is similar. 

At the level of dg Lie algebras, the dg Lie algebra describing mixed BF theory on $D \times \RR$ is quasi-isomorphic to the graded Lie algebra
\[
\cO^{hol}(D) \otimes \fg \ltimes \Omega^{1,hol}(D) [-1] \otimes \fg
\]
where $\fg$ acts on itself via the adjoint representation. 
Hence, classically, the local operators of the theory on $D \times \RR$ are quasi-isomorphic to the Lie algebra cohomology
\[
{\rm C}_{\GF}^* \left(\cO^{hol}(D) \otimes \fg \ltimes \Omega^{1,hol}(D) [-1] \otimes \fg \right) .
\] 

Suppose we fix the flat holomorphic volume element $\d z$ on $\CC$. 
Then, we can identify the above cochain complex with ${\rm C}_{\GF}^* \left(\cO^{hol}(D) \otimes (\fg \ltimes \fg [-1])\right)$. 
We will use the notation $\fg[\epsilon]$ to denote the graded Lie algebra $\fg \ltimes \fg [-1]$, where $\epsilon$ is a formal parameter of degree $+1$. 

There is a certain sub factorization algebra that we are most interested in. 
%Concretely, to work with formal disk we can use the following trick. 
Note that there is a natural action of $S^1$ on the fields of mixed BF theory coming from rotations in $\CC$ centered at the point $z_0 \in \CC$. 
This induces a $\ZZ$-grading on the local operators on $D \times \RR$, where $D$ is a disk centered at $z_0 \in \CC$, by the integral $S^1$-weights. 

\begin{dfn}
The cochain complex $\Obs_{z_0}$ is defined as the direct sum 
\[
\Obs^{cl}_{z_0} \overset{\rm def}{=}  \bigoplus_{k \in \ZZ} {\rm C}_{\GF}^* \left(\cO^{hol}(D) \otimes \fg[\epsilon] \right)^{(k)}
\]
where ${\rm C}_{\GF}^* \left(\cO^{hol}(D) \otimes \fg[\epsilon] \right)^{(k)}$ is the weight $k \in \ZZ$ eigenspace of the $S^1$-action. 
The quantum version $\Obs_{z_0}^{q}$ is defined similarly. 
\end{dfn}

One can heuristically think of $\Obs^{cl}_{z_0}$ as the value of the classical factorization algebra on the {\em formal disk} centered at $z_0$ inside of $\CC$. 
Indeed, there is an isomorphism
\[
\Obs^{cl}_{z_0}  = {\rm C}_{\GF}^* \left(\fg[\epsilon][[z-z_0]]\right) .
\]
The quantum version $\Obs^{cl}_{z_0}$ is an $\hbar$-deformation of this cochain complex, that we proceed to identify. 

\subsection{The classical factorization algebra}

In ordinary Chern-Simons theory, the theory is topological and hence determines a locally constant three-dimensional factorization algebra. 
Mixed holomorphic BF theory is only topological in the real direction $\RR$, and is holomorphic in the complex plane direction $\CC$. 
Hence, if we only consider the factorization product in the real direction, the factorization algebra still behaves topologically and hence determines the structure of an $\EE_1$, or associative, algebra. 

To formalize this setup, consider the projection map 
\[
\pi : \CC \times \RR \to \RR .
\]
This restricts a projection $\pi |_{D} : D \times \RR \to \RR$ for any disk $D \subset \CC$. 

\begin{lem}
For any $D \subset \CC$ centered at $z_0$, the factorization algebra $\left(\pi |_D\right)_*\Obs^{cl}$ is locally constant. 
Moreover, if $I \subset \RR$ is any interval, the map
\[
\Obs_{z_0} \hookrightarrow \left(\pi |_D\right)_*\Obs^{cl} (I) 
\]
is a dense inclusion which is preserved by the factorization product. 
\end{lem}

The lemma implies that the factorization product in the real direction $\RR$ endows $\Obs_{z_0}$ with the structure of an $\EE_1$, or homotopy associative, algebra. 
\brian{filtration}

\begin{prop}
There is an isomorphism of filtered $\EE_1$-algebras 
\[
\Obs_{z_0} \simeq {\rm C}^*_{\GF}(\fg[\epsilon][[z-z_0]]) .
\]
\end{prop}

\subsubsection{Filtered Koszul duality}

\brian{Reference {\rm C}ite{CostelloYangian1}.}


\subsection{The quantum deformation}

Introduce the following notation for observables. 
Fix an element $X \in \fg$, an integer $n \geq 0$, and a position $t_0 \in \RR$. 
Define the linear observable $\partial_z^n \cA_X (t_0)$ by
\[
\partial_z^n \cA_X (t_0) : \alpha \mapsto \frac{\partial^n}{\partial z^n} \langle X, \alpha\rangle(z = 0, t = t_0) 
\]
and the linear observable $\partial_z^n \cB_X (t_0)$ by
\[
\partial_z^n \cB_X (t_0) : \beta \mapsto \frac{\partial^n}{\partial z^n} \langle X, \iota_{\partial_z} \beta\rangle(z = 0, t = t_0) .
\]
Each of these observables are supported on an open set of the form
\[
D \times I \subset \CC \times \RR
\]
where $D$ is any disk containing $0 \in \CC$ and $I$ is any interval containing $t_0 \in \RR$. 

In fact, it is clear that each of the observables define elements in $\Obs_{0}(I)$:
\[
\partial_z^n \cA_X (t_0) , \partial_z^n \cB_X(t_0) \in \Obs_{0}(I) .
\]
Note that for each $n \geq 0$ and $X \in \fg$, the observable $\partial_z^n \cA_X (t_0)$ is of cohomological degree $+1$ and the observable $\partial_z^n \cB_X (t_0)$ is of cohomological degree zero. 

\def\ft{{\frak t}}

Momentarily, we will compute the first-order correction (linear in $\hbar$) to the factorization product.
We have already seen that classically, the factorization algebra of observables is Koszul dual to an enveloping algebra of the form $U(\fg[\epsilon][[z]])$. 
Here is a summary of what we will find in the quantum case, and its meaning on the Koszul dual side. 
\begin{itemize}
\item[(1)] 
There is a relation in the quantum observables of the form
\[
[\cB_X, \cB_Y] = \hbar \partial_z \cB_{[X,Y]} + O(\hbar^2) .
\]
This gives rise to a differential on the Koszul dual side of the form
\[
\d (z \epsilon \ft_a) = \hbar \sum_{i,j,k} f_i^{jk} (\epsilon \ft_{j}) \otimes (\epsilon \ft_{k}) + O(\hbar^2)
\]
where $\{\ft_a\}$ is a basis for $\fg$ and $f_i^{jk}$ are the (dual) Lie algebra structure constants
\item[(2)] 
There is a relation in the quantum observables of the form
\[ 
[\cB_X, \cA_Y] = \hbar \partial_z \cA_{[X,Y]} + O(\hbar^2) .
\]
This gives rise to a differential on the Koszul dual side of the form
\[
\d (z \ft_a) = \hbar \sum_{i,j,k} f_i^{jk} (\epsilon \ft_{j}) \otimes \ft_{k} + O(\hbar^2) .
\]
\end{itemize}

\subsection{Steps in the proof}

We have already seen that $\Obs^q_0$ is a locally constant factorization algebra on $\RR$, and hence is equivalent to an $\EE_1$-algebra that we denote by $\cA$. 
In order to compute the deformation we proceed in the following steps. 
Below, let $I = (-1, 1)$ 

\begin{itemize}
\item[(1)] 
Take two elements $O_1, O_2 \in \cA$.
The commutator in $\cA$ is computed by the following factorization product
\[
[O_1, O_2] = O_1(0) \cdot O_2(t) - O_1(0) \cdot O_2(-t) 
\]
for any $t > 0$.
Here, $\cdot$ denotes the factorization product. 

\item[(2)] We will compute this commutator using the notion of an ``operator product expansion".
Consider the map
\[
t \mapsto O_1(0) \cdot O_2(t) .
\]
This is a locally constant function of the variable $t \ne 0$.
The operator product expansion picks out the singular part of this function.
There is a natural basis for the singular locally constant functions on $\RR \setminus 0$ given by the sign function ${\rm Sign}(t)$. 
We see that the commutator is related to the OPE by
\[
O_1(0) \cdot O_2(t) = {\rm Sign}(t) [O_1, O_2] .
\] 

\item[(3)] We will compute the semi-classical, or first-order, deformation of the commutator, and hence the semi-classical OPE. 
We define $\{A,B\}_{OPE} = \lim_{\hbar \to 0} \hbar^{-1} A \cdot B$. 
Then
\[
\{O_1(0), O_2(t)\}_{OPE} = {\rm Sign}(t) \{O_1, O_2\} 
\]
where $\{O_1,O_2\}$ is the bracket in $\cA$ measuring the first order deformation to be commutative. 

\item[(4)] 
The semi-classical OPE is compute as a sum over tree diagrams
\[
\{O_1(0), O_2(t)\}_{OPE} = \sum_{\Gamma} W_\Gamma
\]
where the sum is over irreducible trees with two special vertices labeled by the linear functionals $O_1(0)$ and $O_2(t)$. 
All other vertices are labeled by the classical interaction and all internal edges are labeled by the propagator $P(\Phi)$ for any parametrix $\Phi$. 
It is a general fact that the singular part of the OPE is independent of the parametrix chosen. 
\end{itemize}

\subsection{Diagram counting}

\subsection{Feynman diagram calculation}

In the last section we saw that the only diagram that contributes to the OPE is the tree with three vertices and two internal edges, see Figure \ref{??}. 
Two of the vertices are labeled by $O_1(0)$ and $O_2(t)$, respectively. 
The final vertex is labeled by the classical Chern-Simons interaction $I = \int \beta \wedge [\alpha, \alpha]$ .
The edges are labeled by the propagator $P_{0 < L}$.

The external edge takes as an input a field $\beta \in \Omega^{1,*}(\CC) \Hat{\otimes}_\pi \Omega^*(\RR) \otimes \fg$. 
For the weight to be nonzero, we must evaluate on the lowest component $\beta \in \Omega^{1,0}(\CC) \Hat{\otimes}_\pi \Omega^*(\RR) \otimes \fg$ which we take to be of the form,
\[
\beta = f(z,t) \d z \otimes Z 
\]
where $f \in C^\infty(\CC \times \RR)$ and $Z \in \fg$. 

The algebraic contribution of the weight, evaluated on such a field, is simply $\langle Z, [X,Y]\rangle$. 

The contribution from the analytic part of the diagram is 
\[
W^{an}_{\epsilon <L} (f) = \int_{(z,t) \in \CC \times \RR} f(z,t) \d z \wedge P_{\epsilon <L} (z,t) \wedge P_{\epsilon < L}(z,t-s)  .
\]
This weight is a continuous function of the input function $f \in C^\infty(\CC \times \RR)$. 
Further, in $(\delbar + \d)$-cohomology we know $f$ must be holomorphic as a function on $\CC$, and constant in the $\RR$ direction. 
Together, these facts allow us to assume that $f$ is polynomial, say $f = z^\ell$. 

\begin{figure}
\begin{center}
\begin{tikzpicture}[line width=.2mm, scale=1]

\draw[fill=black] (0,0) circle (0.05);
\draw[fill=black] (-2,0) circle (0.05);
\draw[fill=black] (2,0) circle (0.05);
 
 \draw[fermion] (-2,0) -- (0,0);
 \draw[fermion] (2,0) -- (0,0);
 \draw[fermion] (0,2) -- (0,0);
 
 \node at (0,2.3) {$\beta$};
 \node at (-1, 0.5) {$P_L$};
 \node at (1, 0.5) {$P_L$};
 \node at (0,-0.25) {$I$};
 \node at (-2, -0.5) {$O_1(0)$};
 \node at (2, -0.5) {$O_2(t_0)$};
 
 \end{tikzpicture}
 \end{center}
 \end{figure}

\end{document}